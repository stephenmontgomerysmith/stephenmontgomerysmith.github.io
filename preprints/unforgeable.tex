\documentclass[12pt]{article}
\usepackage{amsmath,amssymb}

\begin{document}

\title{Unforgeable Marker Sequences}

\author{
D.J. Greaves\\
University of Cambridge Computer Laboratory\\
New Museums Site, Pembroke Street\\
Cambridge CB2 3QG, United Kingdom.\\
\\
S.J. Montgomery-Smith%
\thanks{The second named author was partially supported by the National
Science Foundation}\\
Department of Mathematics\\
University of Missouri\\
Columbia, MO 65211, U.S.A.}

\date{}

\maketitle

A binary number of $n$ bits consists of an ordered sequence of $n$
digits taken from the set $\{0, 1 \}$.  A sequence is said to be an
unforgeable marker if all subsequences of consecutive digits starting
at the left-hand end are dissimilar from the sequence of the same
length which ends at the right-hand end.  Unforgeable marker sequences
are so called because, when misaligned in a shift-register or other
view port of the correct length, there is no possibility of adjacent
random digits impersonating the true sequence.  Such sequences are
used for frame alignment purposes in serial data communications
systems \cite{dig}. In many communications systems, the unforgeable marker is
also a `comma sequence' in that it is guaranteed not to occur in any
aligned or misaligned view of the data between the markers, but this
relies on constraints on the data and does not impact on whether a
sequence is an unforgeable marker or not

A sequence which is unforgeable with respect to right-hand misalignment
is also unforgeable with respect to left-hand misalignment, and so
there is a single set of unforgeable sequences of a given length. 

It is useful to denote a particular sequence by a decimal number
by weighting its digits with power of 2 in the conventional way.
We can then concisely tabulate the unforgeable sequences for $n$ up to 6.

\begin{tabular}{ll}
$n = 1$: & \{0,\,1\}\\
$n = 2$: & \{1,\,2\}\\
$n = 3$: & \{1,\,3,\,4,\,6\}\\
$n = 4$: & \{1,\,3,\,7,\,8,\,12,\,14\}\\
$n = 5$: & \{1,\,3,\,5,\,7,\,11,\,15,\,16,\,20,\,24,\,26,\,28,\,30\}\\
$n = 6$: & \{1,\,3,\,5,\,7,\,11,\,13,\,15,\,19,\,23,\,31,\,32,\,40,\,44,\,48,\,50,\,52,\,56,\,58,\,60,\,62\}\\
\end{tabular}

\noindent
There is an even number of elements in each set because if 
a number is unforgeable, so is the sequence obtained by complementing
every digit.  This property is useful in NRZI (non-return to zero
invert on ones) modulation since, in general, the absolute polarity
of the sequence will not be known by the decoding hardware.  

\subsection*{Number of Unforgeable Sequences}

Let us write $F(n)$ for the number of unforgeable sequences of length
$n$.  We give a table of values for $F(n)$ for values of $n$ between
1 and 20.

It is also of interest to consider what percentage of all sequences
of length $n$ are unforgeable, and we will denote this proportion
by $\rho(n)$.  Thus $\rho(n) = F(n)/2^n$.
\bigskip

\hfill
\begin{tabular}{|c|c|c|}
\hline
$n$ & $F(n)$ & $\rho(n)$ \\
\hline
1 & 2 & 100.00\% \\
2 & 2 &  50.00\% \\
3 & 4 &  50.00\% \\
4 & 6 &  37.50\% \\
5 & 12 &  37.50\% \\
6 & 20 &  31.25\% \\
7 & 40 &  31.25\% \\
8 & 74 &  28.91\% \\
9 & 148 &  28.91\% \\
10 & 284 &  27.73\% \\
\hline
\end{tabular}
\hfill
\begin{tabular}{|c|c|c|}
\hline
$n$ & $F(n)$ & $\rho(n)$ \\
\hline
11 & 568 &  27.73\% \\
12 & 1116 &  27.25\% \\
13 & 2232 &  27.25\% \\
14 & 4424 &  27.00\% \\
15 & 8848 &  27.00\% \\
16 & 17622 &  26.89\% \\
17 & 35244 &  26.89\% \\
18 & 70340 &  26.83\% \\
19 & 140680 &  26.83\% \\
20 & 281076 &  26.81\% \\
\hline
\end{tabular}
\hfill\null

\bigskip\noindent
We see that the $\rho(n)$ is decreasing as 
$n$ gets larger, and that it seems to converge to a 
non-zero limit
as $n$ tends to infinity.  In fact, as we shall show, this
is indeed the case, and the limit is about
26.78\%.  We give a very precise value for this limit
in the appendix.

\subsection*{Calculation of the Number of Unforgeable Sequences}

We have a recurrence relation for $F(n)$ given by

\[
F(n) = \left\{
\begin{tabular}{ll}
  $2$ & for $n=1$ \\
  $2 F(n-1) - F(n/2)$ & for $n$ even \\
  $2 F(n-1)$ & otherwise. \\
\end{tabular}
\right.
\]
The proof is by induction on $n$.  The cases $n=1$ and $n=2$ are taken from
the previous tabulation.  Let us assume that the recurrence relation
holds for $n-2$ and $n-1$.

If $n$ is odd, we can put $n = 2m + 1$.  Now the sequence of length n
\[
  x_1, x_2, ..., x_{m-1}, x_m, x_{m+1}, ..., x_n
\]
is not forgeable if and only if the sequence of length $n-1$
\[
  x_1, x_2, ..., x_{m-1}, x_{m+1}, ..., x_n
\]
is not forgeable (where $x_m$ can be either 0 or 1).  Therefore there
are two unforgeable sequences of length $n$ for every unforgeable
sequence of length $n-1$, that is, $F(n) = 2 F(n-2)$.

If $n$ is even, we can put $n = 2 m$.  It is sufficient to show that
\[
 F(n) = 4 F(n-2) - F(n/2)
\]
It is clear that the sequence of length n
\[
  x_1, x_2, ..., x_{m-1}, x_m, x_{m+1}, ..., x_n
\]
is not forgeable if and only if both
\begin{enumerate}
\item \label{cond1}
the sequence of length $n-2$
\[
  x_1, x_2, ..., x_{m-1}, x_{m+1}, ..., x_n
\]
is not forgeable and,
\item \label{cond2}
the two sequences of length $m$
\[
  x_1, x_2, ..., x_{m-1}  \hbox{\rm\quad and\quad} x_{m+1}, ..., x_n
\]
are not the same.
\end{enumerate}
The number of times condition~(\ref{cond1}) is satisfied is $4 F(n-2)$ since
$x_m$ and $x_{m+1}$ may freely range over the values 0 and 1.  So it only
remains to show that the number of times condition~(\ref{cond1}) is satisfied
while condition~(\ref{cond2}) is not, is given by $F(n/2)$.  However, 
condition~(\ref{cond1}) is true
while condition~(\ref{cond2}) is not if and only if the following two 
subsidiary conditions are met
\begin{enumerate}
\item
$x_1, x_2, ..., x_{m-1}$ and $x_{m+1}, ..., x_n$
are the same, and
\item
the sequence $x_1, x_2, ..., x_{m} $ is unforgeable.
\end{enumerate}
This happens $F(m)$ times as $  x_1, x_2, ..., x_{m}$ range over all possible
values, and $F(m) = F(n/2)$.

\subsection*{Generating Sets of Unforgeable Sequences}

A side-product of this proof is a straightforward procedure for generating
unforgeable sequences.

If $X(n)$ is the set of unforgeable sequences of length $n$, then for odd $n$ we
can generate $X(n)$ from $X(n-1)$ as follows.  There are twice as many elements
in  $X(n)$ as $X(n-1)$ and they can be generated by taking each sequence from
$X(n-1)$, splitting it in half and then twice rebuilding it, once with
a zero in the middle and once with a one.

For even $n$, we can generate $X(n)$ from $X(n-2)$ by splitting
each sequence from $X(n-2)$ and putting a two digit sequence between
the two halves. This gives four times as many elements as in $X(n-2)$,
but some of the new ones are forgeable and must be removed.  These are
the ones where the first half of the sequence is the same as the second half
of the sequence or the half sequences themselves are forgeable (not elements of 
$X(n/2)$).

\subsection*{A Convenient Test for Unforgeability}

This C routine gives returns a non-zero value 
if the sequence held in
the low order {\tt bitwidth} bits of {\tt x} is unforgeable.


\begin{verbatim}
  int unforgeable(long x, int bitwidth)
  {
    long masklo, i, lo=x, hi=x;
    masklo = (1<<bitwidth)-1;
    for(i=1;i<bitwidth;i++)
    {
      masklo >>= 1;
      lo = lo & masklo;
      hi = (hi>>1) & masklo;
      if (hi==lo)
        return 0;
    }
    return 1;
  }
\end{verbatim}

\subsection*{Appendix: Derivation of the ``26.78\%''}

We observed in the table above that $\rho(n)$ converges to a limit,
which we will denote $\rho$.  In fact, $\rho$ can
be calculated to a very high precision using the following, 
rapidly converging, infinite series:
\begin{eqnarray*} 
  \rho 
  &=& 
  \frac27 - \frac{2\cdot8}{7\cdot127} + 
  \frac{2\cdot8\cdot128}{7\cdot127\cdot32767} 
  - \frac{2\cdot8\cdot128\cdot32768}{7\cdot127\cdot32767\cdot(2^{31}-1)}
  \cdots \\
  &=&
  \sum_{n=1}^\infty (-1)^{n-1}
  \left(\frac2{2^{2^{n+1}-1}-1}\right)
  \left(\prod_{m=2}^n \frac{2^{2^m-1}}{2^{2^m-1}-1}\right) \\
  &\approx&
  0.26778684021788911237667140358430255255505989799348
\end{eqnarray*}
We will prove this formula.
First, an easy argument shows that the sequence $\rho(n)$ is a 
decreasing sequence, bounded below by zero.  
Hence the limit $\rho$ exists.  Now define 
a generating function
$$ h(x) = \sum_{n=1}^\infty F(n) x^n .$$
It is clear that $h(x)$ has radius of convergence at least $1/2$.

Next, we show that
\[ \rho = \lim_{x\nearrow1/2} (1-2x) h(x) .\]
To see this, multiply out to obtain
\[ (1-2x) h(x) = F(1) x + \sum_{n=2}^\infty (F(n) - 2 F(n-1)) x^n ,\]
and thus
\[ \lim_{x\nearrow1/2} (1-2x) h(x) = \rho(1)
   + \sum_{n=2}^\infty (\rho(n) - \rho(n-1)) ,\]
which may be seen to be a telescoping series that converges to $\rho$.

Applying the recurrence relation for the sequence $F(n)$,
we see that
\begin{eqnarray*} 
  h(x)
  &=& 
  2x +
  \sum_{n=2}^\infty 2F(n-1) x^n - 
  \sum_{\substack{n=2\\\text{$n$ even}}}^\infty F(n/2) x^n \\
  &=&
  2x +
  \sum_{n=1}^\infty 2 F(n) x^{n+1} - \sum_{n=1}^\infty F(n) x^{2n} \\
  &=&
  2x + 2x h(x) - h(x^2) ,
\end{eqnarray*}
that is
$$ (1-2x) h(x) = 2x - h(x^2) .$$
We see immediately that
$$ \rho = \lim_{x\nearrow1/2} (1-2x) h(x) = 1 - h(\textstyle \frac14) .$$
Applying this relation again, we get that
$$ \textstyle h(\frac14) = 1 - 2 h(\frac1{16}) .$$
Again, we see that
\begin{eqnarray*}
   \textstyle
   h(\frac1{16}) 
   &=& 
   \textstyle
   \frac87(\frac18-h(\frac1{256})) \\
   &=&
   \textstyle
   \frac87(\frac18-\frac{128}{127}(\frac1{128}-h(\frac1{65536}))) \\
   &=&
   \textstyle
   \frac87(\frac18-\frac{128}{127}(\frac1{128}-
   \frac{32768}{32767}(\frac1{3276}-h(2^{-32}))))
\end{eqnarray*}
Continuing in this way, and multiplying out, we see that
\begin{eqnarray*}
  \rho &=& 
  \sum_{n=1}^N (-1)^{n-1}
  \left(\frac2{2^{2^{n+1}-1}-1}\right)
  \left(\prod_{m=2}^n \frac{2^{2^m-1}}{2^{2^m-1}-1}\right) \\
  &+& (-1)^N 2 h(2^{-2^{N+2}})
  \left(\prod_{m=2}^{N+1} \frac{2^{2^m-1}}{2^{2^m-1}-1}\right)  .
\end{eqnarray*}
Letting $N\to\infty$, and noting that $h(2^{-2^{N+2}}) \to h(0)=0$, 
it is readily seen that the last term in the above expression
converges to zero, and hence we obtain the formula for $\rho$ that we 
stated above.

\begin{thebibliography}{999}

\bibitem{dig} P.~Bylanski and D.G.W.~Ingram, Digital transmission systems,
Stevenage Eng.: P.~Peregrinus on behalf of the Institution of 
Electrical Engineers, 1980

\end{thebibliography}

\end{document}
