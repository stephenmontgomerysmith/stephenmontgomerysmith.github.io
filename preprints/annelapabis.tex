
%
%	Template file generated by TeXmenu v2.2, June 1992, patchlevel 431.
%

\input amstex
\documentstyle{amsppt}
\topmatter
\title  Vector-valued weakly analytic measures \endtitle
\author Nakhl\'e H. Asmar, Annela R. Kelly and Stephen
Montgomery-Smith\endauthor
\thanks This research is part of the second author's doctoral
thesis. \endthanks
\endtopmatter

\document


\def\msp{\Omega}
\def\malg{ {\Cal F} }
\def\univf{ {\Cal U} }
\def\haar{\lambda}
\def\gdfncs{{\eusm R}}
\def\xposrl{\overline{\Bbb R}_+ }

\def\tT{\widehat T}
\def\tR{\widehat R}

\def\C{\Bbb C}
\def\N{\Bbb N}
\def\R{\Bbb R}
\def\T{\Bbb T}
\def\Z{\Bbb Z}


\def\Sumspc#1{{\eusm S}^{#1}(\Lambda, I)}
\def\Lspc#1{L^{#1}({\Omega})}
\def\Lpspc#1{L^{#1}({\Omega},\mu,\Bbb F)}
\def\LBspc#1{L^{#1}({\Omega},\mu,X)}
\def\Wkspc#1{L^{#1}_{X^*}(\Omega_0,X)}



\def\prf{ \noindent {\it Proof.}\ \ }
\def\endprf{\hfill $\blacksquare$}


\def\M{M(\msp, Y^*)}
\def\minM{$\mu\in M(\msp, Y^*)$}
\def\Mstuff{$M(\msp, Y^*)$}
\def\MX{M(\msp, X)}
\def\mM{\mu\in M(\msp, Y^*)}


\loadeusm
\loadmsbm
\magnification=\magstep1
\baselineskip=21pt

\specialhead 1. Introduction\endspecialhead
   In this paper we study properties of weakly analytic
vector-valued measures.
 To motivate the discussion, let us start with some  relevant
results  for
 scalar-valued measures.
 A fundamental result in harmonic analysis, the F. and M. Riesz
 Theorem,
states that if a complex Borel measure  $\mu$ on the unit circle $\T$
(in symbols, $\mu\in M(\T)$) is
 analytic that is,
$$\int_{-\pi}^{\pi} e^{-int} d\mu (t) =0, \text{for all } n<0,$$
then $\mu$ is absolutely continuous with respect to Lebesgue measure
$m$, that is $\mu \ll m$.
Furthermore,  the classical result of Raikov  and Plessner [10],
 shows that  $\mu \ll m$ if and only if the measure $\mu$
translates continuously.  That is the mapping
$   t \to \mu(\cdot  +t) \text{ is continuous}$
from $\R$ into $M(\T)$.
Hence, if $\mu$ is analytic, then
$   t \to \mu(\cdot  +t) \text{ is continuous}.$

The F. and M. Riesz Theorem has since been generalized to
groups and measure spaces by Helson and Lowdenslager [6],
de Leeuw and Glicksberg [3],
Forelli [5], Yamaguchi [14], and many other authors.

In recent work in [1], it is shown that under certain conditions on
 $T$, where $T=(T_t)_{ t\in \R}$ is  a collection of uniformly bounded
invertible isomorphisms of the space of measures on a given measure
space,
 if a measure
$\mu$ is analytic, in some weak sense,
then the mapping $t\to T_t \mu$ is continuous.

   However, not as much is known for vector-valued measures.
First, let us consider measures on the unit circle with values in
some dual space $X=Y^*$, that is measures in $M(\T,Y^*)$. R. Ryan
[12] obtained
that if $\mu$ is  weakly analytic i.e.
 $\int_{-\pi}^{\pi}\,e^{-int}\,d\,\langle y,\mu (t)\rangle=0,$
$\text{for all }
 n<0,\,\text{ for all } y\in Y$, then $\mu \ll \lambda$. Unlike the
scalar
case, there exist weakly analytic vector-valued measures that do
not translate
continuously.
As a corollary of our main result, we show that if we further
assume that $Y^*$
has the analytic Radon-Nikod\'ym property (ARNP), then every weakly
analytic measure translates continuously.


   We start with some definitions and notation.
   Denote  the real numbers, the complex numbers  and 	the circle
group $\{
e^{it}: 0\le t<2\pi\}$ by $\R , \C$  and $\T$ respectively. Let
$M(\R)$ be the
Banach space of complex  regular Borel measures on $\R$.
Denote by $ L^{1}({\R})$ the space of Lebesgue integrable functions
and by
$L^{\infty}({\R})$ the space of essentially bounded  Lebesgue measurable
functions. Define the spaces $H^{1}({\R})$ and $H^{\infty}({\R})$
as follows:
$$H^{1}({\R})= \{ f\in L^1(\R ) :\hat  f  (s)=0, s<0\},$$
and
$$H^{\infty}({\R})=\{ f\in L^{\infty}({\R}): \int_{\R} f(t)g(t) dt=0,
\text{ for all }g\in H^{1}({\R})\}.$$
   Throughout this paper, let $\Sigma$  denote a $\sigma$-algebra
of subsets
 of a set $\Omega$, and $X$ will be an arbitrary Banach space
with norm denoted $\|\cdot\|$. If $\mu$\ is a positive
measure on $\Omega$, then a vector-valued function\
$f:\msp \to X$\ is said to be {\it Bochner\/} (or {\it strongly\/})
{\it $\mu$-measurable\/} if there exists a sequence\ $\{f_n\}$\ of
$X$-valued simple functions on $\Omega$ such that\
$\lim\limits_{n\to\infty} f_n(\omega) = f(\omega)$\
$\mu$-a\.e\.~ on $\msp$.
Let $\LBspc 1$ denote the Banach space of all strongly
measurable functions which satisfy
$\int_{\Omega}\|f(\omega)\|\,d\mu(\omega) < \infty$, with norm
$\|f\|=\int_{\Omega}\|f(\omega)\|\,d\mu(\omega)$.
Such functions $f$ are always limits, in the norm of
$L^1(X)$, of simple functions,
and this allows us to define the {\it Bochner integral\/} of $f$,
$\int_\Omega f(\omega)\,d\mu(\omega)$.
For the remainder of this paper we will only consider Bochner integrals.
A function $f:\Omega\to X$ is called {\it weakly measurable\/} if for
each $x^*\in X^*$, the map $t\mapsto x^*(f(t))$ is a  scalar-valued
measurable function.

   Also, denote   by $\MX$  the space of countably additive $X$-valued
measures of bounded variation on  $(\Omega, \Sigma)$, with the
1-variation
norm $\| \mu \|_1 < \infty$ where
$$ \| \mu \|_1 = \sup_{\pi} \sum_{B\in\pi}\mid\mid \mu(B)\mid\mid_{X} ,$$
and the supremum is over all  finite  measurable partitions $\pi$
of $\Sigma$.
   We obtain results for $\MX$, where  $X$ is a dual space, that is
$X=Y^*$,
for some Banach space $Y$.  So for the rest of the paper we will consider
only  \Mstuff.
%%%%%%%%%%%%%%

 From now on, let $T=(T_t)_{t\in \R}$ denote a collection of
uniformly bounded invertible isomorphisms of $\M \,$  such that
            $$\sup_{t\in \R}\|T_t^{\pm 1}\| \le c,\tag1.1 $$
where $c$ is a positive constant.

We can now turn to the notion of analyticity
for measures.  Two definitions are needed.

\definition { Definition 1.1}   Let $(T_t)_{t \in \R}\,$  be a uniformly
bounded
collection of   isomorphisms $\,$  of $\, \M $.
A measure $\mM $  is called {\it weakly measurable\/} if for every
$A\in\Sigma$, the map
$t\to T_t \mu(A)$ is Bochner measurable.\enddefinition

\definition  { Definition 1.2}
Let $(T_t)_{t\in \R}$  be a uniformly bounded
collection of isomorphisms of $\M$.  A weakly measurable $\mM$
is called  {\it weakly $T$-analytic\/} (or simply {\it weakly analytic})
if for all $A\in \Sigma$ and for all $y\in Y$, the map $t\to
y(T_t\mu(A))$ is
in
$H^{\infty}(\R)$.
           \enddefinition

We will repeatedly use the following two lemmas.

\proclaim {Lemma 1.3 ([7], Corollary to Theorem 7.5.11)}
If $T:X\to Y$  is a bounded linear operator, and $f:\Omega \to X$
is a Bochner
integrable function with respect to a positive measure $\lambda$,
then $Tf$ is Bochner integrable with respect to $\lambda$, and
$$ \int_{\Omega} T\biggl (f(t)\biggr) d \lambda (t) =
   T \biggl(\int_{\Omega} f(t) d\lambda (t) \biggr).\tag1.2$$
\endproclaim

\proclaim { Lemma 1.4}
Let $T = (T_t)_{t\in \R}$  be a uniformly bounded
collection of isomorphisms of $\M$, and suppose that
$\mM$  is weakly measurable with respect to $T$.  Then for all $f\in
L_1(\R)$,
and all $y \in Y$,
$$ \int_{\R} y\bigg(T_t \mu(A)\bigg)  f(t) dt =
y\biggl(\int_{\R} T_t \mu(A) f(t) dt\biggr).\tag1.3$$
\endproclaim
\prf  Since
$$\int_{\Omega} \|y\big (T_t \mu (A)\big) f(t)\| d t \le \|y\| \| T\|\
\|\mu\|_1 \|f\| $$
then $T_t \mu(A)$ is Bochner integrable and the result follows from
Lemma 1.3.\endprf

The following is a fundamental property
that was introduced in [1] for
scalar valued measures.

  \definition { Definition 1.5} Let
$T = (T_t)_{t\in \R}$  be a uniformly bounded
collection of isomorphisms of $\M$.
Then we say that $T$ satisfies {\it hypothesis (A)\/} if
whenever  $\mM \,$ is  weakly analytic and is such that
for every $A\in \Sigma$ we have
$T_t \mu (A)=0$ for almost  all $t\in \R$,
then $\mu$ is the zero measure. \enddefinition

 Next we introduce a property that has been studied extensively by
several
mathematicians (for instance see [2],[8]). One way of defining it is the
following.
 Let $\lambda$ denote the normalized Haar measure on $\T$.

  \definition { Definition 1.6} A complex Banach space $X$ is said
to have
the {\it analytic Radon-Nikod\'ym property (ARNP)\/}
if  every  measure $\mu \in M(\T, X), $  such that
$$\int_{\T} e^{-int} d\mu (t) =0,\,\,\, n<0$$
has a Radon-Nikod\'ym derivative in $L^1 (\T, \lambda, X)$.
\enddefinition

   The following theorem shows that ARNP passes from $X$ to
$M(\Omega ,X)$
if $X$ is a dual space.

\proclaim{ Theorem 1.7 [11] } The Banach space  $\M$ has the ARNP
whenever
$Y^*$ does.\endproclaim



\specialhead
2. Main Lemma
\endspecialhead

First we will define a subspace of ${\M}^*$, which
is in effect the simple $Y$-valued functions on $\Omega$.  This concept
was used by Talagrand [13].
Given  $B_1, \dots,B_n \in \Sigma$ and
$y_1,\dots, y_n \in Y $ define $\varphi_{y_1,
y_2,\dots,y_n}^{B_1,B_2,\dots,B_n}
\in M(\Omega,Y^*)^*$ in the following way:
 $$\varphi_{y_1, y_2,\dots,y_n}^{B_1,B_2,\dots,B_n}(\mu)=\sum_{i=1}^n
y_i(\mu(B_i)), \,\,\mu \in M(\Omega,Y^*)  .\tag2.1$$
Let $E$ denote the collection of all such linear functionals.
The set $E$ is a norming subspace, that is for all $\mu\in
M(\Sigma, Y^*)$
$$\| \mu \|_1 = \sup\{ \varphi(\mu):\| \varphi \| =1, \varphi\in E\}.$$


Now we will state the principal result of this section.

\proclaim{ Lemma 2.1} Suppose $Y^*$ has ARNP. Let $E$ be as above,
and let
$f:\R\to \M $ be such that $\sup\limits_{t\in \R} \| f(t)\|_1<\infty$.
Suppose further that for all
$A\in \Sigma$, $t\to f(t)(A)$ is Bochner measurable and that for
all $y\in Y,\,$
 for all $A\in \Sigma$ the map $t\to y(f(t)(A))$ is in
$H^{\infty}(\R)$.
Then there exists a Bochner measurable essentially bounded $g:\R\to \M $
such that for all $\varphi \in E$ we have
$$ \varphi (g(t))=\varphi(f(t)), $$
for almost all $t\in \R$ .\endproclaim
\prf
Let $\Phi (z)=i{{1-z}\over{1+z}}$  be the conformal mapping of the
unit disk
onto the upper half plane, mapping $\T$  onto $\R$. Let $F=f\circ \Phi .$
Consider the expression
$$ y\biggl( \int_A F(\theta)(B) {d\theta\over{2\pi}}\biggr),$$
where $A$ is a Borel subset of $\T ,\, B\in \Sigma$.
Since $F(\theta)(B)$ is Bochner integrable, one can apply Lemma 1.4
to obtain
that $y(F(\theta)(B))$ is Lebesgue integrable and
$$y\biggl(\int_A F(\theta)(B) {{ d\theta}\over{2\pi}} \biggr)=
  \int_A y(F(\theta)(B) ){{ d\theta}\over{2\pi}}. \tag2.2$$
Using (2.2), in the notation of (2.1), we see that for all
$\varphi=\varphi_{y_1, \dots,y_n}^{B_1,\dots,B_n} \in E$:
$$y_1\bigg(\int_A F(\theta)(B_1) {{ d\theta}\over{2\pi}} \bigg)+\dots
+y_n\bigg(\int_A F(\theta)(B_n) {{ d\theta}\over{2\pi}} \bigg)=
\int_A \varphi_{y_1, \dots,y_n}^{B_1,\dots,B_n} \bigg(F(\theta)
\bigg){{ d\theta}\over{2\pi}} , \tag2.3$$
and that $\varphi(F(\theta))$ is
Lebesgue integrable for every $\varphi \in E$.
Hence, $\theta \to \varphi(F(\theta)) \in
H^{\infty}(\T)$ by $(2.3)$ and
 by the assumption that  $y\biggl(f(t)(A)\biggr)\in H^{\infty}(\R).$

  Define a scalar-valued measure $\mu_{\varphi}$ on the Borel sets
of $\T$ by
$$\mu_{\varphi}(A) = \int_A \varphi(F(\theta))
{{d\theta}\over{2\pi}}, \quad (\varphi \in E) \tag2.4$$
Then for all continuous functions $h$ on $\T$, one has that
$$\int_{\T} h(\theta) d\mu_{\varphi} (\theta) = \int_{\T} h(\theta)
\varphi(F(\theta)) {{d\theta}\over{2\pi}}.$$
   Now we will restrict to  $\varphi_y^B \in E,$ for the clarity of
the proof,
and later we will generalize the formulas for any $\varphi \in E$.
    First we want to prove that the $Y^*$-valued set function defined by:
 $$m(B)=\int_A F(\theta)(B) {{ d\theta}\over{2\pi}},\quad (B \in\Sigma)$$
is a countably additive measure of bounded variation, that is $m\in
M(\Omega, Y^*)$  .
To show that $m$ is countably additive, it is enough ([4],
Corollary 1.4.7) to
prove that
the scalar measure, defined by:
$$ \eta_y(B)= y\bigg(\int_A F(\theta)(B) {{d\theta}\over{2\pi}}\bigg), $$
 is countably additive , for every $y\in Y$.
Hence we need to show that for any sequence of disjoint sets
$\{B_k\}$ in $\Sigma$, $\bigcup_{k=1}^{\infty} \,B_k \in \Sigma$:
$$ \lim_{n\to \infty} y\bigg(\int_A F(\theta)( \bigcup_{k=1}^n  B_k)
{{d\theta}\over{2\pi}}\bigg) = y\bigg(\int_A
F(\theta)(\bigcup_{k=1}^{\infty}  B_k)
{{d\theta}\over{2\pi}}\bigg).\tag2.5$$
 Since $F(\theta) \in \M$, then
$$\lim_{n\to \infty} F(\theta)(\bigcup_{k=1}^{n}  B_k) =
   F(\theta)(\bigcup_{k=1}^{\infty}  B_k),\quad ( \theta \in \T)$$
and  also
$$\|F(\theta)(\bigcup_{k=1}^{n}  B_k) \|_{Y^*} \le
  \|F(\theta)\|_1 < \infty, \quad  (n>0). $$
Therefore one can use the Dominated Convergence Theorem ([7],
Theorem 3.7.9) to
obtain
$$\lim_{n\to \infty} \int_A F(\theta)( \bigcup_{k=1}^n  B_k)
{{d\theta}\over{2\pi}} = \int_A F(\theta)(\bigcup_{k=1}^{\infty}  B_k)
{{d\theta}\over{2\pi}}.$$
Next we can apply $y\in Y$ to both sides to get $(2.5)$, which
shows that $m$ is countably additive.
All that is left to show is that $\|m\|_1< \infty$:
$$\split \|m\|_1 &= \sup_{\pi} \sum_{B\in\Sigma}\mid\mid
m(B)\mid\mid_{Y^*} =
\sup_{\pi} \sum_{B\in\Sigma}\mid\mid \int_A
F(\theta)(B){{d\theta}\over{2\pi}}\mid\mid_{Y^*}\\
& \le
\sup_{\pi} \sum_{B\in\Sigma} \int_A
\|F(\theta)(B)\|_{Y^*}{{d\theta}\over{2\pi}} \\
&\le\sup_{\pi}  \int_A \sum_{B\in\Sigma}
\|F(\theta)(B)\|_{Y^*}{{d\theta}\over{2\pi}}\\
 & \le
\sup_{\pi}  \int_A  \|F(\theta)(B)\|_{1}{{d\theta}\over{2\pi}}\\
& \le\sup_{\theta \in \T}  \|F(\theta)\|_1 <\infty .\endsplit   $$
Thus we have shown that $m\in \M$.
 Denoting $\mu (A)= m\in M(\Omega,Y^*)$, it follows from $(2.3)$
and $(2.4)$
that
$$\mu_{\varphi_y^B}(A)=y(m(B)) =\varphi_y^B (m) = \varphi_y^B(\mu(A)).$$
   Similarly,  for any $\varphi \in E$
using the definition of $\mu_\varphi (A)$, and the formula $(2.3)$
we can show that
$$\mu_{\varphi}(A) = \varphi\biggl( \mu(A)\biggr).\tag2.6$$
  We show  now that the above defined $\mu$ is a $\M$-valued  countably
additive measure  of bounded variation. First,  the fact that
$\mu $ is countably additive  follows  easily from the
definition  of $\mu$.
Next, we will obtain
the inequality:
$$\| \mu (A)\|_1 \le \int_A \|  F(\theta)
\|_1{{d\theta}\over{2\pi}}. \tag2.7$$
This follows because $E$ is a norming subspace and hence, given
$A\in \Sigma$ and $\epsilon >0$, there exists $\varphi \in E,$ with
$\|\varphi\|
\le 1$ and $\|\mu(A)\|_1 \le |\varphi (\mu(A))|+\epsilon$, and
hence (2.7) follows since
$$|\varphi(\mu(A))|= |\int_A \varphi(F(\theta)){{d\theta}\over{2\pi}}|
\le \int_A\| F(\theta)\|_1 {{d\theta}\over{2\pi}}.$$
Therefore we have established that $\mu$ is a
measure  of bounded variation, that is $\mu \in M(\T, M(\Omega, Y^*))$.

Next we show that
$$\int_{\T} e^{-i n \theta} d\mu(\theta) =0, \quad (n<0).$$
This follows because $E$ is norming, and if $\varphi\in E$ and
$n<0$, then
$$\varphi \biggl( \int_{\T} e^{-i n \theta} d\mu(\theta)\biggr) =
\int_{\T} e^{-i n \theta} d\mu_{\varphi} (\theta) = \int_{\T} e^{-i
n \theta} \varphi\biggl( F(\theta)\biggr) {{d\theta}\over {2\pi}}
=0.$$
Since we assume that $\M$ has ARNP, one can find a Bochner integrable
function      $G\in L^1(\T,\M)$     such that
$$\mu(A) = \int_A G(\theta)  {{d\theta}\over {2\pi}},\tag2.8$$
for all Borel subsets $A$ of $\T$.  Using $(2.4),\,(2.6),\,$ and
$(2.8)$ one gets for all $A\in \Cal B$, $\varphi \in E$,
$$ \int_A \varphi \biggl( G(\theta)\biggr) {{d\theta}\over
{2\pi}}= \varphi(\mu(A)) = \mu_{\varphi}(A) = \int_A \varphi
\biggl( F(\theta)\biggr) {{d\theta}\over {2\pi}}.$$
Since this is true for all $\varphi \in E$ and $A\in \Cal B$, one can
conclude that for given  $\varphi \in E$,
$$ \varphi (G(\theta)) = \varphi (F(\theta))
   \quad\text{for almost all $\theta$.} $$
Since we have $(2.7),$ and $(2.8)$, we can apply Lemma 2.3 in [1], from
which it follows that $G$ is essentially
bounded.  Let $g(t)=G(\Phi^{-1} (t))$. Then $g$ is a Bochner
measurable, essentially bounded function and for every $\varphi
\in E$, $$\varphi(g(t))= \varphi(G(\Phi^{-1} (t)))=\varphi(F(\Phi^{-1}
(t)))=\varphi(f(t)),\text{for almost all } t\in \R
  $$
completing the proof of the lemma.\endprf

\bigskip

   We will use the lemma in the following setting. Let
$T=\{T_t\}_{t\in \R}$ be
a family of uniformly bounded isomorphisms of $\M$ such that
$\|T_t\| \le c$.
Suppose  that $\mM$ is weakly analytic and let $f(t)=T_t \mu$
for $t\in \R$.
Then we have that $\| f(t)\|_1 \le c\| \mu\|_1$, and that
the map $t\to y\biggl( f(t)(A)\biggr) \in H^{\infty} (\R)$ for all
$ y\in Y$
and for all $A\in \Sigma.$

  \proclaim {Corollary 2.2} Let $Y^*$ have ARNP.
Let $T=\{T_t\}_{t\in \R}$ be a family of uniformly bounded
isomorphisms of
$\M$.
Let $\mM$ be a weakly analytic measure. Then there exists a Bochner
measurable
essentially bounded function $g:\R \to \M$ such that for all $A\in
\Sigma$,
$$g(t)(A)= T_t \mu(A) \quad \text{for almost all $t\in \R$.} $$
\endproclaim

\remark {Remark 2.3} Since the results in this section hold for
Banach spaces
which are dual spaces and have ARNP,  we give some 
examples of
these spaces.
 Examples of Banach spaces which are dual spaces and have ARNP are:
\roster
\item
Orlicz spaces $L^{\Phi}(\mu)$, where $(\Omega,\Sigma,\mu)$ is a
$\sigma$-finite measure space, and $\Phi$ is an Orlicz function
for which it and its Young's complementary function
satisfy the $\Delta_2$-condition (see [4] and [9]).
\item All dual spaces that are Banach lattices not containing  $c_0$, or
preduals of a von Neuman algebra (see [8]).
\endroster\endremark

\specialhead
3. Bochner measurability of weakly analytic measures
\endspecialhead

For all the results of this section, which include
the main results of this paper, we will suppose that
$T=(T_t)_{ t\in \R}$ is a
one-parameter group of  uniformly bounded invertible isomorphisms of
$M(\Sigma)$ for which $(1.1)$ holds.
We will also suppose that
$$T_t^* : E\to \bar E,\tag3.1 $$
where $\bar E$ denotes the closure of $E$ in $M(\Omega,Y^*)^*$.

We begin with some properties of the convolution, which are
technical, but
needed for proving the main theorem.

        Consider  a weakly measurable  $\mu \in \M$ and $\nu \in
M(\R)$, where
$\nu$ is absolutely continuous with respect to Lebesgue measure.  Define:
$$ \nu \ast_T  \mu (A)= \int_{\R} T_{-t} \mu(A) d\nu(t).\tag3.2 $$
When there is no risk of confusion, we will simply write
$ \nu \ast  \mu$ for $ \nu \ast_T  \mu$.
The integral is well defined since $T_{-t} \mu (A)$ is
$\nu$-measurable and
bounded, hence $\nu$-integrable. One can check that
the formula defines a countably additive vector-valued measure of
bounded variation, using  properties of Bochner integral and the
fact that
$\mu\in \M$.  Hence, $\nu \ast_T  \mu \in \M$.

 From  now on assume that  $\nu, \sigma \in M(\R)$ and let
$ \mu \in \M $ be weakly measurable.
 Let $\nu, \sigma$ be absolutely continuous with respect to
Lebesgue measure.
Let $E$ be defined as in the previous section.

  \proclaim{ Lemma 3.1}
Suppose
that $\varphi \in \bar E$.
Then the mapping $t\to \varphi (T_t \mu)$ is
Lebesgue measurable on $\R$. Futhermore,
$$\int_{\R} \varphi (T_{-s}\mu) d\nu(s) = \varphi (\nu \ast_T  \mu
).$$\endproclaim
 \prf
It is sufficient to prove it in the case that $\varphi \in E$.
The mapping $t\to \varphi (T_t \mu)$  is Lebesgue measurable,
because $\mu$ is
weakly measurable.
   Consider $\varphi_y^A \in E$. Then
$$\int_{\R} \varphi_y^A (T_{-s}\mu) d\nu(s)
=y\biggr(\int_{\R}  (T_{-s}\mu)(A) d\nu(s)\biggl)
= y\biggr(\nu \ast_T  \mu (A)\biggl)
   = \varphi_y^A (\nu \ast_T  \mu  ).$$
Similarly, one can show the result holds for any $ \varphi \in E$,
so the lemma
follows.\endprf


  \proclaim {Corollary 3.2} For all $t\in \R$, we have
$$ T_t(\nu  \ast \mu) = \nu \ast (T_t\mu).$$
Moreover, the measure $\nu  \ast \mu$ is weakly measurable.
\endproclaim
\prf   Using lemma 1.3 we have that for any $A\in \Sigma , \, y\in Y$ :
$$\split y \biggr( \nu \ast (T_t\mu)(A)\biggl)&
=y\biggr(\int_{\R} T_{-s+t} \mu(A) d\nu (s)\biggl)\\
& = \int_{\R}  y\biggr(  T_{-s+t} \mu(A)\biggl)  d\nu (s) \\
&= \int_{\R} \varphi_y^A \biggr( T_t (T_{-s}\mu )\biggl) d\nu(s)\\
&=\int_{\R} T_t^* (\varphi_y^A) (T_{-s}\mu) d\nu (s).\endsplit$$
Applying the previous lemma:
$$\split \int_{\R} T_t^* (\varphi_y^A) (T_{-s}\mu) d\nu (s)
&=T_t^* (\varphi_y^A) (\nu \ast_T \mu )\\
&= \varphi_y^A \biggr(T_t( \nu \ast_T \mu )\biggl)\\
& = y\biggr(T_t (\nu\ast_T \mu) (A) \biggl),\endsplit$$
hence $\nu \ast (T_t \mu) = T_t (\nu \ast \mu).$

Next we want to show that $\nu\ast_T \mu $ is weakly measurable, that is
we need to prove that
$f(t)= T_t (\nu \ast \mu) (A)$
is Bochner measurable for all $A\in \Sigma$.  By the previous part
we have that
$$ f(t)= T_t (\nu \ast \mu) (A)= \nu \ast (T_t\mu) (A)= \int_{\R}
T_{t-s} \mu
(A) d\nu(s). $$
By the Pettis measurability theorem [4,Theorem II.2], we need to
show that
$f(t)$ is
weakly measurable and essentially separably valued.
 Since $T_t \mu (A)$ is Bochner measurable, there exists a sequence
of simple
functions $\{ f_n \}$ such that
 $f_n(t) \to T_t \mu (A)$ in $Y^*$ -norm. 	Let
$$f_n(t) = \sum_{i=1}^k y_{i,n}^* \chi_{A_{i,n}} (t).$$
Note that for all $n\in \N$
$$\split\nu \ast f_n(t) &= \int_{\R} \sum_{i=1}^k y_{i,n}^*
\chi_{A_{i,n}}
(t-s) d\nu (s)\\
&=\sum_{i=1}^k y_{i,n}^* \int_{\R} \chi_{A_{i,n}} (t-s)d\nu (s)\\
&=\sum_{i=1}^k y_{i,n}^* \biggl(\chi_{A_{i,n}}\ast \nu (t)
\biggr).\endsplit $$
This calculation shows that $\nu \ast f_n$ has finite dimensional range,
hence $\nu\ast f_n (\R)$ is separable for all $n\in \N$.
It can be easily checked that $ \nu\ast (T_t\mu)(A)=
\lim\limits_{n\to \infty}
(\nu\ast f_n)(t)$ in $Y^*$-norm, therefore $\nu \ast (T_t \mu)
(A)$ has separable range.
To show that  $f(t)$ is weakly measurable, we need to verify that
$y(f(t))$ is measurable for all $y\in Y$ . It follows from
Lemma 1.3 that
$$y(f(t)) = \int_{\R} y\biggl(T_{t-s} \mu (A)\biggr) d\nu(s), $$
and hence $y(f(t))$ is measurable being the convolution of a
measure in $M(\R)$
 and a bounded measurable function on $\R$.
This finishes the proof of the corollary .\endprf

\proclaim{   Corollary  3.3} With the above notation, one has
$$(\sigma \ast \nu)\ast \mu = \sigma \ast (\nu\ast \mu).$$\endproclaim
\prf   For $A\in\Sigma, y\in Y$, one can reduce it to the scalar
case using Lemma 1.3:
$$y\biggl[ (\sigma \ast \nu) \ast \mu (A)\biggr] = y\biggl[\int_{\R}
(T_{-s} \mu (A) d(\sigma \ast \nu) (s)\biggr]
=\int_{\R} y(T_{-s} \mu (A)) d(\sigma\ast\nu )(s).  $$
 Next we can use the results known in scalar case  [5, Theorem
19.10] to obtain that
$$ \int_{\R} y(T_{-s} \mu (A)) d(\sigma\ast\nu )(s)  = \int_{\R}
y\biggl(\nu\ast T_{-s}\mu (A)\biggr) d\sigma
(s).$$
Finally,  using Lemma 1.3 again gives us that:
$$ \int_{\R} y\biggl(\nu\ast T_{-s}\mu (A)\biggr) d\sigma
(s)=y\biggl ( \sigma \ast (\nu\ast \mu)(A)\biggr). $$
\endprf


Let $g$ be the Bochner measurable function defined on $\R$ with values in
$\M$ given by Corollary 2.2.
Let $\mu$ be a weakly analytic measure in $\M$.  For $y>0$, let $P_y$ be
the Poisson kernel on $\R$:
$$P_y(x) ={1\over {\pi}} {y\over{x^2 +y^2}} \quad (x\in \R).$$
We can form the Poisson integral of $g$ as follows :
$$ P_y \ast g(t) = \int_{\R} g(t-x) P_y(x) dx, $$
where the integral exists as a Bochner integral.
   \proclaim {Proposition 3.4}  We have that
$$ \lim_{y\to 0} P_y \ast g(t) = g(t)\tag3.3$$
in $\M$-norm, for almost all $t\in\R$.\endproclaim
\prf  Since $g$ is measurable and essentially bounded, the
proof is similar to the classical proof for   scalar-valued
functions. \endprf

  \proclaim{ Lemma 3.5} For all $t\in \R$, one has that
$$ P_y\ast g(t) = P_y \ast T_t\mu.$$\endproclaim
\prf For any $A\in \Sigma,\,y_1 \in Y$, one has by Lemma 1.3:
$$ y_1\biggl( P_y\ast g(t) (A)\biggr) = y_1\biggl( \int_{\R} g(s)(A)
P_y(t-s)ds\biggr) =\int_{\R} y_1\biggl(  g(s)(A)\biggr)  P_y(t-s)ds.$$
By Lemma 1.3 again,
$$\split \int_{\R} y_1\biggl(  g(s)(A)\biggr)  P_y(t-s)ds
&=\int_{\R} y_1\biggl(  T_s\mu(A)\biggr)  P_y(t-s)ds\\
&=\int_{\R} y_1\biggl(T_{t-s} \mu(A)\biggr)  P_y(s)ds\\
&=y_1\biggl( \int_{\R} T_{t-s} \mu (A) P_y(t-s)ds\biggr)\\
&=y_1\biggl( P_y\ast (T_t\mu)(A)\biggr),\endsplit$$
which completes the proof of the lemma.\endprf

\proclaim{Lemma 3.6} Let $t_0$ be any real number such that $(3.3)$
holds. Then
for all $t\in{\R}$, we have:
$$ \lim_{y\to 0} P_y\ast T_t\mu = T_{t-t_0} \biggl(g(t_0)\biggr)$$
 in $\M$-norm.\endproclaim
   \prf Since $P_y\ast g(t_0) \to g(t_0)$, it follows that
$$ T_{t-t_0} \biggl( (P_y\ast g(t_0)\biggr) \to
T_{t-t_0}\biggl(g(t_0)\biggr).$$
By Lemma 3.5 and Corollary 3.2 we get
$$T_{t-t_0} \biggl( P_y \ast g(t_0)\biggr) = T_{t-t_0} \biggl( P_y\ast
T_{t_0}\mu \biggr)= P_y\ast T_t \mu,$$
establishing the lemma. \endprf



\proclaim{ Theorem 3.7 (Main theorem)} Suppose $T=(T_t)_{t \in \R}$
satisfies hypothesis (A). Let $g$ be as
above. Then we have
$$T_t \mu = g(t),\,\, \text{for almost all } t \in \R. $$
Consequently, the mapping $t\to T_t\mu$ is Bochner
measurable.\endproclaim

\prf It is enough to show that the equality in the theorem
holds for all $t=t_0$ where $(3.3)$ holds. Fix such a $t_0$ and
let $A\in\Sigma$. Since $t\to T_t\mu(A)$ is a bounded, measurable
function on $\R$,
then, as in Proposition 3.4, one can show
that
$$ P_y\ast (T_t\mu)(A) \to T_t\mu(A),\,\, \text{for almost all }t\in
\R.$$
  By Lemma 3.6:
$$ P_y\ast (T_t \mu)(A) \to T_{t-t_0}
\biggl(g(t_0)(A)\biggr),\quad  (t\in \R).$$
Hence
$$ T_t \mu(A) = T_{t-t_0} g(t_0)(A),\,\,\text{for almost all }
t\in\R.\tag3.4$$
 From Lemma 3.5, one has that $T_{t_0} g(t_0) = \lim\limits_{t\to 0}
P_y\ast\mu$ in $\M$ norm.  Since $\mu$ is weakly analytic, it
follows that
$P_y\ast\mu$ is weakly analytic, because for all $y_1 \in Y$ and all
$A\in\Sigma$ and all $h(t)\in H^1 (\R)$ we have that
$$\split \int_{\R} y_1\biggl(T_t(P_y\ast\mu)(A)\biggr) h(t) dt
& =\int_{\R}y_1\biggl( T_t \int_{\R} T_{-s} \mu(A) P_y(s) ds
\biggr) h(t)dt \\
&=\int_{\R} \int_{\R} y_1\biggl(T_{t-s} \mu (A)\biggr) P_y(s)
h(t)ds dt\\
&=\int_{\R} \int_{\R} y_1\biggl(T_{t-s} \mu (A)\biggr) P_y(s)
h(t)dt ds\\
&=\int_{\R}  y_1\biggl(T_{-s}\int_{\R} T_t\mu (A) h(t) dt \biggr) P_y(s)
 ds =0.\endsplit $$
One can also show that $T_{-t_0} g(t_0)$ is weakly analytic,
because $\mu$ is weakly analytic, that is for all $y_1 \in Y$ and all
$A\in\Sigma$ and all $h(t)\in H^1(\R)$ we have that
$$\split \int_{\R} y_1\biggl(T_{t-t_0} g(t_0)(A)\biggr) h(t) dt
&= \int_{\R} y_1\biggl(T_{t}\{\lim_{y\to 0} P_y \ast
\mu(A)\}\biggr) h(t) dt\\
&=\int_{\R} \lim_{y\to 0} y_1\biggl( T_t \biggl(P_y\ast
\mu(A)\biggr)\biggr) h(t) dt\\
&=\lim_{y\to 0} \int_{\R} y_1\biggl( T_t \biggl(P_y\ast
\mu(A)\biggr)\biggr) h(t) dt.\endsplit$$
   Taking $t=0$ in $(3.5)$ one has that
$$ \mu(A) - T_{-t_0} g(t_0) (A) =0,\quad (A\in\Sigma).$$
 Since the measure  $\mu - T_{-t_0} g(t_0)$ is weakly analytic,
then using hypothesis $(A)$, we have that
$$       \mu = T_{-t_0} g(t_0). $$
Applying $T_{t_0}$ to both sides of this equality completes the
proof.\endprf

   \proclaim{Theorem 3.8} Let $T$ and $\mu$ be as in the  main
theorem. Then
$$ \lim_{y\to 0} P_y \ast \mu =\mu\,\text{ in the
$M$-norm}.$$\endproclaim
\prf
Let $t_0$ be such that (3.3) holds.  Then
$$T_{-t_0} (P_y\ast g(t_0))\to T_{-t_0}(g(t_0))$$
in the $\M$-norm.
By Lemma 3.2, we have that $ T_{-t_0} ( P_y \ast g(t_0)) = P_y\ast \mu$.
By Theorem 3.7 we have that $ T_{-t_0} g(t_0) =\mu$. The result
follows.\endprf

  \proclaim {Theorem 3.9} Let $\mu$ and $T$ be as in the main theorem.
Then the mapping $t\to T_t\mu$ is uniformly continuous from $\R$
into $\M$. \endproclaim
\prf For each $y>0$, consider the map $t\to P_y\ast g(t)$. Since
$g$ is essentially bounded, one can easily see that this map is
continuous. Hence by Lemma 3.2 we have  that the map $t\to P_y\ast
T_t\mu=T_t(P_y\ast\mu )$ is continuous. By Theorem 3.5. one has
that
$$ T_t (P_y\ast \mu)\to T_t \mu$$
uniformly in $t$. It follows that
$t\to T_t\mu$ is uniformly continuous.\endprf

 \definition { Definition 3.10 } Let $\mu \in M(\Omega,Y^*)$ ,
where $\Omega $
 is a $\sigma$-field and $\lambda\in M(\Omega)$. The measure $\mu$
is called
 $\lambda$-continuous (in symbols $\mu \ll\lambda$), if $\mu$
vanishes on sets
of
$|\lambda|$-measure zero. \enddefinition

   Let $\Omega =\T$, take $T_t$ to be translation, that is
$T_t\mu(\cdot) =\mu(\cdot +t)$. It is possible to show
 (see [1] and Proposition 1.7, [5]), similarly as in scalar case  that,
the definition of weak analyticity of the measure $\mu$ can be
written as a
 condition:

$$
 \int_{-\pi}^{\pi} e^{-int} d \,\langle y,\mu(t)\rangle =0, \,\forall
 n<0,\,\forall y\in Y.$$
This definition coincides with the definition of the analyticity in
[12] by R.
 Ryan.
Although in [12] it is referred to as a Pettis integral, it is an
example of
 what [4] calls a Gelfand or weak$^*$ integral.
R. Ryan [12] obtained
that if $\mu$ is  weakly analytic, then $\mu \ll \lambda$. Unlike
the scalar
case, there exist weakly analytic vector-valued measures that do
not translate
continuously. For example, let $g(t)=\exp\left({\displaystyle{-i \cot
 \displaystyle{t\over 2}}}\right).$
Then the measure defined by
$d\mu(t)=g(t-\cdot) dt,$ is a weakly
analytic measure in $M(\T,L^{\infty}(\T))$, but it does not
translate continuously, although $\mu \ll\lambda$.
We have the following as a corollary of Theorem 3.9.
\proclaim{Corollary 3.11 } Let  $\mu \in M(\T,Y^*)$. Assume that
$Y^*$ has
ARNP.
If  $\mu$ is a measure such that
$$
 \int_{-\pi}^{\pi} e^{-int} d \,\langle y,\mu  (t)\rangle =0, \,\forall
 n<0,\,\forall y\in Y.$$
then  the mapping  $t\to\mu(e^{i(\cdot +t)})$ is continuous.
\endproclaim
Hence we have obtained that if we further assume that $Y^*$
has the analytic Radon-Nikod\'ym property (ARNP), then every weakly
analytic measure translates continuously.



\Refs

\ref \no1 \by N. Asmar, S. Montgomery-Smith \paper Analytic measures and
Bochner measurability, \jour to appear in Bull. Sc. Math \endref

\ref \no2 \by A.V.Bukhvalov, A.A. Danilevich \paper Boundary
properties of
analytic and harmonic functions with values in Banach space \jour
Mat. Zametki
\vol 31 \yr 1982 \pages 203-214\transl\nofrills English translation
 \jour Mat. Notes  \vol 31 \yr 1982 \pages 104--110 \endref

\ref \no3 \by K.\ de Leeuw and I.\ Glicksberg
\paper Quasi-invariance and analyticity of measures on compact groups
 \jour Acta Math. \vol 109 \yr 1963 \pages 179--205 \endref

\ref \no4 \by J. Diestel and J.J. Uhl, Vector measures
\publ American Mathematical Society \publaddr Providence, Rhode
Island  \yr
1977 \endref

\ref \no5 \by F.Forelli \paper Analytic and quasi-invariant measures
\jour Acta
Math. \vol 118 \yr 1967 \pages 33--59 \endref

\ref \no6 \by H.\ Helson and D.\ Lowdenslager
\paper Prediction theory and Fourier series in several variables
 \jour Acta Math. \vol 99 \yr 1958 \pages 165--202 \endref

\ref \no7 \by E. Hille  \book Methods in Classical and Functional
Analysis
 \publ Addison--Wesley \publaddr Reading, Massachusetts \yr 1972 \endref

\ref \no8 \by N. Ghoussoub, B.Maurey and W. Schachermayer \paper
Pluriharmonically dentable complex Banach spaces
\jour J. reine angew. Math.\vol 402 \yr 1989 \pages 76--127  \endref

\ref \no9 \by M.M. Rao \book The Theory of Orlitz  Spaces
\publ Marcel Dekker Inc.\publaddr New York \yr 1991 \endref

\ref \no10 \by D.Raikov \paper On absolutely continuous set
functions \jour
Doklady Acad. Sci. USSR \vol34 \yr 1942 \pages 239-241 \endref

\ref \no11 \by N.Randrianantoanina and E.Saab  \paper Stability of
some types
 of Radon-Nikod\'ym Properties \jour Illinois Journal of Math.\vol 39
\yr 1995 \pages 416-430 \endref

\ref \no12 \by R.Ryan \paper The F. and M. Riesz Theorem for
vector-measures
\jour Indag.Math \vol 23 \issue 5 \yr 1963 \pages 408-412 \endref

\ref \no13 \by M.Talagrand \paper Weak Cauchy sequences in $L^1
(E)$ \jour Amer. J. Math. \vol 106 \yr 1984 \pages 703--724
\endref

\ref \no14 \by H.Yamaguchi \paper A property of some Fourier-Stieltjes
transforms \jour Pac.\ J.\ Math. \vol 108 \yr 1983 \pages 243--256
\endref

\endRefs

Department of Mathematics, University of Missouri-Columbia,
Columbia, Missouri

\noindent 65211 U\.S\.A.

\enddocument


