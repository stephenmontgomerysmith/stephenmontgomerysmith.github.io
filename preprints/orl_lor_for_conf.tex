% Typeset this file using plain-TeX

\magnification=\magstep1
\baselineskip =5mm
\lineskiplimit =1.0mm
\lineskip =1.0mm

\def\doublespacing{\baselineskip = 7.5mm}

\long\def\comment#1{}

\long\def\blankout #1\eb{}
\def\noblankout{\def\blankout{}\def\eb{}}

\let\properlbrack=\lbrack
\let\properrbrack=\rbrack
\def\ordcomma{,}
\def\ordcolon{:}
\def\ordsemicolon{;}
\def\ordleftparen{(}
\def\ordrightparen{)}
\def\ordleftbrack{\properlbrack}
\def\ordrightbrack{\properrbrack}
\def\rmcomma{\ifmmode ,\else \/{\rm ,}\fi}
\def\rmcolon{\ifmmode :\else \/{\rm :}\fi}
\def\rmsemicolon{\ifmmode ;\else \/{\rm ;}\fi}
\def\rmleftparen{\ifmmode (\else \/{\rm (}\fi}
\def\rmrightparen{\ifmmode )\else \/{\rm )}\fi}
\def\rmleftbrack{\ifmmode \properlbrack\else \/{\rm \properlbrack}\fi}
\def\rmrightbrack{\ifmmode \properrbrack\else \/{\rm \properrbrack}\fi}
\catcode`,=\active 
\catcode`:=\active 
\catcode`;=\active 
\catcode`(=\active 
\catcode`)=\active 
\catcode`[=\active 
\catcode`]=\active 
\let,=\ordcomma
\let:=\ordcolon
\let;=\ordsemicolon
\let(=\ordleftparen
\let)=\ordrightparen
\let[=\ordleftbrack
\let]=\ordrightbrack
\let\lbrack=\ordleftbrack
\let\rbrack=\ordrightbrack
\def\rmpunctuation{
\let,=\rmcomma
\let:=\rmcolon
\let;=\rmsemicolon
\let(=\rmleftparen
\let)=\rmrightparen
\let[=\rmleftbrack
\let]=\rmrightbrack
\let\lbrack=\rmleftbrack
\let\rbrack=\rmrightbrack}

\def\writemonth#1{\ifcase#1
\or January\or February\or March\or April\or May\or June\or July%
\or August\or September\or October\or November\or December\fi}

\newcount\mins
\newcount\minmodhour
\newcount\hour
\newcount\hourinmin
\newcount\ampm
\newcount\ampminhour
\newcount\hourmodampm
\def\writetime#1{%
\mins=#1%
\hour=\mins \divide\hour by 60
\hourinmin=\hour \multiply\hourinmin by -60
\minmodhour=\mins \advance\minmodhour by \hourinmin
\ampm=\hour \divide\ampm by 12
\ampminhour=\ampm \multiply\ampminhour by -12
\hourmodampm=\hour \advance\hourmodampm by \ampminhour
\ifnum\hourmodampm=0 12\else \number\hourmodampm\fi
:\ifnum\minmodhour<10 0\number\minmodhour\else \number\minmodhour\fi
\ifodd\ampm p.m.\else a.m.\fi
}

\font\tenrm=cmr10
\font\smallcaps=cmcsc10
\font\eightrm=cmr8
\font\ninerm=cmr9
\font\sixrm=cmr6
\font\eightbf=cmbx8
\font\sixbf=cmbx6
\font\eightit=cmti8
\font\eightsl=cmsl8
\font\eighti=cmmi8
\font\eightsy=cmsy8
\font\eightex=cmex10 at 8pt
\font\sixi=cmmi6
\font\sixsy=cmsy6
\font\ninesy=cmsy9
\font\seventeenrm=cmr17
\font\bigseventeenrm=cmr17 scaled \magstep 1
\font\twelverm=cmr10 scaled \magstep2
\font\seventeeni=cmmi10 scaled \magstep3
\font\twelvei=cmmi10 scaled \magstep2
\font\seventeensy=cmsy10 scaled \magstep3
\font\twelvesy=cmsy10 at 12pt
\font\seventeenex=cmex10 scaled \magstep3
\font\seventeenbf=cmbx10 scaled \magstep3
\catcode`@=11
\def\eightbig#1{{\hbox{$\textfont0=\ninerm\textfont2=\ninesy
\left#1\vbox to6.5pt{}\right.\n@space$}}}
\catcode`@=12
\def\eightpoint{\eightrm \normalbaselineskip=4.5 mm%
\textfont0=\eightrm \scriptfont0=\sixrm \scriptscriptfont0=\fiverm%
\def\rm{\fam0 \eightrm}%
\textfont1=\eighti \scriptfont1=\sixi \scriptscriptfont1=\fivei%
\def\mit{\fam1 } \def\oldstyle{\fam1 \eighti}%
\textfont2=\eightsy \scriptfont2=\sixsy \scriptscriptfont2=\fivesy%
\def\cal{\fam2 }%
\textfont3=\eightex \scriptfont3=\eightex \scriptscriptfont3=\eightex%
\def\bf{\fam\bffam\eightbf} \textfont\bffam\eightbf
\scriptfont\bffam=\sixbf \scriptscriptfont\bffam=\fivebf
\def\it{\fam\itfam\eightit} \textfont\itfam\eightit
\def\sl{\fam\slfam\eightsl} \textfont\slfam\eightsl
\let\big=\eightbig \normalbaselines\rm
\def\caps##1{\nottencaps{##1}}
}
\def\seventeenpoint{\seventeenrm \baselineskip=5.5mm%
\textfont0=\seventeenrm \scriptfont0=\twelverm \scriptscriptfont0=\sevenrm%
\def\rm{\fam0 \seventeenrm}%
\textfont1=\seventeeni \scriptfont1=\twelvei \scriptscriptfont1=\seveni%
\def\mit{\fam1 } \def\oldstyle{\fam1 \seventeeni}%
\textfont2=\seventeensy \scriptfont2=\twelvesy \scriptscriptfont2=\sevensy%
\def\cal{\fam2 }%
\textfont3=\seventeenex \scriptfont3=\seventeenex%
\scriptscriptfont3=\seventeenex%
\def\bf{\fam\bffam\seventeenbf} \textfont\bffam\seventeenbf
}

\def\setheadline #1\\ #2 \par{\headline={\ifnum\pageno=1 
\hfil
\else \eightpoint \noindent
\ifodd\pageno \hfil \caps{#2}\hfil \else
\hfil \caps{#1}\hfil \fi\fi}}

\def\beginsection{} % So that next def will scan
\def\datedversion{\footline={\ifnum\pageno=1 \fiverm \hfil
Typeset using plain-\TeX\ on
\writemonth\month\ \number\day, \number\year\ at \writetime{\time}\hfil 
\else \tenrm \hfil \folio \hfil \fi}
\def\tempsetheadline##1{\headline={\ifnum\pageno=1 
\hfil
\else \eightpoint \noindent
\writemonth\month\ \number\day, \number\year,
\writetime{\time}\hfil ##1\fi}}
\def\firstbeginsection##1\par{\bigskip\vskip\parskip
\message{##1}\centerline{\caps{##1}}\nobreak\smallskip\noindent
\tempsetheadline{##1}} \def\beginsection##1\par{\vskip0pt
plus.3\vsize\penalty-250 \vskip0pt plus-.3\vsize\bigskip\vskip\parskip
\message{##1}\centerline{\caps{##1}}\nobreak\smallskip\noindent
\tempsetheadline{##1}}}

\def\finalversion{\footline={\ifnum\pageno=1 \eightrm \hfil 
This paper is in final form.\hfil 
\else \tenrm \hfil \folio \hfil \fi}}

\def\preliminaryversion{\footline={\ifnum\pageno=1 \eightrm \hfil 
Preliminary Version.\hfil 
\else \tenrm \hfil \folio \hfil \fi}}

\footline={\ifnum\pageno=1 \eightrm \hfil 
\else \tenrm \hfil \folio \hfil \fi}

\def\moreproclaim{\par}
\def\Head #1: {\medskip\noindent{\it #1}:\enspace}
\def\Proof: {\Head Proof: }
\def\Proofof #1: {\Head Proof of #1: }
\def\endproof{\nobreak\hfill$\sqr$\bigskip\goodbreak}
\def\itemi{\item{i)}}
\def\itemii{\item{ii)}}
\def\itemiii{\item{iii)}}
\def\itemiv{\item{iv)}}
\def\itemv{\item{v)}}
\def\itemvi{\item{vi)}}
\def\itemvii{\item{vii)}}
\def\itemviii{\item{viii)}}
\def\itemix{\item{ix)}}
\def\itemx{\item{x)}}
\def\dt{\it}
\def\ds#1{{\displaystyle{#1}}}
\def\ts#1{{\textstyle{#1}}}

\def\Abstract\par#1\par{\centerline{\vtop{
\eightpoint
\abovedisplayskip=6pt plus 3pt minus 3pt
\belowdisplayskip=6pt plus 3pt minus 3pt
\moreabstract\parindent=0 true in% 
\caps{Abstract}: \ \ #1}}
\abovedisplayskip=12pt plus 3pt minus 9pt
\belowdisplayskip=12pt plus 3pt minus 9pt
\vskip 0.4 true in}
\def\moreabstract{%
\par \hsize = 5 true in \hangindent=0 true in \parindent=0.5 true in}

\def\caps#1{\smallcaps #1}
\def\nottencaps#1{\uppercase{#1}}

\def\firstbeginsection#1\par{\bigskip\vskip\parskip
\message{#1}\centerline{\caps{#1}}\nobreak\smallskip\noindent}

\def\beginsection#1\par{\vskip0pt plus.3\vsize\penalty-250
\vskip0pt plus-.3\vsize\bigskip\vskip\parskip
\message{#1}\centerline{\caps{#1}}\nobreak\smallskip\noindent}

\def\proclaim#1. #2\par{
\medbreak
\noindent{\caps{#1}.\enspace}{\it\rmpunctuation#2\par}
\ifdim\lastskip<\medskipamount \removelastskip
\penalty55\medskip\fi}

\def\Definition: #1\par{
\Head Definition: #1\par
\ifdim\lastskip<\medskipamount \removelastskip
\penalty55\medskip\fi}

\def\Problem #1: #2\par{
\Head Problem #1: #2\par
\ifdim\lastskip<\medskipamount \removelastskip
\penalty55\medskip\fi}

\def\sqr{\vcenter {\hrule height.3mm
\hbox {\vrule width.3mm height 2mm \kern2mm
\vrule width.3mm } \hrule height.3mm }}

\def\references#1{{
\frenchspacing
\eightpoint
\rmpunctuation
\halign{\bf##\hfil & \quad\vtop{\hsize=5.5 true
in\parindent=0pt\hangindent=3mm \strut\rm##\strut\smallskip}\cr#1}}}

\def\ref[#1]{{\bf [#1]}}

\catcode`@=11 % See p363 in `The TeXbook'
\def\vfootnote#1{\insert\footins\bgroup
\eightpoint
\interlinepenalty=\interfootnotelinepenalty
\splittopskip=\ht\strutbox
\splitmaxdepth=\dp\strutbox \floatingpenalty=20000
\leftskip=0pt \rightskip=0pt \spaceskip=0pt \xspaceskip=0pt
\textindent{#1}\footstrut\futurelet\next\fo@t}

\def\footremark{\insert\footins\bgroup
\eightpoint\it\rmpunctuation
\interlinepenalty=\interfootnotelinepenalty
\splittopskip=\ht\strutbox
\splitmaxdepth=\dp\strutbox \floatingpenalty=20000
\leftskip=0pt \rightskip=0pt \spaceskip=0pt \xspaceskip=0pt
\noindent\footstrut\futurelet\next\fo@t}
\catcode`@=12

\def\Bbb{\bf}
\def\E{{\Bbb E}}
\def\R{{\Bbb R}}
\def\Z{{\Bbb Z}}
\def\N{{\Bbb N}}
\def\C{{\Bbb C}}
\font\specialeightrm=cmr10 at 8pt
\def\R{\hbox{\rm I\kern-2pt R}}
\def\Z{\hbox{\rm Z\kern-3pt Z}}
\def\N{\hbox{\rm I\kern-2pt I\kern-3.1pt N}}
\def\C{\hbox{\rm \kern0.7pt\raise0.8pt\hbox{\specialeightrm I}\kern-4.2pt C}} 
\def\E{\hbox{\rm I\kern-2pt E}}

\def\cN{{\cal N}}
\def\F{{\cal F}}

\def\Id{\hbox{\rm Id}}

\def\invp{{1\over p}}
\def\invq{{1\over q}}
\def\invc{{c^{-1}}}
\def\half{{1\over 2}}
\def\smallhalf{\ts\half}
\def\poq{{p\over q}}
\def\qop{{q\over p}}

\def\list#1,#2{#1_1$, $#1_2,\ldots,$\ $#1_{#2}}
\def\lists#1{#1_1$, $#1_2,\ldots}

\def\set#1{\{1$, $2,\ldots,$\ $#1\}}
\def\span#1{\overline{\hbox{\rm span}}\{#1\}}

\def\lnorm{\left\|}
\def\rnorm{\right\|}
\def\normo#1{\lnorm #1 \rnorm}
\def\widedot{\,\cdot\,}
\def\normdot{\normo{\widedot}}
\def\trinormo#1{\left|\left|\left| #1 \right|\right|\right|}
\def\trinormdot{\trinormo{\widedot}}

\def\lmod{\left|}
\def\rmod{\right|}
\def\modo#1{\lmod #1 \rmod}

\def\angleo#1{\left\langle #1 \right\rangle}

\def\dom#1{{\vert_{#1}}}

\def\implies{$\Rightarrow$}
\def\iff{$\Leftrightarrow$}

% document relevant defs

\def\Deltacond{$\Delta_2$-condition}
\def\phifunction{$\varphi$-function}
\def\Nfunction{{\it N}-function}
\def\conditionJ{condition~$(J)$}
\def\conditionL{condition~$(L)$}
\def\em{\mathop{\rm em}\nolimits}
\def\lm{\mathop{\rm lm}\nolimits}
\def\T{{\cal T}}

% document starts here

%\datedversion
\noblankout
%\doublespacing
%\finalversion
%\preliminaryversion

\setheadline Comparison of Orlicz--Lorentz Spaces\\
             Montgomery-Smith

{
\seventeenpoint
\centerline{\bigseventeenrm Orlicz--Lorentz Spaces}
}
\bigskip\bigskip\medskip
\centerline{\caps{S.J.~Montgomery-Smith}%
\footnote{*}%
{Research supported in part by N.S.F.\ Grant DMS 9001796.}%
}
\smallskip
{
\eightpoint
\centerline{\it Department of Mathematics, University of Missouri,}
\centerline{\it Columbia, MO 65211.}
}
\bigskip\bigskip

\footremark{A.M.S.\ (1980) subject classification: 46E30.}

It is a great honor to be asked to write this article for the
Proceedings of the Conference in honor of W.~Orlicz. I cannot say that I knew
him personally, but it is obvious from the many people I have met that knew him
that he had a tremendous influence. Certainly, his ideas have found their way
into much of my own research.

The purpose of this article is to summerize some recent results of the author
about Orlicz-Lorentz spaces --- function spaces that provide a common
generalization of Orlicz spaces and Lorentz spaces.

Let us first introduce the background to these spaces.
The most well known examples of Banach spaces are the $L_p$\ spaces. Their
definition is very well known. We will restrict ourselves to function spaces on
$[0,\infty)$\ with Lebesgue measure $\lambda$. If $1\le p \le
\infty$, then for any measurable function $f$, the $L_p$-norm is
defined to be 
$$ \normo f_p = 
   \left( \int \modo{f(x)}^p \,dx \right)^{1/p} $$
for $p<\infty$, and
$$ \normo f_\infty = \mathop{\hbox{\rm ess sup}}\limits_{0\le x<\infty}
   \modo{f(x)} $$
for $p=\infty$.
The Banach space $L_p$\ is the vector space of
all measurable functions $f$\ for which $\normo f_p$\ is finite.

Now these spaces can be generalized in two different ways. The first
generalization is due to Orlicz \ref[O] (see also \ref[Lu]).  If
$F:[0,\infty)\to[0,\infty)$\ is non-decreasing and convex with $F(0)=0$, we
define the {\dt Luxemburg norm\/} of a measurable function  $f$\ by 
$$ \normo{f}_F = \inf\left\{\, c :
   \int F\bigl(\modo{f(x)}/c\bigr) \,dx \le 1 \,\right\} 
   .$$
We define the {\dt Orlicz space\/}
$L_F$\ to be those measurable functions $f$\ for which $\normo
f_F$\ is finite. We see that the Orlicz space $L_F$\ really is a true
generalization of $L_p$, at least for $p<\infty$: if $F(t) = t^p$, then $L_F =
L_p$\ with equality of norms.

In this article, we will not always require that the Luxemburg norm actually be
a norm, that is, we will not always require the triangle inequality. For this
reason, we will allow $F$\ in the above definition to be a {\dt\phifunction},
namely, that $F$\ be continuous and strictly increasing, and that
$$ F(0) = 0 , \qquad \lim_{n\to\infty} F(t) = \infty .$$
However, we will often desire that the function $F$\ has some control on its
growth, both from above and below. For this reason we will often require that
$F$\ be {\dt dilatory}, that is, for some $c_1,c_2>1$\ we have $F(c_1 t)\ge c_2
F(t)$\ for all $0\le t<\infty$, and that $F$\ satisfy the {\dt \Deltacond},
that is, that $F^{-1}$\ is dilatory.

The second collection of examples are the {\dt Lorentz spaces}. These were
introduced by Lorentz \ref[Lo1], \ref[Lo2]. If
$f$\ is a measurable function, we define the {\dt non-increasing
rearrangement\/} of $f$\ to be  
$$ f^*(x) = \sup\bigl\{\, t : \lambda(\modo f \ge t) \ge x \,\bigr\} .$$  
If
$1\le q < \infty$, and if $w:(0,\infty)\to(0,\infty)$\ is a non-increasing
function, we define the {\dt Lorentz norm\/} of a measurable function $f$\ to be 
$$ \normo f_{w,q} = \left(\int_0^\infty w(x) f^*(x)^q \,dx\right)^{1/q} .$$ 
Then the {\dt Lorentz space\/} $\Lambda_{w,q}$\ is defined to be
the space of those measurable functions $f$\ for which $\normo f_{w,q}$\ is
finite. These spaces also represent a generalization of the $L_p$\ spaces: if
$w(x) = 1$\ for all $0\le x <\infty$, then $\Lambda_{w,p} = L_p$\ with equality
of norms.

There is one, rather peculiar, choice of the function $w$\ which turns out to
be rather useful. If $1\le q \le p <\infty$, we define the spaces $L_{p,q}$\ to
be $\Lambda_{w,q}$\ with $w(x) = \qop x^{q/p-1}$. We can also allow $q>p$, but
at the loss of the triangle inequality. A good reference for a
description of these spaces is Hunt \ref[H]. By a
suitable change of variables, the $L_{p,q}$\ norm may also be defined in the
following fashion: 
$$ \normo f_{p,q} = 
   \left(\int_0^\infty \modo{f^*(x^{p/q})}^q \,dx\right)^{1/q} .$$
Thus $L_{p,p}
= L_p$\ with equality of norms. The reason for this definition is
that for any measurable set $A\in\F$, we have that $\normo{\chi_A}_{p,q} =
\normo{\chi_A}_p = \lambda(A)^{1/p} $. Thus $L_{p,q}$\ is a space identical to
$L_p$\ for characteristic functions, but `glued' together in a $L_q$\ fashion.

Now we come to the object of the article, the {\dt Orlicz--Lorentz spaces}.
These are a common generalization of the Orlicz spaces and the Lorentz spaces.
They have been studied by Masty\l o (see part~4 of \ref[My]), Maligranda
\ref[Ma], and Kami\'nska \ref[Ka1], \ref[Ka2], \ref[Ka3]. For instance,
Kami\'nska calculated many of the isometric properties for these spaces.
However, this author's work is concerned with isomorphic properties.

If $G$\
is an Orlicz function, and if $w:[0,\infty)\to[0,\infty)$\ is a non-increasing
function, we define the {\dt Orlicz--Lorentz norm\/} of a measurable function
$f$\ to be 
$$ \normo f_{w,G} = \inf\left\{\, c :
   \int_0^\infty w(x) G\bigl(f^*(x)/c\bigr) \,dx \le 1 \,\right\} 
   .$$
We define the {\dt Orlicz--Lorentz space\/} $\Lambda_{w,G}$\ to be
the vector space of measurable functions $f$\ for which $\normo f_{w,G}$\ is
finite. 

We shall not work with this definition of the Orlicz--Lorentz space, however, but
with a different, equivalent definition that bears more resemblance to the spaces
$L_{p,q}$. If $F$\ and $G$\ are \phifunction s, we would like to define our
spaces $L_{F,G}$\ to satisfy the following properties:
\itemi that $\normo{\chi_A}_{F,G} = \normo{\chi_A}_F$\ whenever $A$\ is a
measurable subset;
\itemii that $L_{F,G}$\ be glued together in a $L_G$\ fashion.

\noindent
It turns out that the required definition is the following. (In the sequel,
$\tilde F(t)$\ will always denote the function $1/F(1/t)$. Thus
$\normo{\chi_A}_F = \tilde F^{-1}\bigl(\lambda(A)\bigr)$.)

\Definition: If $F$\ and $G$\ are \phifunction s, then we define the {\dt
Orlicz--Lorentz functional\/} of a measurable function $f$\ by 
$$ \normo{f}_{F,G} = \normo{f^*\circ\tilde F\circ\tilde G^{-1}}_G .$$
We define the {\dt Orlicz--Lorentz space},
$L_{F,G}$, to be the vector space of measurable functions $f$\ for which $\normo
f_{F,G} <\infty$, modulo functions that are zero almost everywhere.

We also have the following definition corresponding to the $L_{p,\infty}$\
spaces.

\Definition: If $F$\ is a \phifunction, then we define the ({\dt weak-}){\dt
Orlicz--Lorentz functional\/} by 
$$ \normo{f}_{F,\infty} = \sup_{x\ge0} \tilde F^{-1}(x) f^*(x) .$$
We define the {\dt Orlicz--Lorentz space},
$L_{F,\infty}$, to be the vector space of measurable functions $f$\
for which $\normo f_{F,\infty} <\infty$, modulo functions that are zero almost
everywhere.

We see that $L_{F,F} = L_F$\ with equality of norms, and that if $F(t) =
t^p$\ and $G(t) = t^q$, then $L_{F,G} = L_{p,q}$, and $L_{F,\infty} =
L_{p,\infty}$, also with equality of norms. For this reason, we shall also
introduce the following notation: if $F(t) = t^p$, we shall write
$L_{p,G}$\ for $L_{F,G}$, and $L_{G,p}$\ for $L_{G,F}$.

Now let us provide some examples. We define the {\dt modified
logarithm\/} and the {\dt modified exponential\/} functions by
$$ \eqalignno{
   \lm(t) 
   &=\cases{ 1+\log t                  & if $t\ge1$\cr
             1/\bigl(1+\log(1/t)\bigr) & if $0<t<1$\cr
             0                         & if $t=0$;\cr}\cr
   \em(t) = \lm^{-1}(t)
   &=\cases{ \exp(t-1)                 & if $t\ge1$\cr
             \exp\bigl(1-(1/t)\bigr)   & if $0<t<1$\cr
             0                         & if $t=0$.\cr}\cr}$$
These functions are designed so that for large $t$\ they behave like
the logarithm and the exponential functions, so that $\lm 1=1$\ and $\em 1=1$,
and so that $\widetilde{\lm}=\lm$\ and $\widetilde{\em} = \em$. Then the
functions $t^p(\lm t)^\alpha$\ and $\em(t^p)$\ are \phifunction s whenever
$0<p<\infty$\ and $-\infty<\alpha<\infty$. If the measure space is a
probability space, then the Orlicz spaces created using these functions are also
known as {\dt Zygmund spaces}, and the Orlicz--Lorentz spaces $L_{t^p(\lm
t)^\alpha,q}$\ and $L_{\em(t^p),q}$\ are known as {\dt Lorentz--Zygmund
spaces} (see, for example, \ref[B--S]).

\beginsection Comparison Results

A large part of my research on these spaces has asked the question: what are
necessary and sufficient conditions on $F_1$, $F_2$, $G_1$\ and $G_2$\ so that
the spaces $L_{F_1,G_1}$\ and $L_{F_2,G_2}$\ are {\it equivalent}, that is,
that there is some constant $c<\infty$\ such that
$$ \invc \normo f_{F_1,G_1} \le \normo f_{F_2,G_2} \le
   c \, \normo f_{F_1,G_1} .$$
In answering these questions, it is necessary to assume that $G_1$\ and $G_2$\
be dilatory, and satisfy the \Deltacond. In all our general discussions, we
shall take this as given.

First, by considering characteristic functions, it is easy to see that it must
be that $F_1$\ and $F_2$\ are equivalent as \phifunction s, that is, there is a
constant $c<\infty$\ such that $F_1(\invc t) \le F_2(t) \le F_1(ct)$\ for all
$0\le t<\infty$. In this manner, it is easy to see that without loss of
generality, we may take $F_1 = F_2$. In fact, it is not hard to show that we are
really asking about the equivalence of $L_1$\ and $L_{1,H}$, where $H = G_1
\circ G_2^{-1}$\ or $H = G_2 \circ G_1 ^{-1}$.

Results along these lines have already been obtained by G.~Lorentz, and also by
Y.~Raynaud. I will take the liberty of translating their results into my
notation. (In so doing, it may not be entirely obvious that their result as they
state it, and as it is stated here, are actually the same.)

To state these results we will require some more notation. We will say that
a \phifunction\ $F$\ is an {\dt \Nfunction\/}
if it is equivalent to a \phifunction\ $F_0$\ such that $F_0(t)/t$\ is
strictly increasing, $F_0(t)/t \to \infty$\ as $t\to \infty$, and $F_0(t)/t \to
0$\ as $t\to0$.
We will say that 
a \phifunction\ $F$\ is {\dt complementary\/}
to a \phifunction\ $G$\ if for some $c<\infty$\ we have 
$$ \invc t \le F^{-1}(t) \cdot G^{-1}(t) \le ct 
   \qquad (0\le t<\infty) .$$
If $F$\ is an \Nfunction, we will let $F^*$\ denote the (unique up to
equivalence) function complementary to $F$.

Our definition of a complementary function differs from the usual definition. If
$F$\ is an \Nfunction\ that is convex, then the complementary function is
usually defined by $F^*(t) = \sup_{s\ge0} \bigl(st-F(s)\bigr)$. However, it is
known that $t\le F^{-1}(t) \cdot {F^{*}}^{-1}(t) \le 2t$\ (see \ref[K--R]). Thus
our definition is equivalent.

Finally, we will say that an \Nfunction\ $H$\ satisfies {\dt
\conditionJ\/}\ if 
$$ \normo{1/\tilde H^*{}^{-1}}_{H^*} < \infty .$$
(I call it \conditionJ\ for personal reasons.)

Now we are ready to give the result of G.~Lorentz \ref[Lo3]. 

\proclaim Theorem 1. Suppose that $H$\ is an \Nfunction. Then the following are
equivalent. 
\itemi $L_1$\ and $L_{1,H}$\ are equivalent.
\itemii $H$\ satisfies \conditionJ.

What kinds of \Nfunction s satisfy \conditionJ? They are functions
that satisfy growth conditions that make it `close' to the identity
function. The reader might like to verify that $t (\lm t)^\alpha$\ satisfies
this condition when $\alpha > 0$.
Lorentz gave the following example:
$$ H(t) = \cases{ t^{1+{1\over 1+\log(1+\log t)}} & if $t\ge1$\cr
                  t^{1-{1\over 1+\log(1-\log t)}} & if $t\le1$.\cr} $$
In fact, we will give another characterization that shows
that this example is, in some sense,  on the ``boundary'' of satisfying
\conditionJ.

Raynaud's result \ref[R] allows one to drop the assumption that $H$\ is an
\Nfunction, but at the cost of making the implication go only one way.

\proclaim Theorem 2. Suppose that $H$\ is a \phifunction. Suppose that there
exist \Nfunction s $K$\ and $L$\ satisfying \conditionJ\ such that $H = K \circ
L^{-1}$\ (or $H = K^{-1}\circ L$). Then $L_1$\ and $L_{1,H}$\ are
equivalent.

As applications, one may show that if $0<p<\infty$\ and
$-\infty<\alpha<\infty$, then $L_{t^p(\lm t)^\alpha}$\ and $L_{t^p(\lm
t)^\alpha,p}$\ are equivalent, and that if $\beta>0$, then $L_{\em(t^\beta)}$\
and $L_{\em(t^\beta),\infty}$\ are equivalent. These were shown for probability
spaces by Bennett and Rudnick \ref[B--R] (see also \ref[B--S]).

The author's contribution was to show that the converse result to
Theorem~2 holds.

\proclaim Theorem 3. Suppose that $H$\ is a \phifunction\ such that $L_1$\
and $L_{1,H}$\ are equivalent. Then the following are true.
\itemi There
exist \Nfunction s $K$\ and $L$\ satisfying \conditionJ\ such that $H = K \circ
L^{-1}$
\itemii There
exist \Nfunction s $K$\ and $L$\ satisfying \conditionJ\ such that $H =
K^{-1}\circ L$.

In fact there are many more equivalent conditions, and we will give some more
later. We will not prove any results here --- the interested reader should
consult \ref[Mo1]. However, we will explain
some of the ideas behind them.

First we will describe the simple comparison principles for Orlicz--Lorentz
spaces. If the reader has studied Lorentz spaces, he will already know
that $\normo f_{p,q_1} \le \normo f_{p,q_2}$\ whenever $q_1 \ge q_2$\ (see
\ref[H]). In our more general setting, we have the following result: if $F$\ is
equivalent to a convex function, then $\normo f_F \le c\, \normo f_{F,1}$\ for
all measurable $f$. In fact, it is quite easy to show that if $\normdot$\ is any
{\it norm\/} (the triangle inequality is essential here) such that
$$ \modo f \le \chi A \Rightarrow \normo f \le c_1 \tilde F^{-1}
   \bigl(\lambda(A)\bigr) ,$$
then $\normo f \le c_2 \normo f_{F,1}$. 

From this, we can deduce the following result. Let us say that $G_1$\ is {\dt
equivalently less convex than\/} $G_2$\ (in symbols $G_1 \prec G_2$) if
$G_2\circ G_1^{-1}$\ is equivalent to a convex function. Then 
$$ G_1 \prec G_2 \Rightarrow \normo f_{1,G_1} \ge \invc \normo f_{1,G_2} .$$

However, we can see from Theorems~1 and~2 that this is not the whole story. If
we desire a converse to this implication, we will have to soften the notion of
`less convex than' to `almost less convex than.' It turns out that we can
precisely characterize this notion of `almost convexity.'

Before doing this, let us discuss what it means for a \phifunction\ to be
equivalent to a convex function. Suppose we are given a fixed number $a>1$. It is
quite easy to see that a \phifunction\ $G$\ is completely determined, up to
equivalence, by the values $G(a^n)$\ for $n\in\Z$. In this way, it can be
easily shown that a \phifunction\ $G$\ is equivalent to a convex function if
and only there exists numbers $a>1$\ and
$N\in\N$\ such that for all $m\in\N$, we have that 
$$ G(a^{n+m}) \ge a^{m-N} G(a^n) $$
for all $n\in\Z$. (Here $\N = \{1$, $2$,
$3,\ldots\}$.)

It turns out that the correct definition for `almost convex' is the following.

\Definition: Let $G$\ be a \phifunction. We say that $G$\ is {\dt almost
convex\/} if there are numbers $a>1$, $b>1$\ and $N\in\N$\ such that for all
$m\in\N$, the cardinality of the set of $n\in\Z$\ such that we do {\bf not} have
$$ G(a^{n+m}) \ge a^{m-N} G(a^n) $$
is less than $b^m$.

In the same way, one can get notions of {\it almost concave}, {\it almost
linear}, etc. It turns out that an \Nfunction\ $H$\ satisfies \conditionJ\
if and only if $H^{-1}$\ is almost convex. 

Using these ideas, it is then possible to prove Theorem~3, and indeed to get
the following result, that gives the desired necessary and sufficient
conditions for Orlicz--Lorentz spaces to be equivalent.

\proclaim Theorem 4. Suppose that $F_1$, $F_2$, $G_1$\ and $G_2$\ are
\phifunction s such that at least one of $G_1$\ and $G_2$\ are dilatory, and at
least one of $G_1$\ and $G_2$\ satisfy the \Deltacond. Then the following are
equivalent statements.
\itemi $L_{F_1,G_1}$\ and $L_{F_2,G_2}$\ are equivalent.
\itemii $F_1$\ and $F_2$\ are equivalent, and there exist \Nfunction s $H$\
and $K$\ that satisfy \conditionJ\ such that $G_1 \circ G_2^{-1} = H\circ
K^{-1}$.
\itemiii $F_1$\ and $F_2$\ are equivalent, and there exist \Nfunction s $H$\
and $K$\ that satisfy \conditionJ\ such that $G_1 \circ G_2^{-1} = K^{-1}\circ
H$.
\itemiv $F_1$\ and $F_2$\ are equivalent, and $G_1 \circ G_2^{-1}$\ is almost
convex, and $G_2 \circ G_1^{-1}$\ is almost convex.
\moreproclaim

\beginsection Is Every R.I.\ Space
Equivalent to an Orlicz--Lorentz Space?

Or more precisely, does there exist a rearrangement invariant space $X$\ such
that the $\normdot_X$\ is not equivalent to any Orlicz--Lorentz norm on the
space of simple functions? It turns out that we can find an example to show that
this can happen. To do this, we use the following result, which is a corollary of
the proof of Theorem~4. As before, we refer the reader to \ref[Mo1] for
details. 

\proclaim Theorem 5. Let $F_1$, $F_2$, $G_1$\ and $G_2$\ be \phifunction s.
Suppose that one of $G_1$\ or $G_2$\ is dilatory, and that one of $G_1$\ or
$G_2$\ satisfies the \Deltacond. Then the following are equivalent.
\itemi $L_{F_1,G_1}$\ and $L_{F_2,G_2}$\ are
equivalent.
\itemii For some $c<\infty$ we have that
$\invc \normo f_{F_1,G_1} \le \normo f_{F_2,G_2} \le c\, \normo f_{F_1,G_1}$\
whenever $f$\ is of the following form:\enspace there exist  
$0=a_0<a_1<a_2<\ldots<a_n$\ such that
$$ F\circ f^*(x) = \cases{
   1/a_i & if $a_{i-1}\le x < a_i$\ and $1\le i\le n$ \cr
   0     & otherwise. \cr }$$
\moreproclaim

Thus to compare two Orlicz--Lorentz spaces, we need only
compare their norms on a certain class of test functions. Now it is easy to
prove the desired result.

\proclaim Theorem 6. There is a rearrangement invariant Banach space $X$\ such
that for every Orlicz--Lorentz space $L_{F,G}$, the norms $\normdot_X$\ and
$\normdot_{F,G}$\ are inequivalent on the vector space of simple functions.

\Proof: We define the
following norm for measurable functions $f$:
$$ \normo f_X = \sup \normo{f g}_1/\normo g_2 ,$$
where the supremum is over all $g$\ of the following form:\enspace there
exist $0=a_0<a_1<a_2<\ldots<a_n$\ such that
$$ g^*(x) = \cases{
   1/\sqrt{a_i} & if $a_{i-1}\le x < a_i$\ and $1\le i\le n$ \cr
   0            & otherwise. \cr }$$
Then it is easy to see from Theorem~5 that if $X$\ is equivalent to an
Orlicz--Lorentz space, then it must be equivalent to $L_2$. That this is not
the case is easily shown by the following example:
$$ f(x) = {1\over \sqrt{x \log x}} \qquad \hbox{for $x \ge 2$.} $$
Then $\normo f_X < \infty$, whereas $\normo f_2 = \infty$.
\endproof

\beginsection Boyd Indices of Orlicz--Lorentz Spaces

In studying a particular rearrangement invariant space, it is very important to
know its Boyd indices. Even very obvious questions, like whether it is
equivalent to a normed space, or whether it is $p$-convex/$q$-concave, cannot be
answered except with a knowledge of these indices. 

As their name suggests, they
were first studied by Boyd \ref[Bo]. We will take our
definition from \ref[L--T]. Their definition differs from that
usually used: most references reverse the words `upper' and `lower', and use
the recipricols of the indices used here. 

Essentially, they describe the norms of the
following operators: for each $a>0$\ we let $d_a f(x) = f(ax)$. The {\dt lower
Boyd index\/} is defined to be
$$ p(X) =
   \sup \left\{\, p :
   \hbox{for some $c<\infty$\ we have $\normo{d_a}_{X\to X} \le c a^{-1/p}$\
         for $a<1$} \,\right\} ,$$
and the {\dt upper Boyd index\/} is
$$ q(X) =
   \inf \left\{\, q :
   \hbox{for some $c<\infty$\ we have $\normo{d_a}_{X\to X} \le c a^{-1/q}$\
         for $a>1$} \,\right\} .$$
The reader should appreciate that $p(L_{p,q}) = q(L_{p,q}) = p$.

The hope is that it should be possible to calculate the Boyd indices of
$L_{F,G}$\ simply from knowledge of some appropriate index of $F$. In fact,
this was the question posed by Maligranda \ref[Ma]. What are the appropriate
indices? For a \phifunction\ $F$, we define the {\dt lower Matuszewska--Orlicz
index\/} to be
$$ p_m(F) =
   \sup \left\{\, p :
   \hbox{for some $c>0$\ we have $F(at) \ge c\, a^p F(t)$\
         for $0\le t<\infty$\ and $a>1$} \,\right\} ,$$
and the {\dt upper Matuszewska--Orlicz index\/}
to be
$$ q_m(F) =
   \inf \left\{\, q :
   \hbox{for some $c<\infty$\ we have $F(at) \le c\, a^q F(t)$\
         for $0\le t<\infty$\ and $a>1$} \,\right\} .$$
Thus, for example, $p_m(T^p) = q_m(T^p) = p $. Maligranda's conjecture is the
following:\enspace is $p(L_{F,G}) = p_m(F)$\ and $q(L_{F,G}) = q_m(F)$?

Without going into details, I was able to show that this is not the case.
Briefly, the example is $L_{1,G}$, where $G$\ is a \phifunction\ that spends
some of the time behaving like $T^p$, and some of the time behaving like $T^q$.
We refer the reader to \ref[Mo2] for more details. (The first example of a
rearrangement invariant space where this sort of thing happened is due to
Shimogaki \ref[Sh].)

However, it is possible to obtain the following result without undue stress.

\proclaim Proposition 7. Let $F$\ and $G$\ be \phifunction s. Then
\itemi
$ p_m(F) \ge p(L_{F,G}) \ge p_m(F\circ G^{-1}) p_m(G) \ge
p_m(F) p_m(G)/q_m(G)$; 
\itemii
$ q_m(F) \le q(L_{F,G}) \le q_m(F\circ G^{-1}) q_m(G) \le 
q_m(F) q_m(G)/p_m(G)$.

We are then left with the following question. Given $F$\ and $G$, how exactly
is one to calculate the Boyd indices of $L_{F,G}$? The author does have some
idea for how to approach this problem, at least for giving necessary and
sufficient conditions for the indices of $L_{1,G}$\ to be $1$. The idea is
simple:\enspace we see that if $0<a<\infty$, then  $ a \normo{d_a f}_{1,G} =
\normo f_{1,G_a}$, where $G_a(t) = G(at)$. Then the problem of
determining the Boyd indices becomes a problem of comparing two Orlicz--Lorentz
spaces, and the methods from the above section should apply. One day, the
author will get around to checking these ideas out. But if anyone else would
like to do this, they can, and the author won't
mind. Then {\it they\/} will have the problem of finding a journal that will
accept results from this tiny corner of mathematics.

Finally, I would like to mention some very recent work of Bastero and Ruiz
\ref[Ba--R]. They prove some results about the Hardy transform on
Orlicz--Lorentz spaces. If one looks hard enough at what they did, and then
twists the way they state the results, one can obtain fairly sharp estimates for
Boyd indices in the following manner.  Given \phifunction s $F$\ and $G$, we
define the {\dt modular lower and upper Boyd indices\/} of $L_{F,G}$\ as follows:
$$ \eqalignno{
   p_{\bmod}(L_{F,G}) &=
   \sup \left\{\vphantom{\int}\right.\, p :
   \hbox{for some $c<\infty$\ we have } \cr
   & \qquad \qquad 
         \int G\bigl(f^*(a \tilde F\circ \tilde G^{-1}(x)\bigr) \, dx
         \le \int G\bigl(c a^{-1/p}f^*(\tilde F\circ \tilde G^{-1}(x)\bigr) 
         \hbox{ for $a<1$}\left.\vphantom{\int}\right\} ,\cr
   q_{\bmod}(L_{F,G}) &=
   \inf \left\{\vphantom{\int}\right.\, q :
   \hbox{for some $c<\infty$\ we have } \cr
   & \qquad \qquad 
         \int G\bigl(f^*(a \tilde F\circ \tilde G^{-1}(x)\bigr) \, dx
         \le \int G\bigl(c a^{-1/q}f^*(\tilde F\circ \tilde G^{-1}(x)\bigr) 
         \hbox{ for $a>1$}\left.\vphantom{\int}\right\} .\cr }$$
Then we have the following result.

\proclaim Theorem~8. Let $F$\ and $G$\ be \phifunction s. Then
\itemi
$ p_{\bmod}(L_{F,G}) = p_m(F\circ G^{-1}) p_m(G) $; 
\itemii
$ q_{\bmod}(L_{F,G}) = q_m(F\circ G^{-1}) q_m(G) $.
\moreproclaim

\beginsection The Definition of Torchinsky and Raynaud

Finally, we mention that there is another possible definition for
Orlicz--Lorentz spaces, first given by Torchinsky \ref[T], and investigated in
detail by Raynaud \ref[R]. We define 
$$ \normo f_{F,G}^T
   = \normo{\tilde F^{-1}(e^x) f^*(e^x)}_G ,$$
and call the corresponding space $L_{F,G}^T$\ (my notation).
Raynaud showed that if $F$\ is
dilatory and satisfies the \Deltacond, and if $G$\ is dilatory, then
$$ \normo{\chi_A}_{F,G}^T \approx \tilde F^{-1}\bigl(\lambda(A)\bigr) .$$
Thus these spaces are really quite a good contender for a possible alternative
definition. Also, the problems that I considered are very easy to solve for
these spaces. Raynaud
showed that if $F_1$\ and $F_2$\ are
dilatory and satisfy the \Deltacond, and if $G_1$\  and $G_2$\ are dilatory,
then $L_{F_1,G_1}^T$\ and $L_{F_2,G_2}^T$\ are equivalent if
$F_1$\ and $F_2$\ are equivalent, and the {\it sequence\/} spaces $l_{G_1}$\ and
$l_{G_2}$\ are equivalent. The converse result is also easy to show.

Also, the Boyd indices of these spaces are much easier
to compute. If $F$\ is dilatory and satisfies the \Deltacond, and if $G$\ is 
dilatory,
then $p(L_{F,G}) = p_m(F)$, and $q(L_{F,G}) = q_m(G)$.

The only problem with these spaces is that we do not always have
that $L_{F,F}^T$\ is equivalent to the Orlicz space $L_F$.

\beginsection References

\references{
Ba--R & J.~Bastero and F.J.~Ruiz,\rm\ Interpolation of operators when
the extreme spaces are $L_\infty$,\sl\ preprint.\cr
B--R & C.~Bennett and K.~Rudnick,\rm\ On Lorentz--Zygmund spaces,\sl\
Dissert.\
Math.\ {\bf 175} (1980), 1--72.\cr
B--S & C.~Bennett and R.~Sharpley,\sl\ Interpolation of Operators,\rm\
Academic Press 1988.\cr
Bo & D.W.~Boyd,\rm\ Indices of function spaces and their relationship to
interpolation,\sl\ Canad.\ J.\ Math.\ {\bf 21} (1969), 1245--1254.\cr
H & R.A.~Hunt,\rm\ On $L(p,q)$\ spaces,\sl\ L'Enseignement Math.\ (2)
{\bf 12} (1966), 249--275.\cr
Ka1 & A.~Kami\'nska, \rm\ Some remarks on Orlicz--Lorentz spaces,\sl\ Math.\
Nachr., to appear.\cr
Ka2 & A.~Kami\'nska, \rm\ Extreme points in Orlicz--Lorentz spaces,\sl\
Arch.\ Math., to appear.\cr
Ka3 & A.~Kami\'nska, \rm\ Uniform convexity of generalized Lorentz spaces,\sl\
Arch.\ Math., to appear.\cr
K--R & M.A.~Krasnosel'ski\u\i\ and Ya.B.~Ruticki\u\i,\sl\
Convex Functions and
Orlicz Spaces,\rm\ P.~Noodhoof Ltd., 1961.\cr
L--T & J.~Lindenstrauss and L.~Tzafriri,\sl\ Classical Banach Spaces
II---Fu\-nc\-t\-ion Spa\-ces,\rm\ Springer-Verlag 1979.\cr
Lo1 & G.G.~Lorentz,\rm\ Some new function spaces,\sl\ Ann.\ Math.\ {\bf 51}
(1950), 37--55.\cr
Lo2 & G.G.~Lorentz,\rm\ On the theory of spaces $\Lambda$,\sl\ Pac.\ J.\ Math.\
{\bf 1} (1951), 411--429.\cr
Lo3 & G.G.~Lorentz,\rm\ Relations between function spaces,\sl\ Proc.\ A.M.S.\
{\bf 12} (1961), 127--132.\cr
Lu & W.A.J.~Luxemburg,\sl\ Banach Function Spaces,\rm\ Thesis, Delft Technical
Univ., 1955.\cr
Ma & L.~Maligranda,\rm\ Indices and interpolation,\sl\ Dissert.\ Math.\ {\bf 234}
(1984), 1--49.\cr
My & M.~Masty\l o,\rm\ Interpolation of linear operators in
Calderon--Lozanovskii spaces,\sl\ Comment.\ Math.\ {\bf 26,2} (1986),
247--256.\cr
Mo1 & S.J.~Montgomery-Smith,\rm\ Comparison of Orlicz--Lorentz spaces,\sl\ 
submitted.\cr
Mo2 & S.J.~Montgomery-Smith,\rm\ Boyd Indices of Orlicz--Lorentz spaces,\sl\
in preparation.\cr
O & W.~Orlicz,\rm\ \"Uber eine gewisse Klasse von R\"aumen vom Typus B, \sl\
Bull.\ Intern.\ Acad.\ Pol.\ {\bf 8} (1932), 207--220.\cr
R & Y.~Raynaud,\rm\ On Lorentz--Sharpley spaces,\sl\ Proceedings of the
Workshop ``Interpolation Spaces and Related Topics'', Haifa, June 1990.\cr
S & R.~Sharpley,\rm\ Spaces $\Lambda_\alpha(X)$\ and Interpolation,\sl\ J.\
Funct.\ Anal.\ {\bf 11} (1972), 479--513.\cr
Sh & T.~Shimogaki,\rm\ A note on norms of compression operators on
function
spaces,\sl\ Proc.\ Japan Acad.\ {\bf 46} (1970), 239--242.\cr
T & A.~Torchinsky,\rm\ Interpolation of operators and Orlicz classes,\sl\
Studia Math.\ {\bf 59} (1976), 177--207.\cr
}

\bye

\references{
B--S & C.~Bennett and R.~Sharpley,\sl\ Interpolation of Operators,\rm\
Academic Press 1988.\cr
Bo2 & D.W.~Boyd,\rm\ Indices for the Orlicz spaces,\sl\ Pacific J.\ Math.\ {\bf
38} (1971), 315--323.\cr
H--L--P & G.H.~Hardy, J.E.~Littlewood and G.~P\'olya,\sl\ Inequalities,\rm\
Cambridge University Press, 1952.\cr
H & R.A.~Hunt,\rm\ On $L(p,q)$\ spaces,\sl\ L'Enseignement Math.\ (2)
{\bf 12} (1966), 249--275.\cr
Ka1 & A.~Kami\'nska, \rm\ Some remarks on Orlicz--Lorentz spaces,\sl\ Math.\
Nachr., to appear.\cr
Ka2 & A.~Kami\'nska, \rm\ Extreme points in Orlicz--Lorentz spaces,\sl\
Arch.\ Math., to appear.\cr
Ka3 & A.~Kami\'nska, \rm\ Uniform convexity of generalized Lorentz spaces,\sl\
Arch.\ Math., to appear.\cr
K--R & M.A.~Krasnosel'ski\u\i\ and Ya.B.~Ruticki\u\i,\sl\
Convex Functions and
Orlicz Spaces,\rm\ P.~Noordhoof Ltd., 1961.\cr
Lo & G.G.~Lorentz,\rm\ Some new function spaces,\sl\ Ann.\ Math.\ {\bf 51}
(1950), 37--55.\cr
L--T1 & J.~Lindenstrauss and L.~Tzafriri,\sl\ Classical Banach Spaces
I---Se\-qu\-ence Spa\-ces,\rm\ Springer-Verlag 1977.\cr
Lu & W.A.J.~Luxemburg,\sl\ Banach Function Spaces,\rm\ Thesis, Delft Technical
Univ.\ 1955.\cr
Ma & L.~Maligranda,\rm\ Indices and interpolation,\sl\ Dissert.\ Math.\ {\bf 234}
(1984), 1--49.\cr
My & M.~Masty\l o,\rm\ Interpolation of linear operators in
Calderon--Lozanovskii spaces,\sl\ Comment.\ Math.\ {\bf 26,2} (1986),
247--256.\cr
M--O1 & W.~Matuszewska and W.~Orlicz,\rm\ On certain properties of
$\varphi$-functions,\sl\ Bull.\ Acad.\ Polon.\ Sci., S\'er.\ Sci.\ Math.\
Astronom.\ Phys.\ {\bf 8} (1960), 439--443.\cr
M--O2 & W.~Matuszewska and W.~Orlicz,\rm\ On some classes of functions with
regard to their orders of growth,\sl\ Studia Math.\ {\bf 26} (1965), 11--24.\cr
Mo1 & S.J.~Montgomery-Smith,\sl\ The Cotype of Operators from
$C(K)$,\rm\
Ph.D.\ thesis, Cambridge, August 1988.\cr
O & W.~Orlicz,\rm\ \"Uber eine gewisse Klasse von R\"aumen vom Typus B, \sl\
Bull.\ Intern.\ Acad.\ Pol.\ {\bf 8} (1932), 207--220.\cr
R & Y.~Raynaud,\rm\ On Lorentz--Sharpley spaces,\sl\ submitted to Proceedings of
the Workshop ``Interpolation Spaces and Related Topics'', Haifa, June 1990.\cr
T & A.~Torchinsky,\rm\ Interpolation of operators and Orlicz classes,\sl\
Studia Math.\ {\bf 59} (1976), 177--207.\cr
Z & M.~Zippin,\rm\ Interpolation of operators of weak type between
rearrangement invariant spaces,\sl\ J.\ Functional Analysis {\bf 7} (1971),
267--284.\cr
}

\bye

L--T1 & J.~Lindenstrauss and L.~Tzafriri,\sl\ Classical Banach Spaces
I---Se\-qu\-ence Spa\-ces,\rm\ Springer-Verlag 1977.\cr
L--T2 & J.~Lindenstrauss and L.~Tzafriri,\sl\ Classical Banach Spaces
II---Fu\-nc\-t\-ion Spa\-ces,\rm\ Springer-Verlag 1979.\cr
M--O1 & W.~Matuszewska and W.~Orlicz,\rm\ On certain properties of
$\varphi$-functions,\sl\ Bull.\ Acad.\ Polon.\ Sci., S\'er.\ Sci.\ Math.\
Astronom.\ Phys.\ {\bf 8} (1960), 439--443.\cr
M--O2 & W.~Matuszewska and W.~Orlicz,\rm\ On some classes of functions with
regard to their orders of growth,\sl\ Studia Math.\ {\bf 26} (1965), 11--24.\cr
