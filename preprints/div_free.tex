%%%%%%%%%%%%%%%%%%%%%%%%%%%%%%%%%%%%%%%%%%%%%%%%%%%%%%
%% This paper is typeset in LaTeX.
%%
%%
%%
%%
%%
%%
%%
%%
%%
%%

\documentstyle[12pt,leqno]{article}

\input amssym.def

\def\eqref#1{(\ref{#1})}
\def\operatorname#1{\text{#1}}
\def\text#1{\hbox{\rm #1}}
\def\dfrac#1#2{{\displaystyle{#1\over#2}}}

\newenvironment{pf}{{\em Proof:}}{$\Box$\vspace{14.5pt}}
\newenvironment{equation*}{\[}{\]}

%%%%%%%%%THEOREMS%%%%%%%%%%%%%%%%%%%%%%%%%%%%%%%%%%%%%%
\newtheorem{prop}{Proposition}[section]
\newtheorem{thm}[prop]{Theorem}
\newtheorem{cor}[prop]{Corollary}
\newtheorem{lem}[prop]{Lemma}

%\theoremstyle{definition}

\newtheorem{defn}[prop]{Definition}

%\theoremstyle{remark}
\newtheorem{rem}{Remark} \renewcommand{\therem}{}

%%%%%%%%%%%%%%FONTS%%%%%%%%%%%%%%%%%%%%%%%%%%%%%%%%

\newcommand{\bbN}{{\Bbb{N}}}
\newcommand{\bbR}{{\Bbb{R}}}
\newcommand{\bbZ}{{\Bbb{Z}}}
\newcommand{\bbC}{{\Bbb{C}}}
\newcommand{\bbT}{{\Bbb{T}}}
\newcommand{\bbD}{{\Bbb{D}}}




\newcommand{\calM}{{\cal{M}}}
\newcommand{\calD}{{\cal{D}}}
\newcommand{\calS}{{\cal{S}}}
\newcommand{\calP}{{\cal{P}}}
\newcommand{\calQ}{{\cal{Q}}}
\newcommand{\calL}{{\cal{L}}}
\newcommand{\calF}{{\cal{F}}}
\newcommand{\calE}{{\cal{E}}}
\newcommand{\TX}{{{\cal{T}}X}}

%%%%%%%%%%%%%%%GREEK%%%%%%%%%%%%%%%%%%%%%%%%%%%%%%%%
\newcommand{\al}{\alpha}
\newcommand{\be}{\beta}
\newcommand{\g}{\gamma}
\newcommand{\de}{\delta}
\newcommand{\e}{\varepsilon}
\newcommand{\th}{\theta}
\newcommand{\k}{\kappa}
\newcommand{\la}{\lambda}
\newcommand{\m}{\mu}
\newcommand{\n}{\nu}
\newcommand{\r}{\varrho}
\newcommand{\s}{\sigma}
\newcommand{\f}{\varphi}
\newcommand{\h}{\chi}
\newcommand{\p}{\psi}
\newcommand{\om}{\omega}

\newcommand{\Om}{\Omega}
\newcommand{\La}{\Lambda}
\newcommand{\D}{\Delta}
\newcommand{\G}{\Gamma}
%newcommand{\Sig}{\Sigma}
\newcommand{\Th}{\Theta}
%newcommand{\pa}{\partial}

%%%%%%%%%%%%%%% operatornames %%%%%%%%%%%%%%%%%%%%%%%%%%%%%
\newcommand{\grad}{\operatorname{grad}}
\newcommand{\proj}{\operatorname{proj}}
\newcommand{\mes}{\operatorname{mes}}
\renewcommand{\div}{\operatorname{div}}
\newcommand{\diam}{\operatorname{diam}}
\newcommand{\sdiv}{\displaystyle\div}
\renewcommand{\span}{\operatorname{span}}
\newcommand{\supp}{\operatorname{supp}}
\renewcommand{\Im}{\operatorname{Im}}
\newcommand{\sgn}{\operatorname{sgn}}
\renewcommand{\Re}{\operatorname{Re}}
\newcommand{\dint}{\displaystyle\int}

%%%%%%%%%%%%%%%%%%ABBRS%%%%%%%%%%%%%%%%%%%%%%%%%%%%%
\newcommand{\vl}[1]{{\bf #1}}
\newcommand{\tp}[1]{``{\it #1\/}''}
\newcommand{\disp}{\displaystyle}
\newcommand{\lb}{\label}
\newcommand{\ti}{\tilde}
\newcommand{\bs}{\backslash}
\newcommand{\bi}[1]{\bibitem{#1}}
\newcommand{\emt}{\emptyset}
\newcommand{\no}{\nonumber}
\newcommand{\lra}{\longrightarrow}

%%%%%%%%%%%%%%%%%%%%%%%%NUMBERING%%%%%%%%%%%%%%%%%%%%%%%%
\newcommand{\subfig}{a}

\begin{document}

\makeatletter
\def\@currentlabel{2.1}\label{e:dispaa}
\def\@currentlabel{2.21}\label{e:dispau}
\def\@currentlabel{2.22}\label{e:dispav}
\def\@currentlabel{2.23}\label{e:dispaw}
\def\@currentlabel{2.24}\label{e:dispax}
\def\theequation{\thesection.\@arabic\c@equation}
\def\Qed{\hfill \vrule height4pt width3pt depth2pt}

\def\sap{\sigma_{\mbox{ap}}}
\def\C{{C(X,\TX)}}
\def\CND{{C_{ND}(X,\TX)}}

\makeatother

\makeatletter
\def\alphenumi{%
  \def\theenumi{\alph{enumi}}%
  \def\p@enumi{\theenumi}%
  \def\labelenumi{(\@alph\c@enumi)}}
\makeatother

%%%%%%%%%%%%%%%%%%%%%%%%%%%%%%%%%%%%%%%%%%%%%%%%%%%%%

\def\zzz{\par}

\title{The Spectrum of the
Kinematic Dynamo Operator
for an Ideally Conducting Fluid}

\author{C.~Chicone\thanks{carmen$@$chicone.cs.missouri.edu,  supported by
the NSF grant
DMS-9303767;} ,\,
Y.~Latushkin\thanks{mathyl$@$mizzou1.missouri.edu, supported  by the
NSF grant DMS-9400518 and by the Summer
Research Fellowship of the University of Missouri;} ,\,
 and
S.~Montgomery-Smith\thanks{stephen$@$mont.cs.missouri.edu, supported by
the NSF grant  DMS-9201357;
\zzz
{\em Key words and phrases.} Hyperbolic dynamical systems,
 Mather spectrum,
magnetohydrodynamics,
ideally conducting fluid, spectral
mapping theorems, weighted composition
operators.
\zzz
 {\em Mathematics Subject Classification.} 76W05, 58F99, 58G25.}\\
\\
Department of  Mathematics,
University of Missouri\\ Columbia, MO 65211, USA}

\maketitle

\begin{abstract}
The spectrum of the kinematic dynamo operator
for an ideally
conducting fluid and the spectrum of the
corresponding group acting in the space of
continuous divergence
free vector fields on a compact Riemannian manifold
are described.
We prove that the spectrum
of the kinematic
dynamo operator is exactly one vertical strip whose
boundaries can be determined in terms of
the Lyapunov-Oseledets exponents
with respect to all ergodic measures for the Eulerian flow.
Also, we prove that the spectrum of the corresponding group is
obtained from the spectrum of its generator by
exponentiation. In particular, the growth bound for
the group coincides with the spectral bound for
the generator.
\end{abstract}

\section{Introduction}
In this paper we give a  description of the
spectrum of the kinematic dynamo operator
and of the
corresponding group it generates
for an ideally conducting fluid
in the
space of continuous divergence free vector fields.


Consider a steady incompressible conducting
fluid with Eulerian velocity $v=v(x)$
for $x\in \bbR^3$ and let $\phi^t$ denote the corresponding flow.
The kinematic dynamo equations for the induction of
a magnetic field ${\bf H}$ by the flow has the
following form:
\begin{equation}\label{kdeq}
 \dot{\bf H}  =  \nabla\times (v\times {\bf
H}) +  \varepsilon \Delta {\bf H},\quad
 \div{\bf H}  = 0,
\end{equation}
where $\varepsilon= {{\cal R}e_m}^{-1}$, and ${\cal R}e_m$ is the
magnetic Reynolds number (see, e.g., \cite[Ch.~6]{Moffatt}).
The spectral properties of the kinematic dynamo operator
$L_\varepsilon$, defined by \eqref{kdeq}, have been a subject of
intensive study, in particular, in connection with
the famous dynamo problem (see \cite{Arnold,AZRS,BC,FV,MRS,Vishik}
and references therein).

For the
ideally conducting fluid, $\varepsilon =0$,
these equations become:
\begin{equation}\label{kde}
\dot{\bf H}= - (v,\nabla) {\bf H} +
( {\bf H}, \nabla ) v,
\quad {\bf H}(x,0)={\bf H}_0(x),\quad \div{\bf H}  = 0.
\end{equation}
The last equation has \cite{Moffatt} so-called Alfven solutions
$${\bf H}(x,t)=D\phi^t(\phi^{-t}x){\bf H} (\phi^{-t}x,0),$$
given by the group $\{e^{tL}\}_{t\in\bbR}$ with the generator
$L=L_0$ that acts by the rule
\begin{equation}\label{defL}
L: u\mapsto  (u,\nabla)v - (v,\nabla) u.
\end{equation}

In the present paper the kinematic dynamo
operator $L$ is considered in the following  well known
context. Let $v$ denote a
continuous divergence-free vector field on
a compact Riemannian manifold $X$ without boundary, let
$\phi^t$ denote the flow generated by $v$ and let
$D\phi^t(x)$  denote its differential.
Consider the group $\{e^{tL}\}_{t\in\bbR}$ of push-forward
operators generated by the  Lie derivative $L$ in the direction
$v$.
This group acts on  continuous sections
of the tangent bundle $\TX$,
by the rule
\begin{equation}\label{defetL}
\left(e^{tL}u\right)(x)=
D\phi^t(\phi^{-t}x)u(\phi^{-t}x),\qquad x\in X,\quad t\in\bbR.
\end{equation}
We will consider
the group $\{e^{tL}\}_{t\in\bbR}$ in the space
$C_{ND}(X,\TX)$ of the  continuous
vector fields with zero divergence.

Operators of the form \eqref{defetL} belong to the class of
weighted composition operators. This class has been widely investigated
in connection with hyperbolic dynamical systems
since the celebrated paper by
J.~Mather \cite{Mather},
see also \cite{HPS,Pesin}, the recent papers \cite{CL,LMS1,LMS2,Rau}
and the detailed bibliography in \cite{LS}.
The spectral
properties of these operators in  spaces of continuous or
$p$-summable  vector  fields
are by now well understood.
However, the investigation of their spectral properties in the
space of divergence-free vector fields
was initiated recently by R.~de~la~Llave.
His important work \cite{Rafael}
inspired the present paper.

In Section 2 we  prove the spectral mapping
theorem in $\CND$ for the
group $\{e^{tL}\}_{t\in\bbR}$,
assuming that the aperiodic trajectories of $\phi^t$ are
dense in $X$ and $\dim X \geq 3$.
Theorems of this type for continuous and
for  $L_p$-section spaces over finite dimensional manifolds
were proved in \cite{CS,J} while
similar results for the infinite dimensional
setting were obtained in
\cite{LMS1,LMS2,LS}.

The spectral mapping theorem states that
the spectrum $\sigma(e^{tL})$ of $e^{tL}$
can be obtained  from the spectrum of $L$ by
exponentiation. It shows, in particular, that in the space of
divergence-free vector fields the spectral bound  of
the generator $L$ coincides with the
growth bound  of the
group, see also Remarks~2.9--2.11 below.

Our proof of the spectral mapping theorem in Section~2
exploits the fact (cf. \cite{Mather,Rafael}) that
approximative eigenfunctions of the operator \eqref{defetL} can be
``localized'' along trajectories of the flow.
We also show that the
spectrum of $L$ is invariant under vertical translations of the
complex plane.
The same
idea can be used to analyze
semigroups of weighted composition
operators with general
cocycles on Banach spaces, see \cite{LMSR}.

In Section~3,  we
show that, for $\dim X \geq 2$, the spectrum  $\sigma(e^{tL})$,
$t\neq 0$ in $C_{ND}(X,\TX)$ is exactly one annulus centered
at the origin of the complex plane.
Our result generalizes a theorem
in \cite{Rafael} where this fact was proved under the restriction
that the flow is Anosov with one-dimensional spectral foliations.
The possibility of relaxing the hypotheses of the theorem is
discussed as an open problem in \cite{Rafael}. The relatively
simple proofs of these facts in Section~3
can be read independently of the
more difficult proofs in Section~2.

By Mather's theory
(see, e.g., \cite{LS,Mather,Pesin}), the
spectrum $\sigma(e^{tL})$ on the space
$\C$ is generally the union of several disjoint annuli centered
at the origin. Passing to the space $\CND$ dramatically changes
the spectrum: the gaps, if any, between these annuli are filled.
Also, using the Spectral Mapping Theorem,
the spectrum of $L$ on
$\CND$ is exactly one
vertical strip. Since $Lv=0$, this strip always
contains $i\bbR$, and \eqref{kde} does not have an exponential
dichotomy. Moreover,  the equation \eqref{kde} in the space of
divergence free
vector fields (unlike the situation with Anosov flows, cf.
\cite{Pesin}), does not possess a nontrivial uniform exponential
dichotomy even after ``moding out'' the direction of the flow, see
Remarks~3.7--3.8 below.

Using the results in
\cite{LS}, we give a description of
the spectrum of $L$  in terms of the Lyapunov-Oseledets exponents
over all ergodic
measures on $X$, that is, we will determine
the boundaries of the spectrum of $L$
via the Lyapunov exponents.
This description is related to a theorem by M.~Vishik \cite{Vishik}
that states the ``fast'' dynamo action is
impossible whenever all Lyapunov
numbers are zero.

Finally, we remark that we actually prove more general results than
those just mentioned. In particular, we will only assume the
operator $L$ generates a
$C_0$-group
of weighted composition operators that {\em preserves the set of
divergence free vector fields} and has the form
\begin{equation}\label{defTt}
\left(T^tu\right)(x)=\Phi(\phi^{-t}x,t)u(\phi^{-t}x),
\qquad x\in X,\quad t\in \bbR,
\end{equation}
where $\Phi(x,t)$ is a continuous cocycle over $\phi^t$, that is
for $x\in X$ and $t,\tau\in \bbR$ one has
$\Phi(x,t+\tau)=\Phi(\phi^tx,\tau)\Phi(x,t)$ and $\Phi(x,0)=I$.
Throughout the paper we use $M$ to denote the generator of the
group $\{T^t\}$, $T^t=e^{tM}$. In particular, if
$\Phi(x,t)=D\phi^t(x)$, then \eqref{defTt} is the push-forward
operator \eqref{defetL} and $M=L$ is the Lie derivative.

We also note that our technique can be applied to obtain similar
results for the space $L_2$.

\section{The Spectral Mapping Theorem}
In this section we will prove the Spectral Mapping Theorem.
Throughout the section we suppose a smooth
vector field $v$ to be given
on a compact Riemannian manifold $X$ without boundary, that
$v$ is divergence free with respect to the Riemannian volume
and that the flow $\phi^t$ of $v$ satisfies the following
{\it standing hypotheses}:
the aperiodic trajectories of $\phi^t$ are
dense in $X$ and $n= \dim X \geq 3$.
Let $M$ be the generator of the group
$\{T^t\}_{t\in\bbR}$ of weighted
composition operators, as in \eqref{defTt},
for some continuous cocycle $\Phi(x,t)$
over $\phi^t$. We will assume $T^t$ is bounded on the space $\CND$
of divergence free vector fields.

There are at least two choices for the  space of continuous
divergence free vector fields depending on whether the divergence is
understood in the classical sense or in the sense of distributions.
These spaces are defined, respectively, as follows:
\begin{eqnarray}
&C_{ND}^0(X,\TX) & =\text{ closure }
\{f\in C^\infty(X,\TX): \div f=0\}, \label{class}\\
 & C_{ND}^1(X,\TX) & = \{f\in\C :\nonumber\\
&&\int\limits_X <f,\grad g> d\mu =0 \quad  \forall
g\in C^\infty(X,\bbR)\}.\label{distr}
\end{eqnarray}
The closure in \eqref{class} is taken with respect to the $\sup$-norm
while the scalar product $<\cdot,\,\cdot>$ and $\grad$ in \eqref{distr}
are taken with respect to a  Riemannian metric and volume on $X$.
We note that the space $C_{ND}^1(X,\TX)$
is a closed subspace of $C_{ND}^0(X,\TX)$.

For a linear operator $A$ in a Banach
space $E$, we will sometimes use
$\sigma (A;E)$ to denote the spectrum of $A$ on $E$
and $\sigma_{\mbox{ap}}(A;E)$ to
denote its approximate point spectrum. Since, as noted above,
$$C_{ND}^1(X,\TX)\subset C_{ND}^0(X,\TX)\subset \C$$
and since an $\epsilon$-eigenfunction for $T$
in $C_{ND}^1(X,\TX)$ is an
$\epsilon$-eigenfunction for $T$ in
$C_{ND}^0(X,\TX)$, one has
\begin{equation}\label{apprsp}
\hspace*{.25in}\sigma_{\mbox{ap}}(T;C_{ND}^1(X,\TX))
\subset \sigma_{\mbox{ap}}(T;C_{ND}^0(X,\TX))
\subset \sigma_{\mbox{ap}}(T;C(X,\TX)).
\end{equation}

Throughout the remainder of the paper the space $\CND$ may be
taken to be either
$C_{ND}^0(X,\TX)$ or
$C_{ND}^1(X,\TX)$.








\begin{thm}[Spectral Mapping Theorem]\label{SMT}
In the space of divergence-free
vector fields
$C_{ND}(X,\TX)$ the spectrum $\sigma (M;\CND)$
is invariant under vertical translations of
the complex plane. Moreover, for each $t\ne 0$,
\begin{equation}\label{exp}
\sigma (e^{tM}) = \exp t \sigma (M).
\end{equation}
\end{thm}

Since the proof of Theorem~\ref{SMT}
is quite technical, we pause to
discuss our strategy.
Using  standard facts from the
theory of $C_0$-semigroups and by rescaling, we reduce
the proof of Theorem~\ref{SMT} to the following main assertion
(see our  Lemma~\ref{main} below):
If $1\in\sigma_{\mbox{ap}}(T)$, $T:=T^1$ then
$0\in\sigma_{\mbox{ap}}(M)$.

Our strategy for the
proof of this main assertion develops some ideas
of J.~Mather \cite{Mather}. The fact that
$1\in\sigma_{\mbox{ap}}(T)$ implies  the existence of an
$\epsilon$-eigenfunction $u$ for $T$ for every $\epsilon>0$.
That is,
for every $\epsilon>0$,  there is a vector field $u$,
with unit norm
such that $\|Tu-u\|\le\epsilon$.
As in  \cite{Mather} and \cite{Rafael}, the
$\epsilon$-eigenfunctions of the operator
$T$ have a nice feature: they can be ``localized'' along the
trajectories of the flow. This means that for every $N\in\bbN$
there exist  a point $x^0\in X$, a
small neighborhood $D$ of this point, and a vector field $y$
with $\supp y\subset \cup_{j=-N}^N \phi^j(D)$, such that
$\|y-Ty\|=O(1/N)\|y\|$. In fact,
starting from a given
$\epsilon$-eigenfunction $u$ for $T$,
define a
``bump''-function $\alpha$ supported in $D$ and
let $\gamma(j)=(N-|j|)/N$.
The ``localization'' $y$ is
defined to vanish outside $\cup_{j=-N}^N \phi^j(D)$ and by
$y(x)=\gamma (j) \alpha(\phi^{-j}x)(T^ju)(x)$ for
$x\in \phi^j(D)$ and $|j|\leq N$,
equivalently,
\[
y(x)=\sum_{j=-N}^N \gamma (j)(T^j\alpha u)(x),\quad x\in X.
\]

To prove the main assertion above,
our purpose is to construct a
  vector field $y$ with zero divergence such that
$\|M y\| = O(1/N) \|y\|$. We start with a
divergence-free approximative eigenfunction $u$ of $T$.
Since the set of divergence
free vector fields is not closed under multiplication
by ``bump''-functions, we can not
use Mather's  construction directly.
Instead, we will
construct a divergence free vector-field $w$,
supported in a small neighborhood $D$ of a given point $x^0\in X$
in Lemma~\ref{loc}.
The main part of this construction takes place in a
special neighborhood $D$ of $x^0$,  taken to be a thin and long
``ellipsoid'' with
the longest axis directed along $u(x^0)$.
The required vector field $w$ is constructed in
the form $w(x)=\alpha(x) u(x^0)+w_n(x)$, where $\alpha$ is a
``bump''- function, supported in $D$.
The function $\alpha$ is chosen to have value identically one
on a second thin and long
``ellipsoid'' $B$ contained in $D$. Some ``fluid'' leaks from
the neighborhood $D$, but this can be recycled within a slightly
larger neighborhood.

The desired
almost-eigenfunction $y$ for $M$ is given by the formula
$$
y(x) = \int^{\infty}_{-\infty} \gamma(t) T^tw(x) \,dt,\quad x\in X,
$$
where, as above, $\gamma$ vanishes outside of $[-N,N]$.
Direct calculation shows that
$$
(My)(x) = -\int^{\infty}_{-\infty}
\gamma'(t) T^tw(x) \,dt, \quad x\in X.
$$
Since $\supp w\subset D$, the support of the integrand in each
integral belongs to
$\{t\in \bbR: |t|\le N\,\mbox{and}\,\phi^tx\in D\}$.
To obtain the desired inequality, we estimate
$\|My\|$ from above and $\|y\|$
from below.  This  requires some estimates of the sojourn time
of  the trajectory segment $\{\phi^t(x): |t|\leq N\}$ in
$D$ and $B$.
This is done in Lemma~\ref{bddtime}.

We start, for completeness, from the following simple lemma.
\begin{lem}
If  $x^0\in X$, then there is a coordinate chart at $x^0$,
with coordinate
functions $(x_1,\ldots,x_n)$, such that
the local representation of the volume element on $X$ is just the
usual volume $dx_1\wedge\cdots\wedge dx_n$ on $\bbR^n$.
Moreover, if $z\in T_{x^0}X$, then the coordinates
can be chosen so that the local representative of $z$  is
$\|z\|{\partial}/{\partial x_2}$.
\end{lem}
\begin{pf}
Let $y_1,\ldots,y_n$ denote local coordinates at $x^0$. Clearly,
there is a nonvanishing density function
$\rho:\bbR^n\to \bbR$ such
that volume element is given by
$\rho(y_1,\ldots,y_n)dy_1\wedge\cdots\wedge dy_n$.
We seek new coordinates
in the form
\[
y_1=f(x_1,\ldots,x_n),\quad y_2=x_2,\cdots,y_n=x_n
\]
where the volume element has the form
\[
\rho(f(x_1,\ldots,x_n),x_2,\ldots,x_n)
 \frac{\partial f}{\partial x_1}(x_1,\ldots,x_n)
 dx_1\wedge\cdots\wedge dx_n.
\]
There is a smooth function $f$, defined in a
neighborhood of the origin in $\bbR^n$, such that
\begin{equation*}
\frac{\partial f}{\partial x_1}(0,\ldots,0)\ne 0,\quad
\frac{\partial f}{\partial x_1}(x_1,\ldots,x_n)=
 (\rho(f(x_1,\ldots,x_n),x_2,\ldots,x_n))^{-1}.
\end{equation*}
The first condition together with  the Implicit Function Theorem
implies the change of coordinates is
invertible; the second condition
ensures the volume element in the new coordinates
has the desired form.

For the second statement of the lemma, note that the volume
element is invariant under a rigid rotation of Euclidean space.
\end{pf}

A coordinate chart, as in the lemma,
is called {\em adapted} to the volume on $X$ and the vector $z$.
Of course, in the adapted coordinates,
the Riemannian metric will not be the usual one, rather,
it will have the form $\Sigma g_{ij}(x_1,\ldots,x_n) dx_i\otimes dx_j$
where the components $g_{ij}$ form a positive definite symmetric
matrix of smooth functions. However, we make the following
observation: if
$u=(u_1,\ldots,u_n)$
is a vector field defined
in an adapted  coordinate chart, then
\begin{equation}
\label{divform}
\div u=\sum_{i=1}^n\frac{\partial u_i}{\partial x_i}.
\end{equation}

Suppose $z\in {\cal T}_{x^0}X$ is a tangent vector
and let
$(x_1,\ldots, x_n)$ denote local coordinates at $x^0$ adapted to
the volume on $X$ and the vector $z$.
If the adapted coordinate system is defined in a coordinate ball of
diameter $\delta>0$ and if
$a,b\in \bbR$ are such that
$0<a<b<\delta/8$, we define
\begin{eqnarray*}
D_{a,b}= &\{(x_1,\ldots, x_n): |x_j | \le 4b, \;
j=1,2, \;  |x_j |\le a,\; j=3,\ldots,n\},     \\
B_{a,b}= &\{(x_1,\ldots, x_n): |x_j | \le a/2, \;
j=1,3,\ldots,n, \;  |x_2 |\le b/2\}.
\end{eqnarray*}
Note that the closure of $B_{a,b}$ is contained in $D_{a,b}$.
We say there is an {\em $(a,b)$ divergence-free extension}
of the vector $z$ at $x^0$
if there is a smooth bump-function $\alpha:\bbR^n\to[0,1]$ with
$ \alpha(x) =1$
for $x \in B_{a,b}$ and
$\alpha(x) =0$ for $x\notin D_{a,b}$
and a continuously differentiable
vector field $w_n$ with support in $D_{a,b}$
such that
\begin{itemize}
\item[i)] The vector field
$w(x):=\alpha(x)\|z\|{\partial}/{\partial x_2}+w_n(x)$
is divergence free and has value $\|z\|{\partial}/{\partial x_2}$
in $B_{a,b}$,
% in some neighborhood of the origin,
\item[ii)] There is a number $C>0$ independent of $a,b$ such
that $\|w_n\|\le C a/b$.
\end{itemize}

\begin{lem}
\label{loc}
Every tangent vector on $X$ has an
$(a,b)$ divergence-free extension.
\end{lem}

\begin{pf}
Let $z \in {\cal T}_{x^0}X$.
We will first prove the lemma for the case  $n=2$.
To construct
a vector field $W$ with the required
properties in the $(x_1,x_2)$ coordinate plane, consider
the curves given by
\[
\dfrac{x^4_1}{a^4} + \dfrac{x^4_2}{b^4} =1,\quad
\dfrac{x^4_1}{a^4} + \dfrac{x^4_2}{b^4} = 2.
\]
Let $\rho:\bbR \to [0,1]$ denote a smooth function
such that $\rho(t)=1$
for $t\le 1$,
$\rho(t)=0$ for $t\ge 2$, and
$|\rho'(t) | \le 3$, $t\in \bbR$.
Also, define the sets
\begin{eqnarray*}
R&=&\{(x_1, x_2): |x_1| \le
2^{1/4}a, |x_2| \le 2^{1/4}b\}, \\
S^+&=&\{(x_1, x_2):
\dfrac{(x_1-2^{1/4}a)^4}{a^4} +
  \dfrac{x^4_2}{b^4}\le 2,\,x_1\ge 2^{1/4}a\}, \\
S^-&=&\{(x_1, x_2):
\dfrac{(x_1+2^{1/4}a)^4}{a^4} +
  \dfrac{x^4_2}{b^4}\le 2,\,x_1\le -2^{1/4}a\}
\end{eqnarray*}
and the functions $\theta:\bbR^2\to \bbR$, $f:\bbR^2\to \bbR$ and
$\eta^\pm:\bbR\to\bbR$ by
\begin{eqnarray*}
\theta(x_1, x_2) &=& \rho \big(\dfrac{x^4}{a^4} +
\dfrac{x^4_2}{b^4}\big),         \\
f(x_1,x_2) &=&- \dfrac{4x^3_2
\|z\|}{b^4} \int^{x_1}_0 \rho'
\bigg(\dfrac{s^4}{a^4} + \dfrac{x^4_2}{b^4}\bigg)\,ds, \\
\eta^+(\tau)&=&-\frac{4\| z\|}{b^4} \int^{2^{1/4}a}_0 \rho'
\bigg(\dfrac{s^4}{a^4} + \dfrac{\tau}{b^4}\bigg)\,ds, \\
\eta^-(\tau)&=&\frac{4\| z\|}{b^4} \int^0_{-2^{1/4}a}\rho'
\bigg(\dfrac{s^4}{a^4} + \dfrac{\tau}{b^4}\bigg)\,ds,
\end{eqnarray*}

The vector field $w_2$ is defined in $R$ by
$f(x_1,x_2)\partial/\partial x_1$,
in $S^+$ by
\[
 \eta^+((x_1-2^{1/4}a)^4+x_2^4)x_2^3\frac{\partial}{\partial x_1}
-\eta^+((x_1-2^{1/4}a)^4+x_2^4)(x_1-2^{1/4}a)^3\frac{\partial}{\partial x_2},
\]
in $S^-$ by
\[
 \eta^-((x_1+2^{1/4}a)^4+x_2^4)x_2^3
   \frac{\partial}{\partial x_1}
-\eta^-((x_1+2^{1/4}a)^4+x_2^4)(x_1+2^{1/4}a)^3
   \frac{\partial}{\partial x_2}
\]
and $w_2$ is defined to vanish on the complement of $R\cup S^+\cup S^-$.

We will complete the proof for $n=2$ by showing  the vector field
\[
W(x_1,x_2):=
\theta(x_1,x_2)\|z\|\frac{\partial}{\partial x_2}+w_2(x_1,x_2)
\]
is the required extension of $z$.

A direct computation using \eqref{divform} shows $\div W=0$ in the
coordinate chart. Also, using the definition of $\rho$, we see
the support of $W$ is in $D_{a,b}$. To show $W$ is $C^1$, just observe
that $w_2$ and each of its
first partial derivatives
is continuous on the lines
$x_1=\pm 2^{1/4}a$ and on the boundary of $R\cup S^+\cup S^-$.
(We remark that additional smoothness can be obtained, if desired,
by using the function  $x_1^k/a^k+x_2^k/b^k$ with $k$ a sufficiently
large positive integer in place of the choice $k=4$ used here.)

To obtain the required norm bound, let $G(x_1,\ldots,x_n)$ denote
the matrix of the components $g_{ij}$ of the Riemannian metric
in the adapted coordinates. The square of the norm of a vector $V$ at
$x=(x_1,\ldots,x_n)$ is then given by $\langle G(x)V,V\rangle$. Thus, if
$\|G\|$ denotes the supremum of the matrix norms  over the points
in the chart, we have $\|V\|\le  \|G\||V|$ where the single bars denote
the usual norm in $\bbR^n$.
Then, for example, using the usual estimate for the integral in the
definition of $f$, we estimate the norm of $w_2$ in $R$ by
\[
\|G\|\sup |f(x_1,x_2)|\le \|G\|(4\|z\|2^{3/4}b^3/b^4) 3 (2^{1/4}a)
\le C_1 a/b
\]
where the constant $C_1$  does not depend on $a$ or $b$.
Similarly, we can estimate the norm of $w_2$ in $S^\pm$. For example,
in $S^+$ we find the upper bound
\[
\|G\|\sup\big|\eta^+((x_1-2^{1/4}a)^4+x_2^4)\big|\,
((2^{1/4}b)^6+(2^{1/4}a)^6)^{1/2}
\le C_2 a/b.
\]

To prove the lemma for the case $n\ge 3$, we will show
how to extend the vector field $W$ defined above to a vector field on
$\bbR^n$ with the required properties. To do this,
let $\chi : \bbR \to [0,1]$ denote a smooth ``bump''-function
such that
$\chi (t) =1$ for $|t|\le\dfrac{a}{2}$ and
$\chi (t) =0$ for $|t|\ge a$,  and
define
\[
\Psi(x_3,\ldots,x_n) =  \prod^n_{j=3} \chi (x_j).
\]
The required vector field $w$ is given by
\[
w(x_1,\ldots,x_n)=\Psi(x_3,\ldots,x_n)W(x_1,x_2).
\]

The fact that the new vector field $w$ is continuously
differentiable, agrees with $\|z\|{\partial}/{\partial x_2}$
in $B_{a,b}$,
and is  supported in $D_{a,b}$ is clear.
Also, since the range of $\Psi$ is
the unit interval, the norm bound on $w_n$
is the same as the
norm bound on $w_2$. To complete the proof
we must show $\div w=0$.
But, since the vector field $w$ has nonzero components only
in the first two coordinate directions,
\[\div w(x_1,\ldots,x_n)=\Psi(x_3,\ldots,x_n)\div W(x_1,x_2)=0,\]
as required.
\end{pf}

To estimate the time that trajectories spend in a neighborhood of
$x^0$, we first need the following observation.

\begin{lem}\label{le1} Suppose $x^0\in X$ is not a periodic point
for the vector field $v$  with flow $\phi^t$. If  $N$ is a positive
integer and  $0<s\le N$, then there is a $\delta >0$ such that every
neighborhood $D$ containing $x^0$, with $\diam {D}\le\delta$, has the
following property: if $x\in D$, then $\phi^tx\not\in D$ for
$s\leq |t|\le N$.
\end{lem}
\begin{pf}
Suppose the lemma is false and, for each positive integer $k$,
let $D_k$ denote the ball centered
at $x^0$ with radius $1/k$.
For each $k$, there is some $x_k\in D_k$ and some
$t_k$ in the set $J:=\{t: 0<s\le|t|\le N\}$ with $\phi^{t_k}(x_k)$
in $D_k$. Since $J\times \bar D_1$
for the closure $\bar D_1$ of $D_1$
is compact, there is a convergent
subsequence of the pairs $(t_k,x_k)$. But, by the choice of $D_k$, the
second component of this sequence converges to $x^0$ and,
by the compactness of $J$,
the limit $T$ of the first component satisfies $|T|\ge s>0$.
The continuity of the flow ensures that $\phi^T(x^0)=x^0$,
in contradiction to the fact that $x^0$ is not periodic.
\end{pf}

Consider  a vector field $v$ on $X$ tangent to the flow $\phi^t$.
For each open set $U\subset X$, each non
negative integer $N$ and each
point $x\in X$, define
\begin{eqnarray}\label{defm}
\Theta_{N,U}(x)&:=&\{t\in\bbR: |t|\leq N, \phi^tx\in U\},\\
   m_{N,U}(x)&:=&\mes {(\Theta_{N,U}(x))}.\nonumber
\end{eqnarray}
\begin{lem}\lb{bddtime}
Suppose $X$ has dimension $n\ge 3$.
%%  and $v$ is as in \eqref{defm}.
If $\epsilon>0$, then
there is a constant $K>0$ such that for each non periodic point
$x^0$ and positive integer $N$ there is a pair of  numbers
$a,b$  such that $\epsilon>b>a>0$ and $a/b \leq\epsilon$ together with
a pair of open sets $B,D$ at $x^0$ such that
$B\subset B_{a,b}$ and $D_{a,b}\subset D$, with the following property:
for each  $x\in X$,
\begin{equation}\label{bddn}
\dfrac{m_{N,D}(x)}{m_{N,B}(x^0)} \leq K.
\end{equation}
\end{lem}
\begin{pf}
By Lemma~\ref{le1},
there is a neighborhood $\widehat{D}$ at $x^0$ such that
$\phi^ty\not\in \widehat{D}$ whenever $y\in\widehat{D}$
and $1/2\le |t|\le 2N$. Suppose $K,a,b,B,D$,
with $D\subseteq \widehat{D}$,
are given so that the inequality \eqref{bddn}
holds for $x\in D$.
We claim that \eqref{bddn} holds for all $x\in X$.
To see this, note first that, for $y\in D$, we have
$\Theta_{N,D}(y) = \Theta_{2N,D}(y)$.
Thus,
by our definition,
$m_{N,D}(y)=m_{2N,D}(y)$.
If  $x\in X$ and $D\cap \{\phi^tx: |t|\le N\}=\emptyset$, then
$m_{N,D}(x)=0$ and \eqref{bddn} holds. Otherwise, fix
$y\in D\cap \{\phi^tx: |t|\le N\}$.
Since $\Theta_{N,D}(x)\subset \Theta_{2N,D}(y)$,
we have
$$
m_{N,D}(x) \leq \max_{y\in D} m_{2N,D}(y)=
\max_{y\in D}m_{N,D}(y)\le K\cdot m_{N,D}(x^0),
$$
as required.

To complete the proof,
we will construct $K,a,b,B,D$ so that \eqref{bddn} holds for $x\in D$.
This will require several steps.

{\bf Step 1.} We will work in an adapted coordinate system at $x^0$ with
coordinate functions $(x_1,\ldots,x_n)$.
We will determine the required sets $B,D$ for appropriate $a,b$ in the
form
\begin{eqnarray*}
B= & \left\{x:
\dfrac{x_1^4}{\left(a/2\right)^4}
+\dfrac{x_2^4}{\left(b/2\right)^4}+
\sum_{j=3}^4\dfrac{x_j^4}{\left(a/2\right)^4}\leq 1
\right\}, \\
D= & \left\{x:
\dfrac{x_1^4}{\left(4nb\right)^4}
+\dfrac{x_2^4}{\left(4nb\right)^4}+
\sum_{j=3}^4\dfrac{x_j^4}{\left(na\right)^4}\leq 1
\right\}.
\end{eqnarray*}
If  $a<b$, then, clearly,
$B\subset B_{a,b}$ and $D_{a,b}\subset D$.
We will show that there is a constant $K$ such that for some choice of
$a,b$ and $a/b$ all sufficiently small,
the inequality \eqref{bddn} is valid  for the corresponding set
$D$.

We will use the following auxiliary constructions:

For each $\delta>0$, let  $S_\delta$ denote a section for
$v$ at the origin of the coordinate
system, that is, at $x^0$, such that the Riemannian diameter
$\diam S_\delta<\delta$, and define
$\Sigma_\delta:=\{\phi^t\sigma:\sigma\in S_\delta, |t|\leq\delta\}$.
Consider the local representation of $v$ given by
$\sum_{i=1}^n v_i(x)\partial/\partial x_i$. By a rigid rotation, if
necessary, we can and will arrange the adapted coordinates so that
$v_1(0)=0$. Also, we define $V$ by
\[V(x):=\sum_{i=3}^nv^4_i(x), \quad x\in\Sigma_\delta\]
and the number
$$
M_\delta:=\max \left\{
\max_{x\in\Sigma_\delta} |v_1(x)|,\quad
\max_{x\in\Sigma_\delta}
\left(\sum_{i=3}^nv^4_i(x)\right)^{1/4}\right\}.
$$

If $V(x^0)=0$, then $\lim_{\delta\to 0}M_\delta\to 0$. In case
$V(x^0)=0$ and
$M_\delta\neq 0$ for every $\delta$, we will only
consider $a,b$ such that
\begin{equation}\label{ab}
a=b \cdot \left(M_\delta\right)^{1/2}.
\end{equation}
Of course,  even under the restriction just imposed, $a,b$,
$a/b$ can each
be chosen arbitrary small.
If $V(x^0)\neq 0$ or if  $V(x^0)= 0$ and
$M_\delta\equiv 0$ for
all sufficiently small $\delta$, we ignore this restriction.

{\bf Step 2.} For each $\delta$, Lemma~\ref{le1} and the definition of
$\Sigma_\delta$ together imply there is
an open ball $A_\delta \subset\Sigma_\delta$
at the origin such that,
for each $x\in A_\delta$ and for each time $t$ with $|t|>\delta$,
the point $\phi^tx$ is not in $\Sigma_\delta$.
If $\delta>0$ is given,
choose $a,b$ as required in \eqref{ab}
and so small that $D$ is in
$A_\delta$. If  $x\in D$, let $x'$
denote the point on $\partial D$
where the segment of the trajectory
$\{\phi^tx: |t|\leq N\}$ first enters
$D$ and let $x''=\phi^{t_D}x'$ denote
the point of $\partial D$ where the
segment of the trajectory $\{\phi^tx: |t|\leq N\}$ last exits
$D$. Clearly, $m_{N,D}(x)\leq t_D$.

We use the Mean Value Theorem for integrals
on the $j$th component
of the vector field $v$ to obtain a point $ \xi^j\in\Sigma_\delta$
such that
\[ x''_j=x'_j+\int\limits_0^{t_D} v_j(\phi^t x) \,dt=
x'_j+t_D v_j(\xi^j).\]
For each $j=1,\dots,n$,  let $ v_j^*=v_j(\xi^j)$.
Also, as an abbreviation, define $\alpha_j=4nb$ for $j=1,2$ and
$\alpha_j=na$ for $j=3,\dots, n$.

Since $x'$ and $x''$ both belong to
$\partial D$, we have
$$\sum_{i=1}^n\left(\dfrac{x'_i}{\alpha_i}\right)^4=1
\qquad \text{and}\qquad
\sum_{i=1}^n\left(\dfrac{x'_i+t_Dv_i^*}{\alpha_i}\right)^4=1.
$$
Using a standard inequality for the norm
$\|(\gamma_i)\|=\left(\sum_i|\gamma_i|^4\right)^{1/4}$, we find
\begin{eqnarray*}
& 1=  \sum_{i=1}^n\left(\dfrac{x'_i+t_Dv_i^*}{\alpha_i}\right)^4
\geq  \left(\sum_{i=1}^n\left(\dfrac{t_Dv_i^*}{\alpha_i}\right)^4
\right)^{1/4} -
\left(\sum_{i=1}^n\left(\dfrac{{x'}_i}{\alpha_i}\right)^4
\right)^{1/4}\\
& =  \left(\sum_{i=1}^n\left(\dfrac{t_Dv_i^*}{\alpha_i}\right)^4
\right)^{1/4}-1.
\end{eqnarray*}
This computation yields the estimate
\begin{equation}\label{tD}
t_D^4\sum_{i=1}^n\left(\dfrac{v^*_i}{\alpha_i}\right)^4\leq 2^4.
\end{equation}

{\bf Step 3.} Consider the time  $t_B$  when
the segment of the trajectory $\{\phi^tx^0: |t|\leq N\}$ first
leaves  $D$. Clearly, $m_{N,B}(x^0)\geq t_B$.

Recall that in our local coordinates, $x^0$ resides
at the origin and  define $\tilde{x}=\phi^{t_B}(x^0)$.
As in Step~2,  by the the Mean Value Theorem,
there is some $\eta^j\in\Sigma_\delta$ such that
$$
\tilde{x}_j=\int\limits_0^{t_B}v_j(\phi^t(x^0))\,dt=
t_Bv_j(\eta^j).
$$
Define  $\tilde{v}_j:=v_j(\eta^j)$, the numbers
$\beta_j=a/2$ for $j=1,3,\dots, n$ and $\beta_2=b/2$.
Since $\tilde{x}\in\partial B$, we have
\begin{equation}\label{tB}
t^4_B \sum_{j=1}^n\left(\dfrac{\tilde{v}_j}{\beta_j}\right)^4=1.
\end{equation}

{\bf Step 4.}
In accordance with the previous notation, we define
$$
\tilde{V}=\sum_{i=3}^n\tilde{v}^4_i,\quad
V^*=\sum_{i=3}^n\left(v^*_i\right)^4.
$$
We use \eqref{tD}-\eqref{tB} to obtain the estimate
$(t_D/t_B)^4\leq 2^{16} n^4\cdot d$ where
$$
d:=\dfrac{\tilde{v}_1^4+\left(\dfrac{a}{b}\tilde{v}_2\right)^4+
\tilde{V}}{\left(\dfrac{a}{b}v^*_1\right)^4+
\left(\dfrac{a}{b}v^*_2\right)^4+4^4 V^*}.
$$

We will show that for all sufficiently small $\delta>0$,
there are some choices of  $a,b$ such that $d<2$.
There are several cases.

Case 1. If $V(x^0)\neq 0$, then $\lim_{\delta\to 0}d= 4^{-4}$.

Case 2. If $V(x^0)= 0$ and
$M_\delta\equiv 0$ for all sufficiently
small $\delta$,
then, since $v_2(x^0)\neq 0$, $\lim_{\delta\to 0}d= 1$.

Case 3. Suppose  $V(x^0)= 0$ and $M_\delta\neq 0$. However, note
that we still have
$\lim_{\delta\to 0}M_\delta\to 0$.
Also, the restriction we imposed in \eqref{ab} provides that
\[
\dfrac{b}{a}\tilde{v}_1^{1/2}\leq 1,\quad
\dfrac{b}{a}\tilde{V}^{1/8}\leq 1.
\]
In this case, we have
\begin{equation*}
d  = \dfrac{
\left(\dfrac{b}{a}\tilde{v}_1\right)^4
+\tilde{v}_2^4+\left(\dfrac{b}{a}\right)^4\tilde{V}}
{\left(v_1^*\right)^4+\left(v_2^*\right)^4+
4^4\left(\dfrac{b}{a}\right)^4V^* }
 \leq
\dfrac{
\tilde{v}_1^2+\tilde{v}_2^4+\tilde{V}^{1/2} } {({v}_2^*)^4}.
\end{equation*}
Passing to the limit as $\delta\to 0$, we see that
the last expression converges to $1$.
\end{pf}

We need the following elementary fact.
\begin{lem}\label{opA}
Suppose  $A$ denotes an
invertible bounded operator on a Banach space $E$ and let $N\in \bbZ$.
If $N\geq 2$ and
$1\in\sap(A)$, then there is a vector $u\in E$ with
$\|u\|_E=1$ such that   $\|A^ku\|_E\leq 2$ for each integer $k$ with
$|k|\leq N$.
\end{lem}
\begin{pf} Set
$\displaystyle{\epsilon=\left(\sum_{k=-N}^N\|A^k\|\right)^{-1}}$.
Since $1\in\sap(A)$, there is some
$u\in E$ with $\|u\|_E=1$ such that $\|Au-u\|\leq \epsilon$.
Also, for $1\leq |k| \leq N$, note that
$$
A^k-I=\left(\sum_{j=0}^{k-1}A^j\right)(A-I),\;
k>0,\quad
A^k-I=-\left(\sum_{j=-1}^{k}A^j\right)(A-I),\;
k<0.
$$
Hence, for $ |k|\leq N$, we have
$\displaystyle{ \|A^ku-u\|\leq
\sum_{j=-N}^N\|A^j\|\cdot \|Au-u\|
\leq 1}$
and, as a result,
$\|A^ku\|\leq
\|A^ku-u\|+\|u\|\leq 2$.
\end{pf}

The main result of this section is the following lemma.
\begin{lem}\label{main}
Suppose $e^{tM}=T^t$ is the group defined in
\eqref{defTt}and define $T:=T^1$.
If $1\in\sap(T;\C)$, then $\sap(M;\CND)$
contains the imaginary axis
of the complex plane.
\end{lem}
In accordance to \eqref{apprsp} this lemma also shows that
$1\in\sap(T,\CND)$ implies $0\in\sap(M,\CND)$
for  both cases:
\[ \CND=C_{ND}^0(X,\TX) \text{  or  }
\CND=C_{ND}^1(X,\TX).\]

\begin{pf}
Let $\xi\in\bbR$ and let $K$ be defined as in Lemma~\ref{le1}. Also,
for notational convenience,  define
$\omega:=1/(12K)$.

Since $1\in\sap(T)$,  Lemma~\ref{opA} applied to the bounded
linear operator $T=T^1$ ensures that, for each integer $N\geq 2$,
there is some vector field $u\in C(X,\TX)$ such that
\begin{eqnarray}
\|u\|_{C(X,\TX)}=1, \label{1a}\\
\|T^ku\|_{C(X,\TX)}\leq 2 \mbox{ for } |k|\leq N. \label{1b}
\end{eqnarray}

Since $\{T^t\}$ is a $C_0$-semigroup, $T^tu\to u$ in
$C(X,\TX)$ as $t\to 0$. Thus, there is a real number $s$ with
$0<s\le 2N$ such that
\begin{eqnarray}
\|T^tu-u\|_{C(X,\TX)}\leq \omega \mbox{ for } |t|\leq s, \label{2}\\
| e^{-i\xi t} -1|\le\omega
\mbox{ for } |t|\leq s. \label{5cd}
\end{eqnarray}
Also, there is a smooth function $\gamma:\bbR\to [0,1]$ such that:
\begin{eqnarray}
\gamma(t)=0  \mbox{ for } |t|\geq N, \label{3a}\\
|\gamma'(t)|\leq\dfrac{2}{N}, \mbox{ for } t\in\bbR, \label{3b}\\
\gamma (t)=1  \mbox{ for } |t|\leq s. \label{3c}
\end{eqnarray}

In view of \eqref{1a},
and the fact that the non periodic points are dense in $X$, there is a
non periodic point $x^0\in X$ such that
\begin{equation}\label{4}
\|u(x^0)\|\geq \dfrac12 \|u\|_{C(X,\TX)}= \dfrac12.
\end{equation}
Use Lemma~\ref{bddtime} to find small $a,b$ with small
$a/b$, and neighborhoods $D\supset B\ni x^0$,
such that \eqref{bddn} holds. Moreover, in accordance with
Lemma~\ref{le1}, we can choose $a,b$ sufficiently small
so that for $D:=D_{a,b}$, for
$s$ from \eqref{2}--\eqref{5cd} and with
$c:=\max_{|t|\le 1}\|T^t\|$
we have
\begin{equation}\label{out}
\phi^ty\not\in D
\text{ for any } y\in D \text{ provided }
s\le |t|\le 2N
\end{equation}
and, for some constant $C$, the following inequalities:
\begin{eqnarray}
C \dfrac{a}{b} \left( \max_{|t|\leq N}\|T^t\|
\right)\leq
\omega,\label{5a}\\
\max_{y\in D,  |t|\leq N} \|\Phi(y,t)u(x^0)\|\leq 4c.
\label{5c}
\end{eqnarray}
For the last inequality, we use \eqref{1b} to show
that
\begin{eqnarray*}
&&\max_{|t|\leq N}\|\Phi(x^0,t)u(x^0)\|\leq
\max_{|t|\leq N, x\in X}\|\Phi(x,t)u(x)\|   \\
&& \leq
\max_{|t|\leq N, x\in X}
\|\Phi(\phi^{-t}x,t)u(\phi^{-t}x)\|
=\max_{|k|\leq N-1, |\tau | \leq 1}\|T^{k+\tau}u\|    \\
&& \leq c \max_{|k|\leq N-1}\|T^k\|\leq 2c.
\end{eqnarray*}
Since $\Phi: (x,t)\mapsto \Phi(x,t)$ is uniformly continuous
on the compact set $X\times [-N,N]$,
we have \eqref{5c} for a sufficiently
small neighborhood $D$ of $x^0$.

We use Lemma~\ref{loc} with $z=u(x^0)$. After a
rigid rotation, if necessary, we can arrange the adapted coordinates
so that the component of $v(x^0)$ in the direction of the first coordinate
vanishes.
Then, for this choice of adapted coordinates,
there is  a divergence-free
vector field of the form
\begin{equation}\label{w}
w(x)=\alpha (x)u(x^0)+w_n(x)
\end{equation}
with $\alpha$ and
$w_n$ supported in $D$,
\begin{equation}\label{w0}
\|w_n\|_{C(X,\TX)} \leq C\dfrac{a}{b},
\end{equation}
and $\alpha(x)=1$ for $x\in B_{a,b}$.

Define the  vector field $y$ on $X$ by
$$
y(x) = \int^{\infty}_{-\infty} e^{-i\xi t}
\gamma(t) T^tw(x)\, dt.
$$
 We see that $y$ has zero divergence (for
this remember $T^t$ preserves the divergence-free vector fields).
By easy computations with $My=\left.\frac{d}{d\tau}T^\tau
y\right|_{\tau=0}$ one has:
 $$
My=\int^{\infty}_{-\infty}\left.\dfrac{d}{d\tau}\big(e^{-i\xi
(t-\tau)}\gamma (t-\tau)\big)\right|_{\tau=0}T^twdt =i\xi
y-\int^{\infty}_{-\infty} \gamma '(t)e^{-i\xi t}T^tw\,dt. $$
To complete the proof, we must show that  $i\xi \in
\sigma_{ap}(M;\CND)$. This is an immediate consequence of the
following proposition: There is a number $A>0$  that
does not depend on the choice of $N$ such that
$$\left\|
\int^{\infty}_{-\infty}
\gamma ' (t)e^{-i\xi t} T^tw \,dt\right\|_{C(X,\TX)} \le
\dfrac{A}{N} \|y\|_{C(X,\TX)}.
$$

To prove the proposition, fix $x\in X$ and note that
\begin{equation*}
\|My(x) - i\xi y(x)\| =
  \left\| \int^{\infty}_{-\infty} e^{-i\xi t}
\gamma'(t)(T^tw)(x)\,dt\right \|\le I_1
+I_2
\end{equation*}
where, by \eqref{3a} and \eqref{3b},
\begin{eqnarray*}
I_1 & = \dfrac{2}{N} \int\limits^N_{-N}
\alpha (\phi^{-t}x) \|
\Phi^t(\phi^{-t}x,t)u(x^0)\| \,dt,\\
I_2 & = \dfrac{2}{N} \int\limits^N_{-N}
\|\Phi(\phi^{-t}x,t)\| \;
\|w_n(\phi^{-t}x) \| \,dt.
\end{eqnarray*}
Since $\supp \alpha\subset D$
and $\supp w_n \subset D$,
the integrations $I_1$ and $I_2$
can be restricted to $\Theta_D(x)=\Theta_{N,D}(x)$,
see the notation
in \eqref{defm}.
We use \eqref{5c} to obtain:
\begin{equation}\label{I1}
I_1\leq \dfrac{2}{N}\int\limits_{\Theta_D(x)}
\max_{y\in D,|t|\leq N}\|\Phi(y,t)u(x^0)\|
\leq \dfrac{2}{N} 4 c \cdot m_{N,D}(x).
\end{equation}
We use \eqref{5a} and \eqref{w0} to estimate $I_2$:
\begin{eqnarray}
 I_2 &\leq & \dfrac{2}{N}\int\limits_{\Theta_D(x)}
\max_{y\in D,|y|\leq
N}\|\Phi(y,t)\|\|w_n\|_{C(X,\TX)}\,dt\label{I2} \\
&\leq &
\dfrac{2}{N} \max_{|t|\leq N}\|T^t\| \cdot
C\dfrac{a}{b}\cdot m_{N,D}(x)
 \leq\dfrac{2}{N} \omega m_{N,D}(x). \nonumber
\end{eqnarray}
We obtain the desired upper estimate from \eqref{I1} and \eqref{I2}, namely,
\begin{equation}\label{above}
\|My-i\xi y\|_{C(X,\TX)} \le
\dfrac{A_1}{N}\max_{x\in X}m_{N,D}(x).
\end{equation}

To determine  the lower bound, we define
\begin{eqnarray*}
J_1 &=& \left\| \int^{\infty}_{-\infty}
e^{-i\xi t} \gamma (t) \alpha(\phi^{-t}x^0)
u(x^0) \,dt\right\|, \\
J_2 &=& \left\| \int^{\infty}_{-\infty}
e^{-i\xi t} \gamma (t) \alpha(\phi^{-t}x^0)
\left[ (T^tu)(x^0)-u(x^0)\right] \,dt \right\|, \\
J_3 &=& \left\|\int^{\infty}_{-\infty} e^{-i\xi t} \gamma (t)
(T^tw_n)(x_0)\right\|
\end{eqnarray*}
and note that
$$ \|y\|_{C(X,\TX)}\ge \|y(x_0)\|  \geq
 J_1  - J_2 - J_3.$$
Again, each integral is equal to its restriction to
$\Theta_D(x^0)=\Theta_{N,D}(x^0)$.

As in \eqref{I2}, we use \eqref{w0} and \eqref{5a}
to estimate $J_3$ from above:
\begin{equation}\label{J3}
J_3 \leq
\int\limits_{\Theta_D(x^0)}\|(T^tw_n)(x^0)\| \,dt
\leq \omega m_{N,D}(x^0).
\end{equation}

Next, we use \eqref{2} to estimate $J_2$ from above. For this,
note that from \eqref{out} if $t\in \Theta_D(x^0)$, then $|t|<s$. Thus,
we have
\begin{equation}\label{J2}
J_2\leq
\int\limits_{\Theta_D(x^0)}\|T^tu-u\|_{C(X,\TX)} \,dt\leq
\omega m_{N,D}(x^0).
\end{equation}

Finally, we  estimate $J_1$ from below:
\begin{equation}\label{J1}
J_1=  \|u(x^0)\| \Big|
\int_{\Theta_D(x^0)}
e^{-i\xi t}\gamma (t)
\alpha(\phi^{-t}x^0)\,dt \Big|
\ge J_{11} - J_{12},
\end{equation}
where
\begin{eqnarray*}
J_{11} &=& \big|
\int_{\Theta_D(x^0)}
\gamma (t)\alpha(\phi^{-t}x^0)\,dt
\big|\cdot \|u(x^0)\|, \\
J_{12} &=& \big|
\int_{\Theta_D(x^0)}(e^{-i\xi t}
-1)\gamma (t)\alpha(\phi^{-t}x^0)\,dt \big|
\cdot \|u(x^0)\|.
\end{eqnarray*}

Since, by \eqref{out}, $\phi^tx^0\not\in D$ for $s\le |t|\le 2N$,
equation \eqref{3c} gives $\gamma(t)=1$ for $t\in \Theta_D(x^0)$.
As $\alpha (x)=1$ for $x\in B$, we use \eqref{4} to compute
the estimate:
\begin{equation}\label{J11}
J_{11}\ge \dfrac12 \int\limits_{\Theta_B(x^0)}
\alpha(\phi^{-t}x^0) \,dt \ge
\dfrac12 m_{N,B}(x^0).
\end{equation}

Since $\|u(x^0)\|\le 1$ and $|t|\le s$
for $t\in \Theta_D(x^0)$, the inequality
\eqref{5cd} implies:
\begin{equation}\label{J12}
J_{12}\le
\int\limits_{\Theta_D(x^0)}\left|e^{-i\xi t}-1\right| \alpha
(\phi^{-t}x^0)\,dt \le \omega m_{N,D}(x^0).
\end{equation}

The estimates
\eqref{J11}, \eqref{J12}, \eqref{J2}, \eqref{J3}, and \eqref{bddn}
together with
the our choice of $\omega=1/(12K)$ give the following:
\begin{eqnarray*}
 \|y\|_{C(X,\TX)} & \ge & \dfrac12 m_{N,B}(x^0)- 3\omega
m_{N,D}(x^0)\\
 & \ge & \dfrac{1}{2K}\max_{x\in X}m_{N,D}(x)
-3\omega\max_{x\in X}m_{N,D}(x)\\
& = & \dfrac{1}{4K}\max_{x\in X}m_{N,D}(x).
\end{eqnarray*}
By combining this estimate with \eqref{above}, we have the desired result.
\end{pf}

We are now in the position  to prove Theorem~\ref{SMT}.

\begin{pf}
It is well-known (see, e.g., \cite{Pazy}) that the Spectral
Inclusion Theorem
 \begin{equation}\label{SIT}
\sigma (e^{tM}) \supset \exp t \sigma
(M), \quad t\neq 0
\end{equation}
 holds for any $C_0$-semigroup.
Also, the spectral mapping theorem is true for the point and
residual spectrum. Therefore,
to prove \eqref{exp} one needs to show
that,  in $\CND$,
\[\sap (e^{tM}) \subset \exp t \sap (M), \quad t\neq 0.\]

Fix
$\mu=|\mu|e^{i\theta}\in\sap(e^{tM};\CND)$. Then $\mu=e^{t\lambda}$
for $\lambda=\dfrac{1}{t}\ln |\mu|+i\dfrac{\theta}{t}$. Consider
the cocycle $\tilde{\Phi}(x,t)=e^{-t\lambda}\Phi(x,t)$, and the
group $\{\tilde{T}^t\}$, $\tilde{T}^t=e^{t\tilde{M}}$,
defined by $\tilde{\Phi}(x,t)$ as in
\eqref{defTt}. Then $\mu\in\sap(e^{tM},\CND)$ implies that
$1\in\sap(e^{t\tilde{M}};\CND)$. By \eqref{apprsp}
and Lemma~\ref{main}, we have $0\in\sap(\tilde{M};\CND)$. But, since
$\tilde{M}=M-\lambda$, this implies $\lambda\in\sap(M;\CND)$.

To prove that $\sigma (M)$ is invariant under the translations
along the imaginary axis, we fix $\lambda\in\sap(M)$ and
$\xi\in\bbR$. By the Spectral Inclusion Theorem for $t=1$ we have
$1\in\sap(e^{\tilde{M}})$, also, $\tilde{M}=M-\lambda$. By
Lemma~\ref{main}, one has $i\xi\in\sap(\tilde{M})$ and, as a result,
$\lambda+i\xi\in\sap(M)$. \end{pf}

We will use notations
$$s(A):=\sup\{
\text{Re} z: z\in\sigma(A)\} \text{ and }
\omega(A):=\lim_{t\to\infty}t^{-1}\ln\|e^{tA}\|$$
for the spectral bound of
a generator $A$ and the growth bound of a $C_0$-semigroup
$\{e^{tA}\}$, respectively. Note \cite{Pazy}, that for an arbitrary
$C_0$-semigroup $\{e^{tA}\}$ one has $s(A)\leq \omega(A)$, but,
generally, $s(A)\neq \omega(A)$. The Spectral Mapping Theorem,
however, gives for the group of weighted composition operators the
following fact.
\begin{cor} \label{somega} In the space of continuous
divergence free vector fields the spectral bound and the growth bound
are equal: $s(M)=\omega(M)$.
\end{cor}

\noindent{\bf Remark~2.9.} Consider the kinematic dynamo operator
$L_\varepsilon=L+\varepsilon\Delta$ with $\varepsilon>0$ (see
\eqref{kdeq}). This is an elliptic operator, it generates an
analytic semigroup, and the spectral mapping theorem is
valid \cite{Pazy} for this semigroup. Hence,
$s(L_\varepsilon)=\omega(L_\varepsilon)$ for $\varepsilon>0$. For
$L_0=L$,  Corollary~\ref{somega} shows that this equality is also
valid for $\varepsilon=0$.

\noindent{\bf Remark~2.10.} M.~Vishik \cite{Vishik} has shown that
$\limsup_{\e\to 0}\omega(L_\e)\leq \omega(L_0)$. In view of
Remark~2.9, this theorem can be reformulated as
$\limsup_{\e\to 0}s(L_\e)\leq s(L_0)$. We stress that
the last assertion does not involve the construction of the group
$\{e^{tL_\e}\}$; it is given in the terms of generators only.
See Remark~3.9 below for the connection of this assertion
to the ``fast''-dynamo problem. Our formulation suggests
that the validity of the assertion
$\limsup_{\e\to 0}s(L_\e)= s(L_0)$
can be approached as a problem from the theory
of singular perturbations for the generators of
$C_0$-semigroups.

\noindent{\bf Remark~2.11.} The Spectral Mapping Theorem for
semigroups of weighted composition operators does not hold without
the assumption that aperiodic trajectories are dense in $X$ (see
\cite{CS,LS} for examples). However, in the
space $\C$ (and $L_2$, see  \cite{CS,LS}), this assumption is not
required to prove the following Annular
Hull Theorem:
\begin{equation}\label{AHT}
\exp t \sigma(M)\subset \sigma(e^{tM}) \subset {\cal H}\left(\exp t
\sigma(M)\right), \quad t\neq 0,
\end{equation}
where ${\cal H}(\cdot )$ is the union of the circles centered at
origin, that intersect the set $(\cdot )$. We conjecture the
assertion \eqref{AHT} is valid in $\CND$. Note that the equality
$s(M)=\omega(M)$ is an immediate consequence of \eqref{AHT}.

\section{Description of the Spectrum}
In this section we will describe the spectrum $\sigma(e^{tM})$ in
the space $C_{ND}(X,\TX)$ under the assumptions of the previous
section:
$X$ is a compact  Riemannian manifold
with $\dim X\ge 3$  and the divergence-free
vector field $v$ on $X$ generates
the flow $\phi^t$ whose aperiodic points  are
dense in $X$. However, in fact,
all the results of this section are valid provided $\dim X\ge 2$.

By Theorem~\ref{SMT} it suffices to determine $\sigma(e^{tM})$ for
a single value of $t$, say, for $t=1$.
As a notational convenience, we
define the bounded operator $T$ in $C(X,\TX)$ by
$T=e^M$.  For example, if $M=L$ is the Lie derivative in the
direction $v$, then, for $x\in X$,
\[ (Tu)(x)=D\phi(\phi^{-1}x)u(\phi^{-1}x).\]
Also, we let $T_{ND}:=T \left| C_{ND}(X,\TX)\right.$ denote the
restriction of $T$ to the subspace $C_{ND}(X,\TX)$.

We will prove that $\sigma(T_{ND},\CND)$ is exactly one annulus,
centered at the origin whose inner and outer boundaries
are the boundaries of $\sigma(T,\C)$.
The proof is based on the
following simple idea.  We will show that
both spectra are rotationally invariant and the
approximate point spectra
of $T$ and $T_{ND}$ coincide. Under the assumption that in the space of
divergence-free vector fields
$\sigma(T_{ND},\CND)$ has a gap, we will extend the Riesz
projection for $T_{ND}$ from $\CND$ to $\C$. To construct this
extension, we will approximate a continuous vector field by a
linear combination of locally supported divergence-free vector
fields. The Riesz projection for $T_{ND}$ can
be applied to each such divergence-free vector field and
this extension turns out to be a Riesz
projection for $T$ in $\C$. By Mather's theory this Riesz projection
will be an operator of multiplication by a
continuous matrix-valued function.
This multiplication must preserve $\CND$, a
contradiction.

To approximate a continuous vector field by a
linear combination of locally supported divergence-free vector
fields, we will need the following restricted
form of Lemma~\ref{loc}:
\begin{lem}\label{loc2}
If  $x^0\in X$ is a non periodic point of $v$ and if
$z \in {\cal T}_{x^{0}}X$, then, for each
pair $B,D$ of
sufficiently small neighborhoods with
$D\supset B \ni x^0$,  there is a coordinate chart  with
coordinate functions $(x_1,\ldots,x_n)$ at
$x^0$ containing $D$ and a
vector field $f\in C_{ND}(X,\TX)$ with
$\supp f\subset D$ such that the local representative of $f$
in $B$ is given by the constant vector field
$\displaystyle{\sum_{i=1}^nz_i\frac{\partial}{\partial x_i}}$
whose components, $z_i$,  are the components of the local
representative of the vector $z$.
\end{lem}

By a theorem  of Mather \cite{Mather} (see also \cite{CS,LS}),
the spectrum $\sigma_{ap} (T)$ in $\C$
is  invariant with respect to rotations about the
origin  in the complex plane.
As a corollary of  Theorem~\ref{SMT} we have the following two
assertions. We note, that these two assertions were also proved in
\cite{Rafael}.
\begin{cor}\lb{cor1} $\sigma_{ap} (T_{ND}; \CND)$ is
rotationally invariant. \end{cor}
\begin{cor}\lb{cor2}
$\sigma_{ap} (T; C(X,\TX)) =
\sigma_{ap}
(T_{ND}; C_{ND}(X, \TX)).$
\end{cor}
\begin{pf}
In view of \eqref{apprsp} and the fact that the
spectra $T_{ND}$ and $T$ are rotationally invariant,
the assertion will be proved as soon as we show
the following proposition:  If $1\in\sap(T; C(X,\TX))$, then
$1\in\sap(T_{ND}; C_{ND}(X,\TX))$.
By the Spectral Inclusion Theorem \eqref{SIT}, to prove this
proposition it is enough to show that
$0\in\sap(M_{ND};\CND)$ provided $1\in\sap(T)$ in $ C(X,\TX))$.
This is done in
Lemma~\ref{main}.
\end{pf}

In accordance with \cite{Mather} (see also \cite{CS,LS}),
the set $\sigma(T;\C)$ generally consists of
several disjoint annuli centered at the origin.
Let $r_-$ (resp., $r_+$) denote the radius of the inner most
(resp., outer most) circle in $\sigma(T; C(X,\TX))$.
By Corollary~\ref{cor2}, we have
$\sigma(T_{ND},C_{ND} (X,\TX))\subset \{z: r_-\leq |z|\leq r_+\}$.
We will show the set $\sigma(T_{ND},C_{ND} (X,\TX))$ is exactly
this  annulus. This is the content of the  next theorem.
\begin{thm}\lb{t4}
The spectrum
$\sigma(T_{ND}, C_{ND} (X,\TX))$ of $T$ in  $C_{ND} (X,\TX)$ is
the annulus
$\{z:r_{-}\le |z| \le r_{+} \}$.
\end{thm}
\begin{pf}
Suppose the theorem is not true, then there is a gap in the spectrum
$\sigma(T_{ND})$ in $\CND$. Without
loss of generality,  we can assume
there is an annulus $\{z:r_1 \le |z| \le r_2\}$ in the resolvent
set of $T_{ND}$ containing the unit circle $\bbT$
and $r_- \le r_1 < 1< r_2\le r_{+}$.
In this case,
there is a Riesz projection
$P=P_{ND}$ for the
operator $T_{ND}$ in $C_{ND} (X,\TX)$ corresponding to the
part of $\sigma(T_{ND}, C_{ND} (X,\TX))$ that lies
inside of the unit disc $\bbD$. In addition, there
are positive constants $C_1, C_2$ such that
\begin{eqnarray}
\Im P & = &
 \{f\in C_{ND}(X,\TX): \nonumber\\
&& \|T^nf\|_{\C}
\le C_1 r^n_1\|f\|_{\C},\, n\in\bbN\},\label{imp}\\
\Im{(I-P)}
& = & \{ f\in C_{ND}(X,\TX):  \nonumber\\
&& \| T^{-n}
f\|_{\C} \le C_2
r^{-n}_2 \|f\|_{\C},\, n\in \bbN\}.\label{imq}
\end{eqnarray}

We will construct a projection $\calP$ in $C(X,\TX)$
that commutes with
$T$ and has the following additional properties:
\begin{equation*}
\sigma (T|\Im \calP, C(X,\TX))\subseteq \bbD,\quad
\sigma ([T|\Im{(I-\calP)}]^{-1},C(X,\TX))\subset \bbD.
\end{equation*}
In other words, the operator $T$ is hyperbolic in $C(X,\TX)$, that is,
\[\sigma(T,\C)\cap \bbT=\emptyset,\]
and $\calP$ is the Riesz projection for $T$ in $\C$.
It follows, see \cite{Mather} and
\cite{LS}, that the
projection $\calP$
has a form
$\calP f (x) = P_C(x)f(x)$, where $P_C:X\to {\proj}({\cal T}_xX)$
is a continuous
projection-valued function.

Note, that $\calP=P_{ND}$ on $C_{ND}(X,\TX)$.   Hence,
$\calP$ maps $C_{ND}(X,\TX)$ into itself.
We claim that
this implies $\calP$ is either the identity or the zero
operator, in contradiction to the fact that $T$ is hyperbolic.
To prove the claim, consider local (adapted) coordinates
$(x_1,\ldots,x_n)$ so that
the divergence operator is given as in \eqref{divform}.
The projection $\calP$ is represented by a matrix valued
function with components $\calP_{ij}(x)$. For each divergence-free
vector field $u$, we then have
\begin{equation}
\label{divfreeeq}
0=\div( \calP u)=\sum_{i,j}\frac{\partial\calP_{ij}}{\partial
x_i}u_j +\calP_{ij}\frac{\partial u_{j}}{\partial x_i}.
\end{equation}
For each point $x$ in the coordinate chart and each index pair
${i,j}$ with $i\ne j$, there is a divergence-free
vector field $u$ such that $u(x)=0$ and
$\partial u_{j}(x)/{\partial x_i}=\delta_{ij}$ where $\delta_{ij}$
is Kronecker's delta. With this choice of $u$, \eqref{divfreeeq}
shows $\calP_{ij}=0$ for $i\neq j$.
 Using that fact that $\calP$ is a projection, we
have $\calP(\calP-I)=0$ and it follows that each diagonal element,
$\calP_{ii}(x)$, is either
zero or  one. Since $\calP$ preserves all
divergence-free vector fields, it is easy to see that all diagonal
elements must then be equal  and $\calP$ is as required in the coordinate
chart. The desired result follows by continuity and the connectivity
of $X$.

We will construct the required projection $\calP$ in the space $\C$.

{\bf Step 1.} We introduce ``step-functions'' in $\C$.

Since $X$ is compact, there is a partition of unity
$\{\displaystyle \rho_k\}^{K}_{k=1}$ with $K<\infty$, that is,
for each integer  $0<k\le K$, the function $\rho_k:X\to [0,1]$ is
continuous, and, for each $x\in X$,
\begin{eqnarray}
& \sum^{K}_{k=1} \rho_k (x) = 1, \label{summ}\\
& \supp  \rho_k\setminus
(\cup\sb{\ell \notin k} \supp \rho_{\ell})
\neq \emptyset. \label{suppm}
\end{eqnarray}
In particular, there is some
$x_k\in\supp  \rho_k\setminus\cup\sb{\ell \notin k} \supp \rho_{\ell}$
such that $\rho_k(x_k)=1$.

For each set of  vectors  $u_1,\ldots,u_K$,
the vector field
\[g(x)=\sum^{K}_{k=1} \rho_k(x) u_k\]
is  continuous, and $\|g\|_{C(X,\TX)} =
\sup_k \| u_k \|$.
Indeed, if $\| u_{k'}\| = \sup_k \| u_k\|,$
we can use \eqref{suppm} to choose $x_0$
such that $\rho_{k'}(x_0) =1$ for some
$x_0\in \supp \rho_{k'}\bs \cup\sb{k\notin k'} \supp \rho_k$.
Then,
$$
\| g\| \ge \| g(x_0) \|=\| \sum_k \rho_k
(x_0) u_k\|=\| \rho_{k'} (x_0)
u_{k'}\|=\|u_{k'}\|.
$$
On the other hand, using \eqref{summ},
\begin{equation*}
\|g\|  = \max\sb{x\in X} \| \sum_k \rho_k (x)u_k \|
\le \max\sb{x\in X} \sum_k
\rho_k (x) \| u_k\| \le \sup_k \| u_k \|.
\end{equation*}

It is also easy to see that the set $\frak G$ of all
such ``step-functions'' $g=\sum_k \rho_k u_k$ is dense in
$C(X,\TX)$.

{\bf Step 2.} We will  define $\calP$ for $g\in\frak G$.

Suppose $g=\sum_k \rho_k u_k\in\frak G$.
Without loss of generality we can assume the partition
of unity is so fine that
 Lemma~\ref{loc2} is applicable for each $k$. By this lemma,
 for each
$k$, there is a section $f_k \in C_{ND}(X,\TX)$ such that
$f_k(x)= u_k$ for $x\in \supp \rho_k$ and $\| f_k \|_C = \|u_k \|$.
Then, for each $x\in X$,
\begin{equation}\label{deff}
g(x) = \sum_k \rho_k (x) f_k(x),\quad f_k\in\CND.
\end{equation}
We define $\calP g$ and $\calQ g$ as follows:
\begin{equation}\label{defpq}
\calP g(x) = \sum_k \rho_k(x) (Pf_k)(x), \quad \calQ g(x) =
\sum_k \rho_k (x)[(I-P) f_k](x),
\end{equation}
where $P=P_{ND}$ is the Riesz projection for $T_{ND}$ in $\CND$.
Formally, the decomposition
$g=\calP g+ \calQ g,$  depends upon
the choice of $f_k$. However,
we will show that, in fact, the definition
\eqref{defpq}
does not depend on this  choice and that
$\calP$  is a bounded linear operator on $\frak G$.
Once this is proved,
the unique bounded linear
extension of  $\calP$
to $\C$ is the desired projection.

{\bf Step 3.} Define
\begin{eqnarray*}
F_{+} &: = \{f\in C(X,\TX):\lim\sb{n\to\infty}\| T^n f\| =0\},\\
F_{-} &: = \{f\in C(X,\TX):\lim\sb{n\to\infty}\| T^{-n}f\| =0\}.
\end{eqnarray*}
We will show that $F_{+} \cap F_{-} = \emptyset$.

Clearly, $F_\pm$ are linear (not necessarily closed)
subspaces in
$C(X,\TX)$.
Assume $f\in F_{+} \cap F_{-}$ and $ f\neq 0$.
We have
$$
\lim\sb{n\to\pm\infty}\|T^nf\| =
\lim\sb{n\to\pm\infty}\max_x \| \Phi(x,n)
f(x) \| =0.
$$
In particular, there is some $x^0 \in X$ such that $f(x^0) \neq 0$
and
$$
\sup\sb{n\in\bbZ} \| \Phi(x^0,n) f(x^0) \| < \infty.
$$
This implies (see \cite{CS} or \cite{Mane}) that
$\bbT \subset \sigma_{ap} (T, C(X,\TX))$. Thus,
by Corollary~\ref{cor2}, we have
$\bbT \subset\sigma_{ap} (T_{ND}, C_{ND}(X,\TX))$. But, this
contradicts our assumption that $T_{ND}$ is hyperbolic in $\CND$.

{\bf Step 4.} We show that $\calP$ and $\calQ$ are
well-defined on $\frak G$.

Suppose $g=\sum \rho_k u_k\in\frak G$ and, for each $k$, the
section  $f_k$ is chosen as in \eqref{deff}.
Also, define $g_{+}:=\calP g$ and $ g_{-}:=\calQ g$.
We will show $ g_{\pm}\in F_{\pm}$.

Indeed, as $Pf_k\in \Im P$, using the  inequality
\eqref{imp}, we have
\begin{eqnarray*}
&\| T^n g_{+} \| &=  \| \sum_k \rho_k \circ
\phi^n T^n Pf_k\|_\C \\
&& \le \sup_k \| T^n Pf_k\|_\C
\max_x \sum_k \rho_k \circ \phi^n\\
&& \le  C_1r^n_1\sup_k \| Pf_k\|_\C
\le C_1r^n_1\| P\|\sup_k\| f_k\|_\C =\\
&& =  C_1 r^n_1\|P\|\sup_k \|u_k\| =
C_1 r^n_1\| P\|\cdot\| g\|_\C.
\end{eqnarray*}
In particular, $\lim_{n\to \infty}\| T^n g_{+} \|=0$ and
$g_{+}\in F_{+}$.
Similarly, using \eqref{imq}, we have
\begin{eqnarray*}
&&\|T^{-n}g_{-}\|_\C =  \|\sum_k\rho_k
\circ\phi^{-n} T^{-n} (I-P)f_k\|_\C  \\
&& \le \max_x \sum_k \rho_k\circ
\phi^{-n}(x) \sup_k \|T^{-n} (I-P)f_k\|_\C \\
&&\le  C_2 r^{-n}_2 \|I-P\| \sup_k \| f_k\|_\C = C_2 \| I-P \|
r^{-n}_2 \|g\|_\C.
\end{eqnarray*}
This implies
$\|T^{-n} g_{-} \| \to 0$ as
$n\to\infty$
and $g_{-}\in F_{-}$.

By Step~3, we have  $F_{+} \cap F_{-} = \{0\}$.
Hence, $g_{\pm}$, in the decomposition $g=g_++g_-$ with
$g_{\pm}\in F_{\pm}$, are uniquely defined.
If particular, the definition of $\calP g$ and $\calQ g$ in
\eqref{defpq} does not depend on the choice of $f_k$.

{\bf Step 5.} We extend $\calP$ and $\calQ$ from $\frak G$ to
$\C$.

{}From the  calculations in Step~4 with
$\tilde C_1:= C_1 \|P\|$ and $\tilde C_2:=C_2\| I-P\|$,
we have, for  $n\in\bbN$, that
\begin{equation*}
 \|T^n g_{+}\| C \le \tilde C_1
r^n_1 \|g\|,\quad
 \|T^{-n} g_{-}\|_C \le \tilde C_2
r^{-n}_2 \|g\|.
\end{equation*}
These inequalities, for $n=0$, show that $\calP$ and $\calQ$ are bounded
on $\frak G$. To complete the proof we will show
these operators are linear on $\frak G$.

Indeed, for
\[
g=\sum_{i=1}^{i_0} \rho_i u_i \text{ and  }
\ti{g}=\sum_{j=1}^{j_0} \ti{\rho}_j\ti{u}_j
\]
there are $f_i,\ti{f}_j\in\CND$ such that $g=\sum \rho_if_i$ and
$\ti{g}=\sum\ti{\rho}_j\ti{f}_j$. We define $f_{ij}=f_i$
 and $\ti{f}_{ij}=\ti{f}_j$ for $i=1,\dots,i_0$, $j=1,\dots,j_0$,
and use \eqref{summ} to obtain:
\[ g=\sum_{i,j}\rho_i\ti{\rho}_jf_{ij},\quad
\ti{g}=\sum_{i,j}\rho_i\ti{\rho}_j\ti{f}_{ij}.\]
Then, \eqref{defpq} gives:
\[ \calP (g+\ti{g})=\sum_{i,j}\rho_i\ti{\rho}_j(Pf_{ij}+P\ti{f}_{ij})=
\calP g +\calP\ti{g},\]
as required.
\end{pf}

{}From this theorem and Theorem~\ref{SMT} we conclude, that the spectrum
$\sigma(M)$ in the space $\CND$ of divergence-free vector fields
 is exactly one vertical strip:
\begin{cor}
$\sigma(M;\CND)=\{z: \ln r_{-}
\le \text{ Re }z\leq \ln r_{+} \}$.
\end{cor}

Our next goal is to characterize the spectra $\sigma (T_{ND}; \CND)$ and
$\sigma(M_{ND};\CND)$ via the {\em exact} Lyapunov exponents for the
cocycle $\Phi(x,t)$ with respect to the set of  $\phi^t$-ergodic
measures $\nu\in\calE$. Recall that, by the Multiplicative Ergodic
Theorem \cite{OselMET}, for each ergodic measure
$\nu\in\calE$, there exists a set $X_\nu\subset X$ with $\nu(X_\nu)=1$
such that for each $x\in X$ and $u\in {\cal T}_xX$ there exist {\em
exact} Lyapunov exponents
\begin{equation}\label{eLE}
\lambda_\nu(x,u)=
\lim_{t\to\pm\infty}\dfrac{1}{t}\ln\|\Phi(x,t)u\|.
\end{equation}
For each $\nu$, there may exist  $n'=n'(\nu)\le n$
different Lyapunov exponents; we will denote them by
$\lambda_\nu^1> \lambda_\nu^2 > \dots > \lambda_\nu^{n'}$.
\begin{cor} \label{calcul} The boundary circles of the
spectrum $\sigma (T_{ND};\CND)$ and the spectrum
$\sigma(M_{ND};\CND)$ are given by
\begin{equation*}
 \ln r_{+} = \sup \{\lambda^1_\nu : \nu\in\calE\},\quad
 \ln r_{-} = \inf \{\lambda^{n'}_\nu :
\nu\in\calE\}.
\end{equation*}
There exist measures $\nu_+$ and $\nu_-$ and exact Lyapunov
exponents $\lambda_{\nu_+}(x_+,u_+)$ and
$\lambda_{\nu_-}(x_-,u_-)$, such that the
$\sup$ and $\inf$ above are attained.
 \end{cor}
\begin{pf} For the
boundaries $r_{\pm}$ of the spectrum
$\sigma(T;\C)$ in $\C$
these formulas were obtained in \cite{LS} (see
also \cite{OselDYN}). \end{pf}



\noindent{\bf Remark~3.7.}
The absence of nontrivial
spectral components of $\sigma (L)$ for the space
$\CND$ leads to the following  observation.
 Consider a
 situation when  $L$ acts
in the space $\C$.
After some inessential modifications, we can obtain
a dichotomic (no spectrum on $i\bbR$) operator $L$.
For example, starting with an Anosov flow, as usual in Mather's
theory, such an operator can be obtained by
``moding out'' the direction of the flow.

We note that
``moding out'' the direction of the
flow does not change $\sigma(L)$.
The reason is that the spectrum
of $L$ on the direct sum of the quotient space $C_{ND}(X, \TX
/[v])$ and the space of sections generated by $v$ is the union of the
respective spectra.
An element of the second space must be a divergence-free
vector field of the form $\alpha v$ where $\alpha$ is a function on
the manifold. This implies $\grad \alpha=0$ so that $\alpha$
is constant along the trajectories of $v$. But, then $L\alpha v=0$
and the spectrum of $L$ on this subspace is $\{0\}$.
Since $\sigma(L_{ND};\CND)$ is
invariant with respect to
vertical translations in the complex plane,
the entire imaginary axis must be in $\sigma(L_{ND},\CND)$.
All the
points except the origin must then be in the spectrum of
$L$ restricted
to the quotient space.
But the spectrum of $L$ is closed,
thus the origin is already in the spectrum
on the quotient. In particular, the spectrum of the  quotient is
the same as the spectrum on the original space.

Going back to the  kinematic dynamo equations \eqref{kdeq}, we
make the following concluding remark.

\noindent{\bf Remark~3.8.} Recall (see,
e.g., \cite{Arnold,AZRS,BC}) that kinematic dynamo is called
``fast'' provided $\limsup_{\e\to\infty}\omega(L_\e)$ is positive.
M.~Vishik \cite{Vishik} gave the following sufficient condition for
the non-existence of a fast kinematic dynamo:
Define the Lyapunov numbers
\[
\bar{\lambda}(x,u)=\limsup_{t\to\infty}
\dfrac{1}{t}\ln\|D\phi^t(x)u\|.
\]
If
\begin{equation}\label{vish}
\sup \{
\bar{\lambda}(x,u):
x\in X, u\in \TX
\}\leq 0,
\end{equation}
then there is no fast kinematic dynamo.
The fact that
the spectral bound $\omega(L)$ is less than or
equal to the supremum in
\eqref{vish}, see Remark~2.10, is used in \cite{Vishik}.
Therefore, in view of \cite{Vishik},
our Corollary~\ref{calcul} gives an alternate form of the
sufficient condition for no fast kinematic dynamo.
\section{Acknowledgement}
We thank Misha Vishik for several very helpful conversations.
\begin{thebibliography}{99}

\bi{Arnold} V.~I.~Arnold,
{\em Some remarks on the antidynamo theorem,}
Moscow University Mathem.~Bull., {\bf 6} (1982) 50--57.

\bi{AZRS} V.~I.~Arnold, Ya.~B.~Zel'dovich, A.~A.~Rasumaikin, and
D.~D.~Sokolov, {\em Magnetic field in a stationary flow with
stretching in Riemannian space,} Sov.~Phys.~JETP, {\bf 54} (6)
(1981) 1083--1086.

\bi{BC} B.~J.~Bayly and S.~Childress,
{\em Fast-dynamo action in unsteady flows and maps in three
dimensions,}
Phy.~Rev.~Let. {\bf 59} (14) (1987) 1573--1576.

\bi{CL} C.~Chicone, and Y.~Latushkin,
{\em Quadratic Lyapunov functions for linear skew-product flows and weighted
composition operators,}
J.~Int.~Diff.~Eqns., to appear.

\bibitem{CS} C. Chicone and R. Swanson,
{\em Spectral theory for linearization of dynamical systems,}
J.~Diff.~Eqns.,  {\bf 40} (1981) 155--167.

\bi{FV} S.~Friedlander and M.~Vishik,
{\em Dynamo theory methods for hydrodynamic stability,}
J.~Math.~Pures Appl. {\bf 72} (1993) 145--180.

\bi{HPS} M.~Hirsch, C.~Pugh, and M.~Shub,
{\em Invariant Manifolds,}
Lect. Notes Math. {\bf 583} (1977).

\bi{J} R.~Johnson,
{\em Analyticity of spectral subbundles},
J.~Diff.~ Eqns., {\bf 35} (1980) 366--387.

\bi{LMS1} Y.~Latushkin and S.~Montgomery-Smith,
{\em Evolutionary semigroups and Lyapunov theorems in Banach spaces,}
J.~Funct.~Anal., to appear.


\bi{LMS2} Y.~Latushkin and S.~Montgomery-Smith,
{\em Lyapunov theorems for Banach spaces,}
Bull.~AMS {\bf 31} (1) (1994) 44--49.


\bi{LMSR} Y.~Latushkin, S.~Montgomery-Smith and T.~Randolph,
{\em Evolutionary semigroups and dichotomy of linear skew-product
flows on locally compact spaces with Banach fibres,} in preparation.

\bibitem{LS} Y.~Latushkin and A.~M.~Stepin,
{\em Weighted composition
operators and linear extensions of dynamical systems,}
Russian Math.~Surveys, {\bf 46}, no. 2, (1992) 95--165.

\bi{Mather} J.~Mather,
{\em Characterization of Anosov diffeomorphisms,}
Indag.~Math., {\bf 30} (1968) 479--483.

\bi{Mane} R.~Man\~e,
{\em Quasi-Anosov diffeomorphisms and hyperbolic manifolds,}
Trans.~Amer.~Math.~Soc., {\bf 229} (1977) 351--370.

\bi{Moffatt} H.~K.~Moffatt,
{\em Magnetic Field Generation in
Electrically Conducting Fluids,} Cambridge Univ. Press, Cambridge,
1978.

\bi{MRS} S.~A.~Molchanov, A.~A.~Ruzmaikin, and D.~D.~Sokolov,
{\em Kinematic dynamo in random flow,} Sov.~Phys.~Usp. {\bf 28} (4)
(1985) 307--327.

\bi{OselMET}  V.~I.~Oseledec,
{\em A multiplicative ergodic
theorem: Lyapunov characteristic numbers for dynamical systems,}
Trans.~Moscow Math.~Soc., {\bf 19} (1968) 197--231.


\bi{OselDYN} V.~I.~Oseledec, $\Lambda$-entropy and the anti-dynamo
theorem, In: {\it Proc. 6th Intern. Symp. on Information Theory,}
Part III, Tashkent, (1984) 162--163.

\bi{Pazy} A.~Pazy,
{\em Semigroups of Linear Operators and Applications
to Partial Differential Equations,}
Springer-Verlag, N.Y./Berlin, 1983.

\bi{Pesin} Ya.~B.~Pesin, {\em Hyperbolic theory}, in: Encyclop. Math. Sci.
Dynamical Systems {\bf 2} (1988).

\bi{Rafael}  R.~de~la~Llave,
{\em Hyperbolic dynamical systems and
generation of magnetic fields by perfectly conducting fluids,}
Preprint, 1993.

\bi{Rau}  R.~Rau,
{\em Hyperbolic linear skew-product semiflows,}
subm. to J.~Diff.~Eqns.

\bibitem{Vishik} M.~M.~Vishik,
{\em Magnetic field generation by the motion of
a highly conducting fluid},
Geophys.~Astrophys.~Fluid Dynamics,
{\bf 48} (1989) 151--167.

\end{thebibliography}
\end{document}





