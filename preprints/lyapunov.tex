\magnification=\magstep1
\baselineskip = 0mm

\def\Head#1:{\medskip \noindent {\bf #1:}\ \ }

\def\implies{$\Rightarrow$}

% If you comment out \input amssym.def to %\input amssym.def, it will still
% work, but black board font becomes bold face.  Do this if you do not
% have amssym.def.

\def\Bbb{\bf}
\input amssym.def
\def\R{{\Bbb R}}
\def\D{{\Bbb D}}
\def\T{{\Bbb T}}
\def\Z{{\Bbb Z}}

\def\Re{\mathop{\rm Re}}
\def\Im{\mathop{\rm Im}}
\def\Ker{\mathop{\rm Ker}}
\def\normo#1{\left\|#1\right\|}

\def\moreproclaim{\par}

%This is to get the smaller characters for the references

\font\tenrm=cmr10
\font\eightrm=cmr8
\font\ninerm=cmr9
\font\sixrm=cmr6
\font\eightbf=cmbx8
\font\sixbf=cmbx6
\font\eightit=cmti8
\font\eightsl=cmsl8
\font\eighti=cmmi8
\font\eightsy=cmsy8
\font\eightex=cmex10 at 8pt
\font\sixi=cmmi6
\font\sixsy=cmsy6
\font\ninesy=cmsy9
\catcode`@=11
\def\eightbig#1{{\hbox{$\textfont0=\ninerm\textfont2=\ninesy
\left#1\vbox to6.5pt{}\right.\n@space$}}}
\catcode`@=12
\def\eightpoint{\eightrm \normalbaselineskip=4 mm%
\textfont0=\eightrm \scriptfont0=\sixrm \scriptscriptfont0=\fiverm%
\def\rm{\fam0 \eightrm}%
\textfont1=\eighti \scriptfont1=\sixi \scriptscriptfont1=\fivei%
\def\mit{\fam1 } \def\oldstyle{\fam1 \eighti}%
\textfont2=\eightsy \scriptfont2=\sixsy \scriptscriptfont2=\fivesy%
\def\cal{\fam2 }%
\textfont3=\eightex \scriptfont3=\eightex \scriptscriptfont3=\eightex%
\def\bf{\fam\bffam\eightbf} \textfont\bffam\eightbf
\scriptfont\bffam=\sixbf \scriptscriptfont\bffam=\fivebf
\def\it{\fam\itfam\eightit} \textfont\itfam\eightit
\def\sl{\fam\slfam\eightsl} \textfont\slfam\eightsl
\let\big=\eightbig \normalbaselines\rm
}

\centerline{\bf Lyapunov Theorems for Banach Spaces}

\bigskip

\centerline {Y. Latushkin and S. Montgomery-Smith}


\footnote{}{\noindent 1991 {\it Math.\ Subject
Classification:}\
Primary 47D06, 47B38;
Secondary 34D20, 34G10.}

\footnote {}{\noindent
{\it Keywords:} Hyperbolicity, evolution family,
exponential dichotomy,
weighted composition operators, spectral mapping theorem.}

\bigskip

 {\leftskip=.5in  \rightskip=.5in \noindent {\bf Abstract.}
We present a spectral mapping theorem for semigroups on any Banach
space $E$.  From this, we obtain
a characterization of exponential dichotomy for
nonautonomous differential equations for $E$-valued
functions.  This characterization is given in
terms of the spectrum of the generator of the semigroup
of evolutionary operators.
\par }
\bigskip

\centerline {\bf 1. Introduction}
\medskip

\noindent Let us consider an autonomous
differential equation $y'=Ay$
in a Banach space $E$,
where $A$ is a generator of continuous
semigroup $\{e^{tA}\}_{t\geq 0}$,
that is, the solution to the differential equation
satisfies
$y(t)=e^{tA}y(0),\, t\geq 0$.
A classical result of A.M. Lyapunov (see,
e.g., [DK]) shows that for
{\it bounded} $A$, the spectrum $\sigma (A)$ of
$A$ is responsible for the
asymptotic behavior of $y(t)$. For
example, if $\Re \sigma (A)<0$, then
the trivial
solution is uniformly asymptotically
stable, that is $\|e^{tA}\|\to 0$ as
$t\to\infty$. This fact follows from
the spectral mapping theorem
(see, e.g.,
[N, p.82]): 
$$ \sigma(e^{tA}) \setminus \{0\} =
\exp(t\sigma(A)), \qquad t\ne0,
   \eqno(1) $$
which always holds for bounded $A$.

For unbounded $A$, equation (1) is not always true.
Moreover, there are examples
of generators $A$\ (see [N, p.61]) such that even
$\Re(\sigma(A)) \leq s_0< 0$\
does not guarantee $\sigma(e^A) \subset
\D = \{|z|<1\}$ or
$\normo{e^{tA}} \to 0$\ as $t\to\infty$.
Since $\sigma(A)$\ does not characterize
the asymptotic behavior of the
solutions $y(t)$, we would like to find some other
characterization that still does not
involve solving the differential equation (i.e.
finding  $\sigma(e^{tA})$).

In this article we solve precisely this
problem in the following manner.
Consider the space $L_p(\R;E)$\ of
$E$-valued functions for $1\le p <
\infty$, and the semigroup
$\{e^{tB}\}_{t\ge 0}$\ of evolutionary operators
(also called weighted translation operators)
$$ (e^{tB} f)(x) = e^{tA} f(x-t),~~ t\ge 0 \eqno(2) $$
generated by the operator $\displaystyle{B
= -{d\over dx} + A},\, x\in\R$.
It turns
out that it is $\sigma(B)$\ in $L_p(\R;E)$\
that is responsible for
the asymptotic behavior of $y(t)$\ in $E$.  For example,
$\Re(\sigma(B)) < 0$
on $L_p(\R;E)$
implies that $\normo{e^{tA}} \to 0$\ as
$t\to\infty$ on $E$.


We will also
consider the well posed nonautonomous equation
$y'=A(t)y(t)$.
Instead of the
semigroup given by (2), we consider in $L_p(\R;E)$ 
the semigroup
$$ (e^{tD} f)(x) =
U(x,x-t) f(x-t) \qquad x\in\R, t\ge 0. \eqno(3) $$
Here $U(t,s),~~t\ge s$\ is the
evolutionary family
(propagator) for the nonautonomous equation. We will show that
$\sigma (D)$ characterizes
the asymptotic behavior of $y(t)$.


The order of proofs will be as follows.  We will first
show the spectral
mapping theorem for the autonomous case (2).
We will also
consider
a similar theorem for the evolutionary
semigroup in the space of
the periodic
functions. This theorem will give us a
variant of Greiner's
spectral mapping
theorem (see [N, p.94]) for any $C_0$-semigroup
$\{e^{tA}\}$ in a
Banach space.
This variant also
is a direct generalization of Gerhard's
spectral mapping theorem
in Hilbert space for
generators with resolvent bounded along $i\R$ (see [N, p.95]).
Then we
will obtain the spectral mapping theorem
for the nonautonomous
case (3)
using a
simple change of variables argument
reducing it to the autonomous
case (2).

We will be considering not only stability
but also the exponential
dichotomy (hyperbolicity) for the solutions
of the equation
$y'=A(t)y(t)$ in $E$.
In the theory of differential 
equations with bounded coefficients, exponential dichotomy is
an important tool used, for example, 
in proving unstability theorems for nonlinear equations,
and determining
existence and uniqueness of bounded solutions and Green's
functions (see, e.g., [DK]). The spectral mapping 
theorem given here for the semigroup (3) allows one
to extend these ideas to the case of unbounded coefficients.

It turns out that the condition
$0\notin \sigma(D)$, or equivalently
$\sigma(e^{tD}) \cap \T = \emptyset,t>0,~~\T =
\{|z| = 1\}$ on
$L_p(\R; E)$, is equivalent
to the hyperbolic behavior of a special
kind for the solutions.
We will call this {\it spectral hyperbolicity}.
Note that if the $U(t,s)$ are invertible for
$(t,s)\in \R ^2$, that is (3) is
extendible to a group, then the
spectral hyperbolicity is the same as the
exponential dichotomy (or hyperbolicity)
in usual (see, e.g, [DK]) sense.
Therefore, the spectrum $\sigma (e^{tD})$ for nonperiodic
$A(\cdot)$ plays the same role in the description of 
exponential dichotomy as the spectrum of the monodromy
operator does in the usual Floquet theory for the periodic case.
However, ordinary hyperbolicity is not
equivalent (see [R]) to spectral
hyperbolicity in the {\it
semi}group case and thus cannot be
characterized in the
terms of $\sigma (e^{tD})$
only. For a Hilbert space,
we were able to characterize
hyperbolicity in terms of other spectral
properties of $e^{tD}$.

Finally, the results of this article can be generalized to the case of
variational equation $y'(t) =
A(\varphi^t x) y(t)$\ for a flow
$\{\varphi^t\}$\
on a compact metric space $X$, or for a
linear skew-product flow
$\displaystyle{ \hat\varphi^t:X\times E \to
X\times E :
   (x,y) \mapsto (\varphi^t x,\Phi(x,t)y),
   \quad t \ge 0,}$
(see [CS, H, LS, SS] and references contained therein).
Here $\Phi :X\times \R_+\to L(E)$\ is a cocycle
over $\varphi^t$,
that is,
$\Phi(x,t+s) = \Phi(\varphi^t x,s) \Phi(x,t)$,
$\Phi(x,0) = I$.
Let us recall (see [SS]) that one of the purposes
of the theory of linear skew-product
flows was to aid in studying the
equation $y'=A(t)y$ for the case of almost periodic $A(\cdot )$.
To answer the question when $\hat\varphi ^t$ is
hyperbolic (or
Anosov), instead of (3)
one considers
the semigroup of so called
weighted composition
operators (see [CS, J, LS]) on $L_p(X;\mu;E)$:
$$ (T^t f)(x) = \left( {d \mu \circ \varphi^{-t}
\over d \mu}\right)^{1/p}
   \Phi(\varphi^{-t}x,t) f(\varphi^{-t} x),
   \qquad x\in X,\,t\ge 0. \eqno (4) $$
Here $\mu$\ is a $\varphi^t$-quasi-invariant Borel
measure on $X$. As above, the condition $\sigma (T^t)\cap \T
=\emptyset$ is
equivalent to the spectral hyperbolicity of
the linear skew-product flow $\hat\varphi^t$.
The spectral hyperbolicity
coincides with the usual hyperbolicity if $\Phi(x,t)$, $x\in X, t\ge0$
are invertible or compact operators.
Unlike the finite dimensional case
(see [M]), the hyperbolicity of
$\hat\varphi^t$ does not generally imply the condition
$\sigma (T^t)\cap \T =\emptyset$. For Hilbert space
the condition of
hyperbolicity of $\hat\varphi^t$ can also be described
in the terms of other
spectral properties of $T^t$. We will not include these
generalizations in this
paper.

We point out that the investigation of evolutionary
operators (2) and (3) has a long history. More
recently, significant progress has been made
in [BG, N, P, R] (see
[R] for detailed bibliography),
and this list does not pretend to be complete.
A detailed investigation of weighted composition
operators
(4) for Hilbert
spaces $E$\ and connections with the
spectral theory of linear skew-product flows [SS]
and other questions of dynamical systems  theory
and a bibliography may be found
in [LS].
\smallskip

\noindent {\it Notation:} $L(E)$ (correspondingly
$L_s(E)$) denotes the set of
bounded operators on
$E$ with the uniform (correspondingly strong) topology;
$\rho (A) $ denotes the
resolvent set of the
operator $A$; $|$ denotes the restriction of an operator;
$C_b(\R; E)$ denotes
the space of continuous
bounded $E$-valued functions on $\R$ with the supremum norm, and
$C^0_b(\R; E)$ denotes
the subspace of functions vanishing at infinity;
$L_\infty^s=L_\infty (\R; L_s(E))$ denotes the space of
functions $a: \R\to L(E)$ such that for each $y\in E$
the function $x\mapsto a(x)y$ from $\R$ to $E$ is
strongly measurable and essentially bounded.
\medskip

\centerline {\bf 2.  Autonomous Case}
\medskip

\noindent Let $A$ be a generator of any $C_0$-semigroup
in a Banach
space $E$. Consider the $C_0$-semigroup (2)
on the space $L_p([0,2\pi);E),~~1\le p < \infty$,
that is  $(e^{tB}f)(x)=e^{tA}f\bigl((x-t)({\rm mod}2\pi)\bigr)$.

\proclaim Theorem 1.  The following are equivalent:
\item{1)} $1\in \rho(e^{2\pi A})$\ on $E$;
\item{2)} $0\in \rho(B)$\ on $L_p([0,2\pi);E)$;
\item{3)} $1\in \rho(e^{2\pi B})$\ on $L_p([0,2\pi);E)$.

\noindent In the main part of the proof 2)\implies 1),
we modify the idea of [CS]. Let us assume that 2)
is fulfilled, but for
each $0<\epsilon<1/2$, there is a $y\in E$\ such
that $\normo{e^{2\pi A}y-y} <
\epsilon$\ and $\normo y = 1$ (and hence $\normo
{e^{2\pi A}y }\geq 1/2$). Let
$\rho(x) = {
3\over 2\pi} x - 1$\ for $x \in
[{2\pi/ 3},{4\pi/3})$, $\rho(x) = 0$\ for $x
\in [0,{2\pi/3})$, and
$\rho(x) = 1$\ for $x \in [{4\pi/3},2\pi)$.
Define the function $f\in
L_p([0,2\pi);E)$\ by $\displaystyle {f(x) =
(1-\rho(x)) e^{(2\pi+x)A}y
	  + \rho(x) e^{xA} y,~
%   \qquad
x \in [0,2\pi) .}$
Then  for $c = \max\{\normo{e^{xA}}:x\in[0,2\pi)\}$,
one has $\normo{e^{2\pi A}y}=\normo{e^{(2\pi -x)A}e^{xA}y}
\leq c\normo{e^{xA}y}$.
But, in contradiction with 2),
$\displaystyle{\normo f_p^p \ge
\int\limits^{2\pi}_{{4\pi/3}}\normo{e^{xA}y}^pdx
\geq{2\pi\over 3}c^{-p}\normo {e^{2\pi A}y}^p
\geq {2\pi\over 3}c^{-p}2^{-p},~~{\rm and}~~
\normo{Bf}_p \le {2\pi\over 3} c
\epsilon.}$


Theorem~1 implies the following variant of Greiner's
spectral mapping
theorem (see [N, p.~94]) for a $C_0$-semigroup
$\{e^{tA}\}$ in Banach space $E$.

\proclaim Theorem 2.  $1\in \rho(e^{2\pi A})$\ if
and only if
a) $i\Z \subset \rho(A)$\ and b) there is a constant
$C$\ such that
for any finite sequence $\{y_k\} \subset E$
$$ \normo{\sum_k (A-ik)^{-1}
y_k e^{-ikx}}_{L_p([0,2\pi);E)}
   \le C\,
   \normo{\sum_k y_k e^{-ikx}}_{L_p([0,2\pi);E)} .$$

Obviously, if $E$\ is a Hilbert space and $p=2$,
then Parseval's identity
allows one to replace b) by the condition
$\sup\{\|(A-ik)^{-1}\|:k\in\Z\} < \infty$.  This
gives the famous
spectral mapping theorem of Gerhard ([N, p.~95]).

Let us consider (2) on $L_p(\R;E),~~1\le p < \infty$.
The spectrum
$\sigma(B)$\ now is
invariant under the translations along the imaginary axis.
Moreover
we have the following result.

\proclaim Theorem 3.  For each $t>0$\ the following
are equivalent:
\item{1)} $\sigma(e^{tA}) \cap \T = \emptyset$\ on $E$;
\item{2)} $0 \in \rho(B)$\ on $L_p(\R;E)$;
\item{3)} $\sigma(e^{tB}) \cap \T = \emptyset
$\ on $L_p(\R;E)$.

Thus, the spectral mapping theorem is valid for (2).
If $\{e^{tA}\}_{t\in\R}$\ is a group, then 1)
is the same as
the exponential dichotomy of the autonomous equation $y'=Ay$
on $\R$.
\medskip
\goodbreak
\centerline {\bf 3.   Nonautonomous Case}
\medskip

Consider the well posed nonautonomous equation
$y'(t)=A(t)y(t)$.
By `well posed', we mean that we assume the  existence of a jointly
strongly continuous
evolutionary family
$U(t,s) \in L_s(E),~~t\ge s$\ with the properties
$ U(t,t) = I,~
%    \quad
   U(t,r) = U(t,s) U(s,r)$, and
%    \quad
  $ \normo{U(t,s)} \le C e^{\beta(t-s)},
%   \qquad
   t\ge r\ge s $.
In fact, $U$ is a propagator for the equation
$y'(t)=A(t)y(t)$,
that is, $y(t) =
U(t,s) y(s)$. The spectral mapping theorem is valid for (3).

\proclaim Theorem 4. Let (3) be
a $C_0$-semigroup on $L_p(\R ;E),~1\leq
p<\infty$. Then $\sigma (D)$ is invariant under
translations along the
imaginary axis and the following are equivalent:
\item{1)} $0\in \rho (D)$ on  $L_p(\R ;E);$
\item{2)} $\sigma (e^{tD})\cap \T =\emptyset$ on
$L_p(\R ;E),~~t>0$.

\noindent To outline the proof of 1)\implies 2)
let us consider the semigroup
$(e^{tB}h)(s,x)=
U(x,x-t)h(s-t,x-t)$, $(s,x)\in \R ^2, t>0$
on the space $L_p(\R\times\R;E) = L_p(\R; L_p(\R;E))$,
and perform a change of variables $u=s+x,v=x$.
More precisely, consider the
isometry $J$ on $L_p(\R \times\R;E)$ defined
by $(Jh)(s,x)=h(s+x,x)$. Then one
has that $Je^{tB}=(I\otimes e^{tD})J$\ and $JB=(I\otimes D)J$,
where $A_1\otimes A_2$ means
that $A_1$ acts on
$h(\cdot ,x)$ and $A_2$ acts on $h(s,\cdot)$. Since
$e^{tB}$ can be written as
$(e^{tB}f)(s)=e^{tD}f(s-t)$ for $f:\R\to L_p(\R
;E):s\mapsto h(s,\cdot)$, one can apply to $e^{tB}$ the
part 2)\implies 3) of
Theorem 3.

\proclaim Definition.  {\rm The evolutionary family
$\{U(x,s)\}_{x\ge s}$
is called {\it hyperbolic\/} if
there exists a projection-valued function
$P:\R\to L(E),~~P\in L_\infty^s$\
and $M,\lambda>0$\ such that for all $x\ge s$
\item{1)} $P(x) U(x,s) = U(x,s) P(s)$, and
\item{2)} $\normo{U(x,s)y} \le M
e^{-\lambda(x-s)}\normo{y}$\ if
	  $y\in \Im P(s)$,
\item {}  $\normo{U(x,s)y} \ge M^{-1}
e^{\lambda(x-s)}\normo{y}$\ if
	  $y\in \Ker P(s)$.\
\moreproclaim\noindent
The evolutionary family $\{U(x,s)\}_{x\ge s}$
is called {\it spectrally hyperbolic\/} if,
in addition,
\item{3)} $\Im U(x,s)| \Ker P(s)$ is dense
in $\Ker P(x)$.}

Note that the second inequality in 2)
implies only left invertibility of the
restriction $U(x,s)|\Ker P(s)$,  while 3)
guarantees its invertibility.
If the evolutionary family
$\{U(x,s)\}_{(x,s)\in \R ^2}$ consists of invertible operators,
then spectral
hyperbolicity is equivalent to the hyperbolicity
and is the same as exponential
dichotomy [DK] of the equation $y'=A(t)y$ on $\R$.
If ${\rm dim}\Ker P(s) <d<\infty$, then obviously 2) always
implies 3). This also happens
if the $U(x,s)$ are compact operators in $E$
([R. Rau, private communication]).

\proclaim
Theorem 5. The evolutionary family $\{U(x,s)\}$
on a separable Banach space $E$ is spectrally
hyperbolic if and only if $0\in \rho (D)$ on $L_p(\R; E)$.


\noindent {\bf Remark.}
The space $L_p$\
may be replaced
by the space $C([0,2\pi);E)$ in Theorem 1
and by $C_b^0(\R; E)$ in Theorems 2 to 5.
\medskip

Under some additional conditions
Theorem 5 is valid even if we replace
the condition $P\in L_\infty^s$
in the definition above by the condition $P\in C_b(\R; L_s(E))$.
It is true, in particular, under the assumptions that $E$
is a separable Hilbert space and that
for some $r>0$ the function
$x\mapsto U(x+r,x)$ is a continuous
function from $\R$ to $L(E)$.
From now on we will restrict ourselves to these assumptions.

A remarkable
observation in [R] shows that
the hyperbolicity
of $\{U(x,s)\}_{x\geq s}$ (unlike the spectral hyperbolicity) does not
generally imply 2) in Theorem 4 for infinite dimensional
$E$. However,
we are able to give the following
characterization of hyperbolicity.

\proclaim
Theorem 6. The evolutionary family
$\{U(x,s)\}_{x\geq s}$
is hyperbolic on a separable Hilbert space $E$
if and only if there exists  a
projection ${\cal P}$ on $L_2(\R ;E)$
such that for some $t>0$: 
%the following holds:
\item{1)} $e^{tD}{\cal P}=
{\cal P}e^{tD};$ \item{2)}
$\sigma (e^{tD}|\Im {\cal P})\subset \D;
$ \item{3)} $e^{tD}|\Ker {\cal P}$
is left invertible and $\sigma\left((e^{tD}|\Ker {\cal
P})^\dagger\right)\subset \D;$
\item{4)} $\displaystyle
{\Ker {\cal P}\ominus
\cap _{n\geq 0}\Im (e^{ntD}|\Ker {\cal P})}$
is invariant with respect to multiplications
by 
the functions from 
$C_b(\R; \R)$.
\moreproclaim\noindent Each projection
${\cal P}$ with these properties has a form $({\cal P}f)(x)=P(x)f(x)$
for a projection-valued
function $P\in C_b(\R;L(E))$.

For left invertible operator $T$ notation
$T^\dagger$ above stands for its
Moore-Penrose left inverse: $T^\dagger u=v$
if $u=Tv$, and $T^\dagger u=0$
for
$u\perp \Im T$. Note, that 1), 2), 3) imply
the left invertibility of
$zI-e^{tD}$ for all $z\in \T$, and the formula
$\displaystyle {
{\cal P}=
{1\over
(2\pi i)
}
\int
\nolimits_{\T}
(zI-e^{tD})^\dagger dz,}
$
which gives the Riesz projection on $\sigma
(e^{tD}) \cap \D$ if $\sigma
(e^{tD})\cap \T =\emptyset$.\medskip


\centerline {\bf References}
\smallskip

\eightpoint

\noindent [BG] A. Ben-Artzi, I. Gohberg,
{\it Dichotomy of systems and
invertibility of linear ordinary
differential operators,} Operator Theory
Advances and Applications, {\bf 56} (1992), 90-119.

\noindent [CS] C. Chicone, R. Swanson, {\it
Spectral theory of linearizations of dynamical systems,}
J. Diff. Eqns. {\bf 40}
(1981), 155-167.

\noindent [DK] J. Daleckij, M. Krein,
``Stability of Differential
Equations in Banach Space", Amer. Math.
Soc., Providence, RI, 1974.


\noindent [H] J. Hale, ``Asymptotic
Behavior of Dissipative Systems",
Math. Surv. and Monogr. Vol. 25,
Amer. Math. Soc., Providence, RI, 1988.

\noindent [J] R. Johnson, {\it
Analyticity of spectral subbundles},
J. Diff. Eqns. {\bf 35} (1980), 366-387.

\noindent [LS] Y. Latushkin, A. Stepin,
{\it Weighted translations operators and
linear extensions of dynamical systems,}
Russian Math. Surveys {\bf 46} (1991),
95-165.

\noindent [N] R. Nagel (ed.), ``One
Parameters Semigroups of Positive Operators",
Lect. Notes Math. {\bf 1184},
Springer-Verlag, Berlin, 1984.



\noindent [M] J. Mather, {\it Characterization
of Anosov diffeomorphisms,}
Indag. Math. {\bf 30} (1968), 479-483.


 \noindent [P] K. Palmer, {\it Exponential
dichotomy and Fredholm operators,}
Proc. Amer. Math. Soc. {\bf 104} (1988),
149-156.

\noindent [R] R. Rau, {\it Hyperbolic
evolution groups and exponentially dichotomic
evolution families}, J. Funct. Anal., to appear.


\noindent [SS] R. Sacker, G. Sell, ``Dichotomies for
Linear Evolutionary
Equations in Banach Spaces", IMA Preprint No. 838 (1991).

\smallskip Department of Mathematics,
University of Missouri, Columbia, MO 65211.

{\it E-mail:} mathyl@mizzou1.missouri.edu,
stephen@mont.cs.missouri.edu

\bye
