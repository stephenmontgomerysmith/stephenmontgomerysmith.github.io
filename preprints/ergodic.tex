% 12 March 1996
\documentstyle[12pt]{article}
\setlength\textwidth{6in}
\setlength\oddsidemargin{0.25in}
\topmargin -20mm
\footskip 12mm
%\textwidth 38pc
\textheight 56pc
\def\Bbb#1{{\mathchoice{\mbox{\bf #1}}{\mbox{\bf #1}}%
{\mbox{$\scriptstyle \bf #1$}}{\mbox{$\scriptscriptstyle \bf #1$}}}}
\def\N{\Bbb N}
\def\R{\Bbb R}
\def\C{\Bbb C}
\def\D{\Bbb D}
\def\Z{\Bbb Z}
\def\T{\Bbb T}
\def\Q{\Bbb Q}
\def\E{\Bbb E}
\def\O{\Omega}
\def\sgn{{\rm sgn}}
\def\supp{{\rm supp}}
\def\cH{{\cal H}}
\def\e{\epsilon}
\def\cA{{\cal A}}
\def\cE{{\cal E}}
\def\cL{{\cal L}}
\def\cI{{\cal I}}
\def\cB{{\cal B}}
\def\cM{{\cal M}}
\def\cT{{\cal T}}
\def\Linfg{L^\infty (G)}
\def\Hinfg{H^\infty (G)}
\def\spec{{\rm spec}}

\begin{document}
\title{A transference theorem for 
ergodic $H^1$}
\author{Nakhl\'e Asmar
 and Stephen Montgomery--Smith
\\
Department of Mathematics\\
University of
Missouri--Columbia\\
Columbia, Missouri 65211  U.\ S.\ A.
}
\date{}
\maketitle
			     %
			     %
			     %
			     %
			     %
			     %
			     %
			     %
%\begin{abstract}
%\baselineskip=10 pt
%\end{abstract}
%%%%%%%%%%%%%%%%%%%%%%%%%%%%%%%%%%%%%%%%%%%%%%%%%%%%%%%%%%%%%%%%%%%%%%%%%%%%%%%
%%%%%%%%%%%%%%%%%%%%%%%%%%%%%%%%%%%%%%%%%%%%%%%%%%%%%%%%%%%%%%%%%%%%%%%%%%%%%%%%%%%%%%%%%%%%%%%%%%%%%%%%%%%%%%%%%%%%%%%%%%%%%%%%%%%%%%%%%%%%%%%%%%%%%%%%%%%%%%%%%%%%%%%%%%%%%%%%%%%%%%%%%%%%%%%%%%%%%%%%%%%%%%%%%%%%%%%%%%%%%%%%%%%%%%%%%%%%%%%%%%%%%%%%%%%%%%%%%%%%%%%%%%%%%%%%%%%%%%%%%%%%%%%%%%%%%%%%%%%%%%%%%%%%%%%%%%%%%%%%%%%%%%%%%%%%%%%%%%%%%%%%%%%%%%%%%%%%%%%%%%%%%%%%%%%%%%%%%%%%%%%%%%%%%%%%%%%%%%%%%%%%%%%%%%%%%%%%%%%%%%%%%%%%%%%%%%%%%%%%%%%%%%%%%%%%%%%%%%%%%%%%%%%%%%%%%%%%%%%%%%%%%%%%%%%%%%%%%%%%%%%%%%%%%%%%%%%%%%%%%%%%%%%%%%%%%%%%%%%%%%%%%%%%%%%%%%%%%%%%%%%%%%%%%%%%%%%%%%%%%%%%%%%%%%%%%%%%%%%%%%%%%%%%%%%%%%%%%%%%%%%%%%%%%%%%%%%%%%%%%%%%%%%%%%%%%%%%%%%%%%%%%%%%%%%%%%%%%%%%%%%%%%%%%%%%%%%%%%%%%%%%%%%%%%%%%%%%%%%%%%%%%%%%%%%%%%%%%%%%%%%%%%%%%%%%%%%%%%%%%%%%%%%%%%%%%%%%%%%%%%%%%%%%%%%%%%%%%%%%%%%%%%%%%%%%%%%%%%%%%%%%%%%%%%%%%%%%%%%%%%%%%%%%%%%%%%%%%%%%%%%%%%%%%%%%%%%%%%%%%%%%%%%%%%%%%%%%%%%%%%%%%%%%%%%%%%%%%%%%%%%%%%%%%%%%%%%%%%%%%%%%%%%%%%%%%%%%%%%%%%%%%%%%%%%%%%%%%%%%%%%%%%%%%%%%%%%%%%%%%%%%%%%%%%%%%%%%%%%%%%

%%%%%%%%%%%%%%%%%%%%%%%%%%%%%%%%%%%%%%%%%%%%%%%%%%%%%%%%%%%%%%%%%%%%%%%%%%%%%%%%%%%

\section{Introduction}
%%%%%%%%%%%%%%%%%%%%%%%%%%%%%%%%%%%%%%%%%%%%%%%%%%%%%%%%%%%%%
%%%%%%%%%%%%%%%%%%%%%%%%%%%%%%%%%%%%%%%%%%%%%%%%%%%%%%%%%%%%%%
%%%%%%%%%%%%%%%%%%%%%%%%%%%%%%%%%%%%%%%%%%%%%%%%%%%%%%%%%%%%%
%%%%%%%%%%%%%%%%%%%%%%%%%%%%%%%%%%%%%%%%%%%%%%%%%%%%%%%%%%%%%%
%%%%%%%%%%%%%%%%%%%%%%%%%%%%%%%%%%%%%%%%%%%%%%%%%%%%%%%%%%%%%
%%%%%%%%%%%%%%%%%%%%%%%%%%%%%%%%%%%%%%%%%%%%%%%%%%%

In this paper, we extend
the basic transference theorem for
convolution operators on $L^p$ spaces of
Coifman and Weiss \cite{cw1}
to $H^1$ spaces.  For clarity's sake, we start 
by recalling the Coifman-Weiss
transference theorem for a single convolution
operator.  

Suppose that $k\in L^1(G)$,
where $G$ is a locally compact 
abelian group,
and let $N_p(k)$
denote the norm of the convolution
operator $f\mapsto k*f$,
where $f\in L^p(G,\lambda)$, and
where
$\lambda$ is a fixed Haar measure on $G$.  
Suppose that $R=\{R_u\}_{u\in G}$
is a strongly continuous, uniformly bounded representation
of $G$ acting on a general 
Lebesgue space $L^p(\cM,\mu)=X_p$
where $1\leq p<\infty$.
Let $c_p$ be a positive constant such that
$\|R_u\|\leq c_p$ for all $u\in G$.
We use the Bochner integral to define the
transferred convolution operator for all $f\in X_p$
by
$$T_k(f)=k*_Rf=\int_G R_{-u}(f) k(u) du,$$
where here $du=d\lambda (u)$.
It is straightforward to obtain the inequality
$\|T_k(f)\|_{L^p(\mu)}\leq 
c_p\|k\|_{L^1(G)} \|f\|_{L^p(\mu)}$.
Using the transference methods,
one can show that the operator norm
of $T_k$ does not exceed $c_p^2 N_p(k)$.  
In most cases of interest,
when $1<p<\infty$,  $N_p(k)$ is much smaller
than $\|k\|_{L^1(G)}$, and thus 
there is a clear 
advantage to the transference methods.
By contrast, the case $p=1$ is of little interest since 
we always have $N_1(k)=\|k\|_{L^1(G)}$. 
%
However, important operators in harmonic analysis
are defined on subspaces of $L^1(G)$ and have norms
smaller than $\|k\|_{L^1(G)}$, e.g.,
singular integral operators
on $H^1(\R)$. 
%
One natural question is to ask for the transference
of such operators to appropriate subspaces of $L^1(\mu)$.
When $G=\R$ and
the representation $R$ is given by measure-preserving
transformations, the subspace of $L^1(\mu)$ that is suitable for the
transference of operators on $H^1(\R)$ was introduced 
 by Coifman and Weiss \cite{cw2}, and called ergodic $H^1$.
The note \cite{cw2} contains basic properties of
ergodic $H^1$, derived using 
sophisticated techniques from
\cite{fs}.
Other interesting properties of ergodic $H^1$
are obtained by de la Torre \cite{dlt}, and \cite{abg2}.

Our goal in this paper is to prove a transference
theorem for ergodic $H^1$.
The proofs require new 
techniques, since
the basic averaging process that is behind the 
methods of \cite{cw1} does not work when dealing with 
functions in $H^1$.
The reader
can check that the same
proofs apply when $\R$ is replaced
by $\T$, the circle group.  
%In fact, the theory can be extended to the 
%case when $G$ is a locally compact abelian 
%group with an ordered dual group.
%In this more general setting, one can define $H^1(G)$
%as the closed subspace of $L^1(G)$ consisting of
%functions with Fourier transforms vanishing
%on the negative characters.  (See \cite{ams1}
%for a recent account of analysis on
%groups with ordered duals.) 

Recently several authors contributed to 
 transference from $H^1$.
See Liu and Lu \cite{ll}, and Carro and Soria \cite{cs}.
These papers have points of contact with our results
in the special case of
transference from $H^1(\R)$ to $H^1(\T)$.
We note that the Hardy spaces considered in \cite{cs}
are different from 
our transferred spaces, 
and our main results,
Theorems \ref{single} and \ref{main-theorem} below,
cannot be derived from any of the cited papers.

The plan of the paper is as follows. 
In Section 2, we define and study analytic functions in
$L^1(\mu)$.
In Section 3, we prove a 
transference theorem for
 maximal operators.
In Section 4, we derive some 
applications along the lines of
\cite{cw2} and \cite{abg2}.
%%%%%%%%%%
%%%%%%%%%%
%%%%%%%%%%
%%%%%%%%%%%%%%%%%%%%%%%%%%%%%%%%%%%%%%%%%%%%%%%%%%%%%%%%%%%%%%%
%%%%%%%%%%%%%%%%%%%%%%%%%%%%%%%%%%%%%%%%%%%%%%%%%%%%%%%%%%%%%%
%%%%%%%%%%%%%%%%%%%%%%%%%%%%%%%%%%%%%%%%%%%%%%%%%%%%%%%%%%%%%%%
%%%%%%%%%%%%%%%%%%%%%%%%%%%%%%%%%%%%%%%%%%%%%%%%%%%%%%%%%%%%%%%
%%%%%%%%%%%%%%%%%%%%%%%%%%%%%%%%%%%%%%%%%%%%%%%%%%%%%%%%%%%%%%%
%%%%%%%%%%%%%%%%%%%%%%%%%%%%%%%%%%%%%%%%%%%%%%%%%%%%%%%%%%
%%%%%%%%%%%%%%%%%%%%%%%%%%%%%%%%%%%%%%%%%%%%%%%%%%%%%%%%%%%%%%%
%%%%%%%%%%%%%%%%%%%%%%%%%%%%%%%%%%%%%%%%%%%%%%%%%%%%%%%%%%%%%%
%%%%%%%%%%%%%%%%%%%%%%%%%%%%%%%%%%%%%%%%%%%%%%%%%%%%%%%%%%%%%%%
%%%%%%%%%%%%%%%%%%%%%%%%%%%%%%%%%%%%%%%%%%%%%%%%%%%%%%%%%%%%%%%
%%%%%%%%%%%%%%%%%%%%%%%%%%%%%%%%%%%%%%%%%%%%%%%%%%%%%%%%%%%%%%%
%%%%%%%%%%%%%%%%%%%%%%%%%%%%%%%%%%%%%%%%%%%%%%%%%%%%%%%%%%
%%%%%%%%%%%%%%%%%%%%%%%%%%%%%%%%%%%%%%%%%%%%%%%%%%%%%%%%%%%%%%%
%%%%%%%%%%%%%%%%%%%%%%%%%%%%%%%%%%%%%%%%%%%%%%%%%%%%%%%%%%%%%%
%%%%%%%%%%%%%%%%%%%%%%%%%%%%%%%%%%%%%%%%%%%%%%%%%%%%%%%%%%%%%%%
%%%%%%%%%%%%%%%%%%%%%%%%%%%%%%%%%%%%%%%%%%%%%%%%%%%%%%%%%%%%%%%
%%%%%%%%%%%%%%%%%%%%%%%%%%%%%%%%%%%%%%%%%%%%%%%%%%%%%%%%%%%%%%%
%%%%%%%%%%%%%%%%%%%%%%%%%%%%%%%%%%%%%%%%%%%%%%%%%%%%%%%%%%
%%%%%%%%%%%%%%%%%%%%
\section{Preliminary results}
\newtheorem{defspec}{Definition}[section]
%\newtheorem{scholspec}[defspec]{Scholium}
\newtheorem{useful}[defspec]{Remark}
\newtheorem{proposition1-sec2}[defspec]{Proposition}
\newtheorem{proposition2-sec2}[defspec]{Proposition}
\newtheorem{corollary-prop}[defspec]{Corollary}
\newtheorem{prop-spectrum}[defspec]{Proposition}
\newtheorem{lemma1}[defspec]{Lemma}
\newtheorem{lemma2}[defspec]{Lemma}
\newtheorem{lemma3}[defspec]{Lemma}
\newtheorem{lemma4}[defspec]{Lemma}
\newtheorem{lemma5}[defspec]{Lemma}
%%%%%%%%%%%%%
%%%%%%%%%%%%%%%%%%%%
%%%%%%%%%%%%%%%%%%%%
%%%%%%%%%%%%%
%%%%%%%%%%%%%%%%%%%%
%%%%%%%%%%%%%%%%%%%%
%%%%%%%%%%%%%
%%%%%%%%%%%%%%%%%%%%
%%%%%%%%%%%%%%%%%%%%
%%%%%%%%%%%%%
Throughout this section, 
 $(\O,\cM,\mu)$ denotes a measure space,
where $\mu$ is an arbitrary measure,  
$R:\ u\mapsto R_u$ is a uniformly
bounded, strongly continuous representation of 
$\R$ in $L^1(\mu)$ 
such that
%%%%%%%%%%%%%%%%%%%%%%%%%%%%%%%%%%%%%%%%%%
%%%%%%%%%%%%%%%%%%%%%%%%%%%%%%%%%%%%%%%%%%%
\begin{equation}
\|R_u\|\leq c.
\label{uniform-norm}
\end{equation}
%%%%%%%%%%%%%%%%%%%%%%%%%%%%%%%%%%%%%%%%%%
%%%%%%%%%%%%%%%%%%%%%%%%%%%%%%%%%%%%%%%%%%%
We now review a few useful
facts from spectral analysis.
Let $g \in L^\infty(\R)$, and let 
$[g]$ denote the smallest weak-* closed
translation invariant subspace of $L^\infty(\R)$
that contains $g$.  The {\em spectrum}
of $g$, denoted
$\spec (g)$,
 is the set of characters
of $\R$ that belong to $[g]$.
Define 
$$H^\infty(\R)=\{f\in L^\infty(\R):\ \spec(f)\subset [0,\infty)\}.$$
An equivalent definition of the spectrum 
is obtained as follows.  Let 
$$\cI(g)=\{f\in L^1(\R):\ f*g=0 \},$$
and
$$Z\left(\cI(g) \right)=
\bigcap_{f\in \cI (g)}\{s:\ \widehat{f}(s)=0\}.$$
According to \cite[(40.21)(i)]{hr2}, we also have
%%%%%%%%%%%%%%%%%%%%%%%%%%%%%%%%%%%%%%%%
%%%%%%%%%%%%%%%%%%%%%%%
%%%%%%%%%%%%%%%%%%%%%%%%%%%%%%%%%%%%%%
%%%%%%%%%%%%%%%%%%%%%%
$\spec (g)=
Z\left(\cI(g)\right).$
%%%%%%%%%%%%%%%%%%%%%%%%%%%%%%%%%%%
%%%%%%%%%%%%%%%%%%%%%%%%%%%
%%%%%%%%%%%%%%%%%%%%%%%%%%%%%%%%%%%%%
%%%%%%%%%%%%%%%%%%%%%%%%%
%%%%%%%%%%%%%%%%%%%%%%%%%%%%%%%%%%%%%%
%%%%%%%%%%%%%%%%%%%%%%%%
%%%%%%%%%%%%%%%%%%%%%%%%%%%%%%%
%%%%%%%%%%%%%%%%%%%%%%%%%%%%%%%%%%%%%%%%%%%
%%%%%%%%%%%%%%%%%%
To define the spectrum of a function $f\in L^1(\mu)$, 
let
%%%%%%%%%%%%%%%%%%%%%%%%%%%%%%%
%%%%%%%%%%%%%%%%%%%%%%%%%%%%%%
%%%%%%%%%%%%%%%%%%%%%%%%%%%%%%%%
%%%%%%%%%%%%%%%%%%%%%%%%%%%%%
\begin{equation}
\cI_R(f)=\{h\in L^1(\R): h*_Rf=0\ \mu -{\rm a.e.}\}.
\label{ideal}
\end{equation}
%%%%%%%%%%%%%%%%%%%%%%%%%%%%%%%%%%%
%%%%%%%%%%%%%%%%%%%%%%%%%%
%%%%%%%%%%%%%%%%%%%%%%%%%%
%%%%%%%%%%%%%%%%%%%%%%%%%%%%%%%%%%%
It is easy to check that $\cI_R(f)$ is a closed ideal in 
$L^1(\R)$.
%%%%%%%%%%%%%%%%%%%%%%%%%%%%%%%%%%%%%%%
%%%%%%%%%%%%%%%%%%%%%%
%%%%%%%%%%%%%%%%%%%%%%%%%%%%%%%%%%%%%%%%
%%%%%%%%%%%%%%%%%%%%%
%%%%%%%%%%%%%%%%%%%%%%%%%%%%%%%%%%%%%%%%%
%%%%%%%%%%%%%%%%%%%%
%%%%%%%%%%%%%%%%%%%%%%%%%%%%%%%%%%%%%%%
%%%%%%%%%%%%%%%%%%%%%%
\begin{defspec}
For $f\in L^1(\mu)$, define the spectrum of $f$ by
%%%%%%%%%%%%%%%%%%%%%%
%%%%%%%%%%%%%%%%%%%%%%%%
%%%%%%%%%%%%%%%
%%%%%%%%%%%%%%%%%%%%%%%
%%%%%%%%%%%%%%%%%%%%%%%%%%%%%%%%%%%%%%
\begin{equation}
\spec_R(f)\equiv Z\left(\cI_R (f)\right)\equiv
\bigcap_{h\in \cI_R(f) }
\{s\in\R: \widehat{h}(s)=0\}.
\label{spectrum}
\end{equation}
%%%%%%%%%%%%%%%%%%%%%%%%%%%%%%
%%%%%%%%%%%%%%%%%%%%%%%%%%%%%%%
%%%%%%%%%%%%%%%%%%%%%%%%%%%%%%%%%%%%
%%%%%%%%%%%%%%%%%%%%%%%%%%
We say that $f\in L^1(\mu)$ is analytic,
and write $f\in H^1(\mu)$,
 if $\spec_R(f)\subset [0,\infty)$.  Hence
$$H^1(\mu)=\{f\in L^1(\mu):\ \spec_R(f)\subset [0,\infty)\}.$$
The norm in $H^1(\mu)$ is the $L^1(\mu)$-norm.
\label{defspec}
\end{defspec}
%%%%%%%%%%%%%%%%%%%%%%%%%%%%%%%%%%%%%%%%%
%%%%%%%%%%%%%%%%%%%%%%%%%%%%%%%%%%%%%%
%%%%%%%%%%%%%%%%%%%%%%%
%%%%%%%%%%%%%%%%%%%%%
When $\R$ acts on $L^1(\R,d x)$ by translation, 
it is easy to check that the space of
analytic functions in $L^1(\R)$ consists of those functions
with Fourier transforms supported in $[0,\infty)$.
We set
%%%%%%%%%%%%%%%%%%%%%%%%%%%%%%%%%%%%%%%%%%%%%%%%%%%%%%%%%%%%
%%%%%%%%%%%%%%%%%%%%%%%%%%%%%%%%%%%%%%%%%%%%%%%%
$$H^1(\R)=
\{f\in L^1(\R):\ \supp \widehat{f}\subset [0,\infty)\}.$$
%%%%%%%%%%%%%%%%%%%%%%%%%%%%%%%%%%%%%%%%%%%%%%%%%%%%%%%%%%%%
%%%%%%%%%%%%%%%%%%%%%%%%%%%%%%%%%%%%%%%%%%%%%%%%

Let $E$ be a closed subset of $\R$,
 let
$$\Im(E)=\{ f\in L^1(\R):\ \widehat{f}(E)\subset\{0\}\},$$
 and let ${\cal J}(E)$
denote the closure in $L^1(\R)$ of the set
$$\{f\in L^1(\R):\ \widehat{f}\ 
{\rm vanishes\ on \ an\ open\ set\ containing }\ E\}.$$
%%%%%%%%%%%%%%%%%%%%%%
%%%%%%%%%%%%%%%%%%%%%%%%%%%%%
%%%%%%%%%%%%%%%%%%%%%%%%%%%%%%%%
%%%%%%%%%%%%%%%%%%%%%%%%%%%%%%%%%%%%%%%%%
Both $\Im(E)$ and ${\cal J}(E)$ are closed ideals in $L^1(\R)$,
and we clearly have ${\cal J}(E)\subset \Im(E)$.
The set $E$ is called a {\em set of spectral synthesis}
if ${\cal J}(E)= \Im(E)$\ (see \cite[Theorem (40.10)]{hr2}).  
%%%%%%%%%%%%%%%%%%%%%%%%%%%%%%%%%%%%%%%%%%%%%%%%%
%%%%%%%%%%%%%%%%%%%%%%%%%%%%%%%%%%%%%%%%%%%%%%%%%
%%%%%%%%%%%%%%%%%%%%%%%%%%%%%%%%%%%%%%%%%%%%%%%
The following is a simple consequence of \cite[Theorems (40.8), and (40.10)(iii)]{hr2}, and the 
fact that $[\alpha,\infty)$ and
$(-\infty ,\alpha]$ are sets of spectral synthesis.
%%%%%%%%%%%%%%%%%%%%%%%%%%%%%%%%%%%%%%%%%%%%%%%%
%%%%%%%%%%%%%%%%%%%%%%%%%%%%%%%%%%%%%%%%%%%%%%%%%%%%%%%%%%%%
\begin{useful}
Let $\alpha $ be a fixed real number and let $E$
denote the set $[\alpha,\infty)$ or
$(-\infty ,\alpha]$.
Suppose that $g\in L^\infty(\R)$.
  Then $\spec (g)\subset E$,
if and only if
$$\int_\R f\overline{g} dx=0$$
for all $f\in \Im (E)$.
\label{useful}
\end{useful}
%%%%%%%%%%%%%%%%%%%%%%%%%%%%%%%%%%%%%%%%%%%%%%%%
%%%%%%%%%%%%%%%%%%%%%%%%%%%%%%%%%%%%%%%%%%%%%%%%%%%%%%%%%%%%
%%%%%%%%%%%%%%%%%%%%%%%%%%%%%%%%%%%%%%%%%%%%%%%%
%%%%%%%%%%%%%%%%%%%%%%%%%%%%%%%%%%%%%%%%%%%%%%%%%%%%%%%%%%%%
%%%%%%%%%%%%%%%%%%%%%%%%%%%%%%%%%%%%%%%%%%%%%%%%
Taking $E=[0,\infty)$ in Remark \ref{useful},
we see that a function $g\in L^\infty(\R)$ belongs to
$H^\infty(\R)$ if and only if
\begin{equation}
\int_\R f g d x = 0
\label{hinfty}
\end{equation}
for all $f\in H^1(\R)$.

%%%%%%%%%%%%%%%%%%%%%%%%%%%%%%%%%%%%%%%%%%%%%%%%%%%%%%%%%%%%%
%%%%%%%%%%%%%%%%%%%%%%%%%%%%%%%%%%%%%%%%%%%%%%%%
For $f\in L^1(\mu)$ and $A\in \cM$, consider the 
function defined on
 $\R$ by $t\mapsto \int_A R_tf d \mu$.
It follows easily from the strong
continuity and the uniform
boundedness of $R$ that this function is in $L^\infty(\R)$.
We now present a useful characterization of
$\spec_R$.
%%%%%%%%%%%%%%%%%%%%%%%%%%%%%%%%%%%%%%%%%%%%%%%%%%%%%%%%%%%%%%%
%%%%%%%%%%%%%%%%%%%%%%%%%%%%%%%%%%%%%%%%%%%%
\begin{proposition1-sec2}
Let $f\in L^1(\mu)$, and let $\alpha$
be any real number.  The following are equivalent:\\
(i)  \ $\spec_R(f)\subset [\alpha,\infty)$;\\
(ii)\ for every $A\in\cM$, the spectrum of 
the $L^\infty(\R)$-function $t\mapsto \int_A R_tfd\mu$
is contained in $[\alpha,\infty)$.
\label{proposition1-sec2}
\end{proposition1-sec2}
%%%%%%%%%%%%%%%%%%%%%%%%%%%%%%%%%%%%%%%%%%%%%%%%%%%%%%%%%%%%%%%
%%%%%%%%%%%%%%%%%%%%%%%%%%%%%%%%%%%%%%%%%%%%%%%%%%%%%%%%%%%%%
{\bf Proof.}  Let $g\in L^1(\R)$
be any function in $\cI_R(f)$, so that
$g*_R f=0$.  Then, for all $t\in\R$,
since $g*_R (R_t f)=R_t(g*_Rf)=0$, it follows that,
 for any $A\in\cM$, %%%%%%%%%%%%%%%%%%%%%%%%%%%%%%%%%%%%%%%%%%%%%%%
%%%%%%%%%%%%%%%%
%%%%%%%%%%%%%%%%%%%%%%%%%%%%%%%%%%%%%%%%%%%%%%%
%%%%%%%%%%%%%%
$$
\int_Ag*_R (R_t f) d\mu
			=
\int_\R g(u)\int_A R_{t-u}f d\mu du=0.
$$
%%%%%%%%%%%%%%%%%%%%%%%%%%%%%%%%%%%%%%%%%%%%
%%%%%%%%%%%%%%%%%%
%%%%%%%%%%%%%%%%%%%%%%%%%%%%%%%%%%%%%%%%%%%%
%%%%%%%%%%%%%%%%%%%
%%%%%%%%%%%%%%%%%%%%%%%%%%%%%%%%%%%%%%%%%%%%
From this we see that 
%%%%%%%%%%%%%%%%%%%%%%%%%%%%%%%%%%%%%%%%%%
%%%%%%%%%%%%%%%%%%%%%
%%%%%%%%%%%%%%%%%%%%%%%%%%%%%%%%%%%%%%%%%%%
%%%%%%%%%%%%%%%%%%
$$g*\left(t\mapsto \int_A R_tfd\mu\right)=0,$$
%%%%%%%%%%%%%%%%%%%%%%%%%%%%%%%%%%%%%
%%%%%%%%%%%%%%%%%%%%%%%%%%
%%%%%%%%%%%%%%%%%%%%%%%%%%%%%%%%%%%%%
%%%%%%%%%%%%%%%%%%%%%%%%
which implies that $\cI_R(f)\subset \cI\left(
t\mapsto \int_A R_tfd\mu\right)$.  Hence,
$Z\left(\cI_R(f)\right)\supset Z\left(\cI( t\mapsto \int_A R_tfd\mu)\right)$;
equivalently, 
$\spec_R (f)\supset \spec \left(t\mapsto \int_A R_tfd\mu\right)$.
%%%%%%%%%%%%%%%%%%%%%%%%%%%%%%%%%%%%
%%%%%%%%%%%%%%%%%%%%%%%%%%%
%%%%%%%%%%%%%%%%%%%%%%%%%%%%%%%%%%%%%
%%%%%%%%%%%%%%%%%%%%%%%%
This proves that (i) implies (ii).  For the other direction,
let $E=[\alpha,\infty)$.
It is enough to show that
 $\cI_R(f)\supseteq \Im (E)$, since this will
imply that $\spec_R(f)\subset Z(\Im (E))=E$\ (see
\cite[(39.8)(c)]{hr2}).
For this purpose,
let $A$ be an arbitrary nonvoid subset in $\cM$, and let  
$g\in \Im(E)$ so that $g^\sim$ is also in $\Im(E)$.
%%%%%%%%%%%%%%%%%%%%%
Applying Remark \ref{useful}, it follows that 
$$\int_\R \overline{g(-t)}\overline{
\int_A R_tfd\mu } dt=0,$$
because the spectrum of the function 
$t\mapsto \int_A R_t f d\mu$ is contained
in $ E=[\alpha ,\infty)$,
and $g^\sim \in \Im(E)$.
Taking complex conjugates and using
Fubini's Theorem, we obtain after changing $t$ to $-t$ 
%%%%%%%%%%%%%%%%%%%%%%%%%%%%%%%%%%
%%%%%%%%%%%%%%%%%%%%%%%%%%%%%
%%%%%%%%%%%%%%%%%%%%%%%%%%%%%%%%%%%
%%%%%%%%%%%%%%%%%%%%%%%%%%
$$\int_A
\int_\R g(t) R_{-t}f dtd\mu=0.$$
%%%%%%%%%%%%%%%%%%%%%%%%%%%%%%%%%%%%%%%%
%%%%%%%%%%%%%%%%%%%%%%%
%%%%%%%%%%%%%%%%%%%%%%%%%%%%%%%%%%%%%%%%
%%%%%%%%%%%%%%%%%%%%%
Since this holds for all $A\in\cM$, we conclude that
%%%%%%%%%%%%%%%%%%%%%%%%%%%%%%%%%%%%%%%%%
%%%%%%%%%%%%%%%%%%%%%%
%%%%%%%%%%%%%%%%%%%%%%%%%%%%%%%%%%%%%%%%%
%%%%%%%%%%%%%%%%%%%%
$
\int_\R g(t) R_{-t}f dt=0$
$\mu$-a.\ e.
which is what we want to prove.
%%%%%%%%%%%%%%%%%%%%%%%%%%%%%%%%%%%%%%%%%
%%%%%%%%%%%%%%%%%%%%%%
%%%%%%%%%%%%%%%%%%%%%%%%%%%%%%%%%%%%%%%%%
%%%%%%%%%%%%%%%%%%%%
%%%%%%%%%%%%%%%%%%%%%%%%%%%%%%%%%%%%%%%%%
%%%%%%%%%%%%%%%%%%%%%%
%%%%%%%%%%%%%%%%%%%%%%%%%%%%%%%%%%%%%
%%%%%%%%%%%%%%%%%%%%%%%%

%%%%%%%%%%%%%%%%%%%%%%%%%%%%%%%%%%%
%%%%%%%%%%%%%%%%%%%%%%%%%%%%
%%%%%%%%%%%%%%%%%%%%%%%%%%%%%%%%%%%
%%%%%%%%%%%%%%%%%%%%%%%%%%
The following is a useful characterization of $H^1(\mu)$.
%%%%%%%%%%%%%%%%%%%%%%%%%%%%%%%%%%%
%%%%%%%%%%%%%%%%%%%%%%%%%%%%
%%%%%%%%%%%%%%%%%%%%%%%%%%%%%%%%%%
%%%%%%%%%%%%%%%%%%%%%%%%%%%
%%%%%%%%%%%%%%%%%%%%%%%%%%%%%%%%%%%
%%%%%%%%%%%%%%%%%%%%%%%%%%%%
%%%%%%%%%%%%%%%%%%%%%%%%%%%%%%%%%%%
%%%%%%%%%%%%%%%%%%%%%%%%%%
%%%%%%%%%%%%%%%%%%%%%%%%%%%%%%%%%%%%
%%%%%%%%%%%%%%%%%%%%%%%%%%%
%%%%%%%%%%%%%%%%%%%%%%%%%%%%%%%%%%%
%%%%%%%%%%%%%%%%%%%%%%%%%%
%%%%%%%%%%%%%%%%%%%%%%%%%%%%%%%%%%%
%%%%%%%%%%%%%%%%%%%%%%%%%%%%
%%%%%%%%%%%%%%%%%%%%%%%%%%%%%%%%%
%%%%%%%%%%%%%%%%%%%%%%%%%%%%
%%%%%%%%%%%%%%%%%%%%%%%%%%%%%%%%%%%%%%
%%%%%%%%%%%%%%%%%%%%%%%%%
%%%%%%%%%%%%%%%%%%%%%%%%%%%%%%%%%%%%%
%%%%%%%%%%%%%%%%%%%%%%%%
%%%%%%%%%%%%%%%%%%%%%%%%%%%%%%%%%%%%%
%%%%%%%%%%%%%%%%%%%%%%%%%
%%%%%%%%%%%%%%%%%%%%%%%%%%%%%%%%%%%%%%
%%%%%%%%%%%%%%%%%%%%%%%%%
%%%%%%%%%%%%%%%%%%%%%%%%%%%%%%%%%%%%%%%%%%%%
\begin{proposition2-sec2}
Let $f\in L^1(\mu)$.  
The following are equivalent:\\
(i)  $f\in H^1(\mu)$;\\
(ii)  for every $A\in\cM$, the function 
$t\mapsto \int_A R_tfd\mu$
is in $H^\infty(\R)$;\\
(iii) for every $ A\in\cM$ and every $g\in H^1(\R)$,
 we have
$$\int_\R g(t)\int_A R_t f d\mu dt =0;$$
(iv)  for every $h\in \Im ([0,\infty))$, we have
$h*_Rf=0$\ $\mu$-a.e.
\label{proposition2-sec2}
\end{proposition2-sec2}
%%%%%%%%%%%%%%%%%%%%%%%%%%%%%%%%%%%%%%%
%%%%%%%%%%%%%%%%%%%%%%%
%%%%%%%%%%%%%%%%%%%%%%%%%%%%%%%%%%%%%%%%
%%%%%%%%%%%%%%%%%%%%%%%
%%%%%%%%%%%%%%%%%%%%%%%%%%%%%%%%%%%%%%%%%%%%
{\bf Proof.}
The equivalence (i)$\Leftrightarrow$(ii) follows from 
Proposition \ref{proposition1-sec2} and definitions.
The equivalence (ii)$\Leftrightarrow$(iii)
follows from (\ref{hinfty}).  
To prove (iii) $\Rightarrow$(iv), let $h\in \Im([0,\infty))$.
Since the function $t\mapsto h(-t)$ is in $H^1(\R)$, it
follows from (iii) that
$$\int_\R h(-t) \int_A R_t f d\mu dt=0,$$
for all $A\in \cM$; equivalently,
$$\int_\R h(t) \int_A R_{-t}f d\mu dt=
\int_A\int_\R h(t) R_{-t} f dtd\mu=0,$$
for all $A\in \cM$, which implies (iv).
The proof of (iv)$\Rightarrow$(iii) is simple and will be
omitted.

%%%%%%%%%%%%%%%%%%%%%%%%%%%%%%%%%
%%%%%%%%%%%%%%%%%%%%%%%%%%%%%
%%%%%%%%%%%%%%%%%%%%%%%%%%%%%%%%%%
%%%%%%%%%%%%%%%%%%%%%%%%%%%%%
%%%%%%%%%%%%%%%%%%%%%%%%%%%%%%%%%%%%%%%%%%%%
%%%%%%%%%%%%%%%%%%%%%%%%%%%%%%%%%%%%
%%%%%%%%%%%%%%%%%%%%%%%%%%%
%%%%%%%%%%%%%%%%%%%%%%%%%%%%%%%%%%%%%
%%%%%%%%%%%%%%%%%%%%%%%%

The following
simple proposition is very useful.
%%%%%%%%%%%%%%%%%%%%%%%%%%%%%%%%%%%%
%%%%%%%%%%%%%%%%%%%%%%%%%%
%%%%%%%%%%%%%%%%%%%%%%%%%%%%%%%%%%%%%
%%%%%%%%%%%%%%%%%%%%%%%%%%
%%%%%%%%%%%%%%%%%%%%%%%%%%%%%%%%%%%%%%%%%%%%
\begin{prop-spectrum}
Suppose that $f, f_n\in L^1(\mu)$, and $k\in L^1(\R)$.
Then\\
(i)\  $\spec_R ( k*_R f)\subset {\rm supp}\widehat{k}\cap \spec_R(f).$\\
(ii) Suppose that 
 $\spec_R(f_n)\subset [\alpha,\infty)$ and $f_n\rightarrow f$ in $L^1(\mu)$, then
$\spec_R(f)\subset [\alpha,\infty)$.
\label{prop-spectrum}
\end{prop-spectrum}
%%%%%%%%%%
%%%%%%%%%%
The proof of (i) is simple and will be 
omitted.  For the proof of (ii), use Proposition 
\ref{proposition1-sec2}(ii) and 
Dominated Convergence.
%%%%%%%%%%%%%%%%%%%%%%%%%%%%%%%%%%%%%%%%%%
%%%%%%%%%%%%%%%%%%%%%
%%%%%%%%%%%%%%%%%%%%%%%%%%%%%%%%%%%%%%%%%
%%%%%%%%%%%%%%%%%%%%%
%%%%%%%%%%%%%%%%%%%%%%%%%%%%%%%%%%%%%%%
%%%%%%%%%%%%%%%%%%%%%%%%
%%%%%%%%%%%%%%%%%%%%%%%%%%%%%%%%%%%%%%
%%%%%%%%%%%%%%%%%%%%%%%%%
%%%%%%%%%%%%%%%%%%%%%%%%%%%%%%%%%%%%%%
%%%%%%%%%%%%%%%%%%%%%%%%%

For use in the sequel, we introduce
the space $H^1(\R,L^1(\mu))$
which consists of Bochner integrable functions 
$g$ on $\R$
with values in $L^1(\mu)$ such that
%%%%%%%%%%%%%%%%%
%%%%%%%%%%%%%%%%%
%%%%%%%%%%%%%%%%%
%%%%%%%%%%%%%%%%%%%%%%%%%%%%%%%%%%
%%%%%%%%%%%%%%%%%
\begin{equation}
\int_\R g(x) e^{-ixt} dx=0,\ {\rm for\ all}\ t<0.
\label{vector-valued-H1}
\end{equation}
%%%%%%%%%%%%%%%%
%%%%%%%%%%%%%%%%%
%%%%%%%%%%%%%%%%%
%%%%%%%%%%%%%%%%%
%%%%%%%%%%%%%%%%%%%%%%%%%%%%%%%%%%
From definitions, a function $g\in H^1(\R, L^1(\mu))$
is jointly measurable on $\R\times\O$ and 
belongs to $L^1(\R\times \O,dx\ d\omega)$.
%%%%%%%%%%%%%%%%%
%%%%%%%%%%%%%%%%%
%%%%%%%%%%%%%%%%%
%%%%%%%%%%%%%%%%%
%%%%%%%%%%%%%%%%%%%%%%%%%%%%%%%%%%
%%%%%%%%%%%%%%%%%
\begin{lemma1}
A 
function $g$ 
is in $H^1(\R,L^1(\mu))$ if and only if
for $\mu$-almost all 
$\omega\in \O$ the mapping $x\mapsto g(x,\omega)$
is in $H^1(\R)$.
\label{lemma1}
\end{lemma1}
%%%%%%%%%%%%%%%%%%%
%%%%%%%%%%%%%%%%%%%
{\bf Proof.}  One direction is clear:  if
for almost all $\omega\in \O$ the mapping $x\mapsto g(x,\omega)$
is in $H^1(\R)$, then 
(\ref{vector-valued-H1}) holds and so $g\in H^1(\R,L^1(\mu))$.
Now suppose that $g\in H^1(\R,L^1(\mu))$.  Because of
(\ref{vector-valued-H1}), for each $t<0$, there is a subset
$B_t\subset \O$ such that $\mu(\O\setminus B_t)=0$, and,
for all $\omega\in B_t$, we have
%%%%%%%%%%%%%%%%%%%%%%%%%%%%%%
%%%%%%%%%%%%%%%%%%%%%%%%%%%%%
%%%%%%%%%%%%%%%%%%%
\begin{equation}
\int_\R g(x,\omega)e^{-ixt}dx=0.
\label{lemma1eq1}
\end{equation}
%%%%%%%%%%%%%%%%%%%%%%%%%%%%%%
%%%%%%%%%%%%%%%%%%%%%%%%%%%%%
%%%%%%%%%%%%%%%%%%%
Let $(t_n)=\Q\cap (-\infty,0)$
denote the set of negative rational numbers, and let 
$B=\bigcap_nB_{t_n}$.
%%%%%%%%%%%%%%%%%%%%%%%%%%%%%%
%%%%%%%%%%%%%%%%%%%%%%%%%%%%%
%%%%%%%%%%%%%%%%%%%
Then $\mu(\O\setminus B)=0$, and (\ref{lemma1eq1})
holds for all
$\omega\in B$ and $(t_n)$.  
Now, given an arbitrary 
real number $t< 0$, choose a subsequence
$(t_{n_j})$ from $(t_n)$ such that
$t_{n_j}\rightarrow t$.
Then, for $\omega\in B$, we have
$g(x,\omega)e^{-ixt_{n_j}}\rightarrow  g(x,\omega)e^{-ixt}$
for all $x\in \R$.
Hence, by Dominated Convergence, we have,
for all $\omega\in B$, 
$$0=\int_\R g(x,\omega)e^{-ixt_{n_j}}dx\rightarrow 
\int_\R g(x,\omega)e^{-ixt}dx,$$
implying (\ref{vector-valued-H1}).
 
%%%%%%%%%%%%%%%%%%%%%%%%%%%%%%
%%%%%%%%%%%%%%%%%%%%%%%%%%%%%
%%%%%%%%%%%%%%%%%%%
%%%%%%%%%%%%%%%%%%%%%%%%%%%%%%
%%%%%%%%%%%%%%%%%%%%%%%%%%%%%
Using the representation $R$, for each $\alpha\in\R$,
we define a new representation $e^{i\alpha(\cdot)}R$
by:\ \ $u\in \R\mapsto e^{i\alpha u }R_u$.  
The following simple properties will be very useful.
%%%%%%%%%%%%%%%%%%%%%%%%%%%%%%%%%%%%%%%%%%%%%%%%%
%%%%%%%%%%%%%%%%%%%%%%%%%%%%%
%%%%%%%%%%%%%%%%%%%
\begin{lemma2}
Suppose that $f\in H^1(\mu)$ and $k\in L^1(\R)$, and
let $\alpha\geq 0$.  Then\\
(i)\  $\spec_{e^{i\alpha(\cdot)}R} (f)\subset [\alpha,\infty)$;\\
(ii)\ $\lim_{\alpha\rightarrow 0}
    \left\| k*_{e^{i\alpha(\cdot)}R}f-k*_Rf\right\|_{L^1(\mu)}=0$,
    where here $\alpha\rightarrow 0$ through a countable sequence.
\label{lemma2}
\end{lemma2}
%%%%%%%%%%%%%%%%%%%%%%%%%%%%%%%%%%%%%%%%%%%%%%%%%
%%%%%%%%%%%%%%%%%%%%%%%%%%%%%
%%%%%%%%%%%%%%%%%%%
{\bf Proof.}  To prove (i), it is enough to show that for any
$s_0<\alpha$, we can find a function
$h\in L^1(\R)$ such that  
$h*_{e^{i\alpha(\cdot)}R}f=0$
and
$\widehat{h}(s_0)\neq 0$.
Since
$(s_0 - \alpha )\not\in\spec_R(f)\subset [0,\infty)$,
we can find a function $g\in L^1(\R)$
such that $g*_Rf=0$ and $\widehat{g}(s_0-\alpha)=1$.
We clearly have
$$(e^{i\alpha(\cdot )}g)*_{e^{i\alpha(\cdot)}R}f=
g*_Rf=0,$$
and since 
$\widehat{(e^{i\alpha(\cdot )}g)}(s_0)=\widehat{g}(s_0-\alpha)=1$,
the proof of (i) is complete.
For (ii), we have
%%%%%%%%%%%%%%%%%%%%%%%%%%%%%%%%%%%%%%%%%%%%%%%%%
%%%%%%%%%%%%%%%%%%%%%%%%%%%%%
\begin{eqnarray*}
\left\| k*_{e^{i\alpha(\cdot)}R}f-k*_Rf\right\|_{L^1(\mu)}
			&=&
\int_\O \left|
\int_\R
\left(
e^{-i\alpha u} -1\right) R_{-u}f k(u)du
\right| d|\mu|\\
			&\leq&
\int_\R\int_\O\left|R_{-u}f\right|d|\mu|
\left|e^{-i\alpha u} -1\right||k(u)|du\\
			&\leq&
c\|f\|_{L^1(\mu)}
\int_\R
\left|e^{-i\alpha u} -1\right||k(u)|du.
\end{eqnarray*}
%%%%%%%%%%%%%%%%%%%%%%%%%%%%%%%%%%%%%%%%%%%%%%%%%
%%%%%%%%%%%%%%%%%%%%%%%%%%%%%
Now (ii) follows from the fact that
$\lim_{\alpha\rightarrow 0} 1-e^{-i\alpha u}=0$ for
all $u$, and 
Dominated Convergence.

%The following two results are
%well-known.  We state them here for ease of reference. 
%%%%%%%%%%%%%%%%%%%%%%%%%%%%%%%%%%%%%%%%%%%%%%%%%
%%%%%%%%%%%%%%%%%%%%%%%%%%%%%%%%%%%%%%%%%%%%%%%%%%%%%%%%
%%%%%%%%%%%%%%%%%%%%%%
%%%%%%%%%%%%%%%%%%%%%%%%%%%%%%%%%%%%%%%%%%%%%%%%%%%%%%%%%
%%%%%%%%%%%%%%%%%%%%%
%%%%%%%%%%%%%%%%%%%%%%%%%%%%%%%%%%%%%%%%%%%%%%%%%%%%%%%%%
%%%%%%%%%%%%%%%%%%%%%
%\begin{lemma3}
%Let $(e_n)$ be an approximate identity for $L^1(\R)$
%with $e_n\geq 0$ and $\int_\R e_ndx=1$.  Then,
%for all $f\in L^1(\mu)$ we have
%$\lim_{n\rightarrow\infty}\|e_n*_R f-f\|_{L^1(\mu)}=0$.
%%
%\label{lemma3}
%\end{lemma3}
%%%%%%%%%%%%%%%%%%%%%%%%%%%%%%%%%%%%%%%%%%%%%%%%%%%%%%%%%
%%%%%%%%%%%%%%%%%%%%%
%The proof that we omit 
%is a simple consequence of the strong continuity
%of $R$ and the fact that for any open 
%nonvoid neighborhood $U$ of $0\in\R$,
%we have
%$\lim_{n\rightarrow\infty}\int_{\R\setminus U}e_n dx=0.$
%%%%%%%%%%%%%%%%%%%%%%%%%%%%%%%%%%%%%%%%%%%%%%%%%%%%%%%%%
%%%%%%%%%%%%%%%%%%%%%
%%%%%%%%%%%%%%%%%%%%%%%%%%%%%
%%%%%%%%%%%%%%%%%%%%%%%%%%%%%%%%%%%%%%%%%%%%%%%%%%%%%%%%%
%%%%%%%%%%%%%%%%%%%%%
%\begin{lemma4}
%Let $f\in L^1(\mu)$, $k\in L^1(\R)$.  Then the mapping
%$t\mapsto R_t(k*_R f)$ is continuous from $\R$ into $L^1(\mu)$.
%\label{lemma4}
%\end{lemma4}
%%%%%%%%%%%%%%%%%%%%%%%%%%%%%
%%%%%%%%%%%%%%%%%%%%%%%%%%%%%%%%%%%%%%%%%%%%%%%%%%%%%%%%%
%%%%%%%%%%%%%%%%%%%%%
%{\bf Proof.}  Note that 
%$R_t(k*_R f)=k*_R (R_t f)=k*_R(R_tf)=k_t*_Rf$ 
%where $k_t(x)=k(x+t)$.
%Now the lemma is a simple consequence of the fact that translation 
%is uniformly continuous in $L^1(\R)$.

%Combining the last two lemmas, we
%find that the linear subspace of $L^1(\mu)$ consisting
%of functions $f$ such that $t\mapsto R_tf$ is continuous
%%is dense in $L^1(\mu)$.  In fact, because
%$\spec_R e_n *_R f\subset \spec_R f$, we get the 
%following useful result. 
%%%%%%%%%%%%%%%%%%%%%%%%%%%%%%%%%%%%%%%%%%%%%%%%%%%%%%%%
%%%%%%%%%%%%%%%%%%%%%
%%%%%%%%%%%%%%%%%%%%%%%%%%%%%
%%%%%%%%%%%%%%%%%%%%%%%%%%%%%%%%%%%%%%%%%%%%%%%%%%%%%%%%%
%%%%%%%%%%%%%%%%%%%%%
%\begin{lemma5}
%The linear subspace of $H^1(\mu)$ consisting of functions $f$
%such that $t\mapsto R_t f$ is continuous from $H^1(\mu)$ into 
%itself is dense in $H^1(\mu)$.
%\label{lemma5}
%\end{lemma5}
%%%%%%%%%%%%%%%%%%%%%%%%%%%%%%%%%%%%%
%%%%%%%%%%%%%%%%%%%%%
%%%%%%%%%%%%%%%%%%%%%%%%%%%%%%%%%%%%%
%%%%%%%%%%%%%%%%%%%%%%%%%%
%%%%%%%%%%%%%%%%%%%%%%%%%%%%%%%%%%%%%%
%%%%%%%%%%%%%%%%%%%%%%%%
%%%%%%%%%%%%%%%%%%%%%%%%%%%%%%%%%%%%%%%
%%%%%%%%%%%%%%%%%%%%%%%%
%%%%%%%%%%%%%%%%%%%%%%%%%%%%%%%%%%%%%%%%
%%%%%%%%%%%%%%%%%%%%%%%
%%%%%%%%%%%%%%%%%%%%%%%%%%%%%%%%%%%%%%%
%%%%%%%%%%%%%%%%%%%%%%%%
%%%%%%%%%%%%%%%%%%%%%%%%%%%%%%%%%%%%%%%%
%%%%%%%%%%%%%%%%%%
%%%%%%%%%%%%%%%%%%%%%%%%%%%%%%%%%%%%%%%%
%%%%%%%%%%%%%%%%%%%%%%%
%%%%%%%%%%%%%%%%%%%%%%%%%%%%%%%%%%%%%%%%%
%%%%%%%%%%%%%%%%%%%%%
%%%%%%%%%%%%%%%%%%%%%%%%%%%%%%%%%%%%%%%%
%%%%%%%%%%%%%%%%%%%%%%%
%%%%%%%%%%%%%%%%%%%%%%%%%%%%%%%%%%%%%%%%
%%%%%%%%%%%%%%%%%%%%%%
%%%%%%%%%%%%%%%%%%%%%%%%%%%%%%%%%%%%%%%%
%%%%%%%%%%%%%%%%%%%%%%%
%%%%%%%%%%%%%%%%%%%%%%%%%%%%%%%%%%%%%%%%
%%%%%%%%%%%%%%%%%%
%%%%%%%%%%%%%%%%%%%%
\section{Transference of maximal inequalities}
\newtheorem{single}{Theorem}[section]
\newtheorem{main-theorem}[single]{Theorem}
\newtheorem{vector-version}[single]{Lemma}
\newtheorem{lemma-sec3}[single]{Lemma}
\newtheorem{transference-max-multiplier}[single]{Theorem}
\newtheorem{normalized-version}[single]{Theorem}
%%%%%%%%%%%%%
%%%%%%%%%%%%%%%%%%%%
%%%%%%%%%%%%%%%%%%%%
%%%%%%%%%%%%%
%%%%%%%%%%%%%%%%%%%%
%%%%%%%%%%%%%%%%%%%%
%%%%%%%%%%%%%
%%%%%%%%%%%%%%%%%%%%
%%%%%%%%%%%%%%%%%%%%
%%%%%%%%%%%%%
Throughout this section, 
$(\O,\cM,\mu)$ is a measure space where
$\mu$ is an arbitrary measure.  
Given $k\in L^1(\R)$, we let
$N(k)$ denote the norm of the
convolution operator 
$f\mapsto k*f$ from $H^1(\R)$ into $H^1(\R)$.
All other notation is as in the previous section.
Our transference theorem for a single convolution
operator follows.  
%%%%%%%%%%%%%%%%%%%%%%%%%%%%%%%%%%%
%%%%%%%%%%%%%%%%%%%%%%%%%%%%%%%%%%
%%%%%%%%%%%%%%%%%%%%%%%%%%%%%%%%%
\begin{single}
Let $R$ be a strongly continuous uniformly bounded
representation of $\R$ acting on $L^1(\mu)$ such that
$\|R_u\|\leq c$ for all $u\in \R$, 
where $c$ is a positive constant.
Let $k\in L^1(\R)$.
For all $f\in H^1(\mu)$ we have
$$\|k*_Rf\|_{L^1(\mu)}\leq c^2 N(k)\|f\|_{L^1(\mu)}.$$ 
\label{single}
\end{single}
%%%%%%%%%%%%%%%%%%%%%%%%%%%%%%%%
%%%%%%%%%%%%%%%%%%%%%%%%%%%%%%%
%%%%%%%%%%%%%%%%%%%%%%%%%%%%%%
Under appropriate additional conditions on $R$, this result can 
be extended to maximal operators corresponding to
 sequences of convolution operators.
For later applications we will state and prove
 the more general version
for maximal operators.  
A few more definitions are needed.
(For background and references, see \cite{abg1}.)


A linear mapping
$T$ of $L^1(\mu)$ is called {\em separation-preserving}
(respectively, {\em positivity-preserving})
if whenever $f\in L^1(\mu)$, $g\in L^1(\mu)$, and
$fg=0\ \mu-a.\ e.$, on $\O$, 
(respectively, $f\geq 0,\ \mu- a.\ e.$),
then $(Tf)(Tg)=0\ \mu\ a.\ e.$
on $\O$ (respectively, $Tf\geq 0,\ \mu-a.\ e.$). 
If $T$ is separation-preserving, then there is a 
positivity-preserving operator $|T|$ such that for all
$f\in L^1(\mu)$, we have $|Tf|=|T|(|f|),\ \mu-a.\ e.$.

Let $\{k_n\}\subset L^1(\R)$ and denote by 
$N(\{k_n\})$ the smallest constant
such that
$$ \|\sup_{n\geq 1} |k_n*f|\|_{L^1(\R)}\leq N(\{k_n\})
\|f\|_{L^1(\R)},$$
for $f\in H^1(\R)$.  
%%%%%%%%%%%%%%%%%%%%%%%%%%%%%%%%%%%%%%%%%%%%%%%%%%%%%%%%%%%%%%
%%%%%%%%%%%%%%%%%%%%%%%%%%%%%%%%%%%%%%%%%%%%%%%%%%%%%%%%%%%%%%%%
%%%%%%%%%%%%%%%%%%%%%%%%%%%%%%%%%%%%%%%%%%
%%%%%%%%%%%%%%%%%%%%%%%%%%%%%%%%%%%%%%%%%%%%%%%%%%%%%%%%%%%%%%%%%%%%%%%%%%%%%%%%%%%%%%%%%%%%%%%%%%%%%%%%%%%%%%%%%%%%%%%%%%%%%%%%%%%%%%%%%%%%%%%%%%%%%%%%%%%%%%%%%%%%%%%%%%%%%%%%%%%%%%%%%%%%%%%%%%%%%%%%%%%%%%%%%%%%%%%%%%%%%%%%%%%%%%%%%%%%%%%%%%%%%%%%%%%%%%%%%%%%%%%%%%%%%%%%%%%%%%
\begin{main-theorem}
Suppose that $R$ is a strongly continuous, uniformly 
bounded representation of $\R$ in $L^1(\mu)$ by 
separation-preserving operators.  Then for all $f\in H^1(\mu)$, we have
%%%%%%%%%%%%%%%%%%%%%%%%%%%%%%%%%%%%%%%%%%%%%%%%%%%%%%%%%%%%%%%
%%%%%%%%%%%%%%%%%%%%%%%%%%%%%%%%%%%%%%%%%%%%%%%%%%%%%%%%%%%%%
%%%%%%%%%%%%%%%%%%%%%%%%%%%%%%%%%%%%%%%%%%%%%%%%%%%%%%%%%%%%%%
\begin{equation}
\|\sup_{n\geq 1} |k_n*_Rf|\|_{L^1(\mu)}\leq
c^2 N(\{k_n\}) \|f\|_{L^1(\mu)}.
\label{main-inequality}
\end{equation}
%%%%%%%%%%%%%%%%%%%%%%%%%%%%%%%%%%%%%%%%%%%%%%%%%%%%%%%%%%%%%%%
%%%%%%%%%%%%%%%%%%%%%%%%%%%%%%%%%%%%%%%%%%%%%%%%%%%%%%%%%%%%%
%%%%%%%%%%%%%%%%%%%%%%%%%%%%%%%%%%%%%%%%%%%%%%%%%%%%%%%%%%%%%%
\label{main-theorem}
\end{main-theorem}
%%%%%%%%%%%%%%%%%%%%%%%%%%%%%%%%%%%%%%%%%%%%%%%%%%%%%%%%%%%%%%%
%%%%%%%%%%%%%%%%%%%%%%%%%%%%%%%%%%%%%%%%%%%%%%%%%%%%%%%%%%%%%
The proof of this theorem will be done in
several steps.  The reader can check that 
separation-preserving is only needed for the 
transference of maximal inequalities, and so
the proof that we present applies also to Theorem
\ref{single}.
 
We start with a simple transference result to a 
space of vector-valued functions.
%%%%%%%%%%%%%%%%%%%%%%%%%%%%%%%%%%%%%%%%%%%%%%%%%%%%%%%%%%%%%%%
%%%%%%%%%%%%%%%%%%%%%%%%%%%%%%%%%%%%%%%%%%%%%%%%%%%%%%%%%%%%%
%%%%%%%%%%%%%%%%%%%%%%%%%%%%%%%%%%%%%%%%%%%%%%%%%%%%%%%%%%%%%%
\begin{vector-version}
Suppose that $g$ is a 
function in $H^1(\R,L^1(\mu))$.  Then,
%%%%%%%%%%%%%%%%%%%%%%%%%%%%%%%%%%%%%%%%%%%%%%%%%%%%%%%%%%%%%
%%%%%%%%%%%%%%%%%%%%%%%%%%%%%%%%%%%%%%%%%%%%%%%%%%%%%%%%%%%%%%
\begin{equation}
\int_\R \left\| \max_{1\leq j\leq N}
\left|
\int_\R
g(x-t) k_j(t) dt
\right|
\right\|_{L^1(\mu)}dx\leq
N(\{k_j\}) \|g\|_{L^1(\R,L^1(\mu))}.
\label{main-vector-inequality}
\end{equation}
%%%%%%%%%%%%%%%%%%%%%%%%%%%%%%%%%%%%%%%%%%%%%%%%%%%%%%%%%%%%%
%%%%%%%%%%%%%%%%%%%%%%%%%%%%%%%%%%%%%%%%%%%%%%%%%%%%%%%%%%%%%%
\label{vector-version}
\end{vector-version}
%%%%%%%%%%%%%%%%%%%%%%%%%%%%%%%%%%%%%%%%%%%%%%%%%%%%%%%%%%%%%%%
%%%%%%%%%%%%%%%%%%%%%%%%%%%%%%%%%%%%%%%%%%%%%%%%%%%%%%%%%%%%%
%%%%%%%%%%%%%%%%%%%%%%%%%%%%%%%%%%%%%%%%%%%%%%%%%%%%%%%%%%%%%%
{\bf Proof.}  Using Fubini's Theorem and the fact that $x\mapsto g(x,\omega)$
is in $H^1(\R)$ for $\mu$-almost all $\omega$, we get
%%%%%%%%%%%%%%%%%%%%%%%%%%%%%%%%%%%%%%%%%%%%%%%%%%%%%%%%%%%%%%
%%%%%%%%%%%%%%%%%%%%%%%%%%%%%%%%%%%%%%%%%%%%%%%%%%%%%%%%%%%%%%%
%%%%%%%%%%%%%%%%%%%%%%%%%%%%%%%%%%%%%%%%%%%%%%%%%%%%%%%%%%%%%
%%%%%%%%%%%%%%%%%%%%%%%%%%%%%%%%%%%%%%%%%%%%%%%%%%%%%%%%%%%%%%%
%%%%%%%%%%%%%%%%%%%%%%%%%%%%%%%%%%%%%%%%%%%%%%%%%%%%%%%%%%%%%
\begin{eqnarray*}
\int_\O \int_\R \max_{1\leq j\leq N}
\left| \int_\R g(x-t,\omega) k_j(t)d t\right| d x d\mu
		&\leq&
N(\{k_j\}) \int_\O\int_\R \left| g(x,\omega)\right| d x d \mu\\
		&=& 
N(\{k_j\}) \|g\|_{L^1(\R,L^1(\mu))},
\end{eqnarray*}
%%%%%%%%%%%%%%%%%%%%%%%%%%%%%%%%%%%%%%%%%%%%%%%%%%%%%%%%%%%%%%%
%%%%%%%%%%%%%%%%%%%%%%%%%%%%%%%%%%%%%%%%%%%%%%%%%%%%%%%%%%%%%
which is what we want.
%%%%%%%%%%%%%%%%%%%%%%%%%%%%%%%%%%%%%%%%%%%%%%%%%%%%%%%%%%%%%%%
%%%%%%%%%%%%%%%%%%%%%%%%%%%%%%%%%%%%%%%%%%%%%%%%%%%%%%%%%%%%%
\begin{lemma-sec3}
Let $\epsilon >0$, and suppose that $g\in L^1(\R)$ has the following
properties:\\
(i)\  $g\geq 0$,\\
(ii)\ $\widehat{g}(0)=\int_\R g d x=1$,\\
(iii)  \ $\widehat{g}$ has compact support contained in 
$\left(\frac{-\epsilon}{2},\frac{\epsilon}{2}\right)$.\\
Then for any function $f\in H^1(\mu)$ with
$\spec_R (f)\subset [\epsilon,\infty)$, we have that the
function $u\mapsto g(u) R_u f$ is in $H^1(\R, L^1(\mu))$.
\label{lemma-sec3}
\end{lemma-sec3}
%%%%%%%%%%%%%%%%%%%%%%%%%%%%%%%%%%%%%%%%%%%%%%%%%%%%%%%%%%%%%%%
%%%%%%%%%%%%%%%%%%%%%%%%%%%%%%%%%%%%%%%%%%%%%%%%%%%%%%%%%%%%%
{\bf Proof.}  We need to check that for any $s<0$
%%%%%%%%%%%%%%%%%%%%%%%%%%%%%%%%%%%%%%%%%%%%%%%%%%%%%%%%%%%%%%%
%%%%%%%%%%%%%%%%%%%%%%%%%%%%%%%%%%%%%%%%%%%%%%%%%%%%%%%%%%%%%
$$
\int_\R e^{-isu} g(u) R_u f d u=0\ \mu - a.\ e.
$$
%%%%%%%%%%%%%%%%%%%%%%%%%%%%%%%%%%%%%%%%%%%%%%%%%%%%%%%%%%%%%%%
%%%%%%%%%%%%%%%%%%%%%%%%%%%%%%%%%%%%%%%%%%%%%%%%%%%%%%%%%%%%%
This will follow if we can show that for any
$A\in \cM$ we have
%%%%%%%%%%%%%%%%%%%%%%%%%%%%%%%%%%%%%%%%%%%%%%%%%%%%%%%%%%%%%%%
%%%%%%%%%%%%%%%%%%%%%%%%%%%%%%%%%%%%%%%%%%%%%%%%%%%%%%%%%%%%%
$$
\int_A\int_\R e^{-isu} g(u) R_u f d ud\mu=
\int_\R  e^{-isu} g(u) \int_A R_u f d\mu du
=0.
$$
%%%%%%%%%%%%%%%%%%%%%%%%%%%%%%%%%%%%%%%%%%%%%%%%%%%%%%%%%%%%%%%
%%%%%%%%%%%%%%%%%%%%%%%%%%%%%%%%%%%%%%%%%%%%%%%%%%%%%%%%%%%%%
Equivalently, by taking complex conjugates, 
it suffices to show that
\begin{equation}
\int_\R e^{isu} g(u)
\overline{ \int_A R_u f d\mu} d u=0.
\label{to-check}
\end{equation}
%%%%%%%%%%%%%%%%%%%%%%%%%%%%%%%%%%%%%%%%%%%%%%%%%%%%%%%%%%%%%%%
%%%%%%%%%%%%%%%%%%%%%%%%%%%%%%%%%%%%%%%%%%%%%%%%%%%%%%%%%%%%%
By Proposition \ref{proposition1-sec2}, 
the spectrum of the
the function $u\mapsto \int_A R_u f d\mu$
is contained in $[\epsilon, \infty)$.
%%%%%%%%%%%%%%%%%%%%%%%%%%%%%%%%%%%%%%%%%%%%%%%%%%%%%%%%%%%%%%%
%%%%%%%%%%%%%%%%%%%%%%%%%%%%%%%%%%%%%%%%%%%%%%%%%%%%%%%%%%%%%
Since the support of the Fourier transform
of the function $u\mapsto e^{isu} g(u)$
is contained in $(-\frac{\epsilon}{4}+s,\frac{\epsilon}{4}+s)$,
and $s\leq 0$, we have that 
$e^{isu} g(u)\in \Im([\epsilon,\infty))$,
and (\ref{to-check}) follows from 
Remark \ref{useful}.
%%%%%%%%%%%%%%%%%%%%%%%%%%%%%%%%%%%%%%%%%%%%%%%%%%%%%%%%%%%%%%%
%%%%%%%%%%%%%%%%%%%%%%%%%%%%%%%%%%%%%%%%%%%%%%%%%%%%%%%%%%%%%
%%%%%%%%%%%%%%%%%%%%%%%%%%%%%%%%%%%%%%%%%%%%%%%%%%%%%%%%%%%%%%%
%%%%%%%%%%%%%%%%%%%%%%%%%%%%%%%%%%%%%%%%%%%%%%%%%%%%%%%%%%%%%

The proof of 
(\ref{main-inequality}) will be facilitated
 by the following 
two reductions.
%%%%%%%%%%%%%%%%%%%%%%%%%%%%%%%%%%%%%%%%%%%%%%%%%%%%%%%%%%%%%%%
%%%%%%%%%%%%%%%%%%%%%%%%%%%%%%%%%%%%%%%%%%%%
%%%%%%%%%%%%%%%%%%%%%%%%%%%%%%%%%%%%%%%%%%%%%%%%%%%%%%%%%%%%%%

{\bf First reduction}  In proving (\ref{main-inequality}),
it is enough to assume that the sequence $\{k_n\}$ is finite.
This is a simple consequence of Monotone Convergence.  

Henceforth, we assume
that $n$ ranges from $1$ to $N$, where $N$ is a fixed positive
integer and, instead of (\ref{main-inequality}), 
prove the inequality
\begin{equation}
\|\max_{ 1\leq n\leq N} |k_n*_Rf|\|_{L^1(\mu)}\leq
c^2 N(\{k_n\}) \|f\|_{L^1(\mu)},
\label{main-inequality'}
\end{equation}
for all $f\in H^1(\mu)$.

%%%%%%%%%%%%%%%%%%%%%%%%%%%%%%%%%%%%%%%%%%%%%%%%%%%%%%%%%%%%%%%
%%%%%%%%%%%%%%%%%%%%%%%%%%%%%%%%%%%%%%%%%%%%
%%%%%%%%%%%%%%%%%%%%%%%%%%%%%%%%%%%%%%%%%%%%%%%%%%%%%%%%%%%%%%

{\bf Second reduction}  In proving (\ref{main-inequality'}), 
it is enough
to consider functions $f\in H^1(\mu)$ with
$\spec_R (f)\subset [\epsilon,\infty)$, where $\epsilon >0$.

To justify this reduction, 
suppose that (\ref{main-inequality'})
holds whenever a representation $R$ 
is separation-preserving,
strongly continuous, uniformly bounded with constant $c$,
and $f$ has its spectrum contained in $[\epsilon,\infty)$ where
$\epsilon >0$.
Let $\alpha>0$, and consider the representation
$e^{i\alpha (\cdot )} R$.  It is clear that 
this representation enjoys all the stated properties of $R$
(strong continuity,  uniform boundedness with the 
same constant $c$, and separation-preserving).  
Moreover, if $f\in H^1(\mu)$,
then
$\spec_{e^{i\alpha (\cdot )} R} f\subset [\alpha,\infty)$,
by Lemma \ref{lemma2}(i).
Hence, by our assumption,
%%%%%%%%%%%%%%%%%%%%%%%%%%%%%%%%%%%%%%%%%%%%%%%%%%%%%%%%%%%%%%%%
%%%%%%%%%%%%%%%%%%%%%%%%%%%%%%%%%%%%%%%%%%
\begin{equation}
\|\max_{1\leq n\leq N} 
|k_n*_{e^{i\alpha (\cdot )} R} f|\|_{L^1(\mu)}
\leq c^2 N(\{k_n\}) \|f\|_{L^1(\mu)}.
\label{assumption-second}
\end{equation}
%%%%%%%%%%%%%%%%%%%%%%%%%%%%%%%%%%%%%%%%%%%%%%%%%%%%%%%%%%%%%%%%%%%%%%
%%%%%%%%%%%%%%%%%%%%%%%%%%%%%%%%%%%%%%%%%%%%%%%%%%%%%%%%%%%%%%%%%%%%%%%
%%%%%%%%%%%%%%%%%%%%%%%%%%%%%%%%%%%%%%%%%%%%%%%%%%%%%%%%%%%%%%%%%%%%%%%
Letting $\alpha\downarrow 0$, and using Lemma \ref{lemma2},
we have that,
for each $n\in \{1,2,\ldots, N\}$,
 $k_n*_{e^{i\alpha (\cdot )} R} f\rightarrow k_n*_R f$
in $L^1(\mu)$.  From this and (\ref{assumption-second}),
the inequality (\ref{main-inequality'})
follows easily, establishing the second reduction.
%%%%%%%%%%%%%%%%%%%%%%%%%%%%%%%%%%%%%%%%%%%%%%%%%%%%%%%%%%%%%%%
%%%%%%%%%%%%%%%%%%%%%%%%%%%%%%%%%%%%%%%%%%%%%%%%%%%%%%%%%%%%%
%%%%%%%%%%%%%%%%%%%%%%%%%%%%%%%%%%%%%%%%%%%%%%%%%%%%%%%%%%%%%%

{\bf Proof of Theorem \ref{main-theorem}}
Suppose that $f\in L^1(\mu)$ with
$\spec_R (f)\subset [\epsilon, \infty)$ where $\epsilon $
is a fixed positive number. 
Let $g$ be as in Lemma \ref{lemma-sec3} and let
$F(t)=g(t)R_tf$.  By Lemma \ref{lemma-sec3},
$F\in H^1(\R, L^1(\mu))$, and so, by 
Lemma \ref{vector-version}, we have
%%%%%%%%%%%%%%%%%%%%%%%%%%%%%%%%%%%%%%%%%%%%%%%%%%%%%%%%%%%%%%%
%%%%%%%%%%%%%%%%%%%%%%%%%%%%%%%%%%%%%%%%%%%%%%%%%%%%%%%%%%%%%
\begin{equation}
\int_\R \left\|
\max_{1\leq n\leq N}
\left| \int_\R
F(x-t)k_n(t) dt \right| 
\right\|_{L^1(\mu)}dx
\leq N(\{k_n\}) \|F\|_{L^1(\R,L^1(\mu))}.
\label{*}
\end{equation}
%%%%%%%%%%%%%%%%%%%%%%%%%%%%%%%%%%%%%%%%%%%%%%%%%%%%%%%%%%%%%%%
%%%%%%%%%%%%%%%%%%%%%%%%%%%%%%%%%%%%%%%%%%%%%%%%%%%%%%%%%%%%%
We now proceed to show that (\ref{main-inequality'})
is a consequence of (\ref{*}).  We have
%%%%%%%%%%%%%%%%%%%%%%%%%%%%%%%%%%%%%%%%%%%%%%%%%%%%%%%%%%%%%%%
%%%%%%%%%%%%%%%%%%%%%%%%%%%%%%%%%%%%%%%%%%%%%%%%%%%%%%%%%%%%%
\begin{eqnarray}
\|F\|_{L^1(\R,L^1(\mu))}
&=&
\int_\R \int_\O |g(t)R_tf|d|\mu | dt \nonumber\\
&=&
\int_\R g(t) \int_\O|R_tf|d | \mu | dt \leq c \|f\|_{L^1(\mu)}.
\label{**}
\end{eqnarray}
%%%%%%%%%%%%%%%%%%%%%%%%%%%%%%%%%%%%%%%%%%%%%%%%%%%%%%%%%%%%%%%
%%%%%%%%%%%%%%%%%%%%%%%%%%%%%%%%%%%%%%%%%%%%%%%%%%%%%%%%%%%%%
Using the fact that $R$ is a uniformly bounded and strongly
continuous representation by separation-preserving
operators, we obtain
%%%%%%%%%%%%%%%%%%%%%%%%%%%%%%%%%%%%%%%%%%%%%%%%%%%%%%%%%%%%%%%
\begin{eqnarray*}
\max_{1\leq n\leq N}
\left|
\int_\R F(x-t)k_n(t) dt
\right|
			&=&
\max_{1\leq n\leq N}
\left|
\int_\R
(R_{x-t} f) g(x-t) k_n(t) dt
\right|			\\
			&=&
\max_{1\leq n\leq N}
|R_x|\left(\left|
\int_\R
(R_{-t} f) g(x-t) k_n(t) dt
\right|\right).
\end{eqnarray*}
%%%%%%%%%%%%
%%%%%%%%%%%%%%%%%%%%%%%%%%%%%%%%%%%%%%%%%%
%%%%%%%%%%%%%%%%%%%%%%%%%%%%%%%%%%%%%%%%
Since $|R_{\pm x}|$ is positivity-preserving
and since $|R_x|^{-1}=|R_{-x}|$, we obtain
after applying $|R_{-x}|$ to both sides of
the last equality
%%%%%%%%%%%%
%%%%%%%%%%%%%%%%%%%%%%%%%%%%%%%%%%%%%%%%%%
%%%%%%%%%%%%%%%%%%%%%%%%%%%%%%%%%%%%%%%%
%%%%%%%%%%%%
%%%%%%%%%%%%%%%%%%%%%%%%%%%%%%%%%%%%%%%%%%
%%%%%%%%%%%%%%%%%%%%%%%%%%%%%%%%%%%%%%%%
%%%%%%%%%%%%%%%%%%%%%%%%%%%%%%%%%%%%%%%
%%%%%%%%%%%%%%%%%%%%%%%%%%%%%%%%%%%%%%%
$$
|R_{-x}|
\left(
\max_{1\leq n\leq N}
\left|
\int_\R F(x-t)k_n(t) dt
\right|
\right)			
\geq
\max_{1\leq n\leq N}
\left|
\int_\R
(R_{-t} f) g(x-t) k_n(t) dt
\right|.
$$
Hence, using the last inequality
and the uniform boundedness of $R$,
we obtain
%%%%%%%%%%%%%%%%%%%%%%%%%%%%%%%%%%%%%%%%%%%%%%%%%%%%%%
%%%%%%%%%%%%%%%%%%%%%%%%%%%%%%%%%%%%%%%%%%%%%%%%%%%%%
%%%%%%%%%%%%%%%%%%%%%%%%%%%%%%%%%%%%%%%%%%%%%%%%%%%%%%%%%%%%%
\begin{equation}
c\left\|
\max_{1\leq n\leq N}
\left|
\int_\R F(x-t)k_n(t) dt
\right|
\right\|_{L^1(\mu)}
	\geq
\left\|
\max_{1\leq n\leq N}
\left|
\int_\R
(R_{-t} f) g(x-t) k_n(t) dt
\right|
\right\|_{L^1(\mu)}.
\label{*22-2}
\end{equation}
%%%%%%%%%%%%%%%%%%%%%%%%%%%%%%%%%%%%%%%%%%%%%%%%%%%%%%%%%%%%%%%
%%%%%%%%%%%%%%%%%%%%%%%%%%%%%%%%%%%%%%%%%%%%%%%%%%%%%%%%%%%%%
Integrating both sides of (\ref{*22-2})
over $\R$ in the $x$ variable,
and using (\ref{*}) and (\ref{**}), 
we obtain
%%%%%%%%%%%%%%%%%%%%%%%%%%%%%%%%%%%%%%%%%%%%%%%%%%%%%%%%%%%%%%%
%%%%%%%%%%%%%%%%%%%%%%%%%%%%%%%%%%%%%%%%%%%%%%%%%%%%%%%%%%%%%
\begin{equation}
N(\{k_n\}) c^2 \|f\|_{L^1(\mu)}\geq
 \int_\R \int_\O
\max_{1\leq n\leq N}
\left|
\int_\R
(R_{-t} f) g(x-t) k_n(t) dt
\right|
d|\mu|dx .
\label{**22-2}
\end{equation}
%%%%%%%%%%%%%%%%%%%%%%%%%%%%%%%%%%%%%%%%%%%%%%%%%%%%%%%%%%%%%%%
%%%%%%%%%%%%%%%%%%%%%%%%%%%%%%%%%%%%%%%%%%%%%%%%%%%%%%%%%%%%%
%%%%%%%%%%%%%%%%%%%%%%%%%%%%%%%%%%%%%%%%%%%%%%%%%%%%%%%%%%%%%%%
%%%%%%%%%%%%%%%%%%%%%%%%%%%%%%%%%%%%%%%%%%%%%%%%%%%%%%%%%%%%%
%%%%%%%%%%%%%%%%%%%%%%%%%%%%%%%%%%%%%%%%%%%%%%%%%%%%%%%%%%%%%%%
%%%%%%%%%%%%%%%%%%%%%%%%%%%%%%%%%%%%%%%%%%%%%%%%%%%%%%%%%%%%%
Obvious manipulations with 
(\ref{**22-2}),
 Fubini's Theorem and 
the fact that $g\geq 0$ and $\widehat{g}(0)=\int_\R g =1$,
 yield
%%%%%%%%%%%%%%%%%%%%%%%%%%%%%%%%%%%%%%%%%%%%%%%%%%%%%%%%%%%%%%%
%%%%%%%%%%%%%%%%%%%%%%%%%%%%%%%%%%%%%%%%%%%%%%%%%%%%%%%%%%%%%
\begin{eqnarray*}
N(\{k_n\}) c^2 \|f\|_{L^1(\mu)}
			&\geq&
 \int_\O \int_\R
\max_{1\leq n\leq N}
\left|
\int_\R
(R_{-t} f) g(x-t) k_n(t) dt
\right|
dx d|\mu| \\
			&\geq&
 \int_\O 
\max_{1\leq n\leq N}
\left| \int_\R g(x-t)dx
\int_\R
(R_{-t} f) k_n(t) dt
\right|
d|\mu | \\
			&=&
 \int_\O 
\max_{1\leq n\leq N}
\left| 
\int_\R
(R_{-t} f) k_n(t) dt
\right|
d|\mu| \\
			&=&
\|\max_{1\leq n\leq N} |k_n*_Rf|\|_{L^1(\mu)},
\end{eqnarray*}
%%%%%%%%%%%%%%%%%%%%%%%%%%%%%%%%%%%%%%%%%%%%%%%%%%%%%%%%%%%%%%%
%%%%%%%%%%%%%%%%%%%%%%%%%%%%%%%%%%%%%%%%%%%%%%%%%%%%%%%%%%%%% 
which proves
(\ref{main-inequality'}).
%%%%%%%%%%%%%%%%%%%%%%%%%%%%%%%%%%%%%%%%%%%%%%%%%%%%%%%%%%%%%%%
%%%%%%%%%%%%%%%%%%%%%%%%%%%%%%%%%%%%%%%%%%%%%%%%%%%%%%%%%%%%%
%%%%%%%%%%%%%%%%%%%%%%%%%%%%%%%%%%%%%%%%%%%%%%%%%%%%%%%%%%%%%
%%%%%%%%%%%%%%%%%%%%%%%%%%%%%%%%%%%%%%%%%%%%%%%%%%%%%%%%%%%%%%%
%%%%%%%%%%%%%%%%%%%%%%%%%%%%%%%%%%%%%%%%%%%%%%%%%%%%%%%%%%%%%
%%%%%%%%%%%%%%%%%%%%%%%%%%%%%%%%%%%%%%%%%%%%%%%%%%%%%%%%%%%%%
%%%%%%%%%%%%%%%%%%%%%%%%%%%%%%%%%%%%%%%%%%%%%%%%%%%%%%%%%%%%%%%
%%%%%%%%%%%%%%%%%%%%%%%%%%%%%%%%%%%%%%%%%%%%%%%%%%%%%%%%%%%%%
%%%%%%%%%%%%%%%%%%%%%%%%%%%%%%%%%%%%%%%%%%%%%%%%%%%%%%%%%%%%%%%
%%%%%%%%%%%%%%%%%%%%%%%%%%%%%%%%%%%%%%%%%%%%%%%%%%%%%%%%%%%%%
%%%%%%%%%%%%%%%%%%%%%%%%%%%%%%%%%%%%%%%%%%%%%%%%%%%%%%%%%%%%%%%
%%%%%%%%%%%%%%%%%%%%%%%%%%%%%%%%%%%%%%%%%%%%%%%%%%%%%%%%%%%%%
%%%%%%%%%%%%%%%%%%%%%%%%%%%%%%%%%%%%%%%%%%%%%%%%%%%%%%%%%%%%%%%
%%%%%%%%%%%%%%%%%%%%%%%%%%%%%%%%%%%%%%%%%%%%%%%%%%%%%%%%%%%%%
%%%%%%%%%%%%%%%%%%%%%%%%%%%%%%%%%%%%%%%%%%%%%%%%%%%%%%%%%%%%%%%
%%%%%%%%%%%%%%%%%%%%%%%%%%%%%%%%%%%%%%%%%%%%%%%%%%%%%%%%%%%%%
%%%%%%%%%%%%%%%%%%%%%%%%%%%%%%%%%%%%%%%%%%%%%%%%%%%%%%%%%%%%%%%
%%%%%%%%%%%%%%%%%%%%%%%%%%%%%%%%%%%%%%%%%%%%%%%%%%%%%%%%%%%%%
%%%%%%%%%%%%%%%%%%%%%%%%%%%%%%%%%%%%%%%%%%%%%%%%%%%%%%%%%%%%%%%
%%%%%%%%%%%%%%%%%%%%%%%%%%%%%%%%%%%%%%%%%%%%%%%%%%%%%%%%%%%%%
%%%%%%%%%%%%%%%%%%%%%%%%%%%%%%%%%%%%%%%%%%%%%%%%%%%%%%%%%%%%%%%
%%%%%%%%%%%%%%%%%%%%%%%%%%%%%%%%%%%%%%%%%%%%%%%%%%%%%%%%%%%%%
\section{$H^1(\mu)$ and the ergodic Hilbert transform}
\newtheorem{th1}{Theorem}[section]
\newtheorem{sec4prop1}[th1]{Proposition}
\newtheorem{sec4prop2}[th1]{Proposition}
\newtheorem{sec4prop3}[th1]{Proposition}
\newtheorem{sec4prop4}[th1]{Proposition}
\newtheorem{remark1}[th1]{Remark}
\newtheorem{lemma-th1}[th1]{Lemma}
\newtheorem{th2}[th1]{Theorem}
\newtheorem{corollary-th2}[th1]{Corollary}

In this section we will investigate a connection
between $H^1(\mu)$,
 the space ergodic $H^1$ of \cite{cw2},
and spaces of functions introduced in
\cite{abg2} (Theorem \ref{th1} below).
Throughout, $u\rightarrow R_u$ will denote a
strongly continuous
representation of $\R$ by measure-preserving transformations
on an finite measure space $(\O, \cM,\mu)$.
In particular, $R$ is 
separation-preserving and uniformly bounded with
$c=1$.  (The results of this section apply as well
 in the more general
setting of distributionally controlled 
representations that were introduced in \cite{abg2}.  
For clarity's sake, we will only discuss 
representations given by measure-preserving transformations.)
Since the measure $\mu$ is finite, 
we have the following useful direct sum decomposition
of $L^1(\mu)$:
$$L^1(\mu)=Y \bigoplus Z,$$
where 
$$Y=\{f\in L^1(\mu):\ R_uf=f,\ {\rm for\ all}\ u\in\R\},$$
and $Z$ is the $L^1(\mu)$-closure of the linear subspace of
$L^1(\mu)$ spanned by the ranges of the 
operators $f\mapsto g*_R f$, for all $g\in L^1(\R)$
such that $\R\setminus \{0\}$ contains the
support of $\widehat{g}$.  (See \cite[Proposition (3.19)]{abg2}.)

Let $h(t)=\frac{1}{\pi t}$ for $t\neq 0$ denote the 
Hilbert kernel.  For each $n$,  let $h_n$ denote the $n$th truncate
$h_n(t)=\frac{1}{\pi t}$ if $\frac{1}{n}<|t|<n$, and
$h_n(t)=0$ otherwise.  For $f\in L^1(\R)$,
denote its Hilbert transform by $\widetilde{f}$.
It is a classical fact that if $f\in H^1(\R)$,
then $h_n* f\rightarrow \widetilde{f}$ in $L^1(\R)$,
and in this case, we have $\widetilde{f}=-i f$.
(Recall that $H^1(\R)$ consists of all functions
in $L^1(\R)$ with Fourier transform vanishing on $(-\infty,0]$.)
Since the space $H^1(\R)$ is a Banach space with the
$L^1(\R)$ norm, it follows from the 
Uniform Boundedness Principle that there is 
a positive constant $C$ such that for all
$n$ and all $f\in H^1(\R)$
%%%%%%%%%%%%%%%%%%%%%%%%%%%%%%%%%%%%%%%%%%%%%%%%%%%%%%%%%%
%%%%%%%%%%%%%%%%%%%%%%%%%%%%%%%%%%%%%%%%%%%%%%%%%%%%%%%%
\begin{equation}
\|h_n*f \|_1\leq C\|f\|_1.
\label{uniform-bdness}
\end{equation}
%%%%%%%%%%%%%%%%%%%%%%%%%%%%%%%%%%%%%%%%%%%%%%%%%%%%%%%%%%
%%%%%%%%%%%%%%%%%%%%%%%%%%%%%%%%%%%%%%%%%%%%%%%%%%%%%%%%%

For $f\in L^1(\mu)$, we define the ergodic Hilbert 
transform of $f$ by
$$\cH f=\lim_n h_n*_R f\ \mu\ a.e..$$
It is a well-known consequence of the
transference methods that the limit exists 
$\mu$-a.e. and that the
operator $f\mapsto \cH f$ is of weak type
$(1,1)$ with weak type norm smaller than the weak
type $(1,1)$ norm of the Hilbert transform.
Also, the maximal operator
$f\mapsto \sup_n\left| h_n*_R f\right|$
is of weak type
$(1,1)$ with weak type norm smaller than the
weak type $(1,1)$ norm of the maximal Hilbert transform
on $L^1(\R)$.  (See \cite{cal,cw1}, or \cite{abg2}
for the case of distributionally controlled representations.)

Following Coifman and Weiss \cite{cw2},
we define the space ergodic $H^1$
as the class of all functions of the
form $f+ i \cH f\in L^1(\mu)$.
%%%%%%%%%%%%%%%%%%%%%%%%%%%%%%%%%%%%%%%%
We also recall form
\cite{abg2}, Section 3, the space
$$\cA (R)=\{f\in L^1(\mu):\ \cH f\in L^1(\mu)\}.$$
%%%%%%%%%%%%%%%%%%%%%%%%%%%%%%%%%%%%%%%
%%%%%%%%%%%%%%%%%%%%%%%%
%%%%%%%%%%%%%%%%%%%%%%%%%%
%%%%%%%%%%%%%%%%%%%%%%%%%%%%%%
A connection between $H^1(\mu)$ and ergodic $H^1$
is described by the following theorem.
%%%%%%%%%%%%%%%%%%%%%%%%%%%%%%%%%%%%
%%%%%%%%%%%%%%%%%%%%%%%%%%%
%%%%%%%%%%%%%%%%%%%%%%%%%%
%%%%%%%%%%%%%%%%%%%%%%%%%%%%%%
\begin{th1}
Suppose that $f\in L^1(\mu)$.  Then,\\
(i)  $f\in H^1(\mu)\Rightarrow f\in \cA(R)$ 
and $\|\cH f\|_1\leq C\|f\|_1$ where $C$ is as in 
(\ref{uniform-bdness});\\
(ii)  $f\in \cA(R)\Rightarrow f+i\cH f\in H^1(\mu)$.\\ 
\label{th1}
\end{th1}
%%%%%%%%%%%%%%%%%%%%%%%%%%%%%%%%%%%%%
%%%%%%%%%%%%%%%%%%%%%%%%%%
%%%%%%%%%%%%%%%%%%%%%%%%%%
%%%%%%%%%%%%%%%%%%%%%%%%%%%%%%%%%%%%
%%%%%%%%%%%%%%%%%%%%%%%%%%%
%%%%%%%%%%%%%%%%%%%%%%%%%%%%
We present
the proof in a sequence
of steps.
%%%%%%%%%%%%%%%%%%%%%%%%%%%%%%%%%%%%
%%%%%%%%%%%%%%%%%%%%%%%%%
%%%%%%%%%%%%%%%%%%%%%%%%%%%%%%%%%%%
%%%%%%%%%%%%%%%%%%%%%%%%%%%%
%%%%%%%%%%%%%%%%%%%%%%%%%%%%
\begin{sec4prop1}
If $f\in H^1(\mu)$,
then
$$\sup_n\|h_n*_Rf\|_{L^1(\mu)}\leq C\|f\|_{L^1(\mu)},$$
where $C$ is as in (\ref{uniform-bdness}).
\label{sec4prop1}
\end{sec4prop1}

%%%%%%%%%%%%%%%%%%%%%%%%%%%%%%%%%%%
%%%%%%%%%%%%%%%%%%%%%%%%%%
%%%%%%%%%%%%%%%%%%%%%%%%%%%%%%%%%%
%%%%%%%%%%%%%%%%%%%%%%%%%%%
{\bf Proof.}  As a consequence of 
Proposition \ref{prop-spectrum}(ii),
we have that $H^1(\mu)$ is a Banach space
with the $L^1(\mu)$-norm.  The desired result is now a 
 simple consequence
of the Uniform Boundedness Principle,
Theorem \ref{single}, and (\ref{uniform-bdness}).
%%%%%%%%%%%%%%%%%%%%%%%%%%%%%%%%
%%%%%%%%%%%%%%%%%%%%%%%%%%%%%
%%%%%%%%%%%%%%%%%%%%%%%%%%%%%%%
%%%%%%%%%%%%%%%%%%%%%%%%%%%%%%
%%%%%%%%%%%%%%%%%%%%%%%%%%%%%%
%%%%%%%%%%%%%%%%%%%%%%%%%%%%%%%%%
%%%%%%%%%%%%%%%%%%%%%%%%%%%%

%%%%%%%%%%%%%%%%%%%%%%%%%%%%%%%
%%%%%%%%%%%%%%%%%%%%%%%%%%%%%%
%%%%%%%%%%%%%%%%%%%%%%%%%%%%%
%%%%%%%%%%%%%%%%%%%%%%%%%%%%%%%%
{\bf Proof of Theorem \ref{th1}(i).}  
Let $f\in H^1(\mu)$.  
We have $h_n*_Rf\rightarrow \cH f$
$\mu$-a.e.  Applying Proposition \ref{sec4prop1}
and Fatou's Lemma, we get the desired conclusions.
%%%%%%%%%%%%%%%%%%%%%%%%%%%%%%%%%%%%%%%%
%%%%%%%%%%%%%%%%%%%%%
%%%%%%%%%%%%%%%%%%%%%%%%%%%%%%%%%%%%%%%%%
%%%%%%%%%%%%%%%%%%%%

Toward the proof of 
Theorem \ref{th1}(ii), we present the following result.
%%%%%%%%%%%%%%%%%%%%%%%%%%%%%%%%%%%%%%%
%%%%%%%%%%%%%%%%%%%%%%%%
%%%%%%%%%%%%%%%%%%%%%%%%%%%%
\begin{sec4prop3}
Suppose that $f\in H^1(\mu)$,
and write $f=f_1+f_2$, where 
$f_1\in Y$ and $f_2\in Z$, then $f_2\in H^1(\mu)$.
\label{sec4prop3}
\end{sec4prop3}

%%%%%%%%%%%%%%%%%%%%%%%%%%%%%%%%%
%%%%%%%%%%%%%%%%%%%%%%%%%%%%
%%%%%%%%%%%%%%%%%%%%%%%%%%%%%%%%
%%%%%%%%%%%%%%%%%%%%%%%%%%%%%
{\bf Proof.}  According to 
Proposition \ref{proposition2-sec2}(iv),
it is enough to show that for 
every $h\in \Im ([0,\infty))$ we have
$h*_Rf_2=0$.  Note that for $h\in \Im ([0,\infty))$,
we have $\int_\R h dx =\widehat{h}(0)= 0$.
Now, $h*_Rf_2=h*_R f-h*_R f_1=h*_R f- f_1\int_\R h dx =h*_R f$.
Since $f\in H^1(\mu)$, 
we have by Proposition \ref{proposition2-sec2}(iv)
$h*_Rf=0$ and so $h*_Rf_2=0$.\\
%%%%%%%%%%%%%%%%%%%%%%%%%%%%%%%%%%
%%%%%%%%%%%%%%%%%%%%%%%%%%%
%%%%%%%%%%%%%%%%%%%%%%%%%%%%%%%%
%%%%%%%%%%%%%%%%%%%%%%%%%%%%%
%%%%%%%%%%%%%%%%%%%%%%%%%%%%%%%
%%%%%%%%%%%%%%%%%%%%%%%%%%%%%%
%%%%%%%%%%%%%%%%%%%%%%%%%%%%
%%%%%%%%%%%%%%%%%%%%%%%%%%%%%%%%%
%%%%%%%%%%%%%%%%%%%%%%%%%
%%%%%%%%%%%%%%%%%%%%%%%%%%%%%%%%%%%%
%%%%%%%%%%%%%%%%%%%%%%%%
%%%%%%%%%%%%%%%%%%%%%%%%%%%%%%%%%%%%%
Before we return to the proof of Theorem \ref{th1} we 
recall certain properties from \cite{abg2}.
%%%%%%%%%%%%%%%%%%%%%%%%%%%%%%%%%%%%%%%%%%%%%%%%%%%%%%%%%%%%%
%%%%%%%%%%%%%%%%%%%%%%%%%%%%%%%%%%%%%%%%%%%%%%%%%%%%%%%%%%%%%
%%%%%%%%%%%%%%%%%%%%%%%%%%%%%%%%%%%%%%%%%%%%%%%%%%%%%%%%%%%%%
%%%%%%%%%%%%%%%%%%%%%%%%%%%%%%%%%%%%%%%%%%%%%%%%%%%%%%%%%%%%%
%%%%%%%%%%%%%%%%%%%%%%%%%%%%%%%%%%%%%%%%%%%%%%%%%%%%%%%%%%%%%
%%%%%%%%%%%%%%%%%%%%%%%%%%%%%%%%%%%%%%%%%%%%%%%%%%%%%%%%%%%%%
\begin{remark1}
 (i) 
 Suppose that
$f\in \cA(R)$, then $h_n*_Rf\rightarrow \cH f $
in $L^1(\mu)$ (see \cite[Theorem (3.24)]{abg2}).\\
(ii)  Following \cite[Definition (3.29)]{abg2},
let 
$$H^1(R)=\{f\in L^1(\mu):\ f=i\cH f\}.$$
Then, according to \cite[Theorem (3.34)]{abg2},
we have
$$H^1(R)=\{f+i\cH f:\ f\in Z,\ {\rm and}\ 
\cH f\in L^1(\mu)\}.$$
(iii)  Using (i), we see that
if $f\in H^1(R)$, then $f\in \cA (R)$
and hence, $\mu$-a.\ e.\ , we have
$$f=i \cH f = i \lim_n h_n*_R f.$$
\label{remark1}
\end{remark1}

%%%%%%%%%%%%%%%%%%%%%%%%%%%%%%%%%%%%%%%
%%%%%%%%%%%%%%%%%%%%%%
%%%%%%%%%%%%%%%%%%%%%%%%%%%%%%%%%%%%%%%%
%%%%%%%%%%%%%%%%%%%%%%%
%%%%%%%%%%%%%%%%%%%%%%%%%%%%%%%%%%%%%%%%
%%%%%%%%%%%%%%%%%%%%%
{\bf Proof of Theorem \ref{th1}(ii).}
%%%%%%%%%%%%%%%%%%%%%%%%%%%%%%%%%%%%%%%%%
%%%%%%%%%%%%%%%%%%%%%%
%%%%%%%%%%%%%%%%%%%%%%%%%%%%%%%%%%%%%%%%
%%%%%%%%%%%%%%%%%%%%%
We first show that $H^1(R)\subset H^1(\mu)$.
Let $f\in H^1(R)$.  By Proposition \ref{proposition2-sec2},
it is enough to show that for every
$h\in \Im ([0,\infty))$ we have $h*_Rf=0$.
For $h\in \Im([0,\infty))$,
we easily see that $\widetilde{h}\in L^1(\R)$ and
$\widetilde{h}=i h$.  Hence, 
$h_n*h\rightarrow \widetilde{h}=ih$ in $L^1(\R)$.
Now, using Remark \ref{remark1}(iii) and convergence in $L^1(\mu)$, 
we can write
%%%%%%%%%%%%%%%%%%%%%%%%%%%%%%%%%%%
%%%%%%%%%%%%%%%%%%%%%%%%%%%%
%%%%%%%%%%%%%%%%%%%%%%%%%%%%%%%%%%%%%%%%%%%%%%%%%
%%%%%%%%%%%%%%%%%%%%%%%
\begin{eqnarray*}
h*_R f	&=&	h*_R(i\cH f)\\
	&=& 	i h*_R (\lim_n h_n*_R f)\\
	&=&	i \lim_n(h*h_n)*_R f=-h*_Rf,
\end{eqnarray*}
%%%%%%%%%%%%%%%%%%%%%%%%%%%%%%%%%%%%%%%%%%%%%%%%%%%%
implying that $h*_Rf=0\ \mu$-a.e., and 
so $H^1(R)\subset H^1(\mu)$.
%%%%%%%%%%%%%%%%%%%%%%%%%%%%%%%%%%%%%%%%%%%%%%%%%%%%%%%%%%%%%

Now suppose that $f\in \cA(R)$ and write $f=f_1+f_2$ with
$f_1\in Y, f_2\in Z$.  We have
$f+i\cH f=f_1 + f_2+i \cH f_2$.
We have $f_2+ i\cH f_2\in H(R)\subset H^1(\mu)$.
Also, we trivially have
$f_1\in H^1(\mu)$.  Hence it follows that
$f+i\cH f\in H^1(\mu)$.

%%%%%%%%%%%%%%%%%%%%%%%%%%%%%%%%%%%%%%%%%%%%%%%%%%%%%%%%%%%%%%%
%%%%%%%%%%%%%%%%%%%%%%%%%%%%%%%%%%%%%%%%%%%%%%%%%%%%%%%%%%%%%
{\bf Ergodic $H^1$}  We end the paper by mentioning how Theorem 
\ref{main-theorem} 
can be used to simplify some of the proofs in the 
maximal characterization of ergodic $H^1$ in \cite{cw2}.
As is done in \cite{cw2}, we introduce a
 maximal convolution operator $M$ that
characterizes $H^1(\R)$ and such that the kernels of 
the convolution operators have compact support.  
This operator can be defined by the dilates 
of a single smooth function with compact support
(see \cite[Section 1]{cw2}).  For real-valued functions
$f\in L^1(\R)$, we have
\begin{equation}
c_1 \|Mf\|_1\leq \|f\|_1 +\|\widetilde{f}\|_1\leq c_2\|Mf\|_1,
\label{characterization}
\end{equation}
where $c_1$ and $c_2$ are absolute constants.
%%%%%%%%%%%%%%%%%%%%%%%%%%%%%%%%%%%%%%%%%%%%%%%%%%%%%%%%%%%%%%%
%%%%%%%%%%%%%%%%%%%%%%%%%%%%%%%%%%%%%%%%%%%%%%%%%%%%%%%%%%%%%
Let $M^\sharp$ denote the transferred 
maximal operator defined on $L^1(\mu)$.
Coifman and Weiss \cite{cw2} proved that there are
positive constants $c_1$ and $c_2$ such that
for all real-valued $f\in L^1(\mu)$, we have
%%%%%%%%%%%%%%%%%%%%%%%%%%%%%%%%%%%%%%%%%%%%%%%%%%%%%%%%%%%%%%%
%%%%%%%%%%%%%%%%%%%%%%%%%%%%%%%%%%%%%%%%%%%%%%%%%%%%%%%%%%%%%
\begin{equation}
c_1 \|M^\sharp f\|_1\leq \|f\|_1 +\|\cH f\|_1\leq 
c_2\|M^\sharp f\|_1.
\label{tranferred}
\end{equation}
%%%%%%%%%%%%%%%%%%%%%%%%%%%%%%%%%%%%%%%%%%%%%%%%%%%%%%%%%%%%%%%
%%%%%%%%%%%%%%%%%%%%%%%%%%%%%%%%%%%%%%%%%%%%%%%%%%%%%%%%%%%%%
The second inequality in (\ref{tranferred}) follows 
directly using the transference methods, as shown in
\cite{cw2}, following Lemma 2.7.  
The proof of the first inequality in (\ref{tranferred})
as presented in \cite{cw2} is much more involved.
We will show here that this inequality is a 
simple consequence of Theorem \ref{main-theorem}
and the first inequality in (\ref{characterization}).  
Indeed, suppose that
$f$ is a real-valued function in 
$L^1(\mu)$ and $\cH f$ is also in $L^1(\mu)$.  Then,
by Theorem \ref{th1}, $f + i \cH f\in H^1(\mu)$.
Note that for $f\in H^1(\R)$, 
the first inequality in (\ref{characterization})
states that 
\begin{equation}
c_1\|Mf\|_1\leq 2 \|f\|_1.
\label{*4-12}
\end{equation}
Applying
Theorem \ref{main-theorem} with
$f+i\cH f\in H^1(\mu)$, and using
(\ref{*4-12}) and Theorem \ref{th1}(i), we get
$$
\|M^\sharp f\|_1
		\leq
\|M^\sharp( f+i\cH f)\|_1	
		\leq
 2c_1^{-1}\|f + i \cH f\|_1	
		\leq
 2c_1^{-1}( 1+C) \|f\|_1.$$
%%%%%%%%%%%%%%%%%%%%%%%%%%%%%%%%%%%%%%%%%%%%%%%%%%%%%%%%%%%%%%%
%%%%%%%%%%%%%%%%%%%%%%%%%%%%%%%%%%%%%%%%%%%%%%%%%%%%%%%%%%%%%which is what we want.
%%%%%%%%%%%%%%%%%%%%%%%
%%%%%%%%%%%%%%%%%%%%%%%%%%%%%%%%%%%%%%%%
%%%%%%%%%%%%%%%%%%%%%%%%%%%%%%%%%%%%%
%%%%%%%%%%%%%%%%%%%%%%%%

{\bf Acknowledgements}  The work of the authors was supported
by separate grants from the National Science Foundation (U.S.A.).
%%%%%%%%%%%%%%%%%%%%%%%%%%%%%%%%%%%%%%%%%%%%%%%%%%%%%%%%%%%%%%%
%%%%%%%%%%%%%%%%%%%%%%%%%%%%%%%%%%%%%%%%%%%%%%%%%%%%%%%%%%%%%
%%%%%%%%%%%%%%%%%%%%%%%%%%%%%%%%%%%%%%%%%%%%%%%%%%%%%%%%%%%%%%
%%%%%%%%%%%%%%%%%%%%%%%%%%%%%%%%%%%%%%%%%%%%%%%%%%%%%%%%%%%%%%%
%%%%%%%%%%%%%%%%%%%%%%%%%%%%%%%%%%%%%%%%%%%%
%%%%%%%%%%%%%%%%%%%%%%%%%%%%%%%%%%%%%%%%%%%%%%%%%%%%%%%%%%%%%%
%%%%%%%%%%%%%%%%%%%%%%%%%%%%%%%%%%%%%%%%%%%%%%%%%%%%%%%%%%%%%%%%
%%%%%%%%%%%%%%%%%%%%%%%%%%%%%%%%%%%%%%%%%%
%%%%%%%%%%%%%%%%%%%%%%%%%%%%%%%%%%%%%%%%%%%%%%%%%%%%%%%%%%%%%%%%%%%%%%%%%%%%%%%%%%%%%%%%%%%%%%%%%%%%%%%%%%%%%%%%%%%%%%%%%%%%%%%%%%%%%%%%%%%%%%%%%%%%%%%%%%%%%%%%%%%%%%%%%%%%%%%%%%%%%%%%%%%%%%%%%%%%%%%%%%%%%%%%%%%%%%%%%%%%%%%%%%%%%%%%%%%%%%%%%%%%%%%%%%%%%%%%%%%%%%%%%%%%%%%%%%%%%%

\begin{thebibliography}{Dillo 83}

\bibitem{abg1}  N.\ Asmar, E.\ Berkson, and T.\ A.\ Gillespie,
{\em Transference of strong type maximal 
inequalities by separation-preserving representation}, 
Amer.\ J. Math.\  {\bf 113} (1991),  47--74.

\bibitem{abg2}  N.\ Asmar, E.\ Berkson, and T.\ A.\ Gillespie,
{\em Distributional control and generalized analyticity},
Integral Equations and Operator Theory {\bf 14} (1991), 311--341.

%\bibitem{ams1}  N.\ Asmar and S.\ Montgomery-Smith,
%{\em Hahn's Embedding Theorem for orders
%and harmonic analysis on groups with ordered duals},
%to appear in Colloq. Math.

%\bibitem{bgm1}  E.\ Berkson, T.\ A.\ Gillespie, and P.\ S.\ Muhly,
%{\em $L^p$-multiplier transference induced by
%representations in Hilbert space}, Studia Math. {\bf 94}
%(1989), 51-61. 

%\bibitem{bp}  C.\ Bessaga and A.\ Pe\l czy\'nski, 
%{\em On bases and unconditional convergence of series in Banach spaces}, 
%Studia Math. {\bf 17} (1958), 151--164.


%\bibitem{bukh}  A.\ V.\ Bukhvalov, 
%{\em On the analytic Radon-Nikod\'ym property}, 
%in ``Function Spaces'', Proceedings of the 
%Second International Conference, Pozna\'n, 
%Teubner-Texte zur Mathematik, Band {\bf 120}, Leipzig 1991, 211-228. 

%\bibitem{bd}  A.\ V.\ Bukhvalov and A.\ A.\ Danilevich,
%{\em Boundary properties of analytic and harmonic functions 
%with values in Banach space}, Mat.\ Zametki {\bf 31} 
%(1982), 203--214.  English Translation:  Mat. 
%Notes {\bf 31} (1982) , 104--110.

\bibitem{cal}  A.\ Calder\'{o}n, 
{\em Ergodic theory and translation-invariant operators}, 
Proc.\ Nat.\ Acad.\ Sci.\ , {\bf 157}\ (1971), 137--153.

%\bibitem{cc}  A.\ Calder\'{o}n and O.\ N.\ Capri,
%{\em On the convergence in $L^1$ of singular integrals},
%Studia Math. {\bf 78} (1984), 321--327.

\bibitem{cs}  M.\ J.\ Carro and J.\ Soria,
{\em Transference theory on Hardy and Sobolev spaces},
preprint.

\bibitem{cw1}  R.\ R.\ Coifman and G.\ Weiss,
``Transference methods in analysis'', Regional 
Conference Series in Math. {\bf 31}, Amer.\ Math.\ Soc., 
Providence, R.\ I.\, 1977.

\bibitem{cw2}  R.\ R.\ Coifman and G.\ Weiss,
{\em Maximal functions on $H^p$ spaces defined by 
ergodic transformations}, 
Proc.\ Nat.\ Acad.\ Sci.\ , (U.\ S.\ A.) {\bf 70}\ (1973), 1761--1763.

%\bibitem{cw3}  R.\ R.\ Coifman and G.\ Weiss,
%{\em Operators associated with representations of amenable groups,
%singular integrals induced by ergodic flows, the rotation method 
%and multipliers}, Studia Math. {\bf 47} (1973), 285--303.

%\bibitem{dow}  H.\ R.\ Dowson, ``Spectral Theory of
%Linear Operators'',  London Math. Soc. Monographs 12,
%Academic Press, New York 1978. 

%\bibitem{deleeuwglicksberg}  K.\ De Leeuw and I.\ Glicksberg, 
%{\em Quasi-invariance and measures on compact groups},
%Acta Math., {\bf 109} (1963), 179--205.

%\bibitem{eg}    R.\ E.\ Edwards, and G.\ I.\ Gaudry,
%``Littlewood-Paley and Multiplier Theory'', 
%Ergebnisse der Mathematik
%und ihrer Grenzgebiete, Springer-Verlag, No. 90,  
%Berlin, 1977.

\bibitem{fs}    C.\ Fefferman, and E.\ Stein
{\em $H^p$-spaces in several variables},  Acta Math, 
{\bf 129}, (1972), 137--193.

%\bibitem{forelli}  F.\ Forelli,
%{\em Analytic and quasi-invariant measures}, 
%Acta Math., {\bf 118} (1967), 33--59.



%\bibitem{hl1}  H.\ Helson and D.\ Lowdenslager,
%{\em Prediction theory and Fourier series in several variables}, 
%Acta Math.\ {\bf 99}\ (1958), 165--202.

%\bibitem{hl2}  H.\ Helson and D.\ Lowdenslager,
%{\em Prediction theory and Fourier series in several variables II}, 
%Acta Math.\ {\bf 106}\ (1961), 175--212.

%\bibitem{hr1} E.\ Hewitt and K.\ A.\ Ross,
%``Abstract Harmonic Analysis I,''  $2^{nd}$ Edition, Grundlehren der
%Math.\ Wissenschaften, Band 115, Springer--Verlag, Berlin 1979.

\bibitem{hr2} E.\ Hewitt and K.\ A.\ Ross,
``Abstract Harmonic Analysis II,'' Grundlehren der
Math.\ Wissenschaften in Einzeldarstellungen, Band
152, Springer--Verlag, New York, 1970.

%\bibitem{kt} C.\ Kenig and P.\ Tomas,
%{\em Maximal operators defined by Fourier multipliers},
%Studia Math. {\bf 68} (1980), 79--83.

\bibitem{ll} Z.\ Liu and S.\ Lu,
{\em Transference and restriction of maximal 
multiplier operators on
Hardy spaces}, Studia Math. {\bf 105} (1993), 121--134.

%\bibitem{rud} W. \ Rudin, ``Fourier Analysis on Groups,'' 
%Wiley Interscience Publication, 1962.


%\bibitem{stein} E. \ Stein, ``Harmonic Analysis:
%Real-Variable Methods, Orthogonality, and Oscillatory Integrals,'' 
%Princeton Univ. Press, Princeton, N.\ J., 1993.

%\bibitem{sw} E. \ Stein and G.\ Weiss, ``Introduction 
%to Fourier Analysis on Euclidean Spaces,'' Princeton Math. 
%Series, No. 32, Princeton Univ. Press, Princeton, N.\ J., 1971.

\bibitem{dlt} A.\ de la Torre,
{\em Ergodic $H^1$}, 
Bol. Soc. Mat. Mexicana, {\bf 22} (1977), 10--22.

%\bibitem{yosida} K. Yosida,
%``Functional Analysis,'' 6th Edition,
%Grundlehren der
%Math.\ Wissenschaften in Einzeldarstellungen, Band
%123, Springer--Verlag, New York, 1980.


%\bibitem{yud}  V.\ A. \ Yudin, {\em On Fourier sums in}\ $L^p$,
%Proc. Steklov Inst. Math. {\bf 180} (1989) 279--280. 

%\bibitem{zyg}  A.\ Zygmund,
%`` Trigonometric series'', 2nd Edition, 2 vols.\ , 
%Cambridge University Press, 1959.
\end{thebibliography}
\end{document}

%%%%%%%%%%%%
%%%%%%%%%%%
%%%%%%%%%%%%%%%%%%%%%%%%%%%%%%%%%%%%%%%%%%%%%%%%%%%%%%%%%%%%%%%%%%%%%%%%%%%%%%%
%%%%%%%%%%%%%%%%%%%%%%%%%%%%%%%%%%%%%%%%%%%%%%%%%%%%%%%%%%%%%%%%%%%%%%%%%%%%%%%%%%%%%%%%%%%%%%%%%%%%%%%%%%%%%%%%%%%%%%%%%%%%%%%%%%%%%%%%%%%%%%%%%%%%%%%%%%%%%%%%%%%%%%%%%%%%%%%%%%%%%%%%%%%%%%%%%%%%%%%%%%%%%%%%%%%%%%%%%%%%%%%%%%%%%%%%%%%%%%%%%%%%%%%%%%%%%%%%%%%%%%%%%%%%%%%%%%%%%%%%%%%%%%%%%%%%%%%%%%%%%%%%%%%%%%%%%%%%%%%%%%%%%%%%%%%%%%%%%%%%%%%%%%%%%%%%%%%%%%%%%%%%%%%%%%%%%%%%%%%%%%%%%%%%%%%%%%%%%%%%%%%%%%%%%%%%%%%%%%%%%%%%%%%%%%%%%%%%%%%%%%%%%%%%%%%%%%%%%%%%%%%%%%%%%%%%%%%%%%%%%%%%%%%%%%%%%%%%%%%%%%%%%%%%%%%%%%%%%%%%%%%%%%%%%%%%%%%%%%%%%%%%%%%%%%%%%%%%%%%%%%%%%%%%%%%%%%%%%%%%%%%%%%%%%%%%%%%%%%%%%%%%%%%%%%%%%%%%%%%%%%%%%%%%%%%%%%%%%%%%%%%%%%%%%%%%%%%%%%%%%%%%%%%%%%%%%%%%%%%%%%%%%%%%%%%%%%%%%%%%%%%%%%%%%%%%%%%%%%%%%%%%%%%%%%%%%%%%%%%%%%%%%%%%%%%%%%%%%%%%%%%%%%%%%%%%%%%%%%%%%%%%%%%%%%%%%%%%%%%%%%%%%%%%%%%%%%%%%%%%%%%%%%%%%%%%%%%%%%%%%%%%%%%%%%%%%%%%%%%%%%%%%%%%%%%%%%%%%%%%%%%%%%%%%%%%%%%%%%%%%%%%%%%%%%%%%%%%%%%%%%%%%%%%%%%%%%%%%%%%%%%%%%%%%%%%%%%%%%%%%%%%%%%%%%%%%%%%%%%%%%%%%%%%%%%%%%%%%%%%%%%%%%%%%%%%%%%%%%%%%%%%%%%%%%%%%%%%%%%%%%%%%%%%%%%%%%%%%%%%%%%%%%%%%%%%%%%%%%%%%%%%%%%%%%%%%%%%%%%%%%%%%%
%%%%%%%%%%%%%%%%%%%%
%%%%%%%%%%%%%
%%%%%%%%%%%%%%%%%%%%
%%%%%%%%%%%%%%%%%%%%%%%%%%%%%%%%%%%%%%%%%%%%%%%%%%%%%%%%%%%%%%%
%%%%%%%%%%%%%%%%%%%%%%%%%%%%%%%%%%%%%%%%%%%%%%%%%%%%%%%%%%%%%%
%%%%%%%%%%%%%%%%%%%%%%%%%%%%%%%%%%%%%%%%%%%%%%%%%%%%%%%%%%%%%%%
%%%%%%%%%%%%%%%%%%%%%%%%%%%%%%%%%%%%%%%%%%%%%%%%%%%%%%%%%%%%%%%
%%%%%%%%%%%%%%%%%%%%%%%%%%%%%%%%%%%%%%%%%%%%%%%%%%%%%%%%%%%%%%%
%%%%%%%%%%%%%%%%%%%%%%%%%%%%%%%%%%%%%%%%%%%%%%%%%%%%%%%%%%
%%%%%%%%%%%%%%%%%%%%%%%%%%%%%%%%%%%%%%%%%%%%%%%%%%%%%%%%%%%%%%%
%%%%%%%%%%%%%%%%%%%%%%%%%%%%%%%%%%%%%%%%%%%%%%%%%%%%%%%%%%%%%%
%%%%%%%%%%%%%%%%%%%%%%%%%%%%%%%%%%%%%%%%%%%%%%%%%%%%%%%%%%%%%%%
%%%%%%%%%%%%%%%%%%%%%%%%%%%%%%%%%%%%%%%%%%%%%%%%%%%%%%%%%%%%%%%
%%%%%%%%%%%%%%%%%%%%%%%%%%%%%%%%%%%%%%%%%%%%%%%%%%%%%%%%%%%%%%%
%%%%%%%%%%%%%%%%%%%%%%%%%%%%%%%%%%%%%%%%%%%%%%%%%%%%%%%%%%
%%%%%%%%%%%%%%%%%%%%%%%%%%%%%%%%%%%%%%%%%%%%%%%%%%%%%%%%%%%%%%%
%%%%%%%%%%%%%%%%%%%%%%%%%%%%%%%%%%%%%%%%%%%%%%%%%%%%%%%%%%%%%%
%%%%%%%%%%%%%%%%%%%%%%%%%%%%%%%%%%%%%%%%%%%%%%%%%%%%%%%%%%%%%%%
%%%%%%%%%%%%%%%%%%%%%%%%%%%%%%%%%%%%%%%%%%%%%%%%%%%%%%%%%%%%%%%
%%%%%%%%%%%%%%%%%%%%%%%%%%%%%%%%%%%%%%%%%%%%%%%%%%%%%%%%%%%%%%%
%%%%%%%%%%%%%%%%%%%%%%%%%%%%%%%%%%%%%%%%%%%%%%%%%%%%%%%%%%

