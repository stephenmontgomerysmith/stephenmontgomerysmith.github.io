\documentclass{amsart}
\usepackage{amsbsy,amssymb}
\usepackage{epsfig}
\usepackage{graphicx}

\title{On a Bayesian Approach to Estimating Class Number}

\author{Stephen Montgomery-Smith}
\address{Math Dept, University of Missouri, Columbia, MO 65211, U.S.A.}
\email{stephen@math.missouri.edu}


%\date{\today}

\begin{document}

\begin{abstract}
We examine a Bayesian approach to estimating the number of classes in
a population, in the situation that
we are able to take many independent samples from an
infinite population.\end{abstract}

\maketitle

\subsection*{Introduction}

Suppose we capture 1000 birds from an
effectively infinite population, and
observe 15 different species.  Estimate how many species have not been
observed.  While it is clear that one can only get a lower
estimate (because there may be a very large number of extremely rare species
which we have essentially no hope of observing), nevertheless if, say,
4 of the species appear only once, we would assume that most likely that
there are species not observed, because if 15 were the exact number
we would likely have not observed every one of them.

More generally, suppose that we have an observation $\mathcal O$
of $n$ independent samples from
an infinite population.  
Each sample belongs to one of $M$ classes $C_1,C_2,\dots,C_M$, 
and the probability that a sample belongs to
class $C_i$ is $p_i$, where
$0\le p_i\le 1$ and $\sum_{i=1}^M\,p_i =1$.
The goal of this paper is to present a Bayesian method to 
estimate the class number, $M$.
The difficulty is to estimate the number of classes that have not been
observed.

A survey of various approaches is given in \cite{chao05}.  In particular, we point
out a
rather effective method due to A.~Chao \cite{chao84}.  If
$m_k$ is the number of classes observed $k$ times, 
then a lower estimator for the number of 
classes not observed is $m_1^2/2m_2$.

The Bayesian approach will be to treat $M$ and $p = (p_1,\dots,p_M)$
as random variables with prior distributions.
After computing the posteriors, we will integrate out $p$.
This has already been explored, for example, in \cite{rodrigues et al},
where the final distribution is obtained using Monte-Carlo techniques.
In this paper we present some prior distributions for which
we are able to give explicit formulae for the final distributions.

The purpose
of this paper is not to advocate for the Bayesian approach, but rather to
identify that in this particular situation that the Bayesian approach seems
to give rather poor quality answers.  We will present a computer generated
simulation in which our methods will significantly overestimate.  In 
particular,
the beta prior distributions will be shown to give significantly
bland answers, because the final formulae do not depend upon
the numbers $m_k$, but only upon $m = \sum_{k=1}^n m_k$, the total number
of classes observed, and $n = \sum_{k=1}^n k m_k$, the sample
size.

\subsection*{The posterior}  Let $n_1,n_2,\dots,n_m$ denote the number of
samples from each observed class, that is, a sequence in which each
number $k$ appears exactly $m_k$ times.
Now we do not know which class is which, and so
$$ \Pr(\mathcal O | (M,p)) = 
   \sum_{\sigma \in B_{m,M}} \prod_{i=1}^m p_{\sigma(i)}^{n_i} ,$$
where $B_{m,M}$ denotes the set of one to one mappings
from $\{1,2,\dots,m\}$ to $\{1,2,\dots,M\}$.  (We mention
that this formula assumes that we are also told the order in which
the samples are picked.  If we do not know the order in which they are
picked, and we are simply given the numbers $n_1,n_2,\dots,n_m$, then there
is an extra factor $n!/n_1!\cdots n_m!m_1!\cdots m_n!$.  However this factor is independent of $M$ and so it will 
disappear in the final normalization that takes place in Bayes' Theorem.)

The prior for $p$ given $M$ is a distribution $\mu_M$ on the simplex
$$ \Delta_{M-1} = \left \{p = (p_1,\dots,p_M): p_i \ge 0,\, \sum_{i=1}^M p_i = 1 
   \right\} ,$$
which, since we do not know \emph{a priori} which class is which, must be invariant
under rearrangements of $p_1,\dots,p_M$.
Thus
$$ \Pr(\mathcal O|M) =
   \int_{\Delta_{M-1}} \Pr(\mathcal O|(M,p) \, d\mu_M(p) =
   \frac{M!}{(M-m)!} \int_{\Delta_{M-1}} 
   \prod_{i=1}^m p_i^{n_i} \, d\mu_M(p) .$$

\subsection*{Prior distributions from populations}

The prior distributions of $p$ that we will consider will
be derived from the size of the population of each class,
$X_i$, which are very large
(that is, effectively infinite) random variables, but independent and
identically distributed.  We will assume that each $X_i$ has a
continuous probability distribution 
$\lambda \phi(\lambda x)\, dx$, where 
$\lambda>0$ is very small.  In the appendix we will show that
$$
   \Pr(\mathcal O|M) 
   =
   \frac{M!}{(M-m)!(n-1)!}
   \int_{s=0}^\infty s^{-M-1}
   h_0(s)^{M-m} h_1(s)^{m_1} h_2(s)^{m_2} \cdots h_n(s)^{m_n} \, ds ,$$
where
$$ h_k(s) 
   = \int_0^\infty x^k e^{-x} \phi(x/s) \, dx .$$

\subsection*{Beta prior distributions}
This distribution arises when the population size of
each class is given by the Gamma distribution, namely the distribution
is $\phi(x) = x^a e^{-x}/\Gamma(a+1)$, where $a>-1$.
Then $d\mu_M(p)$ is a beta distribution, that is, it is proportional to 
$(p_1\cdots p_M)^a$ times Lebesgue measure on $\Delta_{M-1}$.
The case $a = 0$ is
the uniform distribution on $\Delta_{M-1}$.  A direct calculation shows
$$ \int_{\Delta_{M-1}} \prod_{i=1}^M p_i^{n_i} \, d\mu_M(p) = 
   \frac{\Gamma(M+Ma+1)}{\Gamma(M+Ma+n+1)}
   \prod_{i=1}^m\frac{\Gamma(n_i+a+1)}{\Gamma(a+1)} .$$
After applying Bayes' Theorem, and remembering the normalization
that takes place, we find that
$$ \Pr(M|\mathcal O) \propto \frac{M!\Gamma(M+Ma+1)}{(M-m)!\Gamma(M+n+Ma+1)}
   \Pr(M) ,$$
where the constant of proportionality is so that the probabilities on
the left hand side add up to $1$.  We notice that the
answer does not depend upon the values of $n_1,n_2,\dots,n_m$, but only
upon the sample size $n$, and the number of classes observed $m$.

\subsection*{Uniform distribution on the sphere}
Here we suppose that the population sizes of each class
have a semi-Gaussian distribution, that is
$\phi(x) = \sqrt{2/\pi} \, e^{-x^2/2}$.  
This gives rise to a distribution which is
formed by projecting the normalized uniform distribution on the positive
sphere
$$ S^+_{M-1} = \left\{q=(q_1,q_2,\dots,q_M):q_i \ge 0, \, \sum_{i=1}^M q_i^2 = 1
   \right\} $$
onto $\Delta_{M-1}$ via the map $q \mapsto q/(q_1+q_2+\cdots+q_M)$.
In this case we obtain
$$ h_k(s) 
   = \sqrt{\frac2\pi} \int_0^\infty x^k e^{-x} \exp(-x^2/2s^2) \, dx .$$
Although $h_k(s)$ can be computed explicitly 
as $(-1)^k s^{k+1}$ times the $k$th derivative of $\exp(s^2/2)
\, \text{erfc}(s/\sqrt2)$, the resulting formula is numerically unstable for large $s$.
Instead it is rather easily computed by a
standard numerical integration package.

\subsection*{Uniform distribution on the rectangle}

This next possibility is if the population of each class is 
uniformly distributed.  In this case
$h_k(s)$ is a lower incomplete Gamma function,
which can 
easily be numerically computed.

\subsection*{A numerical simulation}

A computer simulated a sample of $n=500$ taken
from a population of 100 classes,
of which 50 of the classes occurred with 
a probability of 0.05, and the other 50
classes occurred with a probability of 0.15.  The data obtained had
$m_1 = 12$, $m_2 = 15$, $m_3 = 12$, $m_4 = 8$, 
$m_5 = 6$, $m_6 = 15$, $m_7 = 8$, $m_8 = 4$, $m_9 = 4$, 
$m_{10} = 8$, $m_{11} = 2$, $m_{14} = 1$, $m_{15} = 2$, and all other
$m_k = 0$.  Thus the number of observed classes is $m=97$.  
The Chao estimate
gives 3.2 for the number of unobserved classes, which
is very close to the correct answer.

Using the improper prior distribution where $\Pr(M)$ is constant,
then
the beta with $a=0$, uniform spherical, and uniform
rectangular priors for $p$ respectively 
give a final probability of $M \le 100$ equal to
$3.46\times10^{-6}$, 0.00028, and 0.0048, and a mid-90th
percentile range for $M$ of 121--131, 106--121, and 103--115.
It can
be seen that the uniform rectangular prior seems to work the best, but all
of them significantly overestimate the class number.

\subsection*{Appendix: derivation of prior distribution from populations}

We have that
$$ \int_{\Delta_{M-1}} \prod_{i=1}^M p_i^{n_i} \, d\mu_M(p) = 
   \int_{\mathbb R_+^M}
   \frac{x_1^{n_1} x_2^{n_2} \cdots x_m^{n_m}}
   {(x_1+\cdots+x_M)^n} \lambda^M \phi(\lambda x_1) \cdots \phi(\lambda x_M)
   \, dx .$$
Since
$$ \int_{s=0}^\infty s^{n-1} e^{-s y} \, ds = \frac{(n-1)!}{y^n}  ,$$
the right hand side may be seen to equal
$$ \frac{\lambda^{M+n}}{(n-1)!} \int_{s=0}^\infty s^{n-1}
   \int_{\mathbb R_+^M}
   {x_1^{n_1} x_2^{n_2} \cdots x_m^{n_m}}
   e^{-\lambda (x_1+\cdots+x_M) s}
   \phi(\lambda x_1) \cdots \phi(\lambda x_M)
   \, dx \, ds ,$$
which after making the substitution $x_i \mapsto x_i/\lambda s$ can be
written as
$$ \frac1{(n-1)!} \int_{s=0}^\infty s^{-M-1}
    h_{n_1}(s) \cdots h_{n_m}(s) h_0(s)^{M-m} \, ds .$$

\subsection*{Acknowledgments} The author would like to extend his sincere 
thanks to Anthony Garrett introducing him to the Bayesian approach to
this particular problem.  He also extends his gratitude to Frank Schmidt
for bringing the class number problem and reference \cite{chao84} to
his attention.

\begin{thebibliography}{99}
\bibitem{chao84} Chao, A. 1984. Non-parametric estimation of the number of classes in a population. Scandinavian Journal of Statistics 11, 265-270.
\bibitem{chao05} Chao, A. 2005. Species richness estimation, Pages 7909-7916 in N. Balakrishnan, C. B. Read, and B. Vidakovic, eds. Encyclopedia of Statistical Sciences. New York, Wiley.
\bibitem{rodrigues et al}
Rodrigues J., Milan L.A., Leite J.G. 2001.  Hierarchical Bayesian estimation for the 
number of species. Biometrical J., 43, 737-746.

\end{thebibliography}
\end{document}

