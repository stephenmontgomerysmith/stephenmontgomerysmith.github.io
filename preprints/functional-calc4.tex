\documentclass[12pt,reqno]{amsart}

\usepackage{amsaddr}
\usepackage{url}
\usepackage{bm}
\usepackage{amssymb}

\newcommand\bi{\bm i}
\newcommand\bj{\bm j}
\newcommand\bk{\bm k}

\DeclareMathOperator\realpart{Re}
\DeclareMathOperator\imagpart{Im}

\newtheorem{theorem}{Theorem}
\newtheorem{lemma}[theorem]{Lemma}
\newtheorem{cor}[theorem]{Corollary}
\newtheorem{prop}[theorem]{Proposition}

\begin{document}

\title{Functional calculus for dual quaternions}
\author{Stephen Montgomery-Smith}
\address{Department of Mathematics, University of Missouri, Columbia, MO 65211, USA\\
\rm\url{stephen@missouri.edu}\\
\rm\url{https://stephenmontgomerysmith.github.io}\\
%\rm ORCID: 0000-0003-1979-5520
}

\begin{abstract}  We give a formula for $f(\eta)$, where $f :\mathbb C \to \mathbb C$ is a continuously differentiable function satisfying
$f(\bar z) = \overline{f(z)}$,
and $\eta$ is a dual quaternion.  Note this formula is straightforward or well known if $\eta$ is merely a dual number or a quaternion.  If one is willing to prove the result only when $f$ is a polynomial, then the methods of this paper are elementary.

This version of the article has been accepted for publication, after peer
review (when applicable) but is not the Version of Record and does not reflect post-acceptance
improvements, or any corrections. The Version of Record is available online at:
\url{http://dx.doi.org/10.1007/s00006-023-01282-y}. Use of this Accepted Version is subject to the publisher’s Accepted
Manuscript terms of use \url{https://www.springernature.com/gp/open-research/policies/accepted-
manuscript-terms}.
\end{abstract}

\keywords{Complex valid function, exponential, logarithm, Pauli-Pascal triangle}

\subjclass[2020]{20G20, 46H30}

\maketitle

A \emph{quaternion} is a quadruple of real numbers, written as $A = w + x \bi + y \bj + z \bk$, with the algebraic operations $\bi^2 = \bj^2 = \bk^2 = \bi \bj \bk = -1$.  Its \emph{conjugate} is $\bar A = A = w - x \bi - y \bj - z \bk$.  Its \emph{norm} is $|A| = (w^2+x^2+y^2+z^2)^{1/2}$.  It is called a \emph{pure} quaternion if $w = 0$.  We identify 3-vectors with pure quaternions, by identifying $\bi$, $\bj$, and $\bk$ with the three standard unit vectors.  A \emph{dual number} is a pair of real numbers, written as $\alpha = a + \epsilon b$, with the algebraic operation $\epsilon^2 = 0$.  A \emph{dual quaternion} is a pair of quaternions, written as $\eta = A + \epsilon B$, again with $\epsilon^2 = 0$, and $\epsilon$ commuting with all other elements.  The \emph{conjugate} of this dual quaternion is $\bar \eta = \bar A + \epsilon \bar B$.  The notion of dual quaternion goes back to Clifford \cite{clifford}.  For more information, including how they are used by the graphics card industry and in robotics, we refer the reader to \cite{adorno,agrawal,han-et-al,kavan-et-al,kavan-et-al-2,kenwright,kussaba-et-al,schilling1,schilling2,yang-et-al}.

The symbol $i$ denotes one of the square roots of $-1$, and in this paper won't be identified with $\bi$.

We say that a function $f:\Omega \to \mathbb C$ is \emph{valid} if $\Omega$ is a subset of $\mathbb C$ such that $z\in\Omega \Leftrightarrow \bar z \in \Omega$, and
\begin{equation}
f(\bar z) = \overline{f(z)} \quad \text{for $z \in \Omega$.}
\end{equation}
Note that if $f(x+iy)$ is a polynomial in $x$ and $y$, equivalently a polynomial in $z = x + i y$ and $\bar z$:
\begin{equation}
f(x+iy) = \sum_{m,n} r_{m,n} x^m (iy)^n = \sum_{m,n} s_{m,n} z^m \bar z^n.
\end{equation}
then $f$ is valid if and only if the $r_{m,n}$ are real if and only if the $s_{m,n}$ are real.

We show how to extend the definition of $f$, at least when $f$ is appropriately smooth, to dual quaternions in such a way that it is correct for polynomials.  In all of the statements of our results, when we say that a class of functions extends to a subset of dual quaternions, we mean the following standard definition of a functional calculus.
\begin{enumerate}
\item The class of functions contains $f(z) = 1$, $f(z) = z$, and if $f$ and $g$ are in this class, and $a$ is a real number, then $f+g$, $fg$, $\bar f$, and $a f$ are also in this class.
\item Given any dual quaternion $\eta$ from the prescribed subset, there is a map from the class to dual quaternions, $f \mapsto f(\eta)$, such that $f(z) = 1$ maps to $1$, $f(z) = z$ maps to $\eta$, $f+g$ maps to $f(\eta) + g(\eta)$, $fg$ maps to $f(\eta) g(\eta)$, $\bar f$ maps to $\overline{f(\eta)}$, and $a f$ maps to $a f(\eta)$.
\item \label{compact} If any topology is specified on the class of functions, then the map $\eta \mapsto f(\eta)$ is continuous.
\end{enumerate}
All of the results of this paper could be verified for polynomials simply by looking at the individual monomials, and indeed this is how Lemmas~\ref{f(d)} and~\ref{f(dq) commute} are best proved.  Another method is to verify the results if $f(z) = 1$ and $f(z) = z$, and show that the set of functions for which it is true is closed under addition, multiplication, conjugation, and multiplication by real numbers.  However, for Lemma~\ref{f(dq) anti commute}, these straightforward approaches are nevertheless quite mysterious, and we believe that our approach is more intuitive.

The class of functions, with the exception of Theorem~\ref{f(q)} will be a subset of functions that have one continuous derivative on an open neighborhood of a set $\Omega \subset \mathbb C$.  Using the following version of the Stone-Weierstrass Theorem, which may be found in \cite[Section 1.6.2]{narasimhan}, we see that it is sufficient to prove our results for polynomials.

\begin{theorem}  Let $f:\Omega \to \mathbb R$ be a $k$ times continuously differentiable function, where $\Omega$ is an open subset of $\mathbb R^n$.  Let $K$ be a compact subset of $\Omega$.  Then given $\delta > 0$, there exists a polynomial $p:\mathbb R^n \to \mathbb R$ such that
\begin{equation}
\left|\frac{\partial^{\alpha_1 + \cdots + \alpha_n} f}{\partial^{\alpha_1} x_1 \cdots \partial^{\alpha_n} x_n}(x_1,\dots,x_n) - \frac{\partial^{\alpha_1 + \cdots + \alpha_n} p}{\partial^{\alpha_1} x_1 \cdots \partial^{\alpha_n} x_n}(x_1,\dots,x_n) \right| \le \delta
\end{equation}
for all $(x_1,\dots,x_n) \in \Omega$, and non-negative integers $\alpha_1, \dots, \alpha_n$ satisfying $\alpha_1 + \cdots \alpha_n \le k$.
\end{theorem}

Note that the restriction that $p$ be valid presents no difficulty, since one can always replace $p(z)$ by $\frac12(p(z) + \overline{p(\bar z)})$, and obtain the same approximation.

We should mention that the formula for dual numbers is well known, and we cite it as Lemma~\ref{f(d)}.

We also want to mention the remarkable, and rather different, approach taken by Selig  to this problem \cite{selig}.

\bigskip

If $f$ is continuously differentiable, denote
\begin{equation}
f_x(x+iy) = \dfrac{\partial}{\partial x} f(x+iy) , \quad
f_{iy}(x+iy) = -i\dfrac{\partial}{\partial y} f(x+iy) .
\end{equation}
(Thus the Cauchy-Riemann conditions can be stated as $f$ is analytic if and only if $f_x = f_{iy}$.)  Define the following real valued, continuous, functions, which are even in $y$:
\begin{align}
\label{g}
g(x+iy) &= \dfrac{f(x+iy) + f(x-iy)} 2, \\
\label{h}
h(x+iy) &= \begin{cases}
\dfrac{f(x+iy)-f(x-iy)}{2iy} &\text{if $y \ne 0$} \\
f_{iy} (x) &\text{if $y = 0$} ,
\end{cases}
\end{align}
so that
\begin{equation}
f(x+iy) = g(x+iy) + i y h(x+iy) .
\end{equation}

First we state how to extend $f$ to the quaternions.  The proof is straightforward, because any quaternion that is a unit 3-vector behaves formally exactly like $i$ in $\mathbb C$.

\begin{theorem}
\label{f(q)}
Suppose that $f:\Omega \to \mathbb C$ is a valid function.  Then $f$ extends to quaternions
\begin{equation}
f(a_0 + \bm a_1) = \begin{cases}
g(a_0+i|\bm a_1|) +h(a_0+i|\bm a_1|) \bm a_1 & \text{if $\bm a_1 \ne 0$} \\
f(a_0) & \text{if $\bm a_1 = 0$}, \end{cases}
\end{equation}
where $a_0$ is real and $\bm a_1$ is a 3-vector, whenever $a_0 + i|\bm a_1| \in \Omega$.  Furthermore the norm is preserved in that
\begin{equation}
| f(a_0 + \bm a_1) | = | f(a + i |\bm a_1|) | .
\end{equation}
\end{theorem}

If $f$ is a valid polynomial, then a formula that extends $f$ to all dual quaternions follows from Taylor's series.  If $A$ and $B$ are quaternions, then
\begin{equation}
\label{proto f(dq)}
f(A + \epsilon B) = f(A) + \epsilon \left.\frac d{dr} f(A + rB) \right|_{r=0}.
\end{equation}
If $A$ and $B$ commute, this implies Lemma~\ref{f(dq) commute} below.  But if $A$ and $B$ do not commute, it is not immediately apparent how to use this formula.  Thus we now state the main result of this paper.

\begin{theorem}
\label{f(dq)}
Let $f:\Omega \to \mathbb C$ be a valid continuously differentiable function, where $\Omega$ is open in $\mathbb C$.  Define $h$ by equation~\eqref{h}.  Then $f$ can be extended to a continuous function on all dual quaternions as follows.  Given quaternions $A$ and $B$, decompose $A = a_0 + \bm a_1$ and $B = b_0 + \bm b_1 + \bm b_2$, where $a_0$ and $b_0$ are real, $\bm a_1$, $\bm b_1$ and $\bm b_2$ are 3-vectors, $\bm b_1$ is parallel to $\bm a_1$, and $\bm b_2$ is perpendicular to $\bm a_1$.  Write $B_1 = b_0 + \bm b_1$.  If $a_0 + i |\bm a_1| \in \Omega$, then
\begin{align}
f(A + \epsilon B)
&= g(A) 
+ \epsilon f_x(A) b_0
+ h(A) (1 + \epsilon \bm b_2)
+ \epsilon f_{iy}(A) \bm b_1 \\
&= f(A) 
+ \epsilon f_x(A) b_0
+ \epsilon f_{iy}(A) \bm b_1
+ \epsilon h(A) \bm b_2 .
\end{align}
\end{theorem}

Note that in the statement of the theorem, there is no restriction on the value of $B$.

Before proving this result, let us provide some examples.  First something simple, where we show that the formulas created by the functional calculation agree with the standard definitions given in the literature.

\begin{cor}
With the hypotheses of Theorem~\ref{f(dq)}, the conjugate and norm functions extend to dual quaternions as
\begin{align}
\overline{A + \epsilon B} &= \overline{A} + \epsilon \overline{B} ,\\
|A + \epsilon B| &= |A| + \epsilon \frac{\text{\rm Re}(A \overline B)}{|A|} \quad (A \ne 0) .
\end{align}
\end{cor}

Next, we compute the exponential function.  Formulas for the exponential and logarithm are given in \cite{wang-et-al}, but we believe ours are more explicit.  Another formula for the exponential and the logarithm is given in \cite{selig}, with a correction for the logarithm in \cite{wu-et-al}.  While their formula gives the same result for the exponential, this is not immediately obvious, and we haven't checked for the logarithm.  A formula for the exponential and logarithm is also given in \cite{han-et-al}, but we believe that their formula only works if $A$ and $B$ commute.

\begin{cor}
\label{exp dq}
With the hypotheses of Theorem~\ref{f(dq)} we have
\begin{multline}
\label{exp theta}
\exp(A + \epsilon B ) \\
= e^{a_0} \left(\left(\cos(|\bm a_1|) + \dfrac{\sin(|\bm a_1|)}{|\bm a_1|} \bm a_1 \right) \left(1 + \epsilon B_1\right) + \epsilon\frac{\sin(|\bm a_1|)}{|\bm a_1|} \bm b_2\right),
\end{multline}
where if $\bm a_1 = 0$, we set $\sin(|\bm a_1|)/|\bm a_1| = 1$.
\end{cor}

We can compute the logarithm in the same way, but then we can only get the principal value.  Instead we define a logarithm as a right inverse to the exponential function: $\exp(\log(\eta)) = \eta$.

\begin{cor}
\label{log dq}
Assume the hypotheses of Theorem~\ref{f(dq)}, with $A \ne 0$.  Let $t$ be the angle for a choice of polar coordinates for $(a_0, |\bm a_1|)$.  If $\bm a_1 \ne 0$, then a choice of $\log(A+\epsilon B)$ is
\begin{equation}
\log(|A|) + \dfrac t{|\bm a_1|} (\bm a_1 + \epsilon \bm b_2) + \epsilon \dfrac1{|A|^2} \bar A_1 B_1 .
\end{equation}
In the case $\bm a_1 = 0$, and $a_0 \ne 0$, we can assume $t = n \pi$ for some integer $n$, and $\bm b_2 = 0$.  Let $\bm p$ be any 3-vector perpendicular to $\bm b_1$ if $n \ne 0$, and $\bm p = 0$ if $n = 0$.  Then a choice of $\log(A + \epsilon B)$ is
\begin{equation}
\label{t=n pi}
\log(|a_0|) + n \pi \dfrac{\bm b_1}{|\bm b_1|} + \epsilon \dfrac 1{a_0} B_1 + \epsilon \bm p,
\end{equation}
where if $\bm b_1 = 0$, we interpret $\bm b_1 / |\bm b_1|$ as any unit 3-vector.
\end{cor}

Next we give a formula for powers.  For $\alpha \in \mathbb R$ and $z \in \mathbb C \setminus (-\infty,0]$, let $z^\alpha$ denote the principle value.

\begin{cor}
With the hypotheses of Theorem~\ref{f(dq)}, if $A \notin (-\infty,0]$, then we have
\begin{equation}
(A + \epsilon B )^\alpha =
A^\alpha + \epsilon \alpha A^{\alpha - 1} B_1 + \epsilon \frac{\text{\rm Im}(A^\alpha)}{\text{\rm Im}(A)} \bm b_2 ,
\end{equation}
where if $\text{\rm Im}(A) = 0$, we set ${\text{\rm Im}(A^\alpha)}/{\text{\rm Im}(A)}$ to $\alpha A^{\alpha - 1} $.
\end{cor}

Note that for $\alpha = -1$, it can be shown that this is equivalent to the well known formula
\begin{equation}
(A + \epsilon B)^{-1} = A^{-1} - \epsilon A^{-1} B A^{-1} \quad (A \ne 0).
\end{equation}

Finally, we reproduce Selig's formula for the Cayley Transform \cite{selig}.

\begin{cor}
With the hypotheses of Theorem~\ref{f(dq)}, if $A \ne 1$, then we have
\begin{equation}
\begin{aligned}
\frac{1 + A + \epsilon B}{1 - (A + \epsilon B)}
&=
\frac{1+A}{1-A} + \epsilon \frac{2}{(1-A)^2} \bm b_1 + \epsilon \frac{2}{|1-A|^2} \bm b_2 \\
&=
\frac{1+A}{1-A} + 2 \epsilon \frac{1}{(1-A)} B \frac{1}{(1-A)} .
\end{aligned}
\end{equation}
\end{cor}

Now we start the proof of Theorem~\ref{f(dq)}.  As stated above, these first two results come straight from equation~\eqref{proto f(dq)}.  Lemma~\ref{f(d)} is well known, and is a special case of Lemma~\ref{f(dq) commute}.

\begin{lemma}
\label{f(d)}
Let $f:\Omega \to \mathbb R$ be a continuously differentiable function, where $\Omega$ is open in $\mathbb R$.  Then $f$ extends to dual numbers $a + \epsilon b$ with $a \in \Omega$ by
\begin{equation}
\label{dual number case}
f(a + \epsilon b ) = f(a) + \epsilon f'(a) b .
\end{equation}
\end{lemma}

\begin{lemma}
\label{f(dq) commute}
Let $f:\Omega \to \mathbb C$ be a valid continuously differentiable function, where $\Omega$ is open in $\mathbb C$.  Then $f$ extends to dual quaternions $A + \epsilon B$ where $A$ and $B$ commute with
\begin{equation}
\label{commuting case}
f(A + \epsilon B ) = f(A) + \epsilon f_x(A) b_0
+ \epsilon f_{iy}(A) \bm b_1 ,
\end{equation}
where $A = a_0 + \bm a_1$, $B = b_0 + \bm b_1$, with $a_0$ and $b_0$ being real, $\bm a_1$ and $\bm b_1$ being 3-vectors, and $a_0 + i |\bm a_1| \in \Omega$.
\end{lemma}

\begin{proof}
Note that $A$ and $B$ commute if and only if $\bm a_1$ and $\bm b_1$ commute.  It is sufficient to prove this in the cases that $f(x+iy) = x^m (iy)^n$, where $m$ and $n$ are non-negative integers, when it follows by an application of the binomial theorem.
\end{proof}

\begin{lemma}
\label{f(dq) anti commute}
Let $f:\Omega \to \mathbb C$ be a valid continuously differentiable function, where $\Omega$ is an open subset of $i\mathbb R$.  Define the continuous real valued function
\begin{equation}
\label{h ident 2}
h(iy) = \begin{cases}
\dfrac{f(iy) - f(-iy)}{2iy} &\text{if $y \ne 0$} \\
f_{iy}(0) &\text{if $y = 0$} .
\end{cases}
\end{equation}
Then $f$ extends to dual quaternions of the form $\eta = \bm a + \epsilon \bm b_1 + \epsilon \bm b_2$, where $\bm a$, $\bm b_1$, and $\bm b_2$ are pure, $\bm b_1$ and $\bm a$ are parallel, $\bm b_2$ and $\bm a$ are perpendicular, and $|\bm a| \in \Omega$, by the formula
\begin{equation}
\label{f(dq) anti commute equ}
f(\eta) = f(\bm a + \epsilon \bm b_1) + \epsilon h(\bm a) \bm b_2 ,
\end{equation}
\end{lemma}

\begin{proof}  Without loss of generality, we may assume that $f$ is a polynomial.  Define the function
\begin{equation}
g(iy) = \frac{f(iy) + f(-iy)}2 .
\end{equation}
Since $g$ and $h$ are even polynomials, we can create polynomials $k$ and $l$ on an open subset of $[-\infty,0]$ by
\begin{equation}
k(z) = g(\sqrt z), \quad l(z) = h(\sqrt z) .
\end{equation}
Then we have the identities
\begin{equation}
\label{f g h k l ident}
f(z) = k(z^2) + l(z^2) z , \quad k(z^2) = g(z), \quad l(z^2) = h(z) .
\end{equation}
Let $\alpha = \bm a + \epsilon \bm b_1$, and $\beta = \epsilon\bm b_2$.  Use equation~\eqref{proto f(dq)} to replace $z$ with $\eta$ in equations~\eqref{f g h k l ident}.  Note that $\eta^2 = \alpha^2$.  Hence
\begin{equation}
\begin{aligned}
f(\eta)
&= k(\eta^2) + l(\eta^2) \eta = k(\alpha^2) + l(\alpha^2) \eta \\
&= g(\alpha) + h(\alpha) (\alpha + \beta) = f(\alpha) + h(\alpha) \beta.
\end{aligned}
\end{equation}
Finally, by Lemma~\ref{f(dq) commute}
\begin{equation}
h(\alpha) \beta = (h(\bm a) + \epsilon h_{iy}(\bm a) \bm b_1 ) \epsilon \bm b_2 = \epsilon h(\bm a) \bm b_1 ,
\end{equation}
since $\epsilon^2 = 0$.
\end{proof}

\begin{proof}[Proof of Theorem~\ref{f(dq)}]
%By Lemma~\ref{f(dq) commute}, we only need to show
%\begin{equation}
%f(A+\epsilon B) = f(A + \epsilon B_1) + \epsilon h(A) \bm b_2 .
%\end{equation}
Without loss of generality, assume $f$ is a polynomial.  Define the functions $u,v:(\Omega-a_0) \cap i\mathbb R \to \mathbb C$ by
\begin{align}
u(iy) &= f(a_0 + iy) , \\
v(iy) &= f_{x}(a_0 + iy) b_0,
\end{align}
so that by Lemma~\ref{f(d)}, applied to the real and imaginary parts of $x \mapsto f(x + iy)$, and replacing $x$ by $a_0 + \epsilon b_0$, we have
\begin{equation}
f(a_0 + \epsilon b_0 + i y) = u(iy) + \epsilon v(iy) .
\end{equation}
Applying Lemmas~\ref{f(dq) commute} and~\ref{f(dq) anti commute} to $u$, and setting
\begin{equation}
h_u(iy) = h(a_0 + iy),
\end{equation}
we obtain
\begin{equation}
\begin{aligned}
u(\bm a_1 + \epsilon \bm b_1 + \epsilon \bm b_2) &= u(\bm a_1 + \epsilon \bm b_1) + \epsilon h_u(\bm a_1) \bm b_2 \\
&= u(\bm a_1) + \epsilon u_{iy}(\bm a_1) \bm b_1 + \epsilon h_u(\bm a_0) \bm b_2 \\
&= f(A) + \epsilon f_{iy}(A) \bm b_1 + \epsilon h(A) \bm b_2 .
\end{aligned}
\end{equation}
We obtain a simular equation for $v$, but since $\epsilon^2 = 0$, this simplifies to
\begin{equation}
\epsilon v(\bm a_0 + \epsilon \bm b_0 + \epsilon \bm b_1) = \epsilon v(\bm a_0) = \epsilon f_x(A) b_0 .
\end{equation}
(Here a second derivative of $f$, which appears when initially applying Lemma~\ref{f(dq) commute} to $v$, disappears in the final result.)  The result follows by combining the last two equations.
\end{proof}

During the writing of this paper, we came across a different approach to proving Theorem~\ref{f(dq)} motivated by Chasles' Theorem and the approach taken in \cite{wang-et-al}.  It can be shown that any dual quaternion may be factored as
\begin{equation}
a + \bm a_1 + \epsilon (b + \bm b_1) = (1 + \epsilon \bm r) (a + \bm a_1 + \epsilon \tilde B) (1 - \epsilon \bm r),
\end{equation}
where $\tilde B$ commutes with $\bm a_1$, and
\begin{equation}
\bm r = \begin{cases} \frac{\bm b_1 \bm a_1 - \bm a_1 \bm b_1 }{4|\bm a_1|^2} & \text{if $\bm a_1 \ne 0$} \\ 0 & \text{if $\bm a_1 = 0$} \end{cases}
\end{equation}
is a 3-vector.  (A similar formula is also in \cite{selig}).  Also, if $f:\Omega \to \mathbb C$ is a valid polynomial, then
\begin{equation}
f\bigl((1 + \epsilon \bm r) (A + \epsilon B) (1 - \epsilon \bm r)\bigr) = (1 + \epsilon \bm r) f(A + \epsilon B) (1 - \epsilon \bm r) .
\end{equation}
In this way, the computation is reduced to Lemma~\ref{f(dq) commute}.

\bigskip

Finally, we note that the methods of the proof of Lemma~\ref{f(dq) anti commute} give a cute, albeit simple, result for the Dunford-Riesz functional calculus of analytic functions on a complex Banach algebra with unity $\mathcal A$.  Recall from \cite{dunford-et-al}, that given $A \in \mathcal A$, and $\Omega$ an open subset of $\mathbb C$ containing the spectrum of $A$, that the Dunford-Riesz functional calculus maps analytic functions $f : \Omega \to \mathbb C$ to $f(A) \in \mathcal A$, and satisifes
\begin{enumerate}
\item $f(z) = 1$ maps to the unit of $\mathcal A$, $f(z) = z$ maps to $A$, $f+g$ maps to $f(A) + g(A)$, $fg$ maps to $f(A) g(A)$, and $a f$ maps to $a f(A)$.
\item if $f_n \to f$ uniformly on the spectrum of $A$, then $f_n(A) \to f(A)$ in norm.
\end{enumerate}

\begin{prop}
\label{analytic}
Suppose $f:\Omega \to \mathbb C$ is an analytic function on an open subset $\Omega$ of $\mathbb C$ satisfying $z \in \Omega \Leftrightarrow -z \in \Omega$.  Define the analytic functions on $\{z^2 : z \in \Omega\}$ by
\begin{align}
\label{g a}
k(z) &= \dfrac{f(\sqrt z) + f(-\sqrt z)}{2} ,\\
\label{h a}
l(z) &= \begin{cases}
\dfrac{f(\sqrt z) - f(-\sqrt z)}{2\sqrt z} &\text{if $z \ne 0$} \\
f'(0) &\text{if $z = 0$} ,
\end{cases} \\
\end{align}
If $A$ and $B$ are anti-commuting elements of a complex algebra such that their spectra $\sigma(A) \cup \sigma(B) \cup \sigma(A+B) \subset \Omega$, then
\begin{equation}
f(A+B) = k(A^2+B^2) + (A+B) l(A^2+B^2) .
\end{equation}
\end{prop}

Applying this to $(A+B)^n$, where $n$ is a non-negative integer, gives the coefficients of the so called Pauli-Pascal triangle \cite{horn,sloane}, that is, if $A$ and $B$ anti-commute, and $n$ is a non-negative integer, then
\begin{equation}
(A + B)^n = \sum_{k=0}^n  \genfrac\{\}{0pt}0nk  A^{n-k} B^k ,
\end{equation}
where we define for non-negative integers $n$ and $k$:
\begin{equation}
\genfrac\{\}{0pt}0{2n}{2 k} = \genfrac\{\}{0pt}0{2n + 1}{2 k} =
\genfrac\{\}{0pt}0{2n + 1}{2 k + 1} = \binom n k , \quad
\genfrac\{\}{0pt}0{2n}{2 k + 1}  = 0.
\end{equation}

%\section*{Statements and Declarations}

%The author has no relevant financial or non-financial interests to disclose.  The author has no competing interests to declare that are relevant to the content of this article.  The authors certify that they have no affiliations with or involvement in any organization or entity with any financial interest or non-financial interest in the subject matter or materials discussed in this manuscript, except that this is related to consulting work performed by the author for the NASA Johnson Space Center, Houston, Texas, USA.  The author has no financial or proprietary interests in any material discussed in this article.

There is no additional data associated with this paper.  No funding was provided for this paper.

\begin{thebibliography}{99}

\bibitem{adorno} Bruno Vilhena Adorno, Robot Kinematic Modeling and Control Based on Dual Quaternion Algebra -- Part I: Fundamentals, 2017, hal-01478225.

\bibitem{agrawal} Om Prakash Agrawal. Hamilton operators and dual-number-quaternions in spatial kinematics. Mechanism and Machine Theory, 22(6):569-575, Jan 1987.

\bibitem{clifford} M.A. Clifford, Preliminary Sketch of Biquaternions, Proceedings of the London Mathematical Society, Volume s1-4, Issue 1, November 1871, Pages 381-395, \url{https://doi.org/10.1112/plms/s1-4.1.381}.

\bibitem{dunford-et-al} N. Dunford and J.T. Schwartz, Linear Operators, Part I: General Theory, Interscience, 1958.

\bibitem{han-et-al} Da-Peng Han, Qing Wei, and Ze-Xiang Li, Kinematic Control of Free Rigid Bodies Using Dual Quaternions, International Journal of Automation and Computing
05(3), July 2008, 319-324, DOI: 10.1007/s11633-008-0319-1.

\bibitem{horn} M.E. Horn, The Didactical Relevance of the Pauli Pascal Triangle, arXiv:physics/0611277 [physics.ed-ph], 2006.

\bibitem{kavan-et-al} L. Kavan, S. Collins, J. \u Z\'ara, C. O'Sullivan, Skinning with Dual Quaternions, \url{https://dl.acm.org/doi/pdf/10.1145/1230100.1230107}.

\bibitem{kavan-et-al-2} L. Kavan, S. Collins, J. \u Z\'ara, C. O'Sullivan, Geometric Skinning with Approximate Dual Quaternion Blending, ACM Transactions on Graphics, Vol. 27, No. 4, Article 105, Publication date: October 2008.

\bibitem{kenwright} Ben Kenwright, A Beginners Guide to Dual-Quaternions, What They Are, How They Work, and How to Use Them for 3D Character Hierarchies, \url{https://cs.gmu.edu/~jmlien/teaching/cs451/uploads/Main/dual-quaternion.pdf}.

\bibitem{kussaba-et-al} Hugo T.M. Kussaba, Luis F.C. Figueredo, Jo\~ao Y. Ishihara, and Bruno V. Adorno, Hybrid kinematic control for rigid body pose stabilization using dual quaternions, Journal of the Franklin Institute, 354(7):2769-2787, May 2017.

\bibitem{narasimhan} R. Narasimhan, Analysis on Real and Complex Manifolds, 2nd Edition, 1985, North-Holland Mathematical Library.

\bibitem{schilling1}  M. Schilling, Universally manipulable body models---dual quaternion representations in layered and dynamic MMCs, Auton Robot 30, 399 (2011), \url{https://doi.org/10.1007/s10514-011-9226-3 https://link.springer.com/article/10.1007/s10514-011-9226-3}.

\bibitem{schilling2} M. Schilling, Hierarchical Dual Quaternion-Based Recurrent Neural Network as a Flexible Internal Body Model, 2019 International Joint Conference on Neural Networks (IJCNN), 2019, pp. 1-8, doi: 10.1109/IJCNN.2019.8852328, \url{https://ieeexplore.ieee.org/abstract/document/8852328}.

\bibitem{selig} J.M. Selig, Exponential and Cayley Maps for Dual Quaternions, Advances in Applied Clifford Algebras, 20(3-4):923-936, May 2010.

\bibitem{sloane} N.J.A. Sloane, Online Encyplopedia of Integer Sequences, A051159, Triangular array made of three copies of Pascal's triangle, \url{https://oeis.org/A051159}.

\bibitem{wang-et-al} Xiangke Wang, Dapeng Han, Changbin Yu, and Zhiqiang Zheng, The geometric structure of unit dual quaternions with application in kinematic control, Journal of Mathematical Analysis and Applications 389(2), 2012, 1352-1364.

\bibitem{wu-et-al} Yuanqing Wu, J.M. Selig and Marco Carricato,  (2019) Parallel
Robots with Homokinetic Joints: The Zero-Torsion Case, In: Uhl T. (eds) ``Advances in Mechanism and Machine Science.'' IFToMM WC 2019. Mechanisms and Machine Science, vol 73. Springer, Cham, pp. 269-278.

\bibitem{yang-et-al} XiaoLong Yang, HongTao Wu, Yao Li, Bai Chen, A dual quaternion solution to the forward kinematics of a class of six-DOF parallel robots with full or reductant actuation, Mechanism and Machine Theory 107 (2017) 27-36, \url{http://dx.doi.org/10.1016/j.mechmachtheory.2016.08.003}.
\end{thebibliography}

\end{document}

