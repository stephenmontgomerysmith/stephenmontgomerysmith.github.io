\documentstyle[12pt]{amsart}

%%%%%%%%% THEOREMS %%%%%%%%%%%%%%%%%%%%%%%%%%%%%%%%%%%%%%
%%-default--- \theoremstyle{plain}
\newtheorem{prop}{Proposition}[section]
\newtheorem{thm}[prop]{Theorem}
\newtheorem{hyp}[prop]{Hypothesis}
\newtheorem{cor}[prop]{Corollary}
\newtheorem{ques}[prop]{Question}
\newtheorem{lem}[prop]{Lemma}
\newtheorem{res}[prop]{Result}
\newtheorem{prob}[prop]{Problem}

\newtheorem{introSMT}{Spectral Mapping Theorem}
 \renewcommand{\theintroSMT}{}
\newtheorem{introSPT}{Spectral Projection Theorem}
 \renewcommand{\theintroSPT}{}
\newtheorem{introPDT}{Pointwise Dichotomy Theorem}
 \renewcommand{\theintroPDT}{}

%%
\theoremstyle{definition}
\newtheorem{defn}{Definition}[section]

%%
\theoremstyle{remark}
\newtheorem{claimone}{Claim 1} \renewcommand{\theclaimone}{}
\newtheorem{claimtwo}{Claim 2} \renewcommand{\theclaimtwo}{}
\newtheorem{claim}{Claim}\renewcommand{\theclaim}{}
\newtheorem{casekone}{Case $k=1$}\renewcommand{\thecasekone}{}
\newtheorem{casek}{Case $k>1$}\renewcommand{\thecasek}{}

\newtheorem{conj}{Conjecture}  \renewcommand{\theconj}{}
\newtheorem{exmp}[prop]{Example}  %\renewcommand{\theexmp}{}
\newtheorem{rems}[prop]{Remarks}  \renewcommand{\therems}{}
\newtheorem{rem}{Remark} %\renewcommand{\therem}{}
\newtheorem{ack}{Acknowledgments} \renewcommand{\theack}{}

%%%%%%%%%%%%%% Abbrev. for fonts %%%%%%%%%%%%%%%%%%%%%%%
\newcommand{\bbC}{{\Bbb{C}}}
\newcommand{\bbD}{{\Bbb{D}}}
\newcommand{\bbN}{{\Bbb{N}}}
\newcommand{\bbR}{{\Bbb{R}}}
\newcommand{\bbT}{{\Bbb{T}}}
\newcommand{\bbZ}{{\Bbb{Z}}}
\newcommand{\bbK}{{\Bbb{K}}}

\newcommand{\calP}{{\cal{P}}}
\newcommand{\calQ}{{\cal{Q}}}
\newcommand{\calM}{{\cal{M}}}
\newcommand{\calB}{{\cal{B}}}
\newcommand{\calD}{{\cal{D}}}
\newcommand{\calH}{{\cal{H}}}
\newcommand{\calE}{{\cal{E}}}
\newcommand{\calA}{{\cal{A}}}
\newcommand{\calL}{{\cal{L}}}
\newcommand{\calX}{{\cal{X}}}

\newcommand{\frakA}{{\frak{A}}}
\newcommand{\frakB}{{\frak{B}}}
\newcommand{\frakC}{{\frak{C}}}
\newcommand{\frakD}{{\frak{D}}}

%%%%%%%%%%%%%%% operatornames %%%%%%%%%%%%%%%%%%%%%%
\newcommand{\supp}{\operatorname{supp}}
\renewcommand{\mod}{\operatorname{mod}}
\renewcommand{\Im}{\operatorname{Im}}
\newcommand{\diag}{\operatorname{diag}}
\newcommand{\mes}{\operatorname{mes}}
\newcommand{\Ker}{\operatorname{Ker}}
\newcommand{\esssup}{\operatornamewithlimits{ess\, sup}}
\renewcommand{\max}{\operatornamewithlimits{max}}
\newcommand{\Res}{\operatornamewithlimits{Res}}
\renewcommand{\Re}{\operatorname{Re}}
\newcommand{\loc}{{\operatornamewithlimits{loc}}}
\newcommand{\dist}{{\operatornamewithlimits{dist}}}
\newcommand{\dint}{\displaystyle\int}

%%%%%%%%%%%%%%% GREEK %%%%%%%%%%%%%%%%%%%%%%%%%%%%%%
\newcommand{\TH}{\Theta}
\newcommand{\th}{\theta}
\newcommand{\lam}{\lambda}
\newcommand{\sig}{\sigma}
\newcommand{\eps}{\epsilon}
\newcommand{\gam}{\gamma}
\newcommand{\Gam}{\Gamma}
\newcommand{\vphi}{\varphi}
\newcommand{\Lam}{\Lambda}
\newcommand{\bd}{\text{{\bf d}}}

%%%%%%%%%%%%%%%% other abbreviations %%%%%%%%%%%%%%%%
\newcommand{\scr}{\scriptstyle}
\newcommand{\disp}{\displaystyle}
\newcommand{\lb}{\label}
\newcommand{\ti}{\tilde}
\newcommand{\no}{\nonumber}
\newcommand{\bi}[1]{\bibitem[#1]{#1}}
%\newcommand{\bi}{\bibitem}
\newcommand{\bs}{\backslash}
\newcommand{\hp}{{\hat{\vphi}}}

\newcommand{\co}{C_0(\Theta,X)}
\newcommand{\coR}{C_0(\bbR,X)}
\newcommand{\Bco}{B(C_0(\Theta,X))}
\newcommand{\linf}{\ell_\infty(\bbZ,X)}
\newcommand{\Blinf}{B(\ell_\infty(\bbZ,X))}

\newcommand{\sumk}{\sum_{k=-\infty}^\infty}
\newcommand{\sumj}{\sum_{j=-\infty}^\infty}

%%%%%%%%%%%%%% NUMBERING %%%%%%%%%%%%%%%%%%%

%\setcounter{page}{0}
\setcounter{section}{0}

\renewcommand{\theequation}{\thesection.\arabic{equation}}
\renewcommand{\thesection}{\arabic{section}}
\renewcommand{\theprop}{\arabic{section}.\arabic{prop}}


%------------------- article  ----------------
\begin{document}

\baselineskip=16pt

%\pagestyle{empty}
%\renewcommand{\thepage}{{}}
\title[EVOLUTIONARY SEMIGROUPS AND\\
DICHOTOMY OF LSPF]{EVOLUTIONARY SEMIGROUPS AND\\ DICHOTOMY OF
LINEAR SKEW-PRODUCT FLOWS\\ ON LOCALLY COMPACT SPACES WITH BANACH
FIBERS}
\makeatletter
\author{Y.~Latushkin}
\address{ Department of  Mathematics\\
University of Missouri\\ Columbia, MO
65211}
\email{mathyl@mizzou1.missouri.edu}
\thanks{
Supported by
the National Science
 Foundation under the grant  DMS-9400518 and by the Summer Research
Fellowship of the University of Missouri.}
\author{S.~Montgomery-Smith}
\address{ Department of  Mathematics\\
University of Missouri\\ Columbia, MO 65211}
\email{stephen@mont.cs.missouri.edu}
\thanks{Supported by
the National Science
 Foundation under the grant  DMS-9201357.}
\author{T.~Randolph}
\address{ Department of  Mathematics\\
University of Missouri\\ Rolla, MO 65401}
\email{randolph@umrvmb.umr.edu}
\makeatother
\thanks{Supported by the University of Missouri Research Board.}

\keywords {Evolutionary semigroups, spectral
mapping theorems, linear skew-product flows,
exponential dichotomy}


\subjclass {47D06, 34G10, 34C35}
\maketitle
\begin{abstract}
We study evolutionary semigroups generated by a strongly
continuous semi-cocycle over a locally compact metric space
acting on Banach fibers. This setting simultaneously covers
evolutionary semigroups arising from nonautonomuous
abstract Cauchy problems and $C_0$-semigroups, and linear
skew-product flows.

The spectral mapping theorem for these semigroups is proved. The
hyperbolicity of the semigroup is related to the exponential dichotomy
of the corresponding linear skew-product flow. To this end a Banach
algebra of weighted composition operators is studied. The results
are applied in the study of: ``roughness'' of
the dichotomy, dichotomy and solutions of
nonhomogeneous equations,
Green's function for a linear skew-product flow,
``pointwise'' dichotomy versus ``global'' dichotomy, and
evolutionary semigroups along trajectories of the flow.
\end{abstract}







%------------------- section 1 ---------------
\section{Introduction}

The spectral theory of linear skew-product flows (or processes) with
finite dimensional fibers is, by now, a well-developed area in
asymptotics theory of differential equations (see, e.g.,
\cite{Hale,HL,JPS,Palm2,SSDich,SSSpT,Sam,Sel}). J.~Hale in
\cite[p.60]{Hale} stressed that this theory should be extended to
the infinite dimensional setting. Indeed, in recent years
significant progress has been made in the study of linear
skew-product flows (LSPFs) with Banach fibers
(\cite{ChLe1,ChLe2,ChLe3,Mag,SSBan}) over a compact metric space
(see also \cite{ChLinLu,ChLu,Lin}). An impressive list of possible
applications can be found in \cite{ChLe1,Hale,HL,Henry,SSBan}.

In this paper we consider strongly continuous linear skew-product
semiflows on bundles with Banach fibers over a locally compact
metric space.  Our philosophy is {\em not} to start with a
pointwise construction of stable and unstable foliations (as in
\cite{Sel,ChLe1,ChLe2,Mag,SSDich,SSBan}), but
instead to begin by associating with the linear skew-product
flow an {\it evolutionary semigroup}
of operators, $\{T^t\}_{t\ge0}$, on the space of continuous
sections of the bundle. We prove two  facts
concerning the spectrum of this semigroup
that relate its spectral properties to the asymptotic properties of
the linear skew-product flow.  The first of these
facts is the Spectral Mapping Theorem (Theorem \ref{SMT}) for
$\{T^t\}_{t\ge0}$ and its generator $\Gamma$,
\[ e^{t\sigma(\Gamma)}=\sigma(T^t)\bs \{0\},\quad t>0,\]
while the second concerns the description of the
spectral projections (Theorem \ref{SPT}).

As a result of this approach, we not only answer questions
concerning LSPFs, but also address the
theory of evolutionary semigroups associated with strongly
continuous evolutionary families as studied in the theory
of nonautonomous Cauchy problems and the general theory of
$C_0$ semigroups (see \cite{Evans,Gold,Howland,Lum,Nag,Nei,Rab}).
Consequently,  classical theorems on the
exponential dichotomy  of differential equations
(see, e.g., \cite{DK,MS}) can easily be extended to the case
concerning differential equations with unbounded coefficients.
Some of the results appearing here were announced in
\cite{LMS2}.

The connection between evolutionary semigroups and the dichotomy
of LSPFs is not new. In one of the first papers in this
direction, J.~Mather \cite{Mather} proved that the LSPF
generated by the differential of a diffeomorphism of
a smooth manifold is Anosov
(i.e., is hyperbolic or, in the terminology of the
present paper, exponentially dichotomic) if and only if
the corresponding evolutionary operator is hyperbolic (it's
spectrum does not intersect the unit circle).
This led to the notion of the Mather
spectrum $\calM:=\sigma(T)$.  Here $T=T^1$ and  $\sigma(\cdot)$
denotes the usual spectrum of an operator.
This notion is widely used in the
theory of hyperbolic dynamical systems (see \cite{Pesin} for
further references). The articles \cite{CS,J} proved the spectral
mapping theorem for evolutionary semigroups in the finite
dimensional setting, and they described the spectral subbundles of
these operators via the spectral subbundles of the corresponding LSPF.
Moreover, they related the spectrum of $\Gamma$ to
the dynamical spectrum, $\Sigma$, of the LSPF, as defined by
R.~Sacker and G.~Sell in $\cite{SSSpT}$.

In the present paper we proceed in the opposite direction:
we derive the existence of spectral subbundles from
the existence, and description, of the Riesz projections
corresponding to $T$; the properties of $\Sigma$ are described
via the spectrum of the generator, $\sigma(\Gamma)$.  These techniques
build upon those used in \cite{Ant} which
addresses the finite dimensional setting and uses
some $C^*$-algebraic methods. These methods were also used in \cite{LS}
to prove the results  for a uniformly continuous cocycle
on a Hilbert space.  The results have recently been
generalized in \cite{Rau} to the case involving a
strongly continuous cocycle on a compact metric space
with Banach fibers.  The important articles
\cite{Rau,Rau1,Rau2} linked this theory with the theory of
evolutionary semigroups associated with strongly continuous
evolutionary families (as initiated and developed in
\cite{Evans,Gold,Howland,Lum,Nei}).
As we will see below,
the latter situation corresponds to a LSPF over
the translations of $\bbR$; the evolutionary family can be thought
of as the propagator of a well-posed differential equation on a
Banach space \cite{Nag,Nag2,Tan}.
For this situation, in the Banach space setting,
the Spectral Mapping Theorem has been proven in \cite{LMS1,LMS2}
(see \cite{Henry1},\cite{Nag} and \cite{Pazy} for a general discussion
on this theorem)  and the Spectral Projection Theorem in
\cite{LMS1,LMS2,LatRand,Rab,Rau1,Rau2}.

The Spectral Mapping Theorem appearing in the present paper applies to
{\em any} evolutionary semigroup associated with a
LSPF over a nonperiodic flow
on a locally
compact space, $\TH$, but it also provides a new proof for semigroups
arising from evolutionary families, i.e., when $\TH=\bbR$ and the flow
is translation on $\bbR$.  In fact, the same idea applies to
evolutionary semigroups in the space of divergence-free sections of a
bundle---a situation which is important for applications
to hydromagnetodynamics \cite{CLMS}.
Our proof of the Spectral Projection
Theorem for LSPFs also differs from that in \cite{Rau}.
Here, we extend the algebraic technique used in
\cite{Ant,LS} (for Hilbert spaces) so that it
applies in the general setting of Banach spaces.
One advantage to this approach is that
we are able to discuss the case of ``discrete'' time
$t\in\bbZ$ and ``pointwise dichotomies'' (see Section~5;
cf.~\cite{ChLe1}).

We now make the previous discussion more explicit by
establishing some notation and outlining the main results.

For $t\in\bbR$, let $\vphi^t$ be a continuous flow on
a locally compact metric space  $\TH$.
Throughout the paper, the aperiodic trajectories of
the flow are assumed to be
dense in $\TH$.  Let $X$ be a Banach space and let
$\calL_s(X)$ be the space of bounded linear operators on
$X$  endowed strong-operator topology.
Let $\Phi : \TH\times\bbR_+\to \calL_s(X)$ be a  continuous
(semi)cocycle  over $\vphi^t$.  Since $\TH$ is locally compact, we
will allow $\Phi$ to grow at infinity not faster then exponentially. The
linear skew-product (semi)flow $\hp^t$ is defined by the
following formula (see Section 2)
 \[ \hp^t :\TH\times X\to \TH\times X,
\quad  (\th,x)\mapsto (\vphi^t\th,\Phi(\th,t)x).\]

\begin{exmp}\lb{ex1.1} As a very particular example, which
will be referred as the ``norm continuous compact
setting,'' assume $\TH$ is compact
and $A\colon \TH\to \calL(X)$ is a continuous
operator-valued function. Then, for each $\th\in\TH$, the operator
 $\Phi(\th,t)$
can be thought of as a solving operator for the variational equation
\begin{equation}\label{vareq}
\dfrac{dx}{dt}=A(\vphi^t\th)x(t),
\quad \th\in \TH,\; t\in\bbR.
\end{equation}
Here, $\Phi$ is uniformly continuous and takes
invertible values in $\calL(X)$.
\end{exmp}

\begin{exmp}\lb{ex1.2}
As another example  (see also \cite{ChLe3}), which will be referred to
as the ``norm continuous line setting," assume $\TH=\bbR$,  and let
$A\colon \bbR\to \calL(X)$ be a
bounded, continuous operator-valued function.
Let $\{U(\tau,s)\}_{\tau\ge s}$ be the propagator of the
nonautonomous differential equation
\begin{equation}\label{difeq}
\dfrac{dx}{dt}=A(t)x(t),
\quad t\in\bbR;
\end{equation}
that is, the solution $x(\cdot)$ of \eqref{difeq} satisfies
$x(\tau)=U(\tau,s)x(s)$.  This can be related to the previous example
by defining $\Phi(\th ,t)=U(\th +t,\th)$ and identifying the flow as
translation, $\vphi^t\th=\th+t$ for $\th\in \TH=\bbR$ and $t\in\bbR$.
Since the propagator $U$ satisfies \cite{DK} the
identity $U(\tau,s)= U(\tau,r)U(r,s)$, for  $\tau\geq r\geq s$,
and $U(s,s)=I$, $\Phi$ is a cocycle over $\vphi^t$.
\end{exmp}

These two examples show
(see also \cite{ChLe3}) how both the
variational equation \eqref{vareq} and the nonautonomous
equation \eqref{difeq} can be addressed in terms of
linear skew-product flows on a locally compact
metric space.  We stress that the
{\it strongly} continuous setting considered in the present paper
(as opposed to the norm continuous situation described
above) allows one to study the equations
 \eqref{vareq} and \eqref{difeq}  even in the situation where the operators
given by $A(\cdot)$ are {\em un}bounded.

Given a LSPF $\hp^t$, we associate to it a semigroup
$\{T^t\}_{t\ge0}$ of {\it evolutionary operators} defined
on $\co$, the space of strongly continuous functions vanishing
at infinity (with the $\sup$-norm), by
\[(T^tf)(\th )=\Phi(\vphi^{-t}\th ,t)f(\vphi^{-t}\th ),\quad \th\in \TH,
\quad f\in\co.\]  The generator of such a semigroup will
be denoted by $\Gamma$.
For the line setting this semigroup becomes a well-known
\cite{Evans,Gold,Howland,Lum,Nag2,Nei,Rau1,Rau2} evolutionary
semigroup:
\[(T^tf)(\tau)=U(\tau,\tau-t)f(\tau-t), \quad
\tau\in\bbR,\quad f\in C_0(\bbR,X).\]

The generator, $\Gamma$, of the evolutionary semigroups
arising in the norm continuous setting of examples
\ref{ex1.1} and \ref{ex1.2} can be expressed, respectively, as
$$(\Gamma f)(\th )
=-\left.\dfrac{d}{dt}f\circ\vphi^t(\th)\right|_{t=0}+A(\th)f(\th),
\text{  and  }\; (\Gamma f)(\tau)
=-\dfrac{df}{d\tau}+A(\tau)f(\tau).$$
A discussion on the delicate question about when these formulas hold
for the strongly continuous setting can be found in
\cite{Nag2,NagRh}. If, for the line setting, $A_0\equiv A(\tau)$
generates a $C_0$ semigroup $\{e^{tA_0}\}_{t\ge0}$, then
$(T^tf)(\tau)=e^{tA_0}f(\tau-t)$ and $\Gamma$ is the closure
of $-d/dt+A_0$. If $\Phi$ is the differential of a diffeomorphism on
a smooth manifold $\TH$, then $T^t$ is the ``push-forward'' operator
and $\Gamma$ is the Lie derivative.

In Section~3, for the strongly continuous locally compact
setting, we prove the following theorem.

\begin{introSMT}
The spectrum $\sigma(\Gamma)$ is invariant with respect to
translations along the imaginary axis, and the spectrum
$\sigma(T^t)$, $t>0$, is invariant with respect to
rotations centered at origin. Moreover,
$\sigma(T^t)\bs \{0\}= e^{t\sigma(\Gamma)}$.
\end{introSMT}

The proof of this theorem develops some ideas from
\cite{Mather} on ``localization'' of almost-eigenfunctions for
evolutionary operators.

In Section~4 we discuss exponentially dichotomic LSPFs.
The terminology ``exponential dichotomy," as in \cite{ChLe1} (see also
\cite{HL,Henry,Mag,SSBan}), refers to the existence of a strongly
continuous, bounded projection-valued function $P\colon \TH\to
\calL_s(X)$. This $P(\cdot)$ defines a
$\hp^t$-invariant splitting
$\TH\times X=\calX_P+\calX_Q$ with uniform exponential decay of
$\Phi_P$ on $\calX_P$ and of $\Phi_Q^{-1}$ on $\calX_Q$, $Q:=I-P$.
The  subscripts here denote the restrictions of the LSPF; since we
do not assume that $\Phi$ is invertible, the existence of
$\Phi_Q^{-1}$ is part of the definition of exponential dichotomy.

The main result in Section~4 is the following theorem where
 an operator $T$ is called {\em hyperbolic} if
$\sigma(T)\cap\bbT=\emptyset$. If $T$ is hyperbolic,
its Riesz projection corresponding to
$\sigma(T)\cap\bbD$ is denoted by $\calP$.
(Here, $\bbT$ and $\bbD$ denote the unit circle and unit disk in
$\bbC$.)

\begin{introSPT}
Assume $T$ is hyperbolic.
Then its Riesz projection, $\calP$, is an operator of
multiplication. That is, if $f\in\co$, then $(\calP
f)(\th)=P(\th)f(\th)$ for some strongly continuous bounded
projection-valued function $P(\cdot)\colon \TH\to \calL_s(X)$.
\end{introSPT}

We use this theorem to show that the LSPF $\hp^t$ has exponential
dichotomy with $P(\cdot)$ if and only if $T$ is hyperbolic, and we
relate $P(\cdot)$ with the Riesz projection for $T$, as above.
The related result for compact $\TH$ was proved in \cite{Rau} (see
also \cite{Rab,Rau1,Rau2} for $\TH=\bbR$), where a quite
different method is used.  We stress that our method also
works for ``discrete'' time $t\in\bbZ$, that is, when $\vphi$
is a homeomorphism, and $\Phi:\TH\times\bbZ\to \calL_s(X)$. For
``continuous'' time, $t\in\bbR$, together with the Spectral Mapping
Theorem, this shows that the LSPF $\hp^t$ has exponential dichotomy
with $P(\cdot)$ if and only if $\Gamma$ is invertible. This
generalizes some results in \cite{Gohb,Palm}. In particular (see
\cite{LMS1,LMS2}), for $\TH=\bbR$ and $A(\tau)\equiv A_0$ this shows
that the growth bound for any $C_0$ semigroup $\{e^{tA_0}\}_{t\ge0}$
on $X$ coincides with the spectral bound for $-d/dt+A_0$ on
$C_0(\bbR,X)$. A similar fact (see \cite{LMS1})
for $L_p$-spaces can be used in a
variety of contexts (cf. \cite{Mont,We}).

Our method of proof exploits the invertibility properties of a
particular Banach algebra of weighted translation operators on $\co$.
To prove the Spectral Projection Theorem, we consider for each $\th\in
\TH$ a weighted shift  operator, $\pi_\th(T)$, acting  on the space
$\ell_\infty(\bbZ,X)$ and given by the diagonal matrix:
\[ \pi_\th(T)=\diag \{\Phi(\vphi^{n-1}\th,1)\}_{n\in\bbZ}S;\]
here, $S$ denotes the shift operator on $\linf$:
$(x_n)_{n\in\bbZ}\mapsto (x_{n-1})_{n\in\bbZ}$.
The hyperbolicity of $\pi_\th(T)$ is equivalent to
the existence of the discrete exponential dichotomy for $\hp^t$ along
the trajectory through $\th$. These operators were, in fact, introduced
in \cite{Henry} for the line setting, and in
\cite{Ant,ChLe1,LS} for the
 compact setting.  A byproduct of our approach is the
following theorem.

\begin{introPDT} The LSPF $\hp^t$ has
exponential dichotomy on $\TH$ if and only if $\pi_\th(T)$ is
hyperbolic for every $\th\in \TH$, and
$\sup\{\|(\lambda-\pi_\th(T))^{-1}\|: \lambda\in\bbT,\th\in \TH\}
<\infty.$
\end{introPDT}

The theorems listed above help to unify a number of
ideas concerning ``spectral properties" of dynamical systems.
Recall that the dynamical spectrum, $\Sigma$, for the LSPF $\hp^t$
(see \cite{SSSpT}) is the set of all $\lambda\in\bbR$ such that the
LSPF, $\hp^t_\lambda$, corresponding to the cocycle $e^{-\lambda
t}\Phi(\th,t)$ does not have exponential dichotomy.
The Bohl spectrum, $\calB$, for equation \eqref{vareq}
is the set of $\lambda\in\bbR$ such that the equation
$x'=[A(\tau)-\lam]x$ does not have exponential dichotomy
(see \cite{DK}).
Using the above theorems, one can relate all of these sets
(in the general strongly continuous, locally compact setting) as
follows:
\begin{equation}\label{spectra}
\Sigma=\calB=\ln \calM=\sigma
(\Gamma)\cap\bbR=\ln|\sigma(T)\setminus\{0\}|.
\end{equation}

Finally, consider an autonomous equation $x'(t)=A_0x(t)$,
$t\in\bbR$, where $A_0$  generates a $C_0$ semigroup,
$\{e^{tA_0}\}_{t\ge0}$. Assume, in addition, that the spectral
mapping theorem is valid for $\{e^{tA_0}\}_{t\ge0}$. This happens,
e.g., if $A_0$ is a sectorial operator (see \cite{Nag} for a detailed
discussion). Then each of the sets in \eqref{spectra} coincides with
$\{z+i\xi: z\in\sigma (A_0), \xi\in\bbR\}\cap\bbR$. If $X$ is a
Hilbert space and $A_0$ is self-adjoint, then the last set is just
$\sigma(A_0)$.  This shows that for abstract Cauchy problems for
nonautonomous differential equations and for LSPFs in Banach spaces,
the spectrum of $\Gamma$  should play the same role as the
spectrum of $A_0$ for autonomous equations.

In Section~5 we give several consequences of the results
mentioned above. The list of these topics
includes: ``roughness'' of
the dichotomy, dichotomy and solutions of
nonhomogenious equations \eqref{vareq}--\eqref{difeq},
Green's function for a LSPF,
``pointwise'' dichotomy versus ``global'' dichotomy, and
evolutionary semigroups along trajectories of the flow.
The paper concludes by listing several open problems.

The following notation is used throughout the paper:
$\TH$ denotes a locally compact metric space; for $\th\in\TH$ and
$\delta>0$, $B(\th,\delta)$ denotes an open ball of radius $\delta$
centered at $\th$.  For any Banach space $X$,
$\calL(X)$ denotes the space of bounded linear operators on $X$, and
$\calL_s(X)$ denotes this space endowed with the strong-operator
topology.  For any operator $A$ on $X$, its domain is denoted by
$\calD(A)$ and its spectrum  by $\sigma(A)$.  In  $\bbC$, let $\bbT=\{z
:|z|=1\}$, and $\bbD=\{z: |z|\le 1\}$. For a set of operators
$d^{(n)}$, $n\in\bbZ$, in $\calL(X)$,  $\diag\{d^{(n)}\}_{n\in\bbZ}$
acts on $\linf$ as the infinite matrix consisting of diagonal entries
$d^{(n)}$. For a projection $P$ (or $\calP$, or a projection-valued
function $P(\cdot)$), we use the letter $Q$ (respectively, $\calQ$,
$Q(\cdot)$) to denote the complementary projection: $Q=I-P$.




%------------------- section 2 ----------------

\section{Evolutionary Semigroups: Preliminaries}
\setcounter{equation}{0}

\subsection{Evolutionary semigroups}

Let $\vphi^t$ be a continuous flow on $\TH$, and
let $\Phi\colon \TH\times\bbR_{+}\to \calL(X)$ be a
strongly continuous, exponentially bounded
cocycle over $\vphi^t$.  That is, we assume the function
$$(\th,t) \mapsto \Phi(\th,t)x$$
to be continuous from $\TH\times\bbR_{+}$ to $X$
for each $x\in X$, and there exist $C,\omega>0$ such that
for every $\th\in \TH$,
\newcounter{aa}
\begin{list}{\alph{aa})}{\usecounter{aa}}
\item $\Phi(\th,t+s)=\Phi(\vphi^t\th,s)\Phi(\th,t)\quad$for $t,s>\ge0$;
\item $\Phi(\th,0)=I$;
\item  $\|\Phi(\th,t)\|_{\calL(X)}\le C
e^{\omega t}$ for all $t\in\bbR_+$.
\end{list}
Note that the operators $\Phi(\th,t)$ are not assumed to be invertible.
The last inequality holds automatically if $\TH$ is compact.

The {\em linear skew-product flow} (LSPF) associated with $\Phi$ is
the map
\begin{equation}\lb{LSPF}
\hp^t\colon \TH\times X\ \to \TH\times X,\qquad
\hp^t(\th,x)=(\vphi^t\th,\Phi(\th,t)x),\qquad t\ge 0.
\end{equation}
We note that for exponentially bounded cocycles, the map $\hp^t$ is
continuous if and only if the corresponding cocycle is strongly
continuous.

To a cocycle $\Phi$ over a flow $\vphi^t$, one can
associate the family of operators $\{T^t\}_{t\ge 0}$ in $\Bco$
defined by
\begin{equation}\lb{evolSG}
(T^tf)(\th)=\Phi(\vphi^{-t}\th,t)f(\vphi^{-t}\th),\quad \th\in \TH,
\quad f\in\co.
\end{equation}
It's easy to check that this defines a semigroup of operators:
$T^{t+s}=T^tT^s$, $T^0=I$.  As noted below, this {\em evolutionary
semigroup} is a strongly continuous
semigroup on $\co$ if and only if $\Phi$ is a strongly continuous,
exponentially bounded cocycle.  The generator will be denoted by
$\Gamma$.

\begin{thm}\lb{t2.1}
$\{T^t\}_{t\ge 0}$, as defined in \eqref{evolSG},
is a $C_0$ semigroup
on $\co$ if and only if $\Phi$ is an exponentially bounded,
strongly continuous cocycle.
\end{thm}

\begin{pf}  If $\Phi$ is a strongly continuous cocycle with
$\|\Phi(\th,t)\|\le C e^{\omega t}$ for some $C,\omega>0$,
then $\{T^t\}_{t\ge0}$ can be seen to be strongly continuous
as in \cite{LS} or \cite{Rau}.

Conversely, assume $\{T^t\}_{t\ge 0}$ is a $C_0$ semigroup.
Clearly, $\Phi$ is exponentially bounded.
Fix $\th_0\in \TH$, $t_0\in\bbR$, and $x\in X$.  Let $\epsilon>0$.
For $\th\in \TH$, and $t\in\bbR$, consider
\begin{equation}\lb{t2.1-2}
\begin{aligned}
\|\Phi(\th,t)x-\Phi(\th_0,t_0)x\|
&\le\|\Phi(\th_0,t)x-\Phi(\th_0,t)x\|
+\|\Phi(\th_0,t)x-\Phi(\vphi^{t-t_0}\th_0,t_0)x\|\\
&\qquad+\|\Phi(\vphi^{t-t_0}\th_0,t_0)x-\Phi(\th_0,t_0)x\|.
\end{aligned}
\end{equation}
Let $D$ be a compact set in $\TH$ containing $\th_0$ in its interior.
Choose $\alpha\colon \TH\to[0,1]$ with compact support
such that $\alpha(\th)=1$ for all $\th\in D$.
Define $f\in\co$ as $f(\th)=\alpha(\th)x$, and note that
for $\th\in D$ and $t\in\bbR$,
$$
\Phi(\th,t)x=(T^tf)(\vphi^t\th).
$$
First consider the middle term on the right-hand side of
\eqref{t2.1-2} and choose $\delta_1>0$
such that $\vphi^{t-t_0}\th_0\in D$
for all $|t-t_0|<\delta_1$.  Choose $\delta_2\in(0,\delta_1]$ such
that $|t-t_0|<\delta_2$ implies $\|T^tf-T^{t_0}f\|\le\epsilon/3$.
Then for $|t-t_0|<\delta_2$,
$$\begin{aligned}
\|\Phi(\th_0,t)x-\Phi(\vphi^{t-t_0}\th_0,t_0)x\|
&=\|(T^tf)(\vphi^t\th_0)-(T^{t_0}f)(\vphi^{t_0}(\vphi^{t-t_0}\th_0))\|\\
&=\|(T^tf)(\vphi^t\th_0)-(T^{t_0}f)(\vphi^t\th_0)\|\\
&\le\sup_{\th\in \TH}\|T^tf(\th)-T^{t_0}f(\th)\|<\epsilon/3.
\end{aligned}
$$

Secondly, since $T^tf$ is continuous, there exists $\delta'>0$ such
that $d(\th_1,\th_2)<\delta'$ implies
$\|T^{t_0}f(\th_1)-T^{t_0}f(\th_2)\|<\epsilon/3$.  Since $t\mapsto
\vphi^t\th_0$ is continuous, there exists $\delta_3\in(0,\delta_2]$
such that $|t-t_0|<\delta_3$ implies
$d(\vphi^t\th_0,\vphi^{t_0}\th_0)<\delta'$.
Then, $|t-t_0|<\delta_3$ implies
$$\begin{aligned}
\|\Phi(\vphi^{t-t_0}\th_0,t_0)x-\Phi(\th_0,t_0)x\|
&=\|T^{t_0}f(\vphi^{t_0}(\vphi^{t-t_0}\th_0))-T^{t_0}
  f(\vphi^{t_0}\th_0)\|\\
&=\|T^{t_0}f(\vphi^t\th_0)-T^{t_0}f(\vphi^{t_0}\th_0)\|<\epsilon/3.
\end{aligned}$$

Finally, choose $\delta''>0$ so that $B(\th_0,\delta'')\subset D$, and
$\th\in B(\th_0,\delta'')$ implies
\[ \|(T^tf)(\vphi^t\th)-(T^tf)(\vphi^t\th_0)\|<\epsilon/3.\]
We note that $\delta''=\delta''(t)$ depends on $t$, but on the
compact interval $[t_0-\delta_3,t_0+\delta_3]$, the map
$t\mapsto T^tf(\vphi^t\cdot\,)$ is
uniformly continuous, so $\delta''$ may be chosen independent of $t$.
Therefore, if
$|t-t_0|<\delta_3$ and $\th\in B(\th_0,\delta'')$, then \eqref{t2.1-2}
shows that $\|\Phi(\th,t)x-\Phi(\th_0,t_0)x\|<\epsilon$.
\end{pf}

\subsection{Examples}

We give several examples of cocycles and corresponding
evolutionary semigroups.

\begin{exmp}\lb{exuvareqn} {\it Norm continuous compact setting.}
Let $\vphi^t$ be a continuous flow on a compact metric space $\TH$,
and let $A\colon \TH\to \calL(X)$ be (norm) continuous. For each
$\th\in\TH$ consider the equation \begin{equation}\label{vareqn}
\dfrac{dx}{dt}=A(\vphi^t\th)x(t),
\quad \th\in \TH,\; t\in\bbR.
\end{equation}
Let $\Phi(\th,t)$, $t\in\bbR$, be the solving operator for
\eqref{vareqn}: $x(t)=\Phi(\th,t)x(0)$.
Then $\Phi$ is a cocycle
and $\displaystyle{\left.A(\th)=\frac{d}{dt}\Phi(\th,t)\right|_{t=0}}$.
Denote
$(\bd f)(\th)=\left.\dfrac{d}{dt}f\circ\vphi^t(\th)\right|_{t=0}$.
The generator $\Gamma$ of the group \eqref{evolSG} is given as
follows: \begin{equation}\label{genucc}
 (\Gamma f)(\th)  =
-(\bd f)(\th)+A(\th)f(\th),
\end{equation}
and its domain
$\calD(\Gamma)$ is, in this case,
$\calD(\Gamma)  =
\{f\in\co:\bd f\in\co\}$.

Equations of the type \eqref{vareqn} arise from
two sources. Firstly, they can be thought of as a linearization of a
nonlinear equation in $X$ in the vicinity of a compact invariant
set $\TH$ for the nonlinear equation.
Secondly, $\TH$ might be a compact hull,
$\TH=\text{closure}\{a(\,\cdot\,+\tau): \tau\in\bbR\}$, of a given
function $a\colon\bbR\to \calL(X)$ (see \cite{Hale,SSDich}).
\end{exmp}

\begin{exmp}\lb{exsvareqn} {\it Strongly continuous compact setting.}
We now allow the operators $A(\th)$ in \eqref{vareqn} to be
unbounded. A strongly continuous (semi)cocycle $\Phi$
is said to solve \eqref{vareqn} if: For every $\th\in \TH$ and every
$x_\th\in\calD(A(\th))$ the function $t\mapsto x(t):=\Phi(\th,t)x_\th$
is differentiable for $t>0$, $x(t)\in\calD(A(\vphi^t\th))$, and
$x(\cdot)$ satisfies \eqref{vareqn}. By Theorem~\ref{t2.1} the cocycle
generates an evolutionary semigroup given by \eqref{evolSG}.

This setting might occur if, after the
linearization of a nonlinear equation in $X$, the Frechet derivative
$A(\th)$ is an unbounded operator. Numerous examples of this
strongly continuous setting can be found in \cite{ChLe1}.
\end{exmp}

\begin{exmp}\lb{constcoc}
To be more specific, in the setting of Example~\ref{exsvareqn},
assume $A(\th)\equiv A_0$, $\th\in \TH$, where $A_0$  generates a
$C_0$ semigroup $\{e^{tA_0}\}_{t\ge 0}$ on $X$. The cocycle $\Phi$ for
\eqref{vareqn} is $\Phi(\th,t)=e^{tA_0}$. In the tensor product
$\co=C_0(\TH,\bbR)\bigotimes X$ one can express the
evolutionary semigroup  defined in \eqref{evolSG} by
$T^t=V^t\bigotimes e^{tA_0}$, where $V^tf:=f\circ \vphi^{-t}$. Then
(see, e.g., \cite[p.~23]{Nag}) the generator $\Gamma_0$ of
$\{T^t\}_{t\ge 0}$ is the closure of the operator
${\Gamma'}_0 f  =
-\bd f+A_0f$ with
\begin{equation}\notag
\calD(\Gamma'_0)  =
\{f\in\co:\bd f\in\co,\; f:\TH\to \calD(A_0), \;\bd f-A_0f\in\co\}.
\end{equation}
\end{exmp}

\begin{exmp}\lb{bddpertconstcoc}
Suppose, in the setting of Example~\ref{exsvareqn}, that
\begin{equation}\lb{pertA}
A(\th)=A_0+A_1(\th),\quad \th\in \TH,
\end{equation}
where $A_0$ is a generator of a
$C_0$ semigroup $\{e^{tA_0}\}_{t\ge 0}$ on $X$ and $A_1: \TH\to
\calL_s(X)$ is (strongly) continuous and bounded on $\TH$. Note
that $A_1(\cdot)$ defines an operator $\calA_1\in \Bco$ by the
rule $(\calA_1f)(\th)=A_1(\th)f(\th)$. For $\Gamma_0$ as in
Example~\ref{constcoc} we have:

\begin{prop}\lb{formofgen} Assume that there
exists a strongly continuous cocycle $\Phi$ that solves
\eqref{vareqn} with $A(\cdot)$ as in \eqref{pertA}.
Then the generator $\Gamma$ of the evolutionary semigroup
\eqref{evolSG} is given by
\begin{equation}\lb{A0A1}
(\Gamma f)(\th)=-(\bd f)(\th)+A_0f(\th)+A_1(\th)f(\th),\quad
\calD(\Gamma)=\calD(\Gamma_0).
\end{equation}
\end{prop}

\begin{pf} Since $\Phi$ defines a classical solution of
\eqref{vareqn}--\eqref{pertA}, $\Phi$  also is a mild solution of
\eqref{vareqn}, that is, $\Phi$ satisfies
\begin{equation}\lb{milds}
\Phi(\th,t)x=e^{tA_0}x+\int\limits_0^te^{(t-\tau)A_0}A_1(\vphi^\tau
\th)\Phi(\th,\tau)x\, d\tau,
 \end{equation}
for $x\in X$, $\th\in \TH$, $t\ge 0$.
(We point out the interesting Theorem~5.1 in \cite{ChLe3} where the
existence of a mild solution $\Phi$ was proved for any equations
\eqref{vareqn}--\eqref{pertA}.)

Since $\Gamma_0$ generates a $C_0$ semigroup, and $\calA_1\in
\Bco$, the operator $\Gamma=\Gamma_0+\calA_1$ also
generates (see, e.g., \cite[p.~77]{Pazy}) a $C_0$ semigroup,
$\{S^t\}_{t\ge 0}$, which is the unique solution of the integral
equation
\begin{equation}\lb{sgeq}
S^tf=e^{t\Gamma_0}f+\int\limits_0^te^{(t-\tau)\Gamma_0}\calA_1S^\tau
f\, d\tau,\quad f\in\co, \; t\ge 0.
 \end{equation}
We show that $S^t=T^t$. Indeed,
for $f\in\co$ define
$g=\int\limits_0^te^{(t-\tau)\Gamma_0}\calA_1T^\tau
f\, d\tau$.
Then \eqref{milds} implies
$$\begin{aligned}
g(\th)&=  \int\limits_0^te^{(t-\tau)A_0}A_1(\vphi^\tau(\vphi^{-t}\th))
\Phi(\vphi^{-t}\th,\tau)f(\vphi^{-t}\th)\,d\tau\\
&=\Phi(\vphi^{-t}\th,t)f(\vphi^{-t}\th)-e^{tA_0}f(\vphi^{-t}\th)\\
&=(T^tf)(\th)-\left(e^{t\Gamma_0}f\right)(\th).
\end{aligned}$$
Therefore, $\{T^t\}_{t\ge0}$ satisfies \eqref{sgeq}, and so $T^t=S^t$.
The generator, $\Gamma=\Gamma_0+\calA_1$, of this semigroup is given by
\eqref{A0A1}.
 \end{pf}
\end{exmp}

\begin{exmp}\lb{nacp} {\it Norm continuous line setting.}
Let $\TH=\bbR$, and assume $A\colon\bbR\to \calL(X)$ is a bounded
continuous function. Let $U(\tau,s)$, $\tau,s\in\bbR$, denote the
propagator for the equation
\begin{equation}\label{difeqn}
\dfrac{dx}{dt}=A(t)x(t),
\quad t\in\bbR.
\end{equation}
This means that the solution $x(\cdot)$ of \eqref{difeqn}
satisfies $x(\tau)=U(\tau,s)x(s)$.
Denote $\vphi^t(\tau)=\tau+t$ and $\Phi(\tau,t)=U(\tau+t,\tau)$,
for $\tau,t\in\bbR$ (cf. \cite{ChLe3}). Then
$\Phi$ is a cocycle. In this case, the group
\eqref{evolSG} is given by $(T^tf)(\tau)=U(\tau,\tau-t)f(\tau-t)$,
$f\in\coR$, and the generator $\Gamma$ is given by the formula:
\[(\Gamma f)(\tau)=-\dfrac{df}{d\tau}+A(\tau)f(\tau),
\quad \calD(\Gamma)=\{f\in\coR: f'\in\coR\}.\]
\end{exmp}

\begin{exmp} \lb{exslset}{\it Strongly continuous line setting.}
In the case of $\TH=\bbR$, consider a well-posed differential equation
\eqref{difeqn} with, generally, unbounded operators $A(\tau)$.
Let  $\{U(t,s)\}_{t\ge s}$ be the associated
strongly continuous evolutionary family of
operators in $\calL(X)$.  This means that
$(\tau,s)\mapsto U(\tau,s)x$ is continuous for each $x\in X$,
$U(\tau,s)=U(\tau,r)U(r,s)$  for $\tau\ge r\ge s$, $U(s,s)=I$, and
$\|U(\tau,s)\|\le C e^{\omega (\tau-s)}$ for some
$C,\omega>0$. The flow $\vphi^t$, the cocycle $\Phi$, and the
evolutionary semigroup can be defined exactly as in the previous
example. If, in particular, $A(\tau)\equiv A_0$ is a generator of a
$C_0$ semigroup $\{e^{tA_0}\}_{t\ge0}$ on $X$, then
$U(\tau,s)=e^{(\tau-s)A_0}$, and
$\Gamma$ is the closure of \[\Gamma'=-\dfrac{d}{dt}+A_0,\quad
\calD(\Gamma')=\{f\in\coR: f'+A_0f\in\coR,\: f:\bbR\to\calD(A_0)\}.\]

For $\TH=\bbR$ consider now the case where each of the operators $A(t)$,
$t\in\bbR$ in \eqref{difeqn} is an (unbounded) operator on
$X$ with dense domain.
For each $t\in\bbR$, denote
the domain of $A(t)$\ by $\calD_t:=\calD(A(t))$.
Consider the following nonautonomous abstract Cauchy problem (cf.
\cite{Nag2}):
\begin{equation}\lb{nCP}
\dfrac{dx}{dt}=A(t)x(t) \mbox{  for  } t\ge s\in\bbR,\quad\mbox{
and  } x(s)=x_s,
\end{equation}
  where $x_s\in\calD_s$, $s\in\bbR$.
We say (see \cite{Nag2}) that the evolutionary family
$\{U(\tau,s)\}_{\tau\ge
s}$ solves \eqref{nCP} if $x(\cdot)=U(\cdot\,,s)x_s$ is a unique
solution of \eqref{nCP} for every $x_s\in\calD_s$, that is
$x(\cdot)$ is differentiable, $x(t)\in\calD_t$ for $t\ge s$, and
\eqref{nCP} holds.
\begin{prop}\lb{bestgen} Assume that a strongly continuous
evolutionary family $\{U(\tau,s)\}_{\tau\ge
s}$ solves \eqref{nCP}. Then the generator $\Gamma$ of the
evolutionary semigroup $(T^tf)(\tau)=U(\tau,\tau-t)f(\tau-t)$ on
$\coR$ is the closure of the operator $\Gamma'$,
with domain consisting of differentiable functions $f$\ such
that $f(t) \in \calD_t$ and $\Gamma'f\in\coR$, defined
as follows:
\begin{equation}\lb{gamma'}
(\Gamma'f)(\tau)=-\dfrac{df}{d\tau}+A(\tau)f(\tau),
\quad
\tau\in\bbR.
\end{equation}
 \end{prop}
\begin{pf}
Fix $s\in\bbR$ and $x_s\in\calD_s$. For any smooth
$\alpha:\bbR\to\bbR$ with compact support $\supp\alpha\subset
[s,\infty)$, consider a function $f\in\coR$, defined by
\begin{equation}\lb{deff}
f(\tau)=\alpha(\tau)U(\tau,s)x_s \mbox{  for  } \tau> s \mbox{  and
} f(\tau)=0 \mbox{ for } \tau\le s.
\end{equation}
Then $\Gamma f=\Gamma' f$.  Indeed
\[ (T^tf)(\tau)=\alpha(\tau-t)U(\tau,\tau-t)U(\tau-t,s)x_s=
\alpha(\tau-t)U(\tau,s)x_s
\] for
$\tau-t>s$, and zero otherwise.
Hence,
$$\left.\dfrac{d}{dt}(T^t
f)(\tau)\right|_{t=0}=-\alpha'(\tau)U(\tau,s)x_s,\quad \tau\in\bbR.
$$
On the other hand, since $\tau\mapsto U(\tau,s)x_s$ satisfies
\eqref{nCP}, one has:
\[\dfrac{df}{d\tau}=\alpha'(\tau)U(\tau,s)x_s+
\alpha(\tau)\dfrac{d}{d\tau}U(\tau,s)x_s=
\alpha'(\tau)U(\tau,s)x_s+\alpha(\tau)A(\tau)U(\tau,s)x_s.\]

To finish the proof, we need to show
 that  linear combinations of
functions $f$, as in \eqref{deff}, are dense in $\coR$. To see this,
first observe that the
set of finite sums $\sum\beta_jv_j$ with arbitrary $v_j\in X$ and
smooth $\beta_j:\bbR\to\bbR$ with compact support, is dense in
$\coR$.

Now, consider any function $g=\beta v$ with fixed
$v\in X$ and smooth $\beta:\bbR\to\bbR$ with compact support.
We show that $g$ can be
approximated by a sum of functions $f$ as in \eqref{deff}.

To see that, fix $\epsilon>0$. For every $\tau_0\in\supp\beta$,
the map $(\tau,s)\mapsto U(\tau,s)v$, $\tau\ge s$ is continuous at
the point $(\tau_0,\tau_0)$ by the definition of $\{U(\tau,s)\}$.
Hence, there exist $s_0\le s'_0$ such that for the interval $I_0=I(\tau_0)
:=(s_0,\, s'_0)$, containing $\tau_0$, one has
$\|U(\tau,s_0)v-v\|\le\epsilon/2$ for  $\tau\in I_0$.
Thus, we obtain an open covering of
$\supp\beta$, formed by $I(\tau_0)$, $\tau_0\in\supp\beta$.
Take a finite subcovering $\{I_j\}_{j=1}^n$.  For
$I_j:=(s_j,s'_j)$, one has  $\|U(\tau,s_j)v-v\|\le\epsilon/2$ for
 $\tau\in I_j$, $j=1,\dots, n$.

Consider a smooth partition of
unity $\{\gamma_j\}_{j=1}^n$ for $\{I_j\}_{j=1}^n$, that is, smooth
functions $\gamma_j:\bbR\to [0,1]$ such that
\[ \sum\limits_{j=1}^n \gamma_j(\tau)=1 \mbox{ for }
\tau\in\supp\beta, \mbox{ and } \supp\gamma_j\subset I_j, j=1,\dots, n.\]
Since the $\calD_{s_j}$ are dense in $X$, one can choose
$v_j\in\calD_{s_j}$ such that
$\|v-v_j\|\le\epsilon/2$.

Define the function
\[ h(\tau):=\sum_{j=1}^n \beta (\tau) \gamma_j(\tau) U(\tau,
s_j)v_j,\quad \tau\in\bbR.\]
Since $\supp\gamma_j\subset (s_j,s'_j)$, the function $h$ is a sum
of functions $f$ as in \eqref{deff}.
Also,
\begin{eqnarray*}
\|g(\tau)-h(\tau)\|& = &\left\|
\beta (\tau) \sum_{j=1}^n \gamma_j(\tau) v-
\sum_{j=1}^n \beta (\tau) \gamma_j(\tau) U(\tau,
s_j)v_j\right\| \\
&\le&
|\beta(\tau)|
\left(\sum_{j=1}^n\gamma_j(\tau)\| v-v_j\| +
\sum_{j=1}^n\gamma_j(\tau)\|
v_j-U(\tau,s_j)v_j\|\right)\\
& \le &\epsilon\max_\tau|\beta(\tau)|, \end{eqnarray*}
and the proof is completed.
\end{pf}
\end{exmp}

\subsection{An algebra of weighted translation operators}

In Section~4 we study the spectrum of $T=T^1$, that is, the
invertibility of $b=\lambda I-T$. This operator belongs to an
algebra, $\frakB$, of weighted translation operators which we now
define.

Let $C_b(\TH;\calL_s(X))$ denote the set of strongly continuous and  bounded
functions $a\colon \TH\to \calL_s(X)$.  For $a\in
C_b(\TH,\calL_s(X))$, set  $\|a\|_u:= \sup_{\th\in \TH} \|a(\th)\|_{\calL(X)}$.
Such a function induces a multiplication operator
on $\co$ defined by $(M_af)(\th)=a(\th)f(\th)$.
The mapping $a\mapsto M_a$ is an isometry from the Banach space
$C_b(\TH,\calL_s(X))$ to $\Bco$ and so
the operator $M_a$ will be
denoted simply by $a$, its
norm given by $\|a\|=\|a\|_{\Bco}=\|a\|_u$.
Let $\frak A$ denote
the set of all such
multiplication operators $a=M_a\in\Bco$.

Now let $\vphi=\vphi^1$ be a homeomorphism on $\TH$.
Denote by $V$ the translation
operator on $\co$ given by $(Vf)(\th)=f(\vphi^{-1}\th)$.
Define $\frakB$ to be the set of all operators $b\in\Bco$
such that
\begin{equation}\lb{b}
b=\sumk a_kV^k, \quad\text{ where }a_k\in\frakA,\text{ and }
\|b\|_1:=\sumk \|a_k\| < \infty.
\end{equation}

\begin{prop}
If the set of aperiodic points of
$\vphi$ is dense in $\TH$, then the representation
\eqref{b} of an
element $b\in\frakB$ is unique.
\end{prop}
\begin{pf}
This follows from the observation
that for any polynomial $b_N=\sum^N_{k=-N} a_kV^k$ in $\Bco$, where
$a_k\in\frakA$ and $N\in\bbN$,
\begin{equation}\lb{lemma0}
\|b_N\|_{\Bco} \ge \|a_k\|_{\Bco},\quad |k|\le N.
\end{equation}
To prove \eqref{lemma0}, first note that
by replacing $b$ by $bV^{-k}$ it suffices to consider only the case
$k=0$. Fix $\epsilon > 0$.  For $a\in\frakA$,
$\displaystyle{\|a\|_{\Bco} = \sup_{\th\in \TH} \sup_{\|x\|_X=1}
\|a(\th)x\|_X}$, so there exists a nonperiodic point $\th_0$ of
$\{\vphi^t\}_{t\in\bbR}$, and a $x\in X$,
$\|x\|=1$, such  that
$$
\|a_0(\th_0)x\|_X\ge \|a_0\|_{\Bco} - \epsilon.
$$
Choose $\delta>0$ such that for $B=B(\th_0,\delta)$, $\varphi^k(B)\cap
\varphi^j(B) = \emptyset$ for $k\ne j, \; |k|, \; |j|\le N$.  Choose a
continuous function $\alpha:\TH\to [0,1]$ such that $\alpha(\th_0) =1$
and $\alpha(\th) = 0$ for $\th\notin B$.
Define $f\in \co$ by
$$f(\th)=\cases
\alpha(\varphi^{-k}\th)x,\quad\text{ if }
\th\in\varphi^k(B), \; |k|\le N\\
0\quad\text{ otherwise}.\endcases$$
Then $\|f\|_{\co}=1$ and
$$\begin{aligned}
\|b\|_{\Bco} &\ge \|bf\|_{\co}
  = \max\sb{\th\in \TH}\left\|\sum^N_{k=-N}
     a_k(\th)\alpha(\varphi^{-k}\th) x\right\|_X \\
&=\max\sb{\th\in \TH} \max\sb{|k| \le N} \|a_k(\th)
   \alpha(\varphi^{-k}\th)x\|_X \ge \|a_0(\th_0)x\|_X\\
&\ge \|a_0\|_{\Bco} - \epsilon.
\end{aligned}
$$
\end{pf}

\begin{prop}
$(\frakB,\|\cdot\|_1)$ is a Banach algebra.
\end{prop}
\begin{pf}
The norm $\|\cdot\|_1$ is an algebra norm.
To see that it is
complete, consider a $\|\cdot\|_1$-Cauchy sequence
$\{b^{(n)}\}_{n=1}^\infty$ in $\frakB$, $b^{(n)} =
\displaystyle\sumk a^{(n)}_k V^k$.  Set $a_k =
\|\cdot\|_1$-$\displaystyle\lim\sb{m\to \infty} a^{(m)}_k$,
$k\in\bbZ$.  For every $m,n$ in $\bbZ$,
$$
\left|\sumk\|a_k^{(m)}\|-\sumk\|a^{(n)}_k\|\right| \le
\|b^{(m)}-b^{(n)}\|_1.
$$
So a sequence $(\|a^{(m)}_k\|)_{k\in\bbZ} \in \ell_1(\bbZ,\bbR)$
converges to $\left(\|a_k\|\right)_{k\in\bbZ}$ in $\ell_1(\bbZ,\bbR)$.
Hence $b=\sum\sb{k} a_kV^k$ is an element of $\frakB$.
For every $\epsilon > 0$ and for sufficiently large $m,n$, the
inequality
$$
\epsilon \ge \|b^{(m)}-b^{(n)}\|_1 = \sumk
\|a^{(m)}_k-a^{(n)}_k \| \ge \sum^N_{k=-N} \|a^{(m)}_k-a^{(n)}_k\|
$$
holds for every $N\in \bbN$.
Taking the limit as $n\to\infty$, and then the limit as
$N\to\infty$ shows $\|b^{(m)}-b\|_1\to 0$.
\end{pf}

In  Section~4, we consider the relationship
between the invertibility of a weighted translation
operator $b$ in $\frak B$ and a family of representations
(weighted shift operators), $\pi_\th (b)$, in $\Blinf$.
To define the representation $\pi_\th$ of $\frakB$,
denote by $S$ the shift operator on $\linf$:
\[ S:\; (x_n)_{n\in\bbZ}\mapsto (x_{n-1})_{n\in\bbZ},
\quad (x_n)_{n\in\bbZ}\in \linf.\]
For $a\in\frakA$ and $\th\in \TH$, let
$\pi_\th(a)=\diag\{a(\vphi^n\th)\}_{n\in\bbZ}$.
Defining $\pi_\th(aV)=\pi_\th(a)S$ gives a continuous
homomorphism from $\frakB$ into $\Blinf$:
for $b=\sum_k a_kS^k$ in $\frakB$,
\begin{equation}\lb{pisubx}
\pi_\th(b)=\sumk \pi_\th(a_k)S^k.
\end{equation}

With this in mind, let $\frak C$ denote
the set of operators in
$\Blinf$ of the form
$d=\diag\{d^{(n)}\}_{n\in\bbZ}$, where
$d^{(n)}\in \calL(X)$, and consider the Banach algebra
$(\frak D,\|\cdot\|_1)$ of operators on $\linf$ of
the form
\[d=\sumk d_k S^k\;
\text{ where } \; d_k\in\frakC,\;\text{ and }\;
\|d\|_1 := \sumk\|d_k\|_{\Blinf} < \infty.\]

The following lemma points out, in particular, that $\frakD$
is an inverse-closed subalgebra of $\Blinf$. The proof
follows exactly as in Lemma 1.6 and Remark 1.7 of \cite{LatRand}.

\begin{prop}\lb{p2.8}
Let $\displaystyle{d(\th)=\sumk d_k(\th)S^k}$ be an operator in $\frakD$,
and assume that
\[\displaystyle\sumk\sup\sb{\th\in \TH}\|d_k(\th)\|_{\Blinf} <
\infty.\] If $d(\th)^{-1}\in\Blinf)$ for each $\th\in \TH$
and if for some $C>0$,
$$
\sup\sb{\th\in \TH} \|d(\th)^{-1} \|_{\Blinf} < C,
$$
then $d(\th)^{-1} \in \frakD$ for each $\th\in \TH$.  Moreover, if
$d(\th)^{-1}= \displaystyle\sumk c_k(\th)S^k$, then
$$\sumk \sup\sb{\th\in \TH}
\| c_k(\th)\|_{\Blinf} < \infty.$$
\end{prop}





%------------------- section 3 ----------------
\section{Spectral Mapping Theorem}
\setcounter{equation}{0}

In this section we prove that the spectral mapping property
\begin{equation}\lb{smteq}
e^{t\sigma(\Gamma)}=\sigma(T^t)\bs \{0\},\quad t>0,
\end{equation}
holds for any evolutionary semigroup \eqref{evolSG} with
generator $\Gamma$.
As usual, the nonperiodic points of $\{\vphi^t\}_{t\in\bbR}$
are assumed to be dense in $\TH$.  For a general $C_0$ semigroup
$\{T^t\}_{t\ge0}$ generated by an operator $\Gamma$,
the spectral inclusion
$e^{t\sigma(\Gamma)}\subseteq \sigma(T^t)\bs \{0\}$
is well known, as are examples showing that the inclusion is,
in general, proper.  Moreover, equality holds for the point
spectrum and residual spectrum \cite{Nag}, so to prove
\eqref{smteq} we focus on the approximate point spectrum,
$\sigma_{ap}(\cdot)$.
By a standard rescaling technique \cite{Nag},
it suffices to consider only $T=T^1$.

We build on the ideas of \cite{Mather} and observe, in particular,
that the $\epsilon$-eigenfunctions of $T$ obey the following
``localization" principle:  if $1\in\sigma_{ap}(T)$, then for every
$N\in\bbN$, there exists a point $\th_0\in \TH$, a function
$f\in\co$ with $\|f\|=1$, and an open neighborhood $D$
of $\th_0$ such that $\displaystyle{
\supp f\subseteq\cup_{j=0}^{2N} \vphi^j(D) }$
and $\|Tf-f\|=O({1\over N})$.  Using $f$, we construct a function
$g\in\co$ such that $\|\Gamma g\|=O({1\over N})\|g\|$, hence
showing that $0\in\sigma_{ap}(\Gamma)$.

Before beginning the proof of \eqref{smteq}, we prove two
technical lemmas which verify the existence of an appropriate
$f\in\co$ and aid in the subsequent construction of $g$,
as described above.

\begin{lem}\lb{lemmaA}
Let $A\in \calL(X)$ and assume $1\in\sigma_{ap}(A)$. For each
$N\ge 2$, there exists $x\in X$, $\|x\|=1$, such that
\begin{list}{\alph{aa})}{\usecounter{aa}}
\item $\|A^Nx-x\|_X\le {1 \over 8}$;
\item $\|A^kx\|_X\le 2 \quad\text{for } k=0,1,\ldots,2N$.
\end{list} \end{lem}

\begin{pf}  Set $c=\sum^{2N}_{k=0}\|A\|^k$.  From the
identity $A^k-I=(A^{k-1}+\ldots + A+I) (A-I)$, for
$k=2,\ldots, 2N$, it follows that for any $x\in X$,
\begin{equation}\lb{lA-1}
\|(A^k-I)x\| \le c\|(A-I) x\|, \quad k=0,1,\ldots,2N.
\end{equation}
Since $1\in \sigma_{ap}(A)$, there exists a $x\in X$, $\|x\|=1$,
such that $\|(A-I)x\|\le {1\over 8c}$.
Set $k=N$ in \eqref{lA-1} to obtain $a)$.
Moreover, \eqref{lA-1} shows that for $k=0,1,\ldots,2N$,
\[ \|A^kx\| \le \|(A^k-I)x\|+\|x\|
\le c\|(A-I)x\| + 1 \le c{1\over 8c}+1 \le 2. \]
\end{pf}

Now let $\th_0\in \TH$ be a nonperiodic point of $\vphi^t$, and
fix $s\in(0,1)$ and $N\ge 2$.  If $B$ is a neighborhood of $\th_0$,
then for $\th\in \TH$  we use the notation $R(\th)=\{t\in \bbR:
|t|\le N, \; \vphi^tx\in B \}$. Lebesgue measure on $\bbR$ will be
denoted by $m$.


\begin{lem}\lb{lemmaB}  There exists a sufficiently small open
neighborhood, $B$, of $\th_0$ and a continuous ``bump'' function
$\alpha\colon \TH\to[0,1]$ such that:
\begin{list}{\alph{aa})}{\usecounter{aa}}
\item  $\alpha(\vphi^t\th_0)=1, \quad\text{ for }|t|\le{s\over 4}$;
\item  $\alpha(\vphi^t\th_0)=0, \quad\text{ for } s\le |t|\le 2N$;
\item  $m(R(\th)) \le 2s, \quad\text{ for all } \th\in \TH$.
\end{list}
\end{lem}

For the important case $\TH=\bbR$ with $\vphi^t\th=\th+t$, this is
trivial: for any $\th_0\in\bbR$ take $B=(\th_0-s,\th_0+s)$, and
$\alpha:\bbR\to [0,1]$ with $\supp\alpha\subset B$ and $\alpha(\th)=1$
for $x\in (\th_0-s/4,\th_0+s/4)$.

\begin{pf}  We begin with:

\begin{claimone}  For sufficiently small
$\epsilon_{*} > 0$, and all $\epsilon \le \epsilon_{*}$,
if $\delta<\epsilon$, then
 $\th\in B':=B(\th_0,\delta)$ implies $\vphi^t\th\notin B'$
 provided ${s\over 2} \le |t|\le 5N$.
\end{claimone}

\begin{pf*}{Proof of Claim 1}    Suppose, to the contrary,
that there exists a sequence $\delta_n\downarrow 0$ such that
for some $\th_n \in B_n:=B(\th_0,\delta_n)$ and some $t_n$ with
${s \over 2} \le |t_n| \le 5N$, one has
$\vphi^{t_{n}}\th_n \in B_n$. Passing to a subsequence, one can
assume $t_n \to t_{*}$ for some $t_{*}$ with
${s\over 2} \le |t_{*}| \le 5N$.  Since the map
$(\th,t) \mapsto \vphi^t\th$ from $\TH\times\bbR$ to $\TH$
is continuous at the point $(\th_0, t_{*})$, and since
$\th_n\in B_n$, it follows that $\th_n\to \th_0$.
Thus $t_n\to t_{*}$ implies $\vphi^{t_{n}}\th_n\to
\vphi^{t_{*}}\th_0$.  On the other hand,
$\vphi^{t_{n}}\th_n \in B_n$ implies $\vphi^{t_{n}}\th_n\to \th_0$,
as  $t_n\to t_{*}$.
But then  $\vphi^{t_{*}} \th_0=\th_0$,
which contradicts the assumption that $\th_0$ is a nonperiodic
point of $\vphi^t$.  This proves Claim 1.
\end{pf*}

Now let $\delta<\epsilon_{*}$, and set $B'=B(\th_0,\delta)$ as in
Claim 1.  Choose an open neighborhood $B''$ of $\th_0$
such that $\overline{B''} \subset B'$.  Denote:
\begin{equation}\lb{defBC}
 B:=\bigcup_{|t|\le {s\over 4}} \vphi^t(B'),\quad
 C:=\bigcup_{|t|\le {s\over 4}} \vphi^t(B'').
\end{equation}
Since $\overline{C} \subset B$, there exists a continuous
$\alpha\colon \TH\to [0,1]$ such that $\alpha(\th)=1$ for $\th\in C$,
and $\alpha(\th)=0$ for $\th\notin B$.

Note that since $\th_0 \in B''$, it follows that
$\{\vphi^t\th_0: |t|\le{s\over 4}\} \subset C$ and so
$\alpha(\vphi^t\th_0) =1$ whenever $|t|\le {s\over 4}$.
This proves $a)$.

To prove $b)$ we first prove

\begin{claimtwo} If $\th\in B$ and if $s\le |t|\le 2N$,
then $\vphi^t\th\notin B$.
\end{claimtwo}

\begin{pf*}{Proof of Claim 2}  Suppose, to the contrary,
that there exists a $\th\in B$ and a $t_{*}\in\bbR$ satisfying
$s\le |t_{*}| \le 2N$ and $\vphi^{t_{*}}\th\in B$.
By the definition \eqref{defBC} of $B$,
there exist $\th_1$ and $\th_2$ in $B'$ such that
\[
\th=\vphi^{t_1}\th_1 \quad\text{and}\quad \vphi^{t_{*}}\th
=\vphi^{t_{2}}\th_2,
\]
for some numbers $t_1,\;t_2$ with $|t_1|,\;|t_2|\le\dfrac{s}{4}$.
This implies
\begin{equation}\lb{Cl2-1}
\varphi^{t_{*} + t_{1}-t_{2}}\th_1 \in B'.
\end{equation}
Note that
$$
|t_{*}+t_1-t_2|\le 2N + \dfrac{s}{4} + \dfrac{s}{4} \le 5N
$$
and
$$
|t_{*}+t_1-t_2|\ge |t_{*}|-|t_1|-|t_2|\ge s-\dfrac{s}{4}-
\dfrac{s}{4} = \dfrac{s}{2},
$$
and so \eqref{Cl2-1} contradicts Claim 1.
This proves Claim 2.
\end{pf*}

Since, in particular, $\th_0\in B$ and $\alpha(\th)=0$
for $\th\notin B$, Claim 2 proves part $b)$ of the lemma.

To prove $c)$, fix $\th_1\in \TH$ and denote the orbit by
$O(\th_1) = \{\vphi^t\th_1: |t| \le N\}$.
Clearly, $m(R(\th_1)) =0$ provided $O(\th_1)\cap B=\emptyset$,
so consider the case $O(\th_1) \cap B\ne \emptyset$.
Fix any $\th\in O(\th_1) \cap B$.  Then $\th=\vphi^{t_1}\th_1$
for some $t_1$ with $|t_1| \le N$.
Further, for any $t_{*} \in R(\th_1)$, one has
$|t_{*}-t_1|\le 2N$ and
\begin{equation}\lb{lB-1}
\vphi^{t_{*}}\th_1 = \vphi^{t_{*}-t_1}\th\in B.
\end{equation}
Since $\th\in B$, Claim 2 shows that \eqref{lB-1} can only hold
provided $|t_{*} -t_1|<s$, that is, $t_{*} \in (t_1-s, t_1+s)$.
Since $t_{*}$  was chosen arbitrarily in $R(\th_1)$, this shows
$R(\th_1) \subseteq (t_1-s,t_1+s)$.  Thus, $m(R(\th_1)) \le 2s$.
This completes the proof of the lemma.
\end{pf}

We now proceed with the proof of the Spectral Mapping Theorem.

\setcounter{equation}{0} %%%-reset counter for this eq'n %%%
\begin{thm}\lb{SMT}  Let $\Gamma$ be the generator of the
evolutionary semigroup \eqref{evolSG}.  Then
\begin{equation}
e^{t\sigma(\Gamma)}=\sigma(T^t)\setminus \{0\},\quad t>0.
\end{equation}
Moreover, the spectrum $\sigma(\Gamma)$ is invariant with
respect to translations along the imaginary axis, and the spectrum
$\sigma(T^t)$, $t>0$, is invariant with respect to
rotations centered at origin.
\end{thm}
\setcounter{equation}{5} %%%-return counter to previous value %%%

\begin{pf} As noted in the beginning of this section,  to prove
\eqref{smteq} it suffices to show: $1\in\sigma_{ap}(T)$ implies
$0\in\sigma_{ap}(\Gamma)$.  In fact, we show that
$\sigma_{ap}(\Gamma)$ contains the entire imaginary axis
whenever $1\in\sigma_{ap}(T)$.  Since $\sigma_{ap}(\Gamma)$
contains the boundary of $\sigma(\Gamma)$, all assertions of
the theorem follow from this.

Let $1\in\sigma_{ap}(T)$, and let $\xi\in\bbR$.  We begin by using
Lemma \ref{lemmaA} with $A=T$ to obtain, for any $N\ge 2$, a function
$f\in\co$ satisfying:
\addtocounter{equation}{1}  %%%--these are eq'ns (3.6a-c)%%%
\begin{list}{(\theequation\alph{aa})}{\usecounter{aa}}
\item $\|f\|_{\co} =1$;
\item $\|T^Nf-f\|_{\co} \le {1\over 8}$;
\item $\|T^kf\|_{\co} \le 2,\text{ for } k=0,1,\ldots,2N$.
\end{list}
For this $f$, fix $s\in (0,1)$ such that
\begin{equation}\lb{SMT-2}
\|T^{t+N}f-T^Nf\|_{\co}\le{1\over 16},\quad\text{ for }|t| \le s,
\end{equation}
and at the same time
\begin{equation}\lb{SMT-3}
|e^{-i\xi t}-1|\le{1\over 32},\quad\text{ for }|t| \le s.
\end{equation}


With the goal of constructing an approximate eigenfunction $g$,
for $\Gamma$, corresponding to $i\xi$, choose a smooth
function $\gamma\colon\bbR\to[0,1]$
such that:
\addtocounter{equation}{1} %%%--these are eq'ns (3.9a-c)%%%
\begin{list}{(\theequation\alph{aa})}{\usecounter{aa}}
\item $\gamma(t) = 0, \quad\text{ for } t\notin (0,2N)$;
\item $|\gamma'(t)|\le {2\over N}, \quad\text{ for all } t\in\bbR$;
\item $\gamma(t) =1, \quad\text{ for } |t-N|\le s$.
\end{list}

Now fix  $\th_0\in \TH$, a nonperiodic point of $\vphi^t$ which
satisfies
\begin{equation}\lb{SMT-4}
\|f(\th_0)\| \ge\dfrac{7}{8}\|f\|_{C_0(\TH,X)} = \dfrac{7}{8}.
\end{equation}
Set $\th_N:=\vphi^{-N}\th_0$, and use Lemma \ref{lemmaB}
to obtain an open neighborhood, $B$, of $\th_N$ and a ``bump''function
$\alpha$ with the properties $a)-c)$ listed there.
Define $g\in\co$ by
\begin{equation}\lb{defing}
g(\th) = \int^{\infty}_{-\infty} e^{-i\xi t}\gamma(t) (T^t\alpha
f)(\th)\,dt,\quad \th\in \TH.
\end{equation}
Note that due to (3.9a), the integration goes from $0$ to $2N$.
Also, note that
\[
\Gamma g = i\xi g
-\int^{\infty}_{-\infty}e^{-i\xi t}\gamma'(t)(T^t\alpha f)\,dt.
\]
Indeed,
$$\begin{aligned}
\Gamma g &= \left.\dfrac{d}{d\tau}\right|_{\tau=0} T^{\tau}g
=\left.\dfrac{d}{d\tau}\right|_{\tau=0} \int^{\infty}_{-\infty}
    e^{-i\xi t}\gamma(t)(T^{t+\tau}\alpha f)\,dt\\
&= \left.\dfrac{d}{d\tau}\right|_{\tau=0}\int^{\infty}_{-\infty}
   e^{-i\xi (t-\tau)}\gamma(t-\tau)(T^t\alpha f)\,dt\\
&=\int^{\infty}_{-\infty}
\left[i\xi e^{-i\xi t}\gamma(t) (T^t \alpha
f)-e^{-i\xi t}\gamma'(t) (T^t \alpha f)\right]\,dt.
\end{aligned}$$

We now proceed to show that $\|\Gamma g-i\xi\|=O({1\over N})\|g\|$
with the following two claims.

\begin{claimone}
Set $\displaystyle{C:=\max_{0\le t\le 1}\|T^t\|_{\calL(X)}}$. Then
$\displaystyle{\|\Gamma g-i\xi g\|_{\co} \le \frac{8Cs}{N} }$.
\end{claimone}

\begin{pf*}{Proof of Claim 1} First use (3.9a) and (3.9b) to obtain
\begin{equation}\lb{SMT-6}
\begin{aligned}
\|\Gamma g-i\xi g\| &= \left\|-\int^{2N}_0 e^{-i\xi t}\gamma'(t)
   \alpha(\vphi^{-t}\cdot)T^tf(\cdot)\,dt \right\| \\
&\le \dfrac{2}{N}\cdot \max_{0\le t\le 2N} \|T^tf\|
   \cdot \max_{x\in \TH} \int^{2N}_0 \alpha(\vphi^{-t}\th)\,dt.
\end{aligned}
\end{equation}
Using (3.6c), note that
\begin{equation}\lb{SMT-7}
\max_{0\le t\le 2N} \|T^tf\|
   \le \max_{0\le k\le 2N}\max_{0\le \tau\le 1}
   \|T^{\tau}T^kf\| \le 2C.
\end{equation}
Also, the change of variable $t\mapsto -t+N$ gives
\[
\max_{x\in \TH}\int^{2N}_0\alpha(\vphi^{-t}\th)\,dt
=\max_{x\in \TH}\int^N_{-N}\alpha(\vphi^t(\vphi^{-N}\th))\,dt
=\max_{x\in \TH}\int^N_{-N} \alpha(\vphi^t\th)\,dt.
\]
Now, for fixed $\th\in \TH$, recall from Lemma \ref{lemmaB} that the set
$R(\th)=\{t\in\bbR: |t|\le N,\;\vphi^t\th\in B\}$
 has measure $m(R(\th))\le 2s$.
Consider $|t|\le N$; then $\alpha(\vphi^t\th)\le 1$ for $t\in R(\th)$,
and $\alpha(\vphi^t\th)=0$ for $t\notin R(\th)$, and so
\begin{equation}\lb{SMT-8}
\max_{\th\in \TH} \int^N_{-N} \alpha(\vphi^t\th)\,dt
\le \max_{\th\in \TH}\int_{R(\th)}dt \le 2s.
\end{equation}
Using \eqref{SMT-7} and \eqref{SMT-8} in the inequality \eqref{SMT-6}
gives
$$
\|\Gamma g-i\xi g\| \le \dfrac{2}{N}\cdot 2C\cdot 2s
= \dfrac{8Cs}{N}.
$$
This proves Claim 1.
\end{pf*}

\begin{claimtwo}  $\displaystyle{\|g\|_{\co}\ge \frac{s}{128} }$.
\end{claimtwo}

\begin{pf*}{Proof of Claim 2} Recall $\th_N=\vphi^{-N}\th_0$.
Using (3.9a) and a change of variable $t\mapsto t+N$ gives:
$$\begin{aligned}
g(\th_0) &= \int^{2N}_0  e^{-i\xi t} \gamma(t)
          \alpha(\vphi^{-t}\th_0)(T^tf)(\th_0)\,dt\\
&=\int^N_{-N}
e^{-i\xi(t+N)}\gamma(t+N)\alpha(\vphi^{-t}\th_N)(T^{t+N}f) (\th_0)\,dt.
\end{aligned}$$
And so, Lemma \ref{lemmaB} $b)$ and (3.9c) show that
$$\begin{aligned}
g(\th_0) &= \int^s_{-s}e^{-i\xi(t+N)}\gamma(t+N)
   \alpha(\vphi^{-t}\th_N)(T^{t+N}f)(\th_0)\,dt\\
&=e^{-i\xi N} \int^s_{-s}
e^{-i\xi t}\alpha(\vphi^{-t}\th_N)(T^{t+N}f)(\th_0)\,dt\\
&= e^{-i\xi N}(I_1 +
I_2 + I_3), \end{aligned}$$
where we have denoted for brevity:
$$\begin{aligned}
I_1 &=f(\th_0)\int^s_{-s} e^{-i\xi t}\alpha(\vphi^{-t}\th_N)\,dt,\\
I_2&=(T^N-I)f(\th_0)\int^s_{-s}e^{-i\xi t}
    \alpha(\vphi^{-t}\th_N)\,dt,\\
I_3 &=\int^s_{-s}e^{-i\xi t}\alpha(\vphi^{-t}\th_N)
    [(T^{t+N}-T^N)f](\th_0)\,dt.
\end{aligned}$$
It follows from Lemma \ref{lemmaB} $a)$
(and recall $0\le\alpha(\th)\le 1$), and \eqref{SMT-3} and \eqref{SMT-4}
that

$$\begin{aligned}
\|I_1\| &= \|f(\th_0)\|\left|\int^s_{-s}
        e^{-i\xi t}\alpha(\vphi^{-t}\th_N)\,dt\right| \\
&\ge\|f(\th_0)\|\left(\left|\int^s_{-s}\alpha(\vphi^{-t}\th_N)\,dt\right|
     -\left|\int^s_{-s}(e^{-i\xi t}-1)\alpha(\vphi^{-t}\th_N)\,dt
     \right|\right) \\
&\ge \|f(\th_0)\|\left(\left|\int^{-\frac{s}{4}}_{\frac{s}{4}}
     \alpha(\vphi^{-t}\th_N)\,dt\right|
     -\int^s_{-s}\left|(e^{-i\xi t}-1)\right|\,dt \right) \\
&\ge \|f(\th_0)\|\left( \frac{s}{2}-\frac{1}{32}2s\right)
\ge \frac{7}{8}\left(\frac{7s}{16}\right)=\frac{49s}{128}.
\end{aligned}$$
On the other hand, (3.6b) and \eqref{SMT-2} show that
$$\begin{aligned}
\|I_2\| &\le \left\|(T^N-I)f\right\| \left|\int^s_{-s}
   e^{-i\xi t}\alpha(\vphi^{-t}\th_N)\,dt\right|
    \le \frac{1}{8}2s = \dfrac{s}{4}, \\
\|I_3\| &\le \int^s_{-s}
\left|e^{-i\xi t}\alpha(\vphi^{-t}\th_0)\right|
  \left\|(T^{t+N}-T^N)f\right\|\,dt
  \le \dfrac{1}{16}2s = \dfrac{s}{8}.
\end{aligned}$$
As a result,
\[ \|g\| \ge \|g(\th_0)\|=
  \|I_1+I_2+I_3\|\ge \|I_1\|-\|I_2\|-\|I_3\| \ge
  \dfrac{49s}{128}-\dfrac{s}{4}-\dfrac{s}{8}=\dfrac{s}{128}.\]
This proves Claim 2.
\end{pf*}

Returning to the proof of the theorem, we combine Claim 1
with Claim 2 to obtain $$\|\Gamma g-i\xi g\|\le \dfrac{8Cs}{N}
\le \dfrac{8C}{N}\cdot 128\|g\|.$$
Since $N\ge 2$ was arbitrary, this shows
$i\xi\in\sigma_{ap}(\Gamma)$.
\end{pf}





%------------------- section 4 ----------------
\section{Exponential Dichotomy}
\setcounter{equation}{0}

Let $\{T^t\}_{t\ge0}$ be an evolutionary semigroup
\eqref{evolSG}.  In this section, we relate the hyperbolicity of
the weighted translation operator $T=T^1=aV$, where
$a(\th)=\Phi(\vphi^{-1}\th,1)$, to the exponential dichotomy
of the LSPF $\hp^t$. One can assume in this section, that
$t\in\bbZ$.

We also relate the hyperbolicity of $T$ to the hyperbolicity of
weighted shift operators $\pi_\th(T)$, $\th\in \TH$.  The
representation $\pi_\th$ of $\frakB$ was introduced in Section~2
(see \eqref{pisubx}). For each $\th\in \TH$,
the weighted shift operator $\pi_\th(T)$ acts on $\linf$
by the rule
\[\pi_\th(T)=\diag\{\Phi(\vphi^{n-1}\th,1)\}_{n\in\bbZ}S,
\quad\text{where }
S\colon (x_n)_{n\in\bbZ}\mapsto(x_{n-1})_{n\in\bbZ}.\]
It is useful to make the observation that for $k\in\bbZ$,
\begin{equation}\lb{pixTk}
\pi_\th(T^k)=[\pi_\th(T)]^k
=\diag\{\Phi(\vphi^{n-k}\th,k)\}_{n\in\bbZ}S^k
=S^k\diag\{\Phi(\vphi^n\th,k)\}_{n\in\bbZ}.
\end{equation}

For $\bar x=(x_n)_{n\in\bbZ}$, and $\bar y=(y_n)_{n\in\bbZ}$ in
$\linf$, note that the equation $[I-\pi_\th(T)]\bar x=\bar y$ can be
expressed componentwise as \[ x_{n+1}-\Phi(\vphi^n\th,1)x_n=y_{n+1},
\quad n\in\bbZ.\] As seen in Section 5.4 (see also
\cite[Prop.~3.22]{LS}), the hyperbolicity of $\pi_\th(T)$ is equivalent
to the existence of exponential dichotomy for $\hp^t$ along the orbit
through $\th$, i.e., ``pointwise'' dichotomy (see \cite[Theorem
7.6.5]{Henry} and \cite{ChLe1}).

We begin with the definition of exponential dichotomy on $\TH$
(or ``global'' dichotomy \cite{ChLe1,ChLe2,Henry,HL,Mag,SSBan}).

\begin{defn}\lb{ED}
A linear skew product flow $\hp^t$ has {\em exponential
dichotomy} on $\TH$ if there exists a continuous projection
$P\colon \TH\to \calL_s(X)$ such that for $\th\in\TH$ and $t\ge0$,
\begin{list}{\alph{aa})}{\usecounter{aa}}
\item  $P(\vphi^t\th)\Phi(\th,t) =\Phi(\th,t)P(\th)$;
\item  $\Phi_Q(\th,t)$ is invertible from $\Im Q(\th)$ to
$\Im Q(\vphi^t\th)$;
\item there exist constants $M$, $\beta >0$ such that for $t>0$
$$
\|\Phi_P(\th,t)\| \leq Me^{-\beta t},\quad
\|[\Phi_Q(\th,t)]^{-1}\| \leq Me^{-\beta t}.
$$
\end{list}
\end{defn}

If a projection $P\colon \TH\to \calL(X)$ and a cocycle $\Phi$
satisfy a), above, then $\Phi_P(\th,t)$ and $\Phi_Q(\th,t)$
will be used to denote the restrictions
$\Phi(\th,t)P(\th)\colon \Im P(\th)\to\Im P(\vphi^t\th)$ and
$\Phi(\th,t)Q(\th)\colon \Im Q(\th)\to\Im Q(\vphi^t\th)$, respectively.

The main result of this section follows.  As usual, the set of
nonperiodic points of $\vphi^t$ is assumed to be dense in $\TH$.


\begin{thm}\lb{SPT}
The following are equivalent:
\newcounter{rr}
\begin{list}{\roman{rr})}{\usecounter{rr}}
\item $\sigma (T) \cap \bbT = \emptyset \text{ in  } C_0(\TH,X)$;
\item $\sigma(\pi_\th(T))\cap \bbT = \emptyset$, on $\linf$,  for all
$\th\in \TH$ and there exists a  constant $C>0$,  such that
\begin{equation}\label{unbddpix}
   \|[\pi_\th(T)-\lambda I]^{-1}\|_{\Blinf}\le C
   \text{ for all } \th\in \TH, \; \lambda \in \bbT;\end{equation}
\item The LSPF $\hp^t$ has exponential dichotomy.
\end{list}
\end{thm}

We note (see Remark~\ref{poinwise-global} in Section~5.6) that
condition \eqref{unbddpix} often follows automatically  from
$\sigma(\pi_\th(T))\cap \bbT = \emptyset$, $\th\in \TH$. For
results related to $i)\Leftrightarrow iii)$, see
\cite{Rab,Rau,Rau1,Rau2} where a quite different method was used.

We begin the proof of the theorem with
an observation about the operators $\pi_\th(T)$.

\begin{prop}\label{p4.3}
$\sigma(\pi_\th (T))$ is invariant
with respect to rotations centered at
the origin.  Moreover, for $\sigma (\pi_\th(T)) \cap \bbT =
\emptyset$, \[
\| [\lambda-\pi_\th(T)]^{-1} \|_{\Blinf} = \| [I-\pi_\th(T)]^{-1}
\|_{\Blinf},\quad \lambda\in\bbT.\]
\end{prop}

\begin{pf}
For $\omega\in\bbT$, define $\Lam=\diag\{\omega^n\}_{n\in\bbZ}$.
Then $\Lam$ is an invertible operator on $\linf$, and
$$
\Lam\pi_\th(\lambda-T)
\Lam^{-1}=\lambda-\omega\diag\{a(\varphi^{n}\th)\}_{n\in\bbZ}S
 =\omega(\omega^{-1}\lambda -\pi_\th(T)).
$$
Hence,
$\|[\lambda-\pi_\th(T)]^{-1}\|=\|[\omega^{-1}\lambda-\pi_\th(T)]^{-1}\|$
and the proposition follows. \end{pf}

As a result, statement $ii$) in Theorem~\ref{SPT} can be
replaced by the statement
 \bigskip
\begin{list}{$ii')$}
\item  $I-\pi_\th(T)$ is invertible
for all $\th \in \TH,$ and there exists $C>0$ such that
$\|[I-\pi_\th(T)]^{-1} \|_{\Blinf} \le C$ for all $\th \in \TH$.
\end{list}

\bigskip

The proof of Theorem \ref{SPT} will follow from a series of
lemmas.

The {\em Proof of $i)\Rightarrow ii')$} is
given by the following lemma.

\begin{lem}\lb{l4.3} If $b=I-T$ is an invertible operator on $\co$,
then $\pi_\th(b)$ is invertible on $\linf$ for all $\th\in \TH$.
Moreover, \[\|[\pi_\th(b)]^{-1}\|_{\Blinf}\leq \|b^{-1}\|_{\Bco}\] for
all $\th\in \TH$.
\end{lem}

\begin{pf}  We first show that $\pi_\th(b)$ is injective, uniformly
for $\th\in \TH$,  by showing that for any $\bar x\in\linf$,
\begin{equation}\lb{l4.3-1}
\| \pi_\th(b)\bar x\|_{\linf} \ge
(\|b^{-1}\|_{\Bco})^{-1} \|\bar x\|_{\linf}.
\end{equation}
 It suffices to prove \eqref{l4.3-1} for
finitely supported $\bar x=(x_n)^N_{n=-N}$.

Let $\th_0\in \TH$ be a
nonperiodic point of $\vphi^t$.  Let $\epsilon>0$.  Since $\th\mapsto
a(\th)x$ is continuous for all $x\in X$, there exists $\delta>0$ such
that for $B:=B(\th_0,\delta)$,
\begin{equation}\lb{l4.3-2}
\th\in B\ \text{ implies }
\ \|[a(\th_0)-a(\th)]x_n\|_X\ <\epsilon,\; \text{ for all }|n|\le N.
\end{equation}
Moreover, $\delta$ can be chosen so that, in addition,
$$
\vphi^n(B)\cap\vphi^k(B)=\emptyset\;\text{ for all }k\ne
n,\; |n|,|k|\le N.
$$
Choose a continuous function $\alpha\colon \TH\to[0,1]$ such that
$\alpha(\th_0)=1$, and $\alpha(\th)=0$ for $\th\notin B$.
Define $f\in \co$ by
$$f(\th)=\cases
\alpha(\vphi^{-n}\th)x_n,\quad
\text{if } \th\in\vphi^n(B),\; |n|\le N\\
0,\quad\text{otherwise}.
\endcases$$
Since $b^{-1}\in\Bco$, we have
\begin{equation}\lb{l4.3-3}
\|b^{-1}\| \cdot \|bf\| \ge \|f\|_{\co}
= \sup_{\th\in \TH} \|f(\th)\|_X =\|\bar x\|_{\linf}.
\end{equation}
Also, since
$f(\varphi^{-1}\th)=\alpha(\varphi^{-n}\th)x_{n-1}$
for $\th\in\varphi^n(B), \; |n|\le N$,
and $f(\varphi^{-1}\th)=0$ otherwise, we have
$$\begin{aligned}
\|bf\|_{\co}&=\sup_{\th\in \TH}\|f(\th)-a(\th)f(\varphi^{-1}\th)\|_X\\
&=\sup_{|n|\le N}\sup_{\th\in \TH}\|\alpha(\varphi^{-n}\th)x_n
  -a(\th)\alpha(\varphi^{-n}\th)x_{n-1}\|_X\\
&\le\sup_{|n|\le N}\sup_{\th\in \TH}\|x_n
  -a(\varphi^n\th)x_{n-1}\|_X\\
&\le\sup_{|n|\le N}\sup_{\th\in \TH}
     \{\|x_n-a(\varphi^n\th_0)x_{n-1}\|_X
     +\|a(\varphi^n\th_0)x_{n-1}
      -a(\varphi^n\th) x_{n-1}\|_X\}\\
&<\|\pi_{\th_0}(b)\bar x\|_{\linf} +\epsilon,
\end{aligned}$$
using \eqref{l4.3-2} in the last inequality.  So,
\begin{equation}\lb{l4.3-4}
\|bf\|_{\co}\le\|\pi_{\th_0}(b)\bar x\|_{\linf}.
\end{equation}
Combining \eqref{l4.3-3} and \eqref{l4.3-4} shows
$$
\|\pi_{\th_0}(b)\bar x\|_{\linf}
\ge\|bf\|_{\co}\ge\frac{1}{\|b^{-1}\|_{\Bco}}\|\bar x\|_{\linf}.
$$

We have shown that \eqref{l4.3-1} holds for all nonperiodic points
of $\vphi^t$.  If $\th_0$ is a periodic point of $\vphi^t$, then
choose a sequence of nonperiodic points $\{\th_k\}_{k=1}^\infty$ in
$\TH$ converging to $\th_0$. Since the map
$\th\mapsto \pi_\th(b)\bar x$ is continuous, \eqref{l4.3-1} holds, and
hence $\pi_\th(b)$ is uniformly injective for all $\th\in \TH$.


We now check that $\pi_\th(b)$ is surjective for all $\th\in \TH$.
Let $\bar x\in\linf$.  As before,
consider $\bar x=(x_n)_{n=-N}^N$.
First suppose $\th_0$ is a nonperiodic point of $\vphi^t$.
Define $f\in \co$ as above and note that since $b$ is surjective,
there exists a $g\in\co$ such that $bg=f$; that is,
$$
g(\th)-a(\th)g(\vphi^{-1}\th)=f(\th)\quad\text{ for all }\th\in \TH.
$$
Substituting $\th=\vphi^n\th_0$ into this equation and
defining  $\bar y\in\linf$ by $y_n=g(\vphi^n\th_0)$  for $|n|\le N$,
gives $\pi_{\th_0}(b)\bar y=\bar x$.

To address the case for which $\th_0$ is a periodic point of
$\vphi^t$, so that $\vphi^k\th_0=\th_0$ for some $k\in\bbZ$, we begin
with

\begin{casekone} Assume $\vphi \th_0=\th_0$.  If $T=aV$ is hyperbolic
on $\co$, then $\pi_{\th_0}(T)=\diag\{a(\th_0)\}_{n\in\bbZ}S$ is
hyperbolic on $\linf$.
\end{casekone}

\begin{pf} By hypothesis, $I-aV$ is surjective on $\co$,
and so $I-a(\th_0)$ is surjective on $X$.  Indeed, given $x\in X$,
choose $f\in\co$ with $f(\th_0)=x$ and then choose $g\in\co$ so that
$(I-aV)g=f$; setting $y=g(\th_0)$ gives $(I-a(\th_0))y=x$.   Therefore,
$a(\th_0)$ is hyperbolic on $X$.  As in Lemma 2.4 of \cite{LMS1},
this implies that $\diag\{a(\th_0)\}_{n\in\bbZ}S$ is hyperbolic on
$\linf$.
\renewcommand{\qed}{}\end{pf}

\begin{casek} Assume $\vphi^k\th_0=\th_0$.  If $T$ is hyperbolic on
$\co$, then $\pi_{\th_0}(T)$ is hyperbolic on $\linf$.
\end{casek}

\begin{pf} By standard spectral properties, it
suffices to consider hyperbolicity for $T^k$.  First
 note (see \eqref{pixTk}) that
\[ \pi_{\th_0}(T^k)
=S^k \diag\{\Phi(\vphi^n \th_0,k)\}_{n\in\bbZ}.\]
This can be expressed as the operator
\begin{equation}\lb{l4.3-5}
\hat S\diag\{d\}_{n\in\bbZ},\quad\text{on}\quad
\ell_{\infty}(\bbZ,X^k)\approx\linf,
\end{equation}
 where $\hat S$, on $\ell_{\infty}(\bbZ,X^k)$, and $d$, on
$X^k$, are defined as
$$\left(\begin{array}{ccc}
x_{n+1}\\  \vdots  \\ x_{n+k}
\end{array}\right)_{n\in\bbZ}
\mapsto
\left(\begin{array}{ccc}
x_{n+k+1}\\  \vdots  \\ x_{n+2k}
\end{array}\right)_{n\in\bbZ}
\text{and }
d=\left[\begin{array}{ccc}
\Phi(\th_0,k)&        &  \\
           & \ddots &  \\
           &        & \Phi(\varphi^{k-1}\th_0,k)
\end{array}\right].
$$
As in Lemma~2.4 of \cite{LMS1}, \eqref{l4.3-5} is hyperbolic on
$\ell_{\infty}(\bbZ,X^k)$ provided $d$ is hyperbolic on $X^k$.
Therefore, it suffices to show that
$\Phi(\th_0,k),\ldots,\Phi(\vphi^{k-1}\th_0,k)$ are hyperbolic on $X$.
Applying {\em Case $k=1$} with $\phi:=\vphi^k$
and the representation
$\pi_{\th,\phi}\colon a\mapsto \diag\{a(\phi^{-n}\th)\}_{n\in\bbZ}$
shows that
\[ \pi_{\th_0,\phi}(T^k)
=S^k\diag\{\Phi(\vphi^{kn}\th_0,k)\}_{n\in\bbZ}
=S^k\diag\{\Phi(\th_0,k)\}_{n\in\bbZ} \]
is hyperbolic.  As before, expressing this operator as
$\hat S\diag\{\Phi(\th_0,k)\}_{n\in\bbZ}$ on
$\ell_{\infty}(\bbZ,X^k)$ and applying Lemma~2.4 of \cite{LMS1}
shows that $\Phi(\th_0,k)$ is hyperbolic.  This proves the case
$k>1$
\end{pf}
Thus, the lemma is proved.
\end{pf}

The {\em Proof of $ii)\Rightarrow i)$} requires
showing that the invertibility of $b$ in $\frakB$ can be
derived from the invertibility of all its images $\pi_\th(b)$ in
$\Blinf$.  To do this,
we consider the algebra $\frakD$ of operators in $\Blinf$ as
defined in Section~2.  The next lemma proves a slightly stronger
statement than $ii)\Rightarrow i)$.


\begin{lem}\lb{l4.4} Let $b=\lambda I-T$.  Assume $\pi_\th(b)$ is
invertible in $\Blinf$ and that there exists a $C>0$ such that
$$
\|[\pi_\th(b)]^{-1}\|_{\Blinf}\le C\quad\text{ for all } \th\in \TH.
$$
Then $b^{-1}$ exists and is an element of $\frakB$.
\end{lem}

\begin{pf}  Let $b=I-T$; the proof remains the same
for $b=\lambda I-T$, $\lambda\in\bbT$.
Using the notation of Proposition~\ref{p2.8}, set
$d(\th):=\pi_\th(b)= I-\pi_\th(a)S$.
Since $\sup_{\th\in \TH} \|a(\th)\|_{\calL(X)}<\infty$,
it follows that \[\sum_k\sup_{\th\in \TH}\|d_k(\th)\|_{\Blinf}
=1+\sup_{\th\in \TH}\|\pi_\th(a)\|_{\Blinf} < \infty,\] and hence
$d(\th)$ satisfies the
conditions of Proposition \ref{p2.8}.
Consequently, for each $\th$, $\pi_\th(b)=d(\th)$
has an inverse that is in $\frakD$:
$$
[\pi_\th(b)]^{-1}= \sumk c_k(\th) S^k,
\quad\text{for some } c_k(\th)=\diag\{c_k^{(n)}(\th)\}_{n\in\bbZ}
\text{ in }\frakC, \; k\in\bbZ.$$
Moreover, since $\sup_{\th\in \TH}\|\pi_\th(b)\|_{\Blinf}\le C$,
Proposition \ref{p2.8}, shows that
\[\sum_k \sup_{\th\in \TH} \|c_k(\th)\|_{\Blinf} < \infty.\]

The lemma is proved by showing that for each $k\in\bbZ$,
\begin{equation}\lb{l4.4-1}
\th  \mapsto c^{(0)}_k(\th)
\text{ is a continuous, bounded function from } \TH\to \calL_s(X),
\end{equation}
(in the notation of Section 2, $c_k^{(0)}\in\frakA$)
and then showing that the operator $r:=\sumk c_k^{(0)}V^k$
is in $\frakB$ and satisfies $r=b^{-1}$.

Let $k\in\bbZ$, and fix $x\in X$ and $\th_0\in \TH$.
Define $\bar x=(x_n)_{n\in\bbZ}$
in $\linf$ by $x_n=x$ if $n=-k$, and $x_n=0$ if $n\ne -k$.
Then for any $\th\in \TH$,
$$\begin{aligned}
\|[c^{(0)}_k(\th)-c^{(0)}_k(\th_0)]x\|_X
&=\left\|\sumj[c^{(0)}_j(\th)-c^{(0)}_j(\th_0)]x_{-j}\right\|_X\\
&\le\sup_{n\in\bbZ}\left\|\sumj
   [c^{(n)}_j(\th)-c_j^{(n)}(\th_0)]x_{n-j} \right\|_X \\
&=\left\|\left(\sumj\diag
  \{c^{(n)}_j(\th)-c_j^{(n)}(\th_0)\}_{n\in\bbZ}S^j\right)
  \bar x\right\|_{\linf}\\
&=\left\|\left(\sumj[c_j(\th)-c_j(\th_0)]S^j
  \right)\bar x\right\|_{\linf}\\
&=\left\| [\pi_\th(b)]^{-1}[\pi_{\th_{0}}(b)-
   \pi_\th(b)][\pi_{\th_0}(b)]^{-1}\bar x\right\|_{\linf}\\
&\le C\left\|[\pi_{\th_0}(b)-\pi_\th(b)]\bar y\right\|_{\linf},
\end{aligned}
$$
where the last inequality comes from letting
$\bar y=[\pi_{\th_0}(b)]^{-1}\bar x$ and using the assumption
$\|[\pi_\th(b)]^{-1}\|\le C$.  Since the map
$\th\mapsto \pi_\th(b)\bar y$ is continuous for any $\bar y\in\linf$,
$c_k^{(0)}$ is continuous at $\th_0$.


Moreover, note that $c_k(\th)=\diag\{c_k^{(n)}(\th)\}_{n\in\bbZ}$
satisfies, for each $n\in\bbZ$, the inequality
\[\|c_k(\th)\|_{\Blinf}\ge\|c_k^{(n)}(\th)\|_{\calL(X)}.\]
  Therefore,
 $\sum_k \sup_{\th\in \TH} \|c_k(\th)\|_{\Blinf} < \infty$
implies, in particular, that
$\sup_{\th\in \TH} \|c^{(0)}_k(\th) \|_{\calL(X)} < \infty$, and so
each $c_k^{(0)}$ is bounded.  Thus, \eqref{l4.4-1} holds.


Also, since
$\|c^{(0)}_k\|_{\Bco}=\sup_{\th\in \TH}\|c^{(0)}_k(\th)\|_{\calL(X)}$,
it follows that
$$\sumk \|c^{(0)}_k \|_{\Bco}
=\sumk\sup_{\th\in \TH}\|c^{(0)}_k(\th)\|_{\calL(X)}
\le \sumk\sup_{\th\in \TH}\|c_k(\th)\|_{\Blinf} < \infty.$$
Therefore, the operator $r:=\sumk c^{(0)}_k V^k$ is in $\frakB$.

Now observe that
\begin{equation}\lb{l4.4-2}
c^{(n)}_k(\th) = c^{(0)}_k (\vphi^n\th), \quad n\in \bbZ.
\end{equation}
Indeed, for any $n\in \bbZ$, $\pi_\th(a)S^n=S^n\pi_{\vphi^n\th}(a)$,
and so
\[ \pi_{\vphi^n\th} (b)
 =I-\diag\{a(\varphi^{n+k}\th)\}_{k\in\bbZ}S
 =S^{-n} \pi_\th(b)S^n. \]
Therefore,
$$\begin{aligned}
[\pi_{\vphi^n\th}(b)]^{-1} &= S^{-n}[\pi_\th(b)]^{-1}S^{n}
=S^{-n}\left(\sumk\diag\{c^{(i)}_k(\th)\}_{i\in \bbZ}
  S^k\right)S^{n}\\
&=\sumk\diag\{c^{(i+n)}_k(\th)\}_{i\in\bbZ}S^k.
\end{aligned}$$
On the other hand,
\[
[\pi_{\vphi^n\th}(b)]^{-1}= \sumk c_k(\vphi^n\th)S^k=
\sumk\diag\{c^{(i)}_k
(\vphi^n\th)\}_{i\in\bbZ}S^k,\]
so \eqref{l4.4-2} holds.

As a consequence, $rb=br=I$.  Indeed, using \eqref{l4.4-2},
$$\begin{aligned}
\pi_\th(r) &= \sumk \pi_\th(c^{(0)}_k) S^k
= \sumk \diag\{c^{(0)}_k(\varphi^n\th)\}_{n\in\bbZ}S^k \\
 &= \sumk \diag\{c^{(n)}_k(\th)\}_{n\in\bbZ}S^k
=\sumk c_k(\th)S^k = [\pi_\th(b)]^{-1}.
\end{aligned}$$
Therefore, $I=\pi_\th(r)\pi_\th(b) = \pi_\th(rb)$, and so
$\pi_\th(rb-I) = 0$ for all $\th$.
One checks directly that for
$\pi_\th\colon\frakB\to\frakD$, $\bigcap_{\th\in \TH}\Ker\pi_\th=\{0\}$ and
so $rb=I$. \end{pf}

The  proof $i)\Rightarrow iii)$  uses  the
following corollary to  Lemma~\ref{l4.4}.

\begin{cor}\lb{c4.5} If $\sigma(T)\cap\bbT = \emptyset$,
then the Riesz projection, $\calP$, corresponding to $\sigma(T)
\cap \bbD$ has the form $\calP f(\th)=P(\th)f(\th)$, $f\in\co$, for some
bounded, continuous projection-valued function $P:\TH\to \calL_s(X)$.
\end{cor}

\begin{pf}  If $\sigma(T)\cap\bbT =\emptyset$, then statement $ii)$
of Theorem~\ref{SPT} holds, and so Lemma~\ref{l4.4} applies
and shows $(\lambda I -T)^{-1}$ is in $\frakB$ for all
$\lambda\in\bbT$.  Consequently,
$\calP=\frac{1}{2\pi i}\int_{\bbT}(\lambda I-T)^{-1}\,d\lambda$ is
an element of $\frakB$.  Proceeding exactly as in
\cite{LS} or \cite[Lemma 3]{LATGDDE}, one can show that $\calP \in
\frakA$. \end{pf}

Before proceeding, we make an additional observation.

\begin{prop}\lb{rppi} Assume
$\sigma(T)\cap\bbT = \emptyset$, and let
$\calP$ be the Riesz
projection corresponding to $\sigma(T)\cap\bbD$, where $\calP
f(\th)=P(\th)f(\th)$, as above.
Then the Riesz projection $\calP_\th$ in $\Blinf$
corresponding to $\sigma(\pi_\th(T))\cap\bbD$ is
given by \[\pi_\th(\calP)=\diag\{P(\vphi^n\th)\}_{n\in\bbZ}.\]
\end{prop}

\begin{pf}  As in the previous corollary, $\calP\in\frakB$.
Moreover,
$\pi_\th((\lambda I-T)^{-1})=[\lambda I-\pi_\th(T)]^{-1}$ is in
$\frakD$, for $\lambda\in\bbT$, and the Riesz projection
corresponding to $\sigma(\pi_\th(T))\cap\bbD$, given by
$\calP_\th=\frac{1}{2\pi i} \int_{\bbT}[\lambda
I-\pi_\th(T)]^{-1}\,d\lambda$, is in $\frakD$. Since  $\pi_\th\colon
\frakB\to\frakD$ is a continuous homomorphism,
$\calP_\th=\pi_\th(\calP)$.  Since $\calP$ is given by $P(\cdot)$, if
follows that
$\calP_\th=\pi_\th(\calP)=\diag\{P(\vphi^n\th)\}_{n\in\bbZ}$.
\end{pf}


The {\em Proof of $i)\Rightarrow iii)$} is a consequence of
the next lemma.

\begin{lem}\lb{hypED}  If $\sigma(T)\cap\bbT = \emptyset$, then
the LSPF $\widehat\vphi^t$ has exponential dichotomy.
\end{lem}

\begin{pf}  Let $\calP$ be the Riesz projection corresponding to
$\sigma(T)\cap\bbD$.  By Corollary~\ref{c4.5}, $\calP$ is
 given by $(\calP f)(\th)=P(\th)f(\th)$ for some
projection-valued function $P\colon \TH\to X$.
We show that $P$ satisfies the properties of Definition \ref{ED}.
As already  shown, the statement $\sigma(T)\cap\bbT
= \emptyset$ implies statement $ii)$ of
Theorem~\ref{SPT}.  We use the latter to show that $\Phi(\th,t)$ is
invertible as an operator from $\Im Q(\th)$ to $\Im Q(\vphi^t\th)$.

Fix $\th\in \TH$.  Statement $ii)$ implies that
$\pi_\th(T)\cap\bbT=\emptyset$, and Proposition \ref{rppi} shows that
the corresponding Riesz projection is given by
$\calP_\th=\pi_\th(\calP)= \diag\{P(\vphi^n\th)\}_{n\in\bbZ}$.  Set
$\calQ_\th=I-\calP_\th$, and let $k\in\bbZ$.  Since
$\pi_\th(T)\calQ_\th$ is invertible on $\Im\calQ_\th$, so is
$[\pi_\th(T)\calQ_\th]^k=\pi_\th(T^k)\calQ_\th$.
Hence, for any $\bar y=(y_n)_{n\in\bbZ}\in\Im\calQ_\th$,
there exists a unique $\bar x=(x_n)_{n\in\bbZ}\in\Im\calQ_\th$
such that (see \eqref{pixTk})
$$
\{\Phi(\vphi^{n-k}\th,k)x_{n-k}\}_{n\in\bbZ}=
\pi_\th(T^k)\bar x=\bar y.
$$
The fact $\bar y\in\Im\calQ_\th$ means precisely that
$y_n\in\Im Q(\vphi^n\th)$ for every $n\in\bbZ$.  So fix
$y\in\Im Q(\vphi^k\th)$ and define $\bar y$ by $y_n=y$
for $n=k$, and $y_n=0$, for $n\ne k$.  There exists a unique
$\bar x\in\Im\calQ_\th$ such that
$$
\Phi(\vphi^{n-k}\th,k)x_{n-k}=y_n,\quad n\in\bbZ.
$$
In particular, for $n=k$, $\Phi(\th,n)x_0=y$.  This shows that
$\Phi(\th,n)$ is invertible from $\Im Q(\th)$ to $\Im Q(\vphi^n\th)$ for
all $\th\in \TH$, $n\in\bbZ$.

To see that $\Phi(\th,t)$ is invertible for all $t\in\bbR$, fix $t$
and choose $n\in\bbZ$ such that $t\in[n,n+1)$.  Since
$\Phi(\th,n)$ is invertible, the identity
$$
\Phi(\th,t)=\Phi(\vphi^n\th,t-n)\Phi(\th,n)
$$
shows that it suffices to prove $\Phi(\vphi^n\th,t-n)$ is invertible.
But the identities
$$\begin{aligned}
\Phi(\vphi^n\th,1)&=\Phi(\vphi^t\th,n+1-t)\Phi(\vphi^n\th,t-n)\\
\Phi(\vphi^{t-1}\th,1)&=\Phi(\vphi^n\th,t-n)\Phi(\vphi^{t-1}\th,n-(t-1))
\end{aligned}$$
show, respectively, that $\Phi(\vphi^n\th,t-n)$ has a left and a
right inverse.  Hence $\Phi(\th,t)$ is invertible from $\Im Q(\th)$ to
$\Im Q(\vphi^t\th)$.

Part $b)$ for $t\in\bbZ_+$ follows from \eqref{pixTk} and the fact
that $\pi_\th(T)\calP=\pi_\th(T)\calP$. As in \cite[Prop.~3.10]{LS}, the
statement is seen to hold  for $t\in\bbR_+$.

Finally, since
$\sigma(T^t)\cap \bbT=\emptyset$, there
exist $M,\beta>0$ such that
\[ \sup_{t\in\bbR}\|\Phi(\th,t) P(\th)\|_{\calL(X)}
=\|\calP T^t\calP\|_{\Bco}\le
Me^{-\beta t}.\]  The first inequality in Definition \ref{ED} $c)$
follows. The second inequality is shown similarly.
\end{pf}

The {\em Proof of $iii)\Rightarrow i)$} is
trivial.  This completes the proof of Theorem~\ref{SPT}.

  Let us make the following
observation concerning the dynamical spectrum, $\Sigma$
(see the Introduction for the definition).
 As a result of Theorem~\ref{SPT} and Corollary~\ref{c4.5},
the dynamical spectrum $\Sigma$ coincides with
$\ln|\sigma(T)\setminus\{0\}|$.  An application of Theorem \ref{SMT}
gives the formula \eqref{spectra} in the introduction.
Also, the  spectral subbundles for $\hp^t$ are determined by the
spectral projections for $T$.





%------------------- section 5 ----------------
\section{Consequences}
\setcounter{equation}{0}

In this section we formulate a
variety of consequences which follow, almost immediately, from the
results of the previous sections.

\subsection{Invertibility of $\Gamma$}

An immediate consequence of the Theorem~ \ref{SMT} and
Theorem~\ref{SPT} ( see $i)\Leftrightarrow iii)$) is the fact that a
LSPF has exponential dichotomy if and only if the spectrum of the
generator $\Gamma$ of the corresponding evolutionary semigroup
\eqref{evolSG} satisfies  $\sigma(\Gamma)\cap i\bbR=\emptyset$.
Since $\sigma(\Gamma)$ is invariant with respect to translations
along $i\bbR$, we conclude:

\begin{cor}\lb{c5.1}  The LSPF $\hp^t$ has exponential
dichotomy if and only if $\Gamma$ is invertible.
\end{cor}

Now recall that in the norm continuous compact
setting, the generator of the evolutionary semigroup is given by
\eqref{genucc}.  Consequently, the LSPF generated by the cocycle
in Example~\ref{exuvareqn} is hyperbolic if and only if
the equation
\[\left.\dfrac{d}{dt}f\circ\vphi^t(\th)\right|_{t=0}-A(\th)f(\th)=g(\th)\]
has a unique solution $f$ for every $g\in\co$.

Applying Corollary~\ref{c5.1} to  Example~\ref{exslset}, we
note that an evolutionary family $\{U(\tau,s)\}$, ${\tau\ge s}$
has exponential dichotomy if and only if $\Gamma$ is invertible
on $C_0(\bbR,X)$ (cf. \cite{LMS1,LMS2,LatRand}).  In the special case
where $A(\tau)\equiv A_0$ generates a $C_0$ semigroup
$\{e^{tA_0}\}_{t\ge0}$ on $X$
(here, $U(\tau,s)=e^{(\tau-s)A_0}$),
this gives a result from \cite{Pruss}: a $C_0$ semigroup
$\{e^{tA_0}\}_{t\ge0}$ is hyperbolic if and only if the equation
$f'-A_0f=g$ has a unique solution for every $g\in\coR$.

On the other hand, for the norm continuous line setting
(Example~\ref{nacp}), we conclude that \eqref{difeqn} has
exponential dichotomy if and only if $\Gamma=-d/dt+A(\cdot)$ is
invertible \cite{Gohb,Palm}, or, equivalently,  the equation
$f'(t)-A(t)f(t)=g(t)$, $t\in\bbR$, has a unique solution for every
$g\in C_0(\bbR,X)$ (see \cite{DK,MS}).

\subsection{Roughness of the dichotomy}

In this subsection we give a very short proof of the facts that the
exponential dichotomy persists under small perturbations.

\begin{thm}\lb{roughness}
Assume that the LSPF $\hp^t_1$ over the flow $\varphi^t$
generated by a cocycle $\Phi_1(\th,t)$ has exponential dichotomy.
Then there exists $\epsilon > 0$ such that for every cocycle
$\Phi_2(\th,t)$ satisfying
\begin{equation}\lb{rough-1}
\sup\sb{\th\in \TH} \|\Phi_1(\th,1)-\Phi_2(\th,1) \|_{\calL(X)}<\epsilon,
\end{equation}
the LSPF $\hp^t_2$ over $\vphi^t$ generated by
$\Phi_2(\th,t)$ also has the exponential dichotomy.
\end{thm}

\begin{pf}
By Theorem~\ref{SPT}, $i)\Leftrightarrow iii)$, the operator
$T_1$, $(T_1f)(\th) = \Phi_1(\vphi^{-1}\th,1) f(\varphi^{-1}\th), $ is
hyperbolic in $\co$.  Note that for $(T_2f)(\th)=\Phi_2(\varphi^{-1}\th,
1)f(\varphi^{-1}\th)$,
$$\begin{aligned}
\|T_1 - T_2\|_{\Bco} &= \sup\sb{\|f\|_{\co}=1} \left\|\left[\Phi_1
(\vphi^{-1}\cdot\,,1)-\Phi_2(\vphi^{-1}\cdot\,,1)\right]
f(\vphi^{-1} \cdot\,)\right\|_{\co}\\
&\le \sup_{\th\in \TH}\|\Phi_1(\th,1) -\Phi_2(\th,1)\|_{\calL(X)}.
\end{aligned}$$
For sufficiently small $\epsilon >0$, \eqref{rough-1} implies
$\sigma(T_2)\cap\bbT =\emptyset$.
\end{pf}

>From Corollary~\ref{c5.1} we derive the following.

\begin{cor}  Let $\hp_1^t$ and $\hp_2^t$ be two LSPFs over $\vphi^t$,
and let $\Gamma_1$ and $\Gamma_2$ denote the respective generators of
the corresponding evolutionary semigroups \eqref{evolSG}.  If
$\hp_1^t$  has exponential dichotomy, then there exists
$\epsilon>0$, $\epsilon=\epsilon(\hp_1^t)$ such that
$\|\Gamma_1-\Gamma_2\|<\epsilon$ implies $\hp_2^t$ also has exponential
dichotomy. \end{cor}

Since the ``typical'' generator $\Gamma$ for a variational equation
\eqref{vareqn} has the form \eqref{genucc}, this can be used when
the unbounded operators $A(\th)$ are perturbed by  bounded operators
such that $\Gamma_1-\Gamma_2$ is bounded.

For the strongly continuous setting, we have the following
 roughness result (cf. \cite{Palm2}) for the variational equation in
Example~\ref{bddpertconstcoc}. Recall that in that example the
generator of the evolutionary semigroup is identified in
\eqref{A0A1}.

\begin{cor}  Let $A_0$ be a generator of a $C_0$ semigroup
$\{e^{tA_0}\}_{t\ge0}$ on $X$, and assume
$A_1$ and $A_2\colon \TH\to \calL_s(X)$ be bounded and
continuous. Consider the equations:
\[
\dfrac{dx}{dt}=A_0x(t)+A_1(\vphi^t\th)x(t),\quad
\dfrac{dx}{dt}=A_0x(t)+A_2(\vphi^t\th)x(t), \quad \th\in \TH,\;
t\in\bbR \]
(see \eqref{vareqn}--\eqref{pertA}). Assume that the LSPF $\hp_1^t$ over
$\vphi^t$ generated by the first equation has exponential dichotomy.
Then there exists $\epsilon>0$, $\epsilon=\epsilon(\hp_1^t)$ such that
$\|A_1(\th)-A_2(\th)\|<\epsilon$ implies the LSPF $\hp_2^t$ over
$\vphi^t$ generated by the second equation has exponential dichotomy,
provided the both cocycles are strongly continuous. \end{cor}

This gives an alternate way of addressing the situation of Theorem~5.2
in \cite{ChLe1}.

\subsection{Green's function}

We describe the existence of
exponential dichotomy for $\hp^t$ in terms of the existence and
uniqueness of a Green's function. Let $P\colon \TH\to \calL_s(X)$
be a bounded continuous projection-valued function that satisfies
the following properties for  $\th\in \TH$ and $t>0$:
\begin{list}{\alph{aa})}{\usecounter{aa}}
\item $\Phi(\th,t)P(\th) = P(\varphi^t\th) \Phi(\th,t)$;
\item $\Phi(\th,t)$ is invertible as an operator from
      $\Im Q(\th)$ to $\Im Q(\varphi^t\th).$
\end{list}
Define
\begin{equation}\lb{Gxt}
G(\th,t)=
\begin{cases}
   \Phi_P(\th,t), & t>0\\
   \Phi_Q(\th,t), & t<0.
\end{cases}
\end{equation}
Here we have denoted:
\begin{align*}
\Phi_P(\th,t)  &= P(\vphi^t\th) \Phi(\th,t) P(\th)\; \text{ for } t>0\\
\Phi_Q(\th,t) &= \left[Q(\vphi^{-t}\th) \Phi(\th,-t) Q(\th)\right]^{-1}
\;\text{ for } t<0,
\end{align*}
and used a) and b).

\begin{defn}\lb{Grfcn}  We say that the LSPF $\hp^t$ has a Green's function
if there exists a bounded continuous projection $P\colon \TH\to \calL_s(X)$,
such that for $G$ from
\eqref{Gxt}  the operator
\begin{equation}\lb{Grfcn-1}
(\widehat{G}f)(\th) = \int^{\infty}_{-\infty} G(\vphi^{-t}\th,t)
f(\vphi^{-t}\th)\,dt,\quad \th\in \TH,
\end{equation}
is bounded on $\co$.
\end{defn}
The following result generalizes \cite{Sam} (see also
\cite{LATGDDE}).

\begin{thm}\lb{t5.5}
The LSPF $\hp^t$ has exponential dichotomy if and only if
the Green's function exists and is unique.
\end{thm}

\begin{pf}
For every $P(\cdot)$ satisfying a)-b), define
$\calP\in\Bco$ by $\calP f(\th)=P(\th)f(\th)$, and set
$\calQ=I-\calP$.  Then $\calP$ commutes with $T^t$, and
$T^t_Q\equiv \calQ T^t\calQ$ is invertible in $\Im \calQ$.  We
note that \eqref{Grfcn-1} can be rewritten as
$\widehat{G} = -\tilde{G}$, where
\begin{equation}\lb{t5.5-1}
\tilde{G}f=\int^{\infty}_0 T^{-t}_Qf\,dt
  -\int^{\infty}_0 T^t_Q f\,dt.
\end{equation}

If $\Gamma$ denotes the generator of $T^t$, then the
Spectral Mapping Theorem, \ref{SMT}, shows that
$\Gamma^{-1}\in \Bco$ if and only if $T^t$ is hyperbolic.

Assume the Green's function, $G$, exists and is unique.
Then $\tilde{G}$ is bounded.  By \cite[Lemma 4.2]{LatRand} then
$\Gamma$ is invertible
and $\Gamma^{-1} = \tilde{G}.$  Since $T^t$ is hyperbolic,
$i)\Rightarrow iii)$ of Theorem \ref{SPT} shows that $\hp^t$
has exponential dichotomy.

Conversely, assume $\hp^t$ has exponential dichotomy.
Then $T^t$ is hyperbolic with Riesz projection $\calP=P(\cdot)$.
By \cite{BAG} this projection is unique.  The standard
norm-estimates in \eqref{t5.5-1} shows that $\tilde{G}$ and
$\widehat{G}$ are bounded.
\end{pf}

\subsection{Pointwise dichotomy}

In this subsection we consider the interrelation between
``global'' and ``pointwise'' dichotomies. Let $\bbK$ denote either
$\bbR$ or $\bbZ$. Fix a point $\th_0\in \TH$.
We define an exponential dichotomy of $\hp^t$ over the orbit
through $\th_0$ as  follows.

\begin{defn} \label{pointdich}
The LSPF $\hp^t$ has {\em exponential dichotomy}
over $\bbK$ at $\th_0$ if for all $\tau\in\bbK$, there exists
a projection $P(\vphi^\tau \th_0)\in \calL(X)$  such that
\begin{list}{\alph{aa})}{\usecounter{aa}}
\item  $P(\vphi^\tau \th_0)\Phi(\th_0,\tau) =\Phi(\th_0,\tau)P(\th_0)$;
\item  $\Phi_Q(\vphi^\tau \th_0,t)$ is invertible from
       $\Im Q(\vphi^\tau \th_0)$ to $\Im Q(\vphi^{\tau+t}\th_0)$ for all $t>0$
       and $\tau\in\bbK$;
\item there exist positive constants $M=M(\th_0)$, $\beta=\beta(\th_0)$
such that for all $\tau\in\bbR$ and $t>0$:
$$\|\Phi_P(\vphi^\tau \th_0,t) \|
\leq Me^{-\beta t},\quad
\|\left[\Phi_Q(\vphi^\tau \th_0,t) \right]^{-1}\|
\leq Me^{-\beta t}.
$$
\end{list}
For $\bbK=\bbR$ we require, in addition, that
$\tau\mapsto P(\vphi^\tau \th_0)$ is a (strongly) continuous function
from $\bbR$ to $\calL_s(X)$.
\end{defn}

For invertible-valued cocycles $\Phi\colon \TH\times \bbK\to \calL(X)$,
Definition~\ref{pointdich} is
equivalent to the classical definition of exponential
dichotomy of a LSPF at a point:  There exists a projection $P$ and
constants $M$, $\beta$, such that $\|\Phi( \th_0,s') P\Phi^{-1}(
\th_0,s)\| \leq Me^{-\beta (s'-s)}$ for $s'\geq s$, and
$\|\Phi( \th_0,s') Q\Phi^{-1}( \th_0,s)\|
\leq Me^{-\beta (s-s')}$ for $s\geq s'$
(see, e.g., \cite{SSDich}).

The following assertion has appeared previously.  For the sake of
completeness, we include a brief proof that is consistent with the
approach of the present paper (see also \cite{ChLe1},
\cite[Thm.~7.6.5]{Henry}, and \cite[Thm.~3.22]{LS}).

\begin{lem}\label{invpx} The LSPF $\hp^t$ has exponential dichotomy
over $\bbZ$ at $\th_0\in \TH$ if and only if $\pi_{\th_0}(T)$ is
hyperbolic in $l_\infty(\bbZ;X)$.
\end{lem}

\begin{pf} Indeed, the spectral radius $r$ of
$\pi_{\th_0}(T)\calP_{\th_0}$ is given by the formula
\[r=\lim_{k\to \infty}\left( \sup_{n\in\bbZ}
\|\Phi(\vphi^n\th_0,k)P(\vphi^n\th_0)\|\right)^{1/k},\]
and the exponential dichotomy of $\hp^t$ at $\th_0$ implies
the hyperbolicity of $\pi_{\th_0}(T)$ with Riesz projection
\begin{equation}\label{RPx}
\calP_{\th_0}=\diag \{P_n\}_{n\in\bbZ}
\end{equation}
for $P_n:=P(\vphi^n\th_0)$.

Conversely, assume $\pi_{\th_0}(T)\cap\bbT=\emptyset$. Then for all
$\lambda\in\bbT$, $[\lambda I-\pi_{\th_0}(T)]^{-1}\in\frakD$.
Therefore, as in Corollary \ref{c4.5}, the Riesz projection
$\calP_{\th_0}$ corresponding to  $\sigma(\pi_{\th_0}(T)\cap\bbD$ is
an element of $\frakD$.  Proceeding as in \cite[Lemma~3]{LATGDDE},
one can show that $\calP_{\th_0}\in\frakC$.  I.e., there exist
operators $P_n$ in $\calL(X)$ such that
$\calP_{\th_0}=\diag\{P_n\}_{n\in\bbZ}$.
Now define $P(\vphi^n\th_0):=P_n$ for each $n\in\bbZ$.
This defines projections in $\calL(X)$, which, by the fact that
$\pi_{\th_0}(T)\calP_{\th_0}= \calP_{\th_0}\pi_{\th_0}(T)$, are seen to
satisfy part  $a)$ of Definition \ref{pointdich}:
\[ P(\vphi^\tau \th_0)\Phi(\th_0,\tau) =\Phi(\th_0,\tau)P(\th_0),
\quad \tau\in\bbZ \]
(use \eqref{pixTk}).
Also, as in the proof of Lemma \ref{hypED},
$\Phi_Q(\th_0,k)\colon \Im Q(\th_0)\to\Im Q(\vphi^k\th_0)$
is invertible for all $\th\in \TH$, $k\in\bbZ$.
In particular, $\Phi_Q(\vphi^n\th_0,k)\colon
\Im Q(\vphi^n\th_0)\to\Im Q(\vphi^{n+k}\th_0)$
is invertible for all $n,k\in\bbZ$.

The estimates in Definition \ref{pointdich} $c)$ are also verified
as in Lemma \ref{hypED}.
\end{pf}

Combining this with Theorem~\ref{SPT} gives
the following fact.

\begin{cor}\label{pntw} The LSPF $\hp^t$ has exponential dichotomy
on $\TH$ if and only if it has exponential dichotomy at every
$\th\in \TH$ and $\|[\pi_\th(T)-\lambda I]^{-1}\|\le C $,
$\th\in\TH$, $\lambda\in\bbT$. \end{cor}


\subsection{Evolutionary semigroups along trajectories}

Next we relate the exponential dichotomy of
$\hp^t$ to an associated evolutionary semigroup
defined on $C_0(\bbR,X)$ along each trajectory of the flow
(cf.~\cite{LATGDDE,LS}).  Fix $\th\in \TH$, and consider the semigroup
$\{\Pi_\th^t\}_{t\ge0}$ defined by
$$
(\Pi^t_\th f)(s)=\Phi(\vphi^{s-t}\th, t)f(s-t),\quad s\in\bbR,\; t\ge
0,\; f\in C_0(\bbR,X).
$$
Let $L_\th$ denote the generator.  The semigroup
$\{\Pi_\th^t\}_{t\ge0}$ is, in fact, an
evolutionary semigroup in the sense of the strongly continuous
line setting.  Indeed, for $s\ge \tau$, set $U(s,\tau) =
\Phi(\vphi^{\tau}\th, s-\tau)$. Then $U(s,s) = I$, and for
$s\ge r\ge \tau$,
$$\begin{aligned}
U(s,\tau)=\Phi(\vphi^{\tau}\th, s-t)&=\Phi(\vphi^{r-\tau}
(\vphi^{\tau}\th), s-r) \Phi(\vphi^{\tau}\th, r-t)\\
&=\Phi(\vphi^r\th, s-r) \Phi(\vphi^{\tau}\th, r-\tau) \\
&=U(s,r)U(r,\tau).
\end{aligned}$$
Hence, $\{U(s,\tau)\}_{s\ge \tau}$ is an evolutionary family
on $X$, and $(\Pi^t_\th f)(s) = U(s,s-t) f(s-t)$.
As shown in \cite{LMS1,LMS2} (see also \cite{Rau1,Rau2}), any such
evolutionary semigroup has the following properties:

\begin{lem}\lb{l5.2}
$\sigma(\Pi^t_\th)$ is invariant under rotations centered at origin,
$\sigma(L_\th)$ is invariant under translation along $i\bbR$, and the
spectral mapping theorem holds:
$$
\sigma(\Pi^t_\th) \bs \{0\} = e^{t\sigma(L_\th)}, \quad t> 0.
$$
Further, when $\sigma(\Pi^1_\th)\cap\bbT=\emptyset$, the
corresponding Riesz projection $\calP$ is an operator of
multiplication: $(\calP f)(s)=P(s)f(s)$, for some
bounded, continuous projection-valued $P:\bbR\to \calL_s(X)$.
\end{lem}

Theorem~\ref{SPT} and Lemma~\ref{invpx} give the following
result.
\begin{thm}\lb{t5.3}
The following are equivalent
\setcounter{rr}{0}
\begin{list}{\roman{rr})}{\usecounter{rr}}
\item  The LSPF $\widehat{\vphi}^t$ has exponential dichotomy;
\item  $\sigma(\Pi^1_\th)\cap\bbT = \emptyset$ for all $\th\in \TH$, and
there exists $C>0$ such that
$$
\|\left[\Pi^1_\th-\lambda I\right]^{-1} \|_{B(C_{0}(\bbR; X))}\le C,
  \quad \th\in \TH,\; \lambda\in\bbT;
$$
\item $\sigma(L_\th)\cap i\bbR=\emptyset$ for all $\th\in \TH$, and
there exists $C>0$ such that
$$
\|\left[i\xi-L_\th\right]^{-1}\|_{B(C_{0}(\bbR;X))}\le C,\quad \th\in
\TH,\; \xi\in\bbR.
$$
\end{list}
\end{thm}

As in Theorem~\ref{SPT} (see also [LMS1]), one
can replace the inequalities in $ii)$ and $iii)$ by the
inequalities
$$
\|[\Pi^1_\th-I]^{-1}\|_{B(C_{0}(\bbR;X))} \le C,\;\text{ and }\;
\|L^{-1}_\th \|_{B(C_{0}(\bbR; X))} \le C, \quad \th\in \TH,
$$
respectively. We note that the operators $L_\th$ are
differential operators of the first order, and
$i)\Leftrightarrow iii)$ reduces the problem of the existence of
exponential dichotomy on $\TH$ to the problem of invertibility
of these operators.

\begin{pf} Lemma \ref{l5.2} proves the equivalence of
$iii)$ and $ii)$. To prove $i) \Leftrightarrow ii)$, we
introduce operators on the space
$$
C_0(\bbR\times \TH,X) = C_0(\TH,C_0(\bbR,X)) = C_0(\bbR,C_0(\TH,X))
$$
defined as
$$\begin{aligned}
(\Pi h)(s,\th) & = \Phi(\vphi^{s-1}\th, 1) h(s-1, \th),\\
(\hat{T}h)(s,\th) & = \Phi(\vphi^{-1}\th,1) h(s-1,\vphi^{-1}\th),\\
(Jh)(s,\th) & = h(s,\vphi^s\th),
(J^{-1}h)(s,\th) = h (s,\vphi^{-s}\th),
\end{aligned}$$
for $s\in\bbR$, $\th\in \TH$, $h\in C_0(\bbR\times \TH,X)$.
These operators satisfy
\begin{equation}\lb{5-1}
J^{-1} \Pi J=\hat{T}.
\end{equation}
Indeed, for $r_1(s,\th) = h(s,\vphi^s\th)$ one has $(\Pi r_1) (s,\th) =
\Phi(\vphi^{s-1}\th,1) h(s-1,\vphi^{s-1}\th)$, and $(J^{-1}\Pi J h)
(s,\th) = \Phi (\vphi^{s-1} (\vphi^{-s} \th), 1)h(s-1,\vphi^{s-1}
(\vphi^{-s}\th) = (\hat{T}h)(s,\th).$

Next, note that for a function $F\colon \TH\to C_0(\bbR, X)$, the
operator $\Pi$ acts as the multiplication by $\Pi_\th^1$:
$(\Pi F)(\th) = \Pi_\th^1F(\th)$.  Hence,
for $\lambda\in\bbT$, the operator $\lambda-\Pi$ of multiplication
by $\lambda-\Pi^1_\th$ is invertible on $C_0(\TH,C_0(\bbR,X))$ if
and only if $\lambda-\Pi^1_\th$ is invertible on $C_0(\bbR,X)$ for
each $\th\in \TH$, and $\|(\lambda-\Pi^1_\th)^{-1}\|\le C$ for some
$C>0$. This means, that the
statement $\sigma (\Pi) \cap \bbT = \emptyset$ is equivalent to
$ii)$.

By Theorem~\ref{SPT}, $i)$ is equivalent to the statement $\sigma
({T})\cap\bbT = \emptyset$ on $\co$. From \cite{LMS1} (Theorem 2.5, (1)
$\Leftrightarrow$ (2)), we conclude that $\sigma ({T})\cap\bbT =
\emptyset$ on $\co$ is equivalent to
$\sigma (\hat{T})\cap\bbT = \emptyset$ on $C_0(\bbR,\co)$.  Indeed,
for $f\colon\bbR\to C_0(\TH,X)$, $\hat{T}$ acts as $(\hat{T} f) (s) =
Tf(s-1)$, and this is exactly the case considered in Theorem 2.5 of
\cite{LMS1}. Hence, $i)$ is equivalent to $\sigma (\hat{T})\cap\bbT =
\emptyset$.  Equation \eqref{5-1} shows that
$\sigma(\Pi) = \sigma(\hat{T})$, and hence $i)\Leftrightarrow ii)$.
\end{pf}

\subsection{Open Problems and Concluding Remarks}

\begin{rem}\label{LP} An  open problem is to prove
analogues of Theorems~\ref{SMT} and \ref{SPT} in the
$L_p$-setting:  Let $\mu$ be a Borel measure on $\TH$, positive on
open sets, and quasi-invariant with respect to the flow $\vphi^t$.
The semigroup \eqref{evolSG} could be replaced by
\[(T^tf)(\th)=\left(\dfrac{d\mu\circ \vphi^t}{d\mu}(\th)\right)^{1/p}
\Phi(\vphi^{-t}\th,t)f(\vphi^{-t}\th),\quad f\in L_p(\TH,\mu,X).\]
The proof of Theorem~\ref{SMT} would require minor changes.
However, the proof of Lemma~\ref{l4.3}
should be modified essentially.
Theorem~\ref{SPT} has been proven for the
strongly continuous line setting for $L_p$ \cite{LMS1,LatRand};
in this setting, the statement $i)\Leftrightarrow iii)$
is proved  in \cite{Rab} using an
interesting alternative $C_0$-semigroup approach.
\end{rem}

\begin{rem}\label{Generators} It would be interesting to
determine under which conditions on the cocycle $\Phi$ and in which
sense (in strongly continuous locally compact setting) the
generator $\Gamma$ is given by formula \eqref{genucc} (see
\cite{Nag2,NagRh} for the line setting).
\end{rem}

\begin{rem}\label{poinwise-global} We conjecture that
the condition \eqref{unbddpix} in Theorem~\ref{SPT} and
the corresponding inequalities in Corollary~\ref{pntw}
and Theorem~\ref{t5.3}  are
redundant, at least for the norm continuous compact setting.
This means that the LSPF is exponentially dichotomic on
$\TH$ if and only if it is exponentially dichotomic at every
$\th\in \TH$. This conjecture is
true for finite dimensional $X$ \cite[Lemma 2A]{SSSpT} and for the
norm continuous compact setting where $X$ is a Hilbert space
\cite{LS}. The corresponding algebraic question here is whether the
set of representations $\{\pi_\th\}_{\th\in \TH}$ of the algebra $\frakB$ is
sufficient, that is, whether $b\in\frakB$ is invertible if and only if
$\pi_\th(b)$ is invertible for all $\th\in \TH$.
\end{rem}

\begin{rem}\label{epstraj} A problem related
to Remark~\ref{poinwise-global} is to
consider, instead of $\pi_\th(T)$, weighted shift operators
$\pi_{\bar{\th}}(T)$ along
$\epsilon$-trajectories $\bar{\th}$. Recall,
that a sequence $\bar{\th}=\{\th_n\}_{n\in\bbZ}$ is called an
$\epsilon$-trajectory for $\vphi=\vphi^1$ if $\text{dist}
(\vphi \th_n,\th_{n+1})\leq \epsilon$ for $n\in\bbZ$. The operator
$\pi_{\bar{\th}}(T)$ can be defined on $l_\infty(\bbZ,X)$ as
$\pi_{\bar{\th}}(T)=\diag\{\Phi(\th_{n-1},1)\}_{n\in\bbZ} S$. We
suspect that the LSPF has exponential dichotomy over $\TH$ if and
only if $\pi_{\bar{\th}}(T)$ is hyperbolic for all
$\epsilon$-trajectories  with sufficiently small $\epsilon$.
\end{rem}

\begin{rem}\label{semiaxis} An area open for investigation
is the case when $\Phi$ is a (semi)cocycle over a {\it semiflow}
$\{\vphi^t\}_{t\in\bbR_+}$. For the line setting, this
corresponds to the dichotomy on the semiaxis.
\end{rem}

\begin{rem}\label{stabil} We showed in this paper that
exponential dichotomy persists under small perturbations of the
cocycle. A theorem by R.~Sacker and G.~Sell \cite[Thm. 6]{SSSpT}
says that dichotomy persists under ``small'' perturbations of a
$\vphi^t$-invariant compact subset $\TH_0\subset \TH$. The proof in
\cite{SSSpT} is essentially finite-dimensional. It would be of
interest to know whether the theorem holds in the strongly continuous
setting.  We note that Remark~\ref{epstraj} helps to prove the
theorem for the uniformly continuous setting when $X$ is a
Hilbert space (in preparation).
\end{rem}

\begin{rem}\label{Oseledets} We note that Lyapunov numbers
(see \cite{ChLe1,JPS,SSSpT}) for the cocycle $\Phi$ belong to the
dynamical spectrum $\Sigma=\sigma(\Gamma)\cap\bbR$. As proven
in \cite{JPS} for the finite dimensional setting and in \cite{LS}
for the uniformly continuous setting on a Hilbert space $X$, the
boundaries of $\Sigma$ can be computed via the {\it exact}
Lyapunov-Oseledets exponents given by
the multiplicative ergodic theorem. This theorem is now available
for Banach spaces \cite{Mane}. A natural question then is to
characterize the boundaries of $\Sigma$ for the compact-valued
cocycles on a Banach space.
\end{rem}

\begin{rem}\label{Nonperiodic} Another natural question is to relax
our main assumption that the aperiodic trajectories of $\vphi^t$ are
dense in $\TH$. Without this assumption the Spectral Mapping Theorem
does not hold (see \cite{CS} and \cite{LS}). Let $p(\th)=\inf\{t:
\vphi^t\th=\th\}$ denote the prime period of $\th\in \TH$, and set
\[p_0(\th):=\inf_{U} \sup_{y\in U} p(y),\]
 where $U$ denotes an open set containing $\th$.
Let $\calH(S)$ denote the union of the circles, centered at
origin, intersecting a set $S\subset\bbC$.

Assume $p_0(\th)\geq c>0$ for all $\th\in \TH$ and some $c$. We conjecture
that the following Annular Hull Theorem (see \cite{CS}) is valid:
\[ \exp {t\sigma (\Gamma)} \subset \sigma(T^t)\setminus \{0\}
\subset \calH(\exp {t\sigma (\Gamma)}).\]
There are infinite-dimensional counterexamples (similar to one in
\cite{Mont})
 showing the theorem
fails without the assumption $p_0(\th)\geq c>0$.

To prove this conjecture it might be helpful to consider
the following function:
\[y(\th)=\int\limits_0^{p_0(\th)}\rho(t/p_0(\th))(T^tw)(\th)+
(1-\rho(t/p_0(\th)))(T^{t+p_0(\th)}w)(\th) dt,\]
instead of \eqref{defing} in the proof of the Spectral Mapping
Theorem. The function $\rho:[0,1]\to [0,1]$ here (see \cite[p.~46]{LMS2}) is a
smooth function such that that $\rho(\tau)=0$ for $\tau\in [0,1/3]$ and
$\rho(\tau)=1$ for $\tau\in [2/3,1]$.
\end{rem}

\section{Acknowledgments}

Y.~Latushkin is grateful to Professor Shui-Nee Chow,
Professor Jack Hale, and Professor Yingfei Yi for
the  suggestions, questions, and discussions during
his visit to Georgia Tech in 1994, and to Hugo Leiva for
helpful discussions on Remarks~\ref{poinwise-global} and
\ref{stabil}.



%--------------- Bibliography ----------

\begin{thebibliography}{99}

\bibitem{Ant} A.~B.~Antonevich,
Two methods for investigating the invertibility of operators from
$C^*$-algebras generated by dynamical systems,
{\em Math.~USSR-Sb}
{\bf 52} (1985), 1--20.

\bibitem{BAG}
H.~Bart, I.~Gohberg, and M.~A.~Kaashoek,
 Wiener-Hopf factorization,
inverse Fourier transform and
exponentially dichotomous operators,
{\em J.~Funct.~Anal.}
{\bf  68} (1986),1-42.

\bibitem{Gohb} A.~Ben-Artzi and I.~Gohberg,
 Dichotomy of systems and
invertibility of linear ordinary
differential operators,
{\em Oper.~Theory Adv.~Appl.}
{\bf  56} (1992), 90--119.

%\bibitem{CLat} C.~Chicone and Y.~Latushkin,
% Quadratic Lyapunov functions for
%linear skew-product flows and weighted
%composition operators,
%J.~Int.~Diff.~Eqns., to appear.

\bibitem{CS} C.~Chicone and R.~Swanson,
 Spectral theory for linearization of dynamical systems,
{\em J.~Differential Equations}
{\bf 40} (1981), 155--167.

\bibitem{CLMS} C.~Chicone, Y.~Latushkin and S.~Montgomery-Smith,
 The spectrum of the kinematic dynamo operator for an ideally
conducting fluid,  submitted.

\bibitem{ChLinLu} S.-N.~Chow, X.-B.~Lin, and K.~Lu,
 Smooth invariant foliations in infinite dimensional spaces,
{\em J.~Differential Equations}
{\bf 94} (1991), 266--291.

\bibitem{ChLu} S.-N.~Chow, and K.~Lu,
 Invariant manifolds for flows in Banach spaces,
{\em J.~Differential Equations}
{\bf 74} (1988), 285--317.

\bibitem{ChLe1} S.-N. Chow and H. Leiva,
 Existence and roughness of the exponential dichotomy for
linear skew-product semiflows in Banach spaces,
{\em J.~Differential Equations},
to appear.

\bibitem{ChLe2} S.-N. Chow and H. Leiva,
 Two definitions of the exponential dichotomy for
 skew-product semiflow in Banach spaces,
preprint.

\bibitem{ChLe3} S.-N. Chow and H. Leiva,
Dynamical spectrum for time dependent linear systems in Banach spaces,
{\em Japan J.~Industr. Appl. Math.}, {\bf 11}, no. 3 (1994).


\bibitem{DK} J.~Daleckij and M.~Krein,
``Stability of Differential Equations in Banach Space,"
Amer.~Math.~Soc., Providence, RI, 1974.

\bibitem{Evans}
D.~E.~Evans,  Time dependent perturbations and scattering
of strongly continuous groups on Banach spaces,
{\em Math.~Ann.}
{\bf 221} (1976), 275--290.

%\bibitem{GohLeit}  I.~Gohberg and J.~Leiterer,
%Factorization of operator functions with respect to a
%contour. II. Factorization of operator functions close to the
%identity,  {\em Math. Nach.} {\bf 54} (1973), 41--74.

\bibitem{Gold} J. Goldstein,
Asymptotics for bounded semigroups
on Hilbert space, {\em in} ``Aspects of Positivity in Functional
Analysis," Mathematics Studies, Vol.~122  (1986).

\bibitem{Hale} J.~Hale,
``Asymptotic Behavior of Dissipative Systems,"
Math.~Surveys Mongoraphs Vol.~25,
Amer.~Math.~Soc., Providence, RI, 1988.

\bibitem{HL} J.~Hale and S.~M.~Verduyn Lunel,
``Introduction to Functional Differential Equations,''
Appl. Math. Sci. {\bf 99}, Springer-Verlag, New York, 1993.

\bibitem{Henry} D.~Henry,
``Geometric Theory of Nonlinear Parabolic Equations,"
Lecture Notes in Math., no.~840,  Springer-Verlag, New York, 1981.

\bibitem{Henry1} D.~Henry,
``Topics in analysis,"
{\em Publ.~Mat.}
{\bf 31} (1) (1987), 29--84.

\bibitem{Howland} J.~S.~Howland,
Stationary scattering theory for time-dependent hamiltonians,
{\em Math.~Ann.}
{\bf 207} (1974), 315--335.

\bibitem{J} R.~Johnson,
Analyticity of spectral subbundles,
{\em J.~Diff.~ Eqns.}
{\bf 35} (1980), 366--387.

\bibitem{JPS}  R. Johnson, K. Palmer, and G. Sell,
Ergodic properties of linear dynamical systems,
{\em SIAM J.~Math.~Anal.}
{\bf 1}, no.~1 (1987), 1--33.

\bibitem{LATGDDE} Y.~Latushkin,
Green's function, continual weighted composition operators along
trajectories, and hyperbolicity of linear extensions for dynamical
systems,
{\em J.~Dynamics  Differential Equations}
{\bf 6} (1) (1994), 1--21.

\bibitem{LMS1} Y.~Latushkin and S.~Montgomery-Smith,
Evolutionary semigroups and Lyapunov theorems in Banach spaces,
{\em J.~Funct.~Anal.},
to appear.

\bibitem{LMS2} Y.~Latushkin and S.~Montgomery-Smith,
Lyapunov theorems for Banach spaces,
{\em Bull.~Amer.~Math.~Soc.~(N.S.)},
{\bf 31}, no.~1 (1994), 44--49.

\bibitem{LatRand} Y.~Latushkin and T.~Randolph,
Dichotomy of differential equations on Banach spaces and an
algebra of weighted composition operators,
submitted.

\bibitem{LS} Y.~Latushkin and A.~M.~Stepin,
Weighted composition operators and linear extensions of dynamical
systems,
{\em Russian Math.~Surveys},
{\bf 46}, no.~2 (1992), 95--165.

\bibitem{Lin} X.-B.~Lin,
Exponential dichotomies in intermediate spaces with applications
to a diffusively perturbed predator-prey model,
{\em J.~Differential Equations}
{\bf 108} (1994), 36--63.

\bibitem{Lum}  G.~Lumer,
Equations de diffusion dans le domaines $(x,t)$ non-cylindriques et
semigroupes ``espace-temps",
Lecture Notes in Math., no.~1393,
Springer-Verlag, New York, 1989, 161--179.

\bibitem{Mag} L.~T.~Magalh\~aes,
Persistence and smoothness of hyperbolic
invariant manifold for functional differential equations,
{\em SIAM J.~Math.~Anal.},
{\bf 18} (3) (1987), 670--693.

\bibitem{Mane} R.~M\~an\'e,
Lyapunov exponents and stable manifolds for compact transformations,
{\em in} ``Geometric Dynamics," J.~Palis (ed.),
Lecture Notes in Math., no.~1007,  Springer-Verlag, New York, 1983,
522--577.

\bibitem{Mather} J.~Mather,
Characterization of Anosov diffeomorphisms,
{\em Indag.~Math.},
{\bf 30} (1968), 479--483.

\bibitem{MS} J.~Massera and J.~Schaffer,
``Linear Differential Equations and Function Spaces,"
Academic Press, NY, 1966.

\bibitem{Mont} S.~Montgomery-Smith,
Stability and dichotomy of positive semigroups on $L_p$,
preprint.

\bibitem{Nag} R.~Nagel (ed.)
``One Parameters Semigroups of Positive Operators",
Lecture Notes in Math., no.~1184, Springer-Verlag, Berlin, 1984.

\bibitem{Nag2} R.~Nagel,
Semigroup methods for non-autonomous Cauchy problems,
{\em Tubingen Berichte zur Funktionalanalysis},
Heft 2, Jahrang 1992/93, (1993) 103--118.

\bibitem{NagRh} R.~Nagel and A.~Rhandi,
A characterization of Lipschitz continuous evolution families on
Banach spaces,
{\em Integral Equations Operator Theory},
to appear.

\bibitem{Nei} H.~Neidhardt,
On abstract linear evolution equations, I,
{\em Math.~Nachr.}
{\bf 103} (1981), 283--298.

\bibitem{Palm} K. Palmer,
Exponential dichotomy and Fredholm operators,
{\em Proc.~Amer.~Math.~Soc.},
{\bf 104} (1988), 149-156.

\bibitem{Palm2} K. Palmer,
Exponential dichotomies and transversal homoclinic points,
{\em J.~Differential Equations}
{\bf 55} (1984), 225--256.

\bibitem{Pazy} A.~Pazy,
``Semigroups of Linear Operators and Applications to Partial
Differential Equations,"
Springer-Verlag, N.Y./Berlin, 1983.

\bibitem{Pesin}  Ya.~B.~Pesin,
``Hyperbolic theory,"
{\em in} Encyclop.~Math.~Sci.~Dynamical Systems {\bf 2} (1988).

\bibitem{Pruss} J.~Pr\"uss,
On the spectrum of $C_0$-semigroups,
{\em Trans.~Amer.~Math.~Soc.}
{\bf  284}, no. 2 (1984), 847--857.

\bibitem{Rab} F.~R\"{a}biger and R.~Schnaubelt,
A spectral characterization of exponentially dichotomic and hyperbolic
evolution families,
preprint.

\bibitem{Rau}  R.~Rau,
Hyperbolic linear skew-product semiflows,
preprint.

\bibitem{Rau1}  R.~Rau,
Hyperbolic evolutionary semigroups on vector-valued function spaces,
{\em Semigroup Forum},
{\bf 48} (1994), 107--118.

\bibitem{Rau2} R.~Rau,
Hyperbolic evolution groups and exponential dichotomy of evolution
families,
{\em J.~Funct.~Anal.}, to appear.

\bibitem{SSDich} R.~Sacker and G.~Sell,
Existence of dichotomies and invariant splitting for linear
differential systems, I,II,III
{\em J.~Differential Equations}
{\bf 15, 22} (1974,1976) 429--458, 478--522.

\bibitem{SSSpT}  R.~Sacker and G.~Sell,
A spectral theory for linear differential systems,
{\em J.~Differential Equations}
{\bf  27} (1978), 320--358.

\bibitem{SSBan} R.~Sacker and G.~Sell,
``Dichotomies for Linear Evolutionary Equations in Banach Spaces,"
IMA Preprint No.~838 (1991).

\bibitem{Sam} Y.~A.~Mitropolskij, A.~M.~Samojlenko, and V.~L.Kulik,
Issledovanija Dichotomii Lineinyh Sistem  Differencial'nyh Uravnenij,
Russian (Dichotomy of Systems of Linear Differential Equations),
Naukova Dumka, Kiev, 1990.

\bibitem{Sel} J.~Selgrade,
Isolated invariant sets for flows on vector bundles,
{\em Trans.~Amer.~Math.~Soc.}
{\bf 203} (1975) 359--390.

\bibitem{Tan}  H.~Tanabe,
``Equations of Evolution,"
Pitman, London, 1979.

\bibitem{We} L.~Weis,
The stability of positive semigroups on $L_p$ spaces,
{\em Proc.~Amer.~Math.~Soc.}, to appear.

\end{thebibliography}

\end{document}







